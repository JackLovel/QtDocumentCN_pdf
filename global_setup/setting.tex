\ctexset{section/format += \raggedright}

% \newfontfamily\consolas{Fira Code}
\newfontfamily\consolas{Source Code Pro}
% \newfontfamily\consolas{JetBrainsMono-Regular}

% green
\definecolor{mygreen}{rgb}{0, 0.6, 0}
% 设定版本宽度和高度
\geometry{papersize={35cm, 25cm}}
% 页边距
\geometry{left=2.5cm,right=2.5cm,top=2.0cm,bottom=2cm}

% 页眉页脚
\pagestyle{empty}

% 设定代码环境
% \newfontfamily\fira{Fira Mono}
\newfontfamily\lstFont{Sarasa Fixed CL Light}
\lstset{
	basicstyle=\small\lstFont,
	showstringspaces=false, % 支掉空体系时产生的下划线的空格标志
	numberstyle=\tiny\lstFont, 
	keywordstyle=\color{blue!70},
	commentstyle=\color{mygreen},
	frame=shadowbox, 
	rulesepcolor= \color{ red!20!green!20!blue!20},
	columns=flexible, 
	linewidth=.8\linewidth,
	numbers=left,
	tabsize=2, % set tab size= 2 space
	% 整体距左侧边线的距离为2em
	% xleftmargin=2em,
	% xrightmargin=2em,
} 


% 定义表格列宽
\newcolumntype{L}[1]{>{\vspace{0.5em}\begin{minipage}{#1}\raggedright\let\newline\\
\arraybackslash\hspace{0pt}}m{#1}<{\end{minipage}\vspace{0.5em}}}
\newcolumntype{R}[1]{>{\vspace{0.5em}\begin{minipage}{#1}\raggedleft\let\newline\\
\arraybackslash\hspace{0pt}}m{#1}<{\end{minipage}\vspace{0.5em}}}
\newcolumntype{C}[1]{>{\vspace{0.5em}\begin{minipage}{#1}\centering\let\newline\\
\arraybackslash\hspace{0pt}}m{#1}<{\end{minipage}\vspace{0.5em}}}

% 中文设置
\setCJKmainfont{SourceHanSerifCN-Light}

% 英文设置
% \setmainfont{JetBrains Mono}
\setmainfont{Source Code Pro}

% 标题页
\title{Qt中文文档翻译}
\author{作者:\href{https://github.com/QtDocumentCN}{QtDocumentCN}}
\ctexset{ today = small }
% 标题左对齐
\CTEXsetup[format={\Large}]{section}
% \ctexset{ section = {format = {\Large}}
\ctexset{
% chapter/format = \sffamily\raggedright,
section/format += \sffamily\Large,
% subsection/format += \fbox,
}
% 首行缩进两个字符
%\setlength{\parindent}{2em}
%\CTEXindent 
\CTEXnoindent % 取消首行缩进

\setlength{\headheight}{20pt}
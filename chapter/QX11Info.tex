\chapter{QX11Info}

QX11Info


提供有关X11相关的相关配置信息(就是linux下的x11相关的配置信息

\begin{tabular}{|r|l|}
	\hline
	属性 & 方法 \\
	\hline
	头文件 & \#include <QX11Info>\\      
	\hline
	qmake & QT += x11extras\\      
	\hline
    Since: & Qt5.1\\
    \hline
\end{tabular}


简述

\begin{tabular}{|r|l|}
\hline
类型 & 函数名 \\
\hline
int	& appDpiX(int screen = -1)\\
\hline
int	& appDpiY(int screen = -1) \\
\hline
unsigned long&	appRootWindow(int screen = -1) \\
\hline
int	&appScreen() \\
\hline
unsigned long&	appTime() \\ 
\hline
unsigned long&	appUserTime() \\
\hline
xcb\_connection\_t *	& connection() \\ 
\hline
Display *	&display()\\
\hline
unsigned long &	getTimestamp()\\
\hline
bool &	isCompositingManagerRunning(int screen = -1)\\
\hline
bool &	isPlatformX11() \\ 
\hline
QByteArray	& nextStartupId() \\
\hline
void &	setAppTime(unsigned long time)\\
\hline
void&	setAppUserTime(unsigned long time)\\
\hline
void&	setNextStartupId(const QByteArray \&id)\\
\hline
\end{tabular}

详细说明
该类提供了关于 x window相关的显式配置信息

该类提供了两类API:一种是提供特定的widget或者特定的pixmap相关的非静态函数,一种是为应用程序提供默认信息的静态函数。(这个分类简直了!!!)

成员函数

int QX11Info::appDpiX(int screen = -1) static函数

返回指定屏幕的水平分辨率。

参数screen是指哪个x屏幕(比如两个的话,第一个就是0,第二个就是1)。请注意,如果用户使用的系统是指Xinerama(而不是传统的x11多屏幕),则只有一个x屏幕。请使用QDesktopWidget来查询有关于Xinerama屏幕的信息。

另参阅apDipY();

int QX11Info::appDpiY(int screen = -1) static函数

返回指定屏幕的垂直分辨率。

参数screen是指哪个x屏幕(比如两个的话,第一个就是0,第二个就是1)。请注意,如果用户使用的系统是指Xinerama(而不是传统的x11多屏幕),则只有一个x屏幕。请使用QDesktopWidget来查询有关于Xinerama屏幕的信息。

另参阅apDipX();

unsigned long QX11Info::appRootWindow(int screen = -1) static函数

返回指定屏幕应用程序窗口的句柄

参数screen是指哪个x屏幕(比如两个的话,第一个就是0,第二个就是1)。请注意,如果用户使用的系统是指Xinerama(而不是传统的x11多屏幕),则只有一个x屏幕。请使用QDesktopWidget来查询有关于Xinerama屏幕的信息。

int QX11Info::appScreen() static函数

返回应用程序正在显示的屏幕编号。 此方法是指每个原始的X11屏幕使用不同的DISPLAY环境变量。只有当您的应用程序需要知道它在哪个X屏幕上运行时,这个信息才有用。 在典型的多个物理机连接到一个X11屏幕中时。意味着这个方法对于每台物理机来讲都是相同的编号。在这样的设置中,如果您对X11的RandR拓展程序感兴趣,可以通过QDesktopWidget和QScreen获得。

unsigned long QX11Info::appTime() static函数

返回X11的时间

unsigned long QX11Info::appUserTime() static函数

返回X11的用户时间

xcb\_connection\_t *QX11Info::connection() static函数

返回应用程序默认的XCB信息。

Display *QX11Info::display() static函数

返回应用程序默认的显式屏幕

unsigned long QX11Info::getTimestamp() static函数

从X服务器上获取当前X11的时间戳。 此方法创建一个事件来阻塞住X11服务器,直到它从X服务器接受回来。 这个函数是从Qt5.2中引入的。

bool QX11Info::isCompositingManagerRunning(int screen = -1) static函数

如果屏幕的合成管理器在运行时,则返回 true (ps,合成管理器运行会有一些特殊的效果,比如一些透明色的绘制,可以用这个函数判断下。),否则则返回 false。 这个函数是从Qt5.7中引入的。

bool QX11Info::isPlatformX11() static函数

如果应用程序运行在X11上则返回true。 这个函数是从Qt5.2开始引入的。

QByteArray QX11Info::nextStartupId()

返回此进程显式的下一个窗口的启动ID。 显式下一个窗口后,下一个启动ID则为空。

(Qt官网很少给这种链接啊) http://standards.freedesktop.org/startup-notification-spec/startup-notification-latest.txt

这个函数在Qt5.4引入。

void QX11Info::setAppTime(unsigned long time) static函数

将X11时间设置成指定的值。

void QX11Info::setAppUserTime(unsigned long time) static函数

设置X11用户的时间

void QX11Info::setNextStartupId(const QByteArray \&id) static函数

设置下一个启动程序的ID。 第一个窗口的启动ID来自环境变量DESKTOP\_STARTUP\_ID。当请求来自另一个进程(比如通过QDus)时,此方法对于后续窗口很有用。

这个函数是从Qt5.4中引用的。






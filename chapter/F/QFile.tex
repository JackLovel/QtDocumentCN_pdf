\chapter{QFile}

\begin{tabular}{|l|l|}
\hline
属性 	&方法\\
\hline
Header: &	\#include <QFile>\\
\hline
qmake: &	QT += core\\
\hline
Inherits: &	QFileDevice\\
\hline
Inherited By:& 	QTemporaryFile\\
\hline
\end{tabular}

\begin{compactitem}
\item 包含继承成员的成员列表
\item 废弃的成员
\end{compactitem}

\begin{notice}
	类中所有函数都是 可重入的。
\end{notice}
 
\splitLine

\section{公共成员类型}

\begin{tabular}{|l|l|}
\hline
类型& 	方法\\
\hline
typedef& 	DecoderFn \\
\hline
\end{tabular}

\splitLine

\section{公共成员函数}

\begin{longtable}[l]{|l|l|}
\hline
类型& 	方法\\
\hline
	&QFile(const QString \&name, QObject *parent)\\
\hline
	&QFile(QObject *parent)\\
\hline
	&QFile(const QString \&name)\\
\hline
	&QFile()\\
\hline
virtual& 	$\sim$QFile()\\
\hline
bool& 	copy(const QString \&newName)\\
\hline
bool& 	exists() const\\
\hline
bool& 	link(const QString \&linkName)\\
\hline
bool& 	moveToTrash()\\
\hline
bool& 	open(FILE *fh, QIODevice::OpenMode mode,
      QFileDevice::FileHandleFlags handleFlags = DontCloseHandle)\\
\hline
bool& 	open(int fd, QIODevice::OpenMode mode, QFileDevice::FileHandleFlags handleFlags = DontCloseHandle)\\
\hline
bool& 	remove()\\
\hline
bool& 	rename(const QString \&newName)\\
\hline
void& 	setFileName(const QString \&name)\\
\hline
QString& 	symLinkTarget() const\\
\hline
\end{longtable}

\splitLine

\section{重写公共函数}


\begin{tabular}{|l|l|}
\hline
类型& 	方法\\
\hline
virtual& QString 	fileName() const override\\
\hline
virtual& bool 	open(QIODevice::OpenMode mode) override\\
\hline
virtual& QFileDevice::Permissions 	permissions() const override\\
\hline
virtual& bool 	resize(qint64 sz) override\\
\hline
virtual& bool 	setPermissions(QFileDevice::Permissions permissions)
         override\\
\hline
virtual qint64 &	size() const override\\
\hline
\end{tabular}

\splitLine

\section{静态公共成员}

\begin{tabular}{|l|l|}
\hline
类型& 	方法\\
\hline
bool& 	copy(const QString \&fileName, const QString \&newName)\\
\hline
QString& 	decodeName(const QByteArray \&localFileName)\\
\hline
QString& 	decodeName(const char *localFileName)\\
\hline
QByteArray& 	encodeName(const QString \&fileName)\\
\hline
bool& 	exists(const QString \&fileName)\\
\hline
bool& 	link(const QString \&fileName, const QString \&linkName)\\
\hline
bool& 	moveToTrash(const QString \&fileName, QString *pathInTrash =
      nullptr)\\
\hline
QFileDevice::Permissions &	permissions(const QString \&fileName)\\
\hline
bool& 	remove(const QString \&fileName)\\
\hline
bool& 	rename(const QString \&oldName, const QString \&newName)\\
\hline
bool& 	resize(const QString \&fileName, qint64 sz)\\
\hline
bool& 	setPermissions(const QString \&fileName, QFileDevice::Permissions permissions)\\
\hline
QString& 	symLinkTarget(const QString \&fileName)\\
\hline
\end{tabular}


\splitLine

\section{详细描述}

QFile 是用于读写文本及二进制的文件及资源的I/O设备。 一个QFile可以单独使用,或者更简单的,可以与 QTextStream 或 QDataStream 一同使用。

文件名通常在构造时传递,但也可以在随时使用 setFileName()设置。QFile 需要目录分隔符为 '/' 而不是依照操作系统。其他分隔符 (如 '\') 不受支持。

您可以通过 exists() 判断文件是否存在。(更多操作系统相关的操作在 QFileInfo 和 QDir 中提供)

文件通过 open() 打开,通过 close() 关闭,通过 flush() 刷新。数据通常使用 QDataStream or QTextStream 读写,但您也可以使用 由 QIODevice 的继承函数 read(), readLine(), readAll(), write()。单字符的操作也可以使用 getChar(), putChar(), and ungetChar()。

size() 返回文件大小。您可以通过 pos() 获取当前文件位置,或通过 seek()
移动到新的位置(译者注:此句中的“位置”指文件内操作的字节位置)。当您读
到文件结尾, atEnd() 返回 \hl{true}。

直接读文件

如下例子逐行地直接读取文本文件:

\begin{cppcode}
QFile file("in.txt");
if (!file.open(QIODevice::ReadOnly | QIODevice::Text))
    return;

while (!file.atEnd()) {
     QByteArray line = file.readLine();
     process_line(line);
}
\end{cppcode}

QIODevice::Text flag传递给 open() ,其告诉Qt将Windows风格的换行符
("$\backslash$r$\backslash$n") 转换为 C++风格的换行符("$\backslash$n")。默认情况下,QFile 假设为二进制模式读取,不做字节转换。

通过流来读文件

如下例子逐行地通过 QTextStream 读取文本文件:

\begin{cppcode}
    QFile file("in.txt");
    if (!file.open(QIODevice::ReadOnly | QIODevice::Text))
        return;

    QTextStream in(&file);
    while (!in.atEnd()) {
        QString line = in.readLine();
        process_line(line);
    }
\end{cppcode}

QTextStream 会特意把8位文件中字节数据转换为QString中16位UTF-16字符。默认情况下,其假设用户使用系统默认编码(例如unix平台上是UTF-8;详情参看 QTextCodec::codecForLocale() )。编码可以通过 QTextStream::setCodec() 改变。

要写入文本,您可以使用左移运算符运算符 operator<<(),在 QTextStream 中,
其重载用于讲右侧内容追加的左侧,示例如下:

\begin{cppcode}
    QFile file("out.txt");
    if (!file.open(QIODevice::WriteOnly | QIODevice::Text))
        return;

    QTextStream out(&file);
    out << "The magic number is: " << 49 << "\n";

\end{cppcode}

QDataStream 和上文很相似,详情请见相当应类的文档。

当你使用 QFile, QFileInfo 以及 QDir 来访问系统中文件,你可以使用Unicode文件名。在Unix平台,文件名会转换为8位编码。如果您想使用C++标准API (<cstdio> 或 <iostream>) 或平台相关API来访问文件而不是使用 QFile,你可以使用 encodeName() 和 decodeName() 来在Unicode文件名和8位文件名间转换。

在Unix平台,有一些特殊的系统文件 (例如 /proc 下的文件) ,对于这些文件,
size() 会返回0,但你依然可以读取更多数据;这些数据在你调用 read() 时即
时产生。在这种情况下,您便不能使用 atEnd() 来判断是否已经没有更多数据。
(因为 atEnd() 通过文件大小是否到达结尾)。然而您可以通过连续调用
readAll(), read() 或 readLine() 指导没有数据来替代此功能。下面的例子使
用 QTextStream 逐行读取 \hl{/proc/modules}:

\begin{cppcode}
    QFile file("/proc/modules");
    if (!file.open(QIODevice::ReadOnly | QIODevice::Text))
        return;

    QTextStream in(&file);
    QString line = in.readLine();
    while (!line.isNull()) {
        process_line(line);
        line = in.readLine();
    }
\end{cppcode}

\textbf{信号}

不像其他 QIODevice 的实现,例如 QTcpSocket,QFile 不会发出 aboutToClose(), bytesWritten(), 及 readyRead() 这些信号。这个实现细节意味着 QFile 不适用于读写某些特定类型的文件,比如Unix上的设备文件。

\textbf{平台相关信息}

文件权限和Unix和Windows上的处理并不相同。在Unix平台上,一个非 可写入 的目录,文件无法创建。但对于Windows并不一定如此,例如 'My Documents' (我的文档)目录通常不可写入,但是在其中依然可以创建文件。

Qt对于文件权限的理解有局限,尤其对于 QFile::setPermissions() 有影响。在Windows上,仅当没有任何 Write* flags被设置时,Qt 会设置旧版的只读 flag。Qt不会操作访问过滤表(access control lists , ACLs)这是的此函数在NTFS卷上基本上没什么用。对于VFAT格式的U盘,倒是有可能可用。POSIX 的 ACLs 也不会被修改。


\begin{seeAlso}
QTextStream, QDataStream, QFileInfo, QDir, 以及 The Qt
Resource System。
\end{seeAlso} 

\splitLine

\section{成员类型文档}

typedef QFile::DecoderFn

这个类型定义了一个如下形式的函数的指针:


\begin{cppcode}
QString myDecoderFunc(const QByteArray &localFileName);
\end{cppcode}


\begin{seeAlso}
setDecodingFunction().
\end{seeAlso} 

\splitLine

\section{成员函数文档}

QFile::QFile(const QString \emph{\&name}, QObject \emph{*parent})

构造基于给定的父对象 parent 、文件名 name 指定的QFile对象。

QFile::QFile(QObject *parent)

构造基于给定的父对象 parent 的QFile对象。

QFile::QFile(const QString \emph{\&name})

构造文件名 name 指定的QFile对象。

QFile::QFile()

构造一个QFile对象。

[virtual]QFile::$\sim$QFile()

销毁此QFile对象,如需要会自动关闭文件。

bool QFile::copy(const QString \emph{\&newName})

将当前 fileName() 指定的文件复制为文件名 newName 指定的文件。如果成功,返回 true ;否则返回 false。


\begin{notice}
如果 newName 文件名的文件已存在,函数不会覆盖,直接返回 false 。
\end{notice} 

源文件会在复制前关闭。

\begin{seeAlso}
setFileName().
\end{seeAlso} 

[static]bool QFile::copy(const QString \emph{\&fileName}, const QString \emph{\&newName})

这是一个重载函数。

将文件 fileName 复制为文件名 newName。如果成功,返回 true ;否则返回 false。


\begin{notice}
如果 newName 文件名的文件已存在,函数不会覆盖,直接返回 false 。
\end{notice} 

\begin{seeAlso}
rename().
\end{seeAlso} 

\hl{[static]}QString QFile::decodeName(const QByteArray \emph{\&localFileName})

和 QFile::encodeName() 操作恰恰相反。返回 localFileName 的Unicode形式。


\begin{seeAlso}
encodeName().
\end{seeAlso} 

\hl{[static]}QString QFile::decodeName(const char \emph{*localFileName})

这是一个重载函数。返回 localFileName 的Unicode形式。


\begin{seeAlso}
 encodeName() 。
\end{seeAlso} 


[static]QByteArray QFile::encodeName(const QString \emph{\&fileName})

基于用户区域设置,将 fileName 转换为本地的8为表示。这对于用户选择的文件名足够使用。硬编码到程序中的文件名应当只使用7位ASCII字符。


\begin{seeAlso}
decodeName().
\end{seeAlso} 

[static]bool QFile::exists(const QString \&fileName)

如果 fileName 对应的文件存在,返回 true 否则返回 false。



\begin{notice}
如果 fileName 是指向不存在文件的符号链接,返回 false。
\end{notice} 

bool QFile::exists() const

这是一个重载函数。

如果 fileName() 对应的文件存在,返回 true 否则返回 false。



\begin{seeAlso}
fileName() and setFileName()。
\end{seeAlso} 

[override virtual]QString QFile::fileName() const

重写函数: QFileDevice::fileName() const.

返回 setFileName() 或构造函数设置的文件名。


\begin{seeAlso}
setFileName() and QFileInfo::fileName().
\end{seeAlso} 

bool QFile::link(const QString \emph{\&linkName})

创建一个名为 linkName 的、指向 fileName() 文件的链接。链接的形式取决于底层文件系统(Windows上的快捷方式或Linux下的符号链接symlink)。如果成功,返回 true ;返回 false。

此函数不会覆盖文件系统上已经存在的链接;如果已存在,link() 将返回 false 并设置 error() 为 RenameError。


\begin{notice}
对于Windows平台,一个合法的链接名称必须包含 .lnk 后缀名。
\end{notice} 


\begin{seeAlso}
setFileName().
\end{seeAlso} 

[static]bool QFile::link(const QString \emph{\&fileName}, const QString \emph{\&linkName})

这是一个重载函数。

创建一个名为 linkName 的、指向 fileName 文件的链接。
链接的形式取决于底层文件系统(Windows上的快捷方式或Linux下的符号链接symlink)。
如果成功,返回 true ;否则返回 false。

\begin{seeAlso}
link().
\end{seeAlso} 

bool QFile::moveToTrash()

将 fileName() 文件移入回收站。如果成功返回 true ,并将 fileName() 设置为回收站中对应文件的路径;否则返回 false。
 

\begin{notice}
在API不能返回回收站中文件的路径的操作系统中,一旦文件被移动 fileName() 会被设置为空字符串。在没有回收站的操作系统,此函数总返回 false。
\end{notice} 

此函数引入自: Qt 5.15.

[static]bool QFile::moveToTrash(const QString \&fileName, QString *pathInTrash = nullptr)

这是一个重载函数。

将 fileName 文件移入回收站。如果成功返回 true ,并将 *pathInTrash 设置为回收站中对应文件的路径字符串的指针;否则返回 false。

\begin{notice}
在API不能返回回收站中文件的路径的操作系统中,一旦文件被移动 *pathInTrash 会被设置为空字符串。在没有回收站的操作系统,此函数总返回 false。
\end{notice} 

此函数引入自: Qt 5.15.

[override virtual]bool QFile::open(QIODevice::OpenMode mode)

重写函数: QIODevice::open(QIODevice::OpenMod
e mode)。

使用 OpenMode mode 模式打开文件,如果成功,返回 true ;否则返回 false。

模式 mode 必须是 QIODevice::ReadOnly, QIODevice::WriteOnly, 或 QIODevice::ReadWrite。也可以有附加flags,例如 QIODevice::Text 和 QIODevice::Unbuffered。

\begin{notice}
在 WriteOnly 或 ReadWrite 模式,如果相关文件不存在,此函数会尝试在打开前新建。
\end{notice} 



\begin{seeAlso}
QIODevice::OpenMode and setFileName()。
\end{seeAlso} 

bool QFile::open(FILE \emph{*fh}, QIODevice::OpenMode \emph{mode}, QFileDevice::FileHandleFlags \emph{handleFlags} = DontCloseHandle)

这是一个重载函数。

使用给出的模式 \emph{mode} 打开已有的文件句柄 \emph{fh}。\emph{handleFlags} 可能会被用于指定附加选项。如果成功,返回 \hl{true};否则返回 \hl{false}。

例如:


\begin{cppcode}
#include <stdio.h>

void printError(const char* msg)
{
    QFile file;
    file.open(stderr, QIODevice::WriteOnly);
    file.write(msg, qstrlen(msg));        // 写入 stderr
    file.close();
}
\end{cppcode}

当一个 QFile 通过此函数被被打开,close() 的行为由 AutoCloseHandle flag决定。如果指定了 AutoCloseHandle ,且此函数执行成功,那么 close() 会关闭传入的句柄。否则,close() 不会关闭文件,只会刷新数据(flush)。

\begin{warning}

\end{warning} 

\begin{compactitem}
\item 如果 \emph{fh} 并非指向常规文件,例如 \hl{stdin}, \hl{stdout}, 或 \hl{stderr},你可能不能够使用 seek(),且size() 返回0。详见 QIODevice::isSequential()。
\item 由于使用此函数打开的文件没有指定文件名,你不能通过 QFileInfo 读取相关信息。
\end{compactitem}

Windows平台的注意事项

当访问文件或其他随机存取设备时,\emph{fh} 必须以二进制模式打开(也就是
\hl{fopen()} 的模式串必须包含'b')。Qt 会转换行末字节如果您指定
QIODevice::Text 给 \emph{
mode}。顺序读取设备,例如标准输入输出,不受此限制影响。

您需要启用控制台应用支持,才能在控制台使用标准输入输出。要启用,可以在
项目文件中加入:

\begin{cppcode}
CONFIG += console
\end{cppcode}

\begin{seeAlso}
close().
\end{seeAlso}

bool QFile::open(int \emph{fd}, QIODevice::OpenMode \emph{mode}, QFileDevice::FileHandleFlags 
\emph{handleFlags} = DontCloseHandle)

这是一个重载函数。

使用给出的模式 mode 打开已有的文件描述符 fh。
handleFlags 可能会被用于指定附加选项。
如果成功,返回 true ;否则返回 false。

当一个 QFile 通过此函数被被打开,close() 的行为由 AutoCloseHandle flag决定。
如果指定了 AutoCloseHandle ,且此函数执行成功,那么 close() 会关闭传入的句柄。
否则,close() 不会关闭文件,只会刷新数据(flush)。

通过此函数打开的文件会被自动设置为 raw 模式;
这意味着文件I/O函数会很慢。如果您遇到了性能问题,可以尝试其他 open() 函数。

\begin{warning}

\end{warning} 

\begin{compactitem}
\item 如果 \emph{fd} 不是一个常规文件,例如 0 (\hl{stdin}), 1 (\hl{stdout}), 或 2 (\hl{stderr}),你可能不能够使用 seek(),且size() 返回0。详见 QIODevice::isSequential()。
\item 由于使用此函数打开的文件没有指定文件名,你不能通过 QFileInfo 读取相关信息。
\end{compactitem}


\begin{seeAlso}
 close()。
\end{seeAlso}

[override virtual]QFileDevice::Permissions QFile::permissions() const

重写函数: QFileDevice::permissions() const.

\begin{seeAlso}
setPermissions().
\end{seeAlso}

[static]QFileDevice::Permissions QFile::permissions(const QString \emph{\&fileName})

这是一个重载函数。

返回 fileName 文件经 OR(位或)后的权限 QFile::Permission 组合。

bool QFile::remove()

删除文件名 fileName() 的文件。

如果成功,返回 true ;否则返回 false。

文件会在删除前关闭。

\begin{seeAlso}
setFileName().
\end{seeAlso}

[static]bool QFile::remove(const QString \emph{\&fileName})

这是一个重载函数。

删除文件名 fileName 的文件。

如果成功,返回 true ;否则返回 false。

\begin{seeAlso}
remove().
\end{seeAlso}

bool QFile::rename(const QString \emph{\&newName})

把文件 fileName() 重命名为 newName。如果成功,返回 true ;否则返回 false。

注意如果 newName 文件名的文件已存在,函数不会覆盖,直接返回 false 。

文件在重命名前关闭。

如果直接重命名失败,Qt会尝试拷贝数据到 newName 新文件并删除旧文件来实现重命名。如果拷贝或删除失败,Qt会撤回新文件创建,返回原先状态。



\begin{seeAlso}
setFileName().
\end{seeAlso}

[static]bool QFile::rename(const QString \emph{\&oldName}, const QString \emph{\&newName})

这是一个重载函数。

把文件 oldName 重命名为 newName。如果成功,返回 true ;否则返回 false。


\begin{notice}
如果 newName 文件名的文件已存在,函数不会覆盖,直接返回 false 。
\end{notice}



\begin{seeAlso}
rename().
\end{seeAlso}

\hl{[override virtual]}bool QFile::resize(qint64 \emph{sz})

重写函数: QFileDevice::resize(qint64 \emph{sz}).

\hl{[static]}bool QFile::resize(const QString \&fileName, qint64 \emph{sz})

这是一个重载函数。

这是文件名 fileName 文件的大小为 \emph{sz} 字节。如果修改大小成功返回 \hl{true},否则返回 \hl{false}。如果 \emph{sz} 比当前文件大小大,后面的数据会填充0;如果 \emph{sz} 比当前文件大小小,会裁剪文件数据。



\begin{warning}
如果文件不存在,调用会失败。
\end{warning}

\begin{seeAlso}
resize().
\end{seeAlso}

void QFile::setFileName(const QString \emph{\&name})

设置文件名 name。名称可以不包含目录,包含相对目录或绝对目录。

请不要在文件已经打开后调用此函数。

如果文件名不包含路径,或者是相对路径,路径会基于应用程序调用 open() 时的当前路径。

例如:

\begin{cppcode}
QFile file;
QDir::setCurrent("/tmp");
file.setFileName("readme.txt");
QDir::setCurrent("/home");
file.open(QIODevice::ReadOnly);      // 打开Unix下文件 "/home/readme.txt"
\end{cppcode}



\begin{notice}
Qt中目录分隔符统一使用"/".
\end{notice}

\begin{seeAlso}
fileName(), QFileInfo, and QDir.
\end{seeAlso}

\splitLine

\hl{[override virtual]}bool QFile::setPermissions(QFileDevice::Permissions permissions)

重写函数: QFileDevice::setPermissions(QFileDevice::Permissions permissions).

为文件设置 permissions 权限。如果成功返回 true ,如果权限不能修改返回 false 。



\begin{warning}
此函数不会操作修改 ACLs,这会限制函数功能。
\end{warning}

\begin{seeAlso}
permissions() and setFileName().
\end{seeAlso}

\splitLine

[static]bool QFile::setPermissions(const QString \emph{\&fileName}, QFileDevice::Permissions \emph{permissions})

这是一个重载函数。

为文件名 fileName 的文件设置 permissions 权限。

\hl{[override virtual]}qint64 QFile::size() const

重写函数: QFileDevice::size() const.

\splitLine

\hl{[static]}QString QFile::symLinkTarget(const QString \emph{\&fileName})

返回符号链接(Unix上的symlink或Windows上快捷方式)fileName 指向的文件或目录的绝对路径。如果 fileName 不是一个符号链接,返回空字符串。

名称可能并不是一个存在的文件,只是一个字符串路径。QFile::exists() 可以用来判断是否存在。

此函数引入自: Qt 4.2.

\splitLine

QString QFile::symLinkTarget() const

这是一个重载函数。

返回QFile对象对应的符号链接(Unix上的symlink或Windows上快捷方式)指向的文件或目录的绝对路径。
如果 fileName 不是一个符号链接,返回空字符串。

名称可能并不是一个存在的文件,只是一个字符串路径。QFile::exists() 可以用来判断是否存在。

此函数引入自: Qt 4.2.

\begin{seeAlso}
fileName() and setFileName().
\end{seeAlso}

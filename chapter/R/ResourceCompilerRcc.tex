\chapter{ResourceCompilerRcc}

\hl{rcc} 工具用于在编译期将资源数据集成至 Qt 应用。它基于 Qt 资源文件(\hl{.qrc}) 来生成包含资源数据的 C++ 源文件。

使用方式

\begin{lstlisting}
rcc [选项列表] <输入列表>
\end{lstlisting}

RCC 接受如下命令行选项:

\begin{longtable}{|l|l|m{25em}|}
\hline
选项 	& 参数  &	描述 \\ 
\hline
-o &	file &	将输出信息写入 file,而非打印至 stdout。\\ 
\hline
-name &	name &	创建名为 name 的外部初始化函数。 \\ 
\hline
-threshold 	& level &	指定值为 level(百分比)的阈值,以用于判断是否需要压缩文件。若压缩掉的文件尺寸大于该阈值 level,则会执行压缩,否则会存储未压缩数据。默认阈值是 70\%,即当压缩后的尺寸小于等于原尺寸的 30\%,则会存储为压缩数据。 \\ 
\hline
-compress-algo 	& algorithm &	压缩文件使用的算法,支持 zstd,zlib 和 none,即指通过 Zstandard 库或 zlib 库进行压缩,亦或不进行压缩。默认情况下,若编译期可以找到 zstd 库则使用该算法,佛祖额使用 zlib。\\ 
\hline
-compress &	level &	通过压缩等级 level 压缩输入文件,不同算法有不同的的等级范围。若使用 zstd 算法,有效等级是 1 至 19,另外特殊值 0 和 -1 代指 libzstd 和 rcc 的默认压缩等级。若使用 zlib 算法,有效等级是 1 至 9。对于这两种算法,等级 1 都代表最低压缩率但最快压缩速度,等级 9 或 19 则是最高压缩率但最慢压缩速度。。若要关闭压缩,则使用 -no-compress。level 的默认值是 -1。\\
\hline
-root &	path &	将 path 附加至资源访问路径的前缀,默认为无前缀。 \\ 
\hline
-no-compress & &		禁用压缩。\\
\hline
-binary 		& & 输出至二进制文件,以用作动态资源。\\
\hline
-version 		& &显示版本信息。\\
\hline
-help & &		显示使用方式。\\
\hline
-t, --temp <file> 	& &	通过临时文件 <file> 处理大体积资源。\\
\hline
--namespace 	& &	关闭命名空间宏。\\
\hline
--verbose 	& &	启用详细输出。\\ 
\hline
--list 	& &	仅列出 .qrc 中的文件列表,不生成代码文件。\\ 
\hline
-project 	& &	生成一个包含当前目录中所有文件的资源文件。\\ 
\hline
\end{longtable}

\begin{seeAlso}
Qt 资源系统 以获取集成资源至 Qt 应用程序的更多信息。
\end{seeAlso}
\chapter{QReadWriteLock}

QReadWriteLock 类提供读写锁定。更多...

\begin{tabular}{|l|l|}
\hline
属性 &	方法\\
\hline
头文件:& 	\#include <QReadWriteLock>\\
\hline
qmake:& 	QT += core\\
\hline
\end{tabular}

\begin{notice}
此类中所有函数都是线程安全的。
\end{notice}

\section{公共成员类型}

\begin{tabular}{|l|l|}
\hline
类型 &	名称\\
\hline
enmu 	& RecursionMode \{ Recursive, NonRecursive\}\\
\hline
\end{tabular}


\section{公共成员函数}

\begin{tabular}{|l|m{27em}|}
\hline
返回类型 &	函数 \\ 
\hline
& QReadWriteLock(QReadWriteLock::RecursionMode \emph{recursionMode} = NonRecursive) \\
\hline
& $\sim$QReadWriteLock() \\ 
\hline
void &	lockForRead() \\ 
\hline
void &	lockForeWrite() \\ 
\hline
bool &	tryLockForRead()\\ 
\hline
bool &	tryLockForRead(int \emph{timeout}) \\ 
\hline
bool &	trylockForWrite() \\ 
\hline
bool &	trylockForWrite(int \emph{timeout}) \\ 
\hline
void &	unlock() \\ 
\hline
\end{tabular}

\section{详细描述}

读写锁是一种用于保护读写资源的同步工具。
如果希望允许多个线程同时进行只读访问,那么这种类型的锁很有用,但是只要一个线程想写入资源,那么它就必须阻止所有其他线程,直到写入完成。
  
在许多情况下,QReadWriteLock 是 QMutex 强有力的竞争对手。
QReadWriteLock 在多并发读,少量写的情况下,有较好的执行效果。

例如:

\begin{lstlisting}[language=C++]
QReadWriteLock lock;

void ReaderThread::run()
{
    ...
    lock.lockForRead();
    read_file();
    lock.unlock();
    ...
}
    
void WriterThread::run()
{
    ...
    lock.lockForWrite();
    write_file();
    lock.unlock();
    ...
}
\end{lstlisting}

为了确保写者永远不会被读者阻塞,如果有写者被阻塞,则读者将无法获取锁,
即使该锁当前正被其他读者访问。另外,如果一个写者正在写,而另一个写者想进入,
则该写者将优先于其他可能正在等待的读者。

与 QMutex 类似,当使用 QReadWriteLock::Recursive 作为 QReadWriteLock::RecursiveMode 参数构造时,
QReadWriteLock 可以被同一线程递归锁定。在这种情况下,
unlock() 的调用次数必须与 lockForWrite() 或 lockForRead() 的调用次数相同。
请注意,当尝试递归锁定时,不能更改锁类型,也就是说,在已经为写入而锁定的线程中,不可能为读取而锁定(反之亦然)。

\begin{seeAlso}
QReadLock、QWriteLock、QMutex、QSemaphore。
\end{seeAlso}

\section{成员类型文档}

enum QReadWriteLock::RecursionMode

\begin{tabular}{|l|l|l|}
\hline
常量 &	值  &	描述 \\ 
\hline
QReadWriteLock::Recursive  &	1  &	  在这种模式下,线程可以多次锁定同一个 QReadWriteLock。QReadWriteLock 在执行相应数量的 unlock() 调用之前不会解锁。 \\ 
\hline
QReadWriteLock::NonRecursive &	0 	&  在这种模式下,线程只能锁定 QReadWriteLock 一次。 \\ 
\hline
\end{tabular}

\begin{seeAlso}
QReadWriteLock()。
\end{seeAlso}

\section{成员函数文档}

QReadWriteLock::QReadWriteLock(QReadWriteLock::RecursionMode \emph{recursionMode} = NonRecursive)

构造一个 \hl{recursionMode} 模式的 \hl{QReadWriteLock}。
默认 NonRecursive。

在 Qt 4.4 引入该函数。


\begin{seeAlso}
lockForRead()、lockForWrite()、RecursiveMode。
\end{seeAlso} 

QReadWriteLock::$\sim$QReadWriteLock()

解锁,并析构。

\begin{warning}
销毁正在使用的读写锁可能会导致未定义的行为。
\end{warning} 

void QReadWriteLock::lockForRead()

读锁定。如果另一个线程正写锁定,则此函数将阻塞当前线程。
如果线程已写锁定,则不可读锁定。

\begin{seeAlso}
unlock()、lockForWrite()、tryLockForWrite()。
\end{seeAlso} 

void QReadWriteLock::lockForWrite()

写锁定。如果另一个线程(包括当前线程)为读或写而锁定(除非锁是使用 QReadWriteLock::Recursive 模式创建的),则此函数将阻塞当前线程。   如果线程已读锁定,则不可写锁定。

\begin{seeAlso}
unlock()、lockForWrite()、tryLockForWrite()。
\end{seeAlso} 

bool QReadWriteLock::tryLockForRead()

尝试读锁定。如果获得了锁,此函数将返回 true,否则将返回 false,而不是等待锁变为可用,即不阻塞。
如果另一个线程已写锁定,则读锁定尝试将失败。
如果获得了锁,请务必使用 unlock() 解锁,然后另一个线程才能成功地将其写锁定。
如果线程已写锁定,则不可读锁定。

\begin{seeAlso}
unlock()、lockForRead()。
\end{seeAlso} 

bool QReadWriteLock::tryLockForRead(int \emph{timeout})

重载函数。尝试读锁定。如果获得了锁,此函数返回 true,否则返回 false。如果另一个线程已写锁定,则此函数最多将等待 timeout 毫秒。

\begin{notice}
传递一个负数作为超时值相当于调用 lockForRead(),即当 timeout 值为负数时,此函数将永远等待,直到读锁定为止。
\end{notice} 

如果获得了锁,请务必使用 unlock() 解锁,然后另一个线程才能成功地将其写锁定。
如果线程已读锁定,则不可写锁定。

\begin{seeAlso}
unlock()、lockForRead()。
\end{seeAlso} 

bool QReadWriteLock::tryLockForWrite()

尝试锁定以进行写入。如果获取的锁为 true,则立即返回 false。
如果另一个线程已锁定,则写锁定尝试将失败。

如果获得了锁,请务必使用 unlock() 解锁,然后另一个线程才能成功地将其锁定。
如果线程已读锁定,则不可写锁定。

\begin{seeAlso}
unlock()、lockForWrite()。
\end{seeAlso}

bool QReadWriteLock::tryLockForWrite(int \emph{timeout})

重载函数。尝试写锁定。如果获得了锁,此函数返回 true,否则返回 false。如果另一个线程已写锁定,则此函数最多将等待 timeout 毫秒。

\begin{notice}
传递一个负数作为超时值相当于调用 lockForWrite(),即当 timeout 值为负数时,此函数将永远等待,直到读锁定为止。
\end{notice}

如果获得了锁,请务必使用 unlock() 解锁,然后另一个线程才能成功地将其写锁定。
如果线程已读锁定,则不可写锁定。

\begin{seeAlso}
unlock()、lockForWrite()。
\end{seeAlso}

void QReadWriteLock::unlock()

解锁。 试图解锁未锁定的锁是错误的,将导致程序终止。
 
\begin{seeAlso}
lockForRead()、lockForWrite()、tryLockForRead()、trylockForWrite()。
\end{seeAlso}
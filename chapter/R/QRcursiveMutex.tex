\chapter{QRcursiveMutex}

QRcursiveMutex类提供线程间的访问序列化。更多...

\begin{tabular}{|l|l|}
\hline
属性 &	方法\\
\hline
头文件:& 	\#include <QRcursiveMutex>\\
\hline
qmake:& 	QT += core\\
\hline
加入版本 & 	Qt 5.14 \\ 
\hline
继承自 	& QMutex(private) \\ 
\hline
\end{tabular}

\begin{notice}
此类中所有函数都是线程安全的。
\end{notice}

\section{公共成员函数}

\begin{tabular}{|l|l|}
\hline
返回类型 &	函数 \\ 
\hline
& QRecursiveMutex() \\ 
\hline
& $\sim$QRecursiveMutex() \\ 
\hline
\end{tabular}

\section{详细描述}

QRecursiveMutex 类是一个互斥体,与 QMutex 类似,与之API兼容。
它与 QMutex 的不同之处在于,它可多次接受来自同一线程的 lock() 调用。
QMutex 在这种情况下会死锁。
QRecursiveMutex 的构造和操作成本要高很多,因此尽可能使用纯 QMutex。
然而,有时一个公共函数调用另一个公共函数,它们都需要锁定同一个互斥体。在这种情况下,您有两个选项:

\begin{compactitem}[\arr]
\item 将需要互斥锁保护的代码分解到私有函数中,私有函数假设在调用互斥体时 保留互斥体,并在调用私有实现函数之前在公共函数中锁定一个纯QMutex。
\item 或者使用递归互斥锁,所以第二个公共函数希望锁定互斥锁,与第一个公共函数是否锁定没有多大关系了。
\end{compactitem}

\begin{seeAlso}
QMutex,QMutexLocker,QReadWriteLock,QSemaphore 和 QWaitCondition。
\end{seeAlso}


\section{成员函数文档}

QRecursiveMutex::QRecursiveMutex()

构造一个新的递归互斥锁。互斥锁是在解锁状态下创建的。

\begin{seeAlso}
lock()、unlock()
\end{seeAlso}

QRecursiveMutex::$\sim$QRecursiveMutex()

析构。

\begin{warning}
销毁锁定的互斥锁可能会导致未定义的行为。
\end{warning}
\chapter{Qt资源系统}

Qt 资源系统是一套平台独立的机制,用于将二进制文件存储至应用程序的可执行文件中。
这在您的应用始终依赖一组特定文件(如图标、翻译文件等),并且不想承担丢失这些文件的风险时,会非常有用。

资源系统基于 qmake、rcc (Qt 的资源编译器) 和 QFile 的密切协作。

\section{资源汇总文件}

通过 \hl{.qrc} 文件来指定应用程序所关联的资源,该文件基于 XML 格式,包含了磁盘中的文件列表,并为它们标注可选的别名,以供应用程序来获取资源内容。

下文为一个 \hl{.qrc} 文件范例:

\begin{lstlisting}[language=XML]
<!DOCTYPE RCC><RCC version="1.0">
<qresource>
<file>images/copy.png</file>
<file>images/cut.png</file>
<file>images/new.png</file>
<file>images/open.png</file>
<file>images/paste.png</file>
<file>images/save.png</file>
</qresource>
</RCC>
\end{lstlisting}

.qrc 文件中列举的资源文件是应用程序资源树的一部分,其中指定的路径为 .qrc 文件所在目录的相对路径。注意,列举的资源文件必须位于 .qrc 文件所在目录或其子目录。

资源数据可以被编译进二进制程序,从而在运行时可被立即获取;也可以生成为二进制文件,随后在应用程序代码中注册至资源系统。

默认情况下,资源文件在应用程序中,可使用它们在资源树中的路径,附加上 :/ 前缀来访问,也可以通过名为 qrc 的 URL Scheme 来访问。

例如,文件路径 :/images/cut.png,或 链接地址 qrc:///images/cut.png,可以用于访问 cut.png 文件,该文件在应用程序资源树中的路径为 images/cut.png。该路径也可以通过 file 标签的 alias 属性进行修改:

\begin{lstlisting}
<file alias="cut-img.png">images/cut.png</file>
\end{lstlisting}

此时,该文件可通过 :/cut-img.png 路径访问。也可以通过 qresource 标签的 prefix 属性为 .qrc 文件中的所有资源文件设置路径前缀:

\begin{lstlisting}
<qresource prefix="/myresources">
    <file alias="cut-img.png">images/cut.png</file>
</qresource>
\end{lstlisting}

在此场景下,该文件可通过 :/myresources/cut-img.png 路径访问。

某些资源需要随用户的区域设置发生改变,如翻译文件或图标。这可以通过在 qresource 标签中添加 lang 属性,为其指定对应的区域名称字符串来实现,例如:

\begin{lstlisting}
<qresource>
    <file>cut.jpg</file>
</qresource>
<qresource lang="fr">
    <file alias="cut.jpg">cut_fr.jpg</file>
</qresource>
\end{lstlisting}

若用户区域为法国(即 QLocale::system().name() 函数返回 "fr\_FR"),则 :/cut.jpg 会引用至 cut\_fr.jpg,其它情况下使用 cut.jpg。

\begin{notice}[另请参阅]
QLocale 文档以获取区域字符串格式的说明。
\end{notice}

\begin{notice}[另请参阅]
QFileSelector 文档以了解另一个基于区域选取资源的机制,该机制可基于操作系统等更多附加信息来进行选取。
\end{notice}

\section{外部二进制资源}

若要生成外部二进制资源,您需要传递 -binary 选项至 rcc 来创建资源数据文件(通常使用 .rcc 扩展名)。二进制资源文件创建完毕后,可以通过 QResource 接口进行注册。

例如,.qrc 文件中指定的一组资源数据,可以通过如下方式进行编译:

\begin{lstlisting}
rcc -binary myresource.qrc -o myresource.rcc
\end{lstlisting}

在应用程序中,通过如下代码注册资源:

\begin{lstlisting}
QResource::registerResource("/path/to/myresource.rcc");
\end{lstlisting}

\section{内建资源}

若要将资源编译至二进制程序内,您需要在应用程序的 .pro 文件中指定 .qrc 文件,从而让 qmake 可以感知到它。例如:
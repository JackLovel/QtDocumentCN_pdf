\chapter{QSerialPortInfo}

QSerialPortInfo 类提供了系统中现有串口的相关信息。 更多内容...

\begin{tabular}{|r|l|}
	\hline
	属性 & 方法 \\
	\hline
	头文件 & \#include <QSerialPortInfo>\\      
	\hline
	qmake & QT += serialport\\      
	\hline
	始自: &	Qt 5.1\\
	\hline
\end{tabular}

\begin{compactitem}[\arr]
\item 列出所有成员, 包括子类成员
\item 废弃的类
\end{compactitem}

\section{公共成员函数}

\begin{longtable}[l]{|r|m{30em}|}
\hline 
返回类型 &	函数名	\\ 
\hline 
&QSerialPortInfo(const QSerialPortInfo \&other) \\
\hline
&QSerialPortInfo(const QString \&name)  \\
\hline
&QSerialPortInfo(const QSerialPort \&port) \\ 
\hline
&QSerialPortInfo() \\ 
\hline
QSerialPortInfo & 	operator=(const QSerialPortInfo \&other) \\
\hline
& $\sim$QSerialPortInfo() \\
\hline
QString 	&description() const \\
\hline
bool &	hasProductIdentifier() const\\
\hline
bool 	&hasVendorIdentifier() const \\
\hline
bool &isNull() const \\
\hline
QString &	manufacturer() const \\
\hline
QString &	portName() const \\
\hline
quint16 &	productIdentifier() const \\
\hline
QString &	serialNumber() const\\
\hline
void 	&swap(QSerialPortInfo \&other) \\
\hline
QString 	&systemLocation() const\\
\hline
quint16 &	vendorIdentifier() const\\
\hline
\end{longtable}

\section{静态公共成员函数}

\begin{tabular}{|r|l|}
\hline
返回类型 	&函数名 \\ 
\hline
QList &	availablePorts() \\ 
\hline
QList &	standardBaudRates() \\ 
\hline
\end{tabular}

\section{详细描述}

使用静态函数生成 QSerialPortInfo 类实例对象列表。
列表中的每一个对象代表一个串口设备,可以通过串口名、系统地址、设备描述以及制造商查询串口。 
QSerialPortInfo 类还可以用作 QSerialPort 类成员方法 setPort() 的输入参数。

\begin{seeAlso}
QSerialPort。
\end{seeAlso}


\section{成员函数文档}

QSerialPortInfo::QSerialPortInfo(const QSerialPortInfo \emph{\&other})

构造 QSerialPortInfo 类实例 other 的副本。

QSerialPortInfo::QSerialPortInfo(const QString \emph{\&name})

构造串口名为 name 的 QSerialPortInfo 类实例。

该构造函数在现有的串口设备中按照名称检索名为 name 的串口,找到后为那个串口构造串口信息类实例。

QSerialPortInfo::QSerialPortInfo(const QSerialPort \emph{\&port})

从串口 port 构造 QSerialPortInfo 类实例。

QSerialPortInfo::QSerialPortInfo()

构造一个空的 QSerialPortInfo 类实例。

\begin{seeAlso}
isNull()。
\end{seeAlso}

QSerialPortInfo \&QSerialPortInfo::operator=(const QSerialPortInfo \emph{\&other})

将 QSerialPortInfo 类实例 other 赋值给另一个 QSerialPortInfo 类实例。

QSerialPortInfo::$\sim$QSerialPortInfo()

销毁 QSerialPortInfo 类实例,销毁后该实例的引用无效。

[static] QList QSerialPortInfo::availablePorts()

该函数返回系统中现有串口设备列表。

QString QSerialPortInfo::description() const

该函数返回描述串口的字符串,若没有描述字符串,则返回空字符串。

\begin{seeAlso}
manufacturer() 和 serialNumber()。
\end{seeAlso}

bool QSerialPortInfo::hasProductIdentifier() const

若串口设备有16位产品编号,函数返回true,否则返回false。

\begin{seeAlso}
productIdentifier(), vendorIdentifier(), 和 hasVendorIdentifier()。
\end{seeAlso}

bool QSerialPortInfo::hasVendorIdentifier() const

若串口设备有16位厂商编号,函数返回true,否则返回false。

\begin{seeAlso}
vendorIdentifier(), productIdentifier(), 和 hasProductIdentifier()。
\end{seeAlso}

bool QSerialPortInfo::isNull() const

若 QSerialPortInfo 类实例没有串口定义,函数返回true,否则返回false。

\begin{seeAlso}
isBusy()。
\end{seeAlso}

QString QSerialPortInfo::manufacturer() const

返回串口设备制造商字符串,若该串口没有制造商字符串,则返回空字符串。

\begin{seeAlso}
description() 和 serialNumber()。
\end{seeAlso}

QString QSerialPortInfo::portName() const

返回串口名。

\begin{seeAlso}
systemLocation()。
\end{seeAlso}

quint16 QSerialPortInfo::productIdentifier() const

若串口设备有16位产品编号,函数返回产品编号,否则返回0。

\begin{seeAlso}
hasProductIdentifier(), vendorIdentifier(), 和 hasVendorIdentifier()。
\end{seeAlso}

QString QSerialPortInfo::serialNumber() const

若串口设备有序列号,函数返回序列号字符串,否则返回空字符串。

\begin{notice}
串口序列号可能包含字母。
\end{notice}

该函数从 Qt 5.3 开始使用。

\begin{seeAlso}
description() 和 manufacturer()。
\end{seeAlso}

[static] QList QSerialPortInfo::standardBaudRates()

返回目标操作系统支持的标准串口通信波特率列表。

void QSerialPortInfo::swap(QSerialPortInfo \emph{\&other})

用 QSerialPortInfo 类实例 other 与当前 QSerialPortInfo 类实例互换。此操作非常快并且从不失败。

QString QSerialPortInfo::systemLocation() const

返回串口设备的系统地址。

\begin{seeAlso}
portName()。
\end{seeAlso}

quint16 QSerialPortInfo::vendorIdentifier() const

若串口设备有16位厂商编号,函数返回厂商编号,否则返回0。

\begin{seeAlso}
hasVendorIdentifier(), productIdentifier(), 和 hasProductIdentifier()。
\end{seeAlso}
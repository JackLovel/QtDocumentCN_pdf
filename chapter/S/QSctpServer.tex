\chapter{QSctpServer}

QSctpServer 提供了一个基于 QSCTP 协议的服务器。

\begin{tabular}{|r|l|}
	\hline
	属性 & 方法 \\
	\hline
	头文件 & \#include <QSctpServer>\\      
	\hline
	qmake & QT += network\\      
	\hline
	引入版本 &	Qt5.8 \\ 
	\hline
继承自	 & QTcpServer \\ 
	\hline
\end{tabular}

该类最初在 Qt5.8 版本时引入 。

\section{公共成员函数}

\begin{tabular}{|r|l|}
	\hline 
	类型	& 函数名 \\ 
	\hline
	& QSctpServer(QObject \emph{*parent} = nullptr)  \\ 
	\hline
	virtual	& $\sim$QSctpServer() \\ 
	\hline
	int	& maximumChannelCount() const  \\ 
	\hline
	QSctpSocket* &	nextPendingDatagramConnection() \\ 
	\hline
	void	& setMaximumChannelCount(int \emph{count}) \\ 
	\hline
\end{tabular}


\section{重写保护成员函数}

\begin{tabular}{|r|l|}
	\hline 
	类型	& 函数名 \\ 
	\hline
	virtual void	& incomingConnection(qintptr \emph{socketDescriptor}) override \\ 
	\hline
\end{tabular}

\section{详细介绍}

SCTP(流控制传输协议)是一种传输层协议,其作用与流行的协议 TCP 和 UDP 类似。像 UDP 一样,
SCTP 也是面向消息的,但是它可以通过与 TCP 类似的拥塞控制来确保消息的可靠性,让消息按顺序传输。
您可以在 QSctpSocket 类文档中找到关于 SCTP 协议的更多的信息。

QSctpServer 是 QTcpServer 的便利子类,它允许您以 TCP 模拟或数据报模式接受传入的 SCTP 套接字连接。

QSctpServer 最常见的用法是构造一个 QSctpServer 类的对象并调用 setMaximumChannelCount() 函数设置该服务器准备支持的最大通道数。
您可以在调用 setMaximumChannelCount() 函数时传入一个负值,这样便可以让服务端在 TCP 模拟模式下运行。
另外,如果在 setMaximumChannelCount() 函数调用时传入0(0也是默认值),
这会指定服务端使用对等端设置的最大通道数目。新的传入连接从服务器套接字描述符继承此数字,
并根据远程端点设置对其进行调整。

在 TCP 模拟模式下,服务端允许客户端使用一条连续的字节流来进行数据传输。
此时 QSctpSocket 的操作模式与普通的 QTcpServer 类似。
您可以调用 nextPendingConnection() 函数将一个待处理的连接作为一个已连接的 QTcpSocket 套字节来接受。
您可以使用该 QTcpSocket 套接字与客户端进行通信。此模式仅能使用基本的 SCTP 协议功能。
套接字在系统层面通过 IP 传输 SCTP 数据包,并通过 QTcpSocket 接口与应用程序进行交互。

与 TCP 模拟模式相反,数据报模式是面向消息的,并且在端点之间提供了多个数据流的完全同时传输。
您可以调用 nextPendingDatagramConnection() 函数将待处理的数据报模式的连接作为一个已连接的 QSctpSocket 套接字接受。


\begin{notice}
WIndows系统不支持该特性。
\end{notice}

\begin{seeAlso}
QTcpServer ,QSctpSocket 和 QAbstractSocket。
\end{seeAlso}

\section{成员函数文档}

QSctpServer::QSctpServer(QObject \emph{*parent} = nullptr)

构造函数。构造一个 QSctpServer 类型的对象并将其设置为数据报模式。

\emph{parent} 参数将传递到 QObject 构造函数中。

\begin{seeAlso}
setMaximumChannelCount(),listen() 和 setSocketDescriptor()。
\end{seeAlso}

[virtual] QSctpServer::$\sim$QSctpServer()

析构函数。
销毁 QSctpServer 类型的对象。
如果服务端仍然在监听客户端连接,该套接字会自动关闭。


\begin{seeAlso}
close()。
\end{seeAlso}

[override virtual protected] void QSctpServer::incomingConnection(qintptr \emph{socketDescriptor})

重写 QTcpServer::incomingConnection(qintptr socketDescriptor) 函数。

int QSctpServer::maximumChannelCount() const

返回套接字能够支持的最大通道数。

如果返回值为0(默认值)意味着支持的最大通道数将由远程端点设置。

如果 QSctpServer 运行在 TCP 模拟模式,该函数返回-1。

\begin{seeAlso}
setMaximumChannelCount()。
\end{seeAlso}

QSctpSocket *QSctpServer::nextPendingDatagramConnection()

将下一个待处理的数据报模式的连接作为一个已连接的 QSctpSocket 对象返回。

数据报模式的连接提供了一个面向消息的、多数据流的通信。

该返回的套接字将作为服务端的子类创建,这意味着当 QSctpServer 对象销毁时,该套接字也将销毁。当您使用完一个对象后显式地删除该对象是一个好的做法,这能避免浪费内存。

如果没有待处理的数据报模式的连接,该函数将返回空。

\begin{notice}
返回的 QSctpSocket 对象不能被其他线程使用。
如果想在其他线程中使用到达的连接,您需要重写 incomingConnection() 函数。
\end{notice}

\begin{seeAlso}
hasPendingConnections() ,nextPendingConnection() 和 QSctpSocket。
\end{seeAlso}

void QSctpServer::setMaximumChannelCount(int \emph{count})

设置在数据报模式下服务端准备支持的最大通道数为 \emph{count} 。

如果 \emph{count} 为 0,服务端将使用其他端点的最大通道数。

如果 \emph{count} 为一个负值,则 QSctpServer 将会在 TCP 模拟模式下运行。

\begin{seeAlso}
maximumChannelCount() 和 QSctpSocket。
\end{seeAlso}
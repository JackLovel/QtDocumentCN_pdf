\chapter{信号与槽}

信号(Signals)和槽(Slots)被用于在 Qt 对象之间通信。
信号槽机制是 Qt 的核心特性,同时也可能是与其他框架的类似特性区别最大的一部分。
信号槽使得 Qt 的元对象系统成为可能。

\section{介绍}

在 GUI 编程中,当我们修改某个控件后,我们通常希望另一个控件可以收到通知。
更普遍地,我们希望任何类型的对象都可以和另一个对象进行通信。
例如,如果用户点击了关闭按钮,我们可能希望该窗口的 close() 函数被调用。

其它开发工具中,使用回调来实现此类通信。
回调是指函数指针——当您希望某个处理函数通知您某些事件时,
通过传递一个函数指针(即回调)至该处理函数来实现,处理函数会在适合的时机调用这个回调。
尽管许多优秀的框架的确在使用此方法,但回调依然是一种非常不直观的手段,
而且可能会遭遇回调参数类型正确性校验等方面的问题。

\section{信号槽}

在 Qt 中,我们有回调技术的替代品:信号槽。信号会在特定事件发生时被发射;
Qt 的控件包含大量预定义的信号,并且我们也可以编写控件的子类来为它们添加自定义的信号。
槽是指会在响应对应信号时被自动调用的函数;Qt 的控件包含大量预定义的槽,并且,
编写控件的子类来添加自定义槽也是常见的做法,这样您便可以处理感兴趣的信号。
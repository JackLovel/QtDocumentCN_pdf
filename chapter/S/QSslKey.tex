\chapter{QSslKey}

QSslKey 类为私钥、公钥提供了相应的接口。

\begin{tabular}{|r|l|}
	\hline
	属性 & 方法 \\
	\hline
	头文件 & \#include <QSslKey>\\      
	\hline
	qmake & QT += network\\      
	\hline
	引入 &	Qt4.3 \\ 
	\hline
\end{tabular}

该类最初在 Qt 4.3 版本引入。

\begin{notice}
该类所有的函数都是可重入的。
\end{notice}

\section{公共成员函数}

\begin{longtable}[l]{|l|m{30em}|}
\hline 
类型 &	函数名\\ 
\hline 
&QSslKey(const QSslKey \&other) \\ 
	\hline
	&QSslKey(Qt::HANDLE handle, QSsl::KeyType type = QSsl::PrivateKey) \\ 
	\hline
	& QSslKey(QIODevice *device, QSsl::KeyAlgorithm algorithm, QSsl::EncodingFormat encoding = QSsl::Pem, QSsl::KeyType type = QSsl::PrivateKey, const QByteArray \&passPhrase = QByteArray()) \\ 
	\hline
	& QSslKey(const QByteArray \&encoded, QSsl::KeyAlgorithm algorithm, QSsl::EncodingFormat encoding = QSsl::Pem, QSsl::KeyType type = QSsl::PrivateKey, const QByteArray \&passPhrase = QByteArray()) \\
	\hline
	& QSslKey() \\ 
	\hline
QSslKey \& 	& operator=(const QSslKey \&other) \\ 
\hline
& $\sim$QSslKey() \\ 
	\hline
QSsl::KeyAlgorithm &	algorithm() const \\ 
\hline
void 	& clear() \\ 
\hline
Qt::HANDLE &	handle() const \\ 
\hline
bool &	isNull() const \\ 
\hline
int &	length() const \\ 
\hline
void &	swap(QSslKey \&other) \\ 
\hline
QByteArray &	toDer(const QByteArray \&passPhrase = QByteArray()) const \\ 
\hline
QByteArray& 	toPem(const QByteArray \&passPhrase = QByteArray()) const \\ 
\hline
QSsl::KeyType &	type() const \\ 
\hline
bool 	&operator!=(const QSslKey \&other) const \\ 
\hline
bool &	operator==(const QSslKey \&other) const \\
\hline 
\end{longtable}

\section{详细描述}

QSslKey 类为密钥管理提供了简单的 API 支持。

另外您也可以在 QSslSocket ,QSslCertificate 和 QSslCipher 类文档中找到相关介绍。

\section{成员函数文档}

QSslKey::QSslKey(const QSslKey \&other)

拷贝构造函数。

由 other 拷贝一份新的 QSslKey 对象。

QSslKey::QSslKey(Qt::HANDLE handle, QSsl::KeyType type = QSsl::PrivateKey)

从一个可以用的本地句柄 handle 构造一个 QSslKey 对象。type 参数用来指定该密钥是公钥还是私钥。

QSsl 将会接管该密钥,您不得使用本机库来释放该密钥。

该函数最初在 Qt 5.0版本引入。

QSslKey::QSslKey(QIODevice \*device, 
    QSsl::KeyAlgorithm algorithm, 
	QSsl::EncodingFormat encoding = QSsl::Pem,
	QSsl::KeyType type = QSsl::PrivateKey, 
	const QByteArray \&passPhrase = QByteArray())

使用 algorithm 指定的算法和 encoding 指定的格式从 device 指定的设备中读取并解码数据,
构造 QSslKey 对象。type 参数指定了该密钥是公钥还是私钥。

如果该密钥被加密,则将使用 passPhrase 来解密。

构造后该对象后,您可以使用 isNull() 函数来检查 device 设备是否提供了一个有效的密钥。

QSslKey::QSslKey(const QByteArray \&encoded, 
	QSsl::KeyAlgorithm algorithm, 
	QSsl::EncodingFormat encoding = QSsl::Pem, 
	QSsl::KeyType type = QSsl::PrivateKey, 
	const QByteArray \&passPhrase = QByteArray())

使用 algorithm 指定的算法和 encoding 指定的格式从字符数组 encoded 读取并解密字符串,构造 QSslKey 对象。 type 将指定该密钥是公钥还是私钥。

如果该密钥被加密,则将使用 passPhrase 解密。

构造后该对象后,您可以使用 isNull() 函数来检查 encoded 字符数组是否包含了一个有效的密钥。

QSslKey::QSslKey()

构造一个空的密钥。

另外您可以在 isNull() 函数介绍中找到更多相关信息。

QSslKey \&QSslKey::operator=(const QSslKey \&other)

将 other 中的内容拷贝到此密钥中,是等式左右值相等。

返回该对象的引用。

QSslKey::$\sim$QSslKey()

析构函数。销毁该 QSslKey 对象。

QSsl::KeyAlgorithm QSslKey::algorithm() const

返回该密钥加密算法。

void QSslKey::clear()

清除该密钥的内容,将密钥变为空密钥。

\begin{seeAlso}
isNull()。
\end{seeAlso}

Qt::HANDLE QSslKey::handle() const

若本地密钥句柄存在,则返回指向本地密钥句柄的指针,否则返回nullptr。

你可以将此句柄与本地 API 结合使用,获得关于该密钥更多的信息。

\begin{warning}
使用该函数有很大的可能并不能移植,并且该函数的返回值可能会在不同平台之间或者不同 Qt 发行版本之间变化。
\end{warning}

bool QSslKey::isNull() const

如果该密钥为空则返回true ,否则返回false。

\begin{seeAlso}
clear() 。
\end{seeAlso}

int QSslKey::length() const

返回密钥的长度(以位为单位);如果密钥为空,则返回-1。

void QSslKey::swap(QSslKey \&other)

与 oher 交换密钥信息。该函数执行速度很快并且保证操作一定成功。

该函数最初在 Qt 5.0 版本引入。

QByteArray QSslKey::toDer(const QByteArray \&passPhrase = QByteArray()) const

以 DER 编码格式返回该密钥。

您应该忽略 passPhrase 参数,因为 DER 不能被加密。未来的 Qt 版本中将移除该参数。

QByteArray QSslKey::toPem(const QByteArray \&passPhrase = QByteArray()) const

以 PEM 编码格式返回该密钥。如果该密钥为私钥并且 passPhrase 不为空,该密钥将利用 passPhrase 加密。

QSsl::KeyType QSslKey::type() const

返回该密钥的类型(即公钥或私钥)。

bool QSslKey::operator!=(const QSslKey \&other) const

若该密钥与 other 不相等则返回true ,否则返回false 。

bool QSslKey::operator==(const QSslKey \&other) const

若该密钥与 other 相等则返回true ,否则返回false 。
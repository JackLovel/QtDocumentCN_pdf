\chapter{QSemaphore}

QSemaphore类提供了一个通用的信号量。\href{}{更多内容...}

\begin{tabular}{|r|l|}
	\hline
	属性 & 方法 \\
	\hline
	头文件 & \#include <QSemaphore>\\      
	\hline
	qmake & QT += core\\      
	\hline
\end{tabular}

\begin{warning}
此类中所有函数都是线程安全的。
\end{warning}

\section{公共成员函数}

\begin{tabular}{|r|l|}
	\hline 
	返回类型	& 函数名 \\ 
	\hline 
	& QSemaphore(int \emph{n} = 0) \\ 
	\hline
	& $\sim$QSemaphore() \\
	\hline
	void	& acquire(int \emph{n} = 1) const \\ 
	\hline
	int	& available() const \\ 
	\hline
	void &	release(int \emph{n} = 1) \\
	\hline
	bool &	tryAcquire(int \emph{n} = 1) \\ 
	\hline 
	bool	&tryAcquire(int \emph{n}, int \emph{timeout}) \\ 
	\hline
\end{tabular}


\section{详细描述}

信号量是互斥量的泛化。虽然互斥量只能锁定一次,但可以多次获取信号。信号量通常用于保护一定数量的相同资源。

信号量支持两个基本操作,acquire() 和 release():

\begin{compactitem}
\item acquire(n) 尝试获取 n 个资源。如果没有那么多可用资源,调用将被阻塞,直到能够获取。
\item release(n) 释放 n 个资源。
\end{compactitem}

还有 tryAcquire() 函数,如果无法获取资源,它将立即返回;还有一个 available() 函数,可以随时返回可用资源的数量。

例如:

\begin{lstlisting}[language=C++]
QSemaphore sem(5);      // sem.available() == 5

sem.acquire(3);         // sem.available() == 2
sem.acquire(2);         // sem.available() == 0
sem.release(5);         // sem.available() == 5
sem.release(5);         // sem.available() == 10

sem.tryAcquire(1);      // sem.available() == 9, returns true
sem.tryAcquire(250);    // sem.available() == 9, returns false
\end{lstlisting}

信号量的一个典型应用场景,是控制生产者线程和消费者线程对共享循环缓冲区的访问。生产者消费者示例展示了如何使用 QSemaphore 来解决这个问题。

一个非计算信号量(non-computing)的例子是在餐馆就餐。用餐厅里的椅子数初始化信号量。
当人们到达时,他们需要要一个座位。入座后,available() 将减少。当人们离开时,available() 会增加,允许更多的人进入。
如果有一个 10 人的聚会,但是餐馆只有 9 个座位,那这 10 个人会等待空出位置。
但是如果此时有一个 4 人的聚会,那么他们会入座(将可用的座位减少到 5 个,10 人聚会将等待更长时间)。

\begin{seeAlso}
QSemaphoreReleaser,QMutex,QWaitCondition,QThread 和 生产者消费者示例。
\end{seeAlso}

\section{成员类型文档}

QSemaphore::QSemaphore(int \emph{n} = 0)

构造一个信号量并将其保护的资源数量初始化为 \hl{n}(默认为0)。

\begin{seeAlso}
release(),available()。
\end{seeAlso}

QSemaphore::$\sim$QSemaphore()

析构。

\begin{warning}
析构正在使用的信号量可能会导致未定义的行为。
\end{warning}

void QSemaphore::acquire(int \emph{n} = 1)

尝试获取由信号量保护的 n 个资源。如果 \hl{n > available()},则此调用将阻塞,直到有足够的资源可用。

\begin{seeAlso}
release(),available(),tryAcquire()。
\end{seeAlso}

int QSemaphore::available() const

返回信号量当前可用的资源数。返回值永远不会是负数。

\begin{seeAlso}
release(),available()。
\end{seeAlso}

void QSemaphore::release(int \emph{n} = 1)

释放 n 个信号量保护的资源。

这个函数也可以用来“创建”资源。例如:

\begin{lstlisting}[language=C++]
QSemaphore sem(5);      // 信号量保护 5 个资源  sem.avilable() == 5
sem.acquire(5);         // 请求 5 个资源       sem.avilable() == 0
sem.release(5);         // 释放 5 个资源       sem.avilable() == 5
sem.release(10);        // “创建” 10 个新资源  sem.avilable() == 15
\end{lstlisting}

QSemaphoreReleaser 是一个围绕此函数的 RAII 包装器。

\begin{seeAlso}
acquire(),available(),QSemaphoreReleaser。
\end{seeAlso}

bool QSemaphore::tryAcquire(int \emph{n} = 1)

尝试获取由信号量保护的 n 个资源,并在成功时返回 \hl{true}。如果 \hl{available() < n},此调用立即返回 \hl{false},而不获取任何资源。

例如:

\begin{lstlisting}[language=C++]
QSemaphore sem(5);      // sem.available() == 5
sem.tryAcquire(250);    // sem.available() == 5, returns false
sem.tryAcquire(3);      // sem.available() == 2, returns true
\end{lstlisting}

\begin{seeAlso}
acquire()。
\end{seeAlso}

bool QSemaphore::tryAcquire(int \emph{n}, int \emph{timeout})

尝试获取由信号量保护的 n 个资源,并在成功时返回 \hl{true}。如果 \hl{available() < n},此调用将等待 \hl{timeout} 毫秒,以尝试获取。

\begin{notice}
传递一个负数作为超时相当于调用 \hl{acquire()},也就是说,如果 \hl{timeout} 为负数,这个函数将永远等待资源可用。
\end{notice}

例如:

\begin{lstlisting}[language=C++]
QSemaphore sem(5);            // sem.available() == 5
sem.tryAcquire(250, 1000);    // sem.available() == 5, 等待 1000 毫秒,returns false
sem.tryAcquire(3, 30000);     // sem.available() == 2, 立即返回,returns true
\end{lstlisting}

\begin{seeAlso}
acquire()。
\end{seeAlso}


\chapter{QSslSocket}

QSslSocket 类为客户端和服务端提供了一个 SSL 加密的套接字。

\begin{tabular}{|r|l|}
	\hline
	属性 & 方法 \\
	\hline
	头文件 & \#include <QSslSocket>\\      
	\hline
	qmake & QT += network\\      
	\hline
	引入 &	Qt4.3 \\ 
	\hline
\end{tabular}

该类最初在 Qt 4.3 版本引入。

您可以在 QSslSocket\_Obsolete 界面找到过时的成员函数介绍。

\begin{notice}
该类所有的函数都是可重入的。
\end{notice}


%%%%%%%555

\section{公共成员类型}

\begin{tabular}[l]{|l|m{30em}|}
\hline 
类型 &	属性\\ 
\hline 
enum 	&PeerVerifyMode \{ VerifyNone, QueryPeer, VerifyPeer, AutoVerifyPeer \} \\ 
\hline  
enum &	SslMode \{ UnencryptedMode, SslClientMode, SslServerMode \} \\ 
	\hline
\end{tabular}

%%%%%%%

\section{公共成员函数}

\begin{longtable}[l]{|l|m{30em}|}
\hline 
类型 &	函数名\\ 
\hline 
& QSslSocket(QObject *parent = nullptr) \\ 
\hline
virtual &	$\sim$QSslSocket() \\ 
\hline
void 	&abort() \\ 
\hline
void &	connectToHostEncrypted(const QString \&hostName, quint16 port, QIODevice::OpenMode mode = ReadWrite, QAbstractSocket::NetworkLayerProtocol protocol = AnyIPProtocol) \\ 
\hline
void &	connectToHostEncrypted(const QString \&hostName, quint16 port, const QString \&sslPeerName, QIODevice::OpenMode mode = ReadWrite, QAbstractSocket::NetworkLayerProtocol protocol = AnyIPProtocol) \\ 
\hline
qint64 	&encryptedBytesAvailable() const \\ 
\hline
qint64 &	encryptedBytesToWrite() const \\ 
\hline
bool &	flush() \\ 
\hline
void &	ignoreSslErrors(const QList<QSslError> \&errors) \\ 
\hline
bool 	&isEncrypted() const \\ 
\hline
QSslCertificate &	localCertificate() const \\ 
\hline
QList<QSslCertificate> &	localCertificateChain() const \\ 
\hline
QSslSocket::SslMode &	mode() const \\ 
\hline
QVector<QOcspResponse> &	ocspResponses() const \\ 
\hline
QSslCertificate &	peerCertificate() const \\ 
\hline 
QList<QSslCertificate> 	&peerCertificateChain() const \\ 
\hline
int 	&peerVerifyDepth() const \\ 
\hline
QSslSocket::PeerVerifyMode &peerVerifyMode() const \\ 
\hline
QString &	peerVerifyName() const \\ 
\hline 
QSslKey &	privateKey() const \\ 
\hline
QSsl::SslProtocol &	protocol() const \\ 
\hline
QSslCipher 	&sessionCipher() const \\ 
\hline 
QSsl::SslProtocol 	&sessionProtocol() const \\
\hline
void &	setLocalCertificate(const QSslCertificate \&certificate) \\ 
\hline
void 	&setLocalCertificate(const QString \&path, QSsl::EncodingFormat format = QSsl::Pem) \\ 
\hline
void &	setLocalCertificateChain(const QList<QSslCertificate> \&localChain) \\
\hline
void &	setPeerVerifyDepth(int depth) \\ 
\hline
void &	setPeerVerifyMode(QSslSocket::PeerVerifyMode mode) \\
\hline
void &	setPeerVerifyName(const QString \&hostName) \\ 
\hline
void &	setPrivateKey(const QSslKey \&key) \\ 
\hline 
void 	&setPrivateKey(const QString \&fileName, QSsl::KeyAlgorithm algorithm = QSsl::Rsa, QSsl::EncodingFormat format = QSsl::Pem, const QByteArray \&passPhrase = QByteArray()) \\ 
\hline
void &	setProtocol(QSsl::SslProtocol protocol) \\ 
\hline
void &	setSslConfiguration(const QSslConfiguration \&configuration) \\ 
\hline
QSslConfiguration &	sslConfiguration() const \\ 
\hline
QList<QSslError> 	&sslHandshakeErrors() const \\ 
\hline
bool 	&waitForEncrypted(int msecs = 30000) \\ 
\hline 
\end{longtable}


%%%%%%%%%%%%%%%%%%

\section{重写公共成员函数}

\begin{longtable}[l]{|l|m{30em}|}
\hline 
类型 &	函数名\\ 
\hline 
virtual bool &	atEnd() const override \\ 
\hline
virtual qint64 	&bytesAvailable() const override \\ 
\hline
virtual qint64 &	bytesToWrite() const override \\ 
\hline
virtual bool &	canReadLine() const override \\ 
\hline
virtual void &	close() override \\
\hline
virtual void &	resume() override \\ 
\hline
virtual void &	setReadBufferSize(qint64 size) override \\ 
\hline
virtual bool &	setSocketDescriptor(qintptr socketDescriptor, QAbstractSocket::SocketState state = ConnectedState, QIODevice::OpenMode openMode = ReadWrite) override \\ 
\hline
virtual void &	setSocketOption(QAbstractSocket::SocketOption option, const QVariant \&value) override \\
\hline
virtual QVariant &	socketOption(QAbstractSocket::SocketOption option) override \\ 
\hline
virtual bool &	waitForBytesWritten(int msecs = 30000) override \\ 
\hline
virtual bool &	waitForConnected(int msecs = 30000) override \\ 
\hline
virtual bool &	waitForDisconnected(int msecs = 30000) override \\ 
\hline
virtual bool &	waitForReadyRead(int msecs = 30000) override \\ 
	\hline
\end{longtable}

\section{公共槽函数}

\begin{tabular}[l]{|l|m{30em}|}
	\hline 
	类型 &	函数名\\ 
	\hline 
	void &	ignoreSslErrors() \\ 
	\hline
	void &	startClientEncryption() \\ 
	\hline
	void &	startServerEncryption() \\ 
	\hline
\end{tabular}



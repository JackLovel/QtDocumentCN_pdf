容器

\splitLine

引言

Qt 提供了一系列基于模板的通用容器类,可以用于存储指定类型的元素。例如,如果你需要一个可动态调整大小的 QString 数组,那么你可以使用 QVector<QString>。

这些类的设计目标是比 STL 容器更轻量,更安全,更易用。如果你不熟悉 STL,或想更有“ Qt 范”,使用这些类代替 STL 会是更好的选择。

这些类都是隐式共享和可重入的,并且针对几个方面做了优化,一是速度,二是较低的内存占用,三是尽可能少的内联代码,减少生成程序的体积。另外,在所有线程都以只读的方式访问容器时,这些类是线程安全的。

要遍历容器中的元素,你可以使用两种风格迭代器:Java 风格迭代器和 STL 风格迭代器。Java 风格迭代器有更好的易用性和更高级的函数,而 STL 风格迭代器则在效率上会略有优势,并且可以用于 Qt 和 STL 提供的泛型算法中。

Qt 还提供了 foreach 关键字,可以方便地遍历容器。

注:从 Qt 5.14 开始,绝大部分容器类已经支持范围构造函数,但需要注意的
是 QMultiMap 是一个例外。我们推荐使用该特性代替各种 from/to 方法。例如:

\begin{quote}
译者注:这里的 from/to 方法指的是 Qt 容器类提供的以 from/to 开头的转换方法,如QVector::toList
\end{quote}

\begin{lstlisting}[language=C++]
QVector<int> vector{1, 2, 3, 4, 4, 5};
QSet<int> set(vector.begin(), vector.end());
/*
    将会生成一个 QSet,包含元素 1, 2, 4, 5。
*/
\end{lstlisting}

容器类

\splitLine

Qt 提供了以下几种顺序容器:QList,QLinkedList,QVector,QStack 和 QQueue。对于大多数的应用,QList 是最适用的。虽然其基于数组实现,但支持在头部和尾部快速插入。如果确实需要一个基于链表的列表,你可以使用 QLinkedList;如果要求元素以连续内存的形式保存,那么可以使用 QVector。QStack 和 QQueue提供了 LIFO 和 FIFO 语义的支持。

Qt也提供了一系列关联容器:QMap,QMultiMap,QHash,QMultiHash 和 QSet。"Multi" 容器可以方便地支持键值一对多的情形。“Hash” 容器提供了快速查找的能力,这是通过使用哈希函数代替对有序集合进行二分查找实现的。

较为特殊的是 QCache 和 QContiguousCache,这两个类提供了在有限的缓存中
快速查找元素的能力。


\begin{tabular}{|l|l|}
\hline
类&综述 \\
\hline
QList&	这是目前使用最普遍的容器类,其保存了一个元素类型为T的列表,支持通过索引访问。QList 内部通过数组实现,以确保基于索引的访问足够快。
元素可以通过 QList::append() 和 QList::prepend() 插入到首尾,也可以通过 QList::insert() 插入到列表中间,和其他容器类不同的是,QList 为生成尽可能少的代码做了高度优化。QStringList 继承于 QList<QString>。
QLinkedList)	这个类和 QList 很像,不同的是这个类使用迭代器进行而不是整形索引对元素进行访问。和 QList 相比,其在中间插入大型列表时其性能更优,而且其具有更好的迭代器语义。(在 QLinkedList 中,指向一个元素的迭代器只要该元素存在,则会一直保持有效,而在 QList 的迭代器则可能会在任意的元素插入或删除后失效。)
QVector	这个类以数组的形式保存给定类型的元素,在内存中元素彼此相邻。在
       一个 vector 的头部或中部插入可能会相当慢,因为这可能会导致大量
       元素需要在内存中移动一个位置。\\
\hline
QVarLengthArray<T, Prealloc>&	这个类提供了一个底层的变长数组,在速度极其重要的情况下可以用来代替 QVector\\
QStack&	这个类继承于 QVector,用于为”后进,先出”(LIFO )提供便捷的语
        义支持。其为 QVector 添加了以下方法:QVector::push(),pop() 和
        top()\\
\hline
QQueue&	这个类继承于 QVector,用于为”先进,先出”(FIFO )提供便捷的语
        义支持。其为 QVector 添加了以下方法:QList::enqueue(),
        dequeue() 和 head()\\
\hline
QSet&	这个类提供了一个单值数学集合,支持快速查找\\
\hline
QMap<Key, T>&	这个类提供了一个将类型为Key的键映射到类型为T的值的字典(关联数组)。通常情况下键值是一一对应的。QMap 根据Key进行排序,如果排序无关紧要,使用 QHash 代替速度会更快\\
\hline
QMultiMap<Key, T>&	这个类继承于 QMap,其为诸如键值一对多的多值映射提供了一个友好的接口\\
\hline
QHash<Key, T>&	这个类几乎与 QMap 有完全一致的 API ,但查找效率会有明
               显的提高。QHash 的数据是无序的。\\
\hline
QMultiHash&	这个类继承于 QMap,其为多值映射提供了一个友好的接口\\
\hline
\end{tabular}

容器可以嵌套。例如在键为 QString,值为 QList 时,完全可以使用 QMap<QString, QList>。

容器类的定义位于和容器同名的独立头文件中(例如,<QLinkedList>)。为了方便,在 <QtContainerFwd> 中对所有容器类进行了前置声明。

保存在各个容器中的值类型可以是任意可复制数据类型。为了满足这一要求,该类型必须提供一个复制构造函数和一个赋值运算符。某些操作可能还要求类型支持默认构造函数。对于大多数你想要在容器中保存的类型都满足这些要求,包括基本类型,如 int, double,指针类型,以及 Qt 数据类型,如 QString,QDate 和 QTime,但并不包括 QObject 及其子类(QWidget,QDialog,QTimer 等)。如果你尝试实例化一个 QList<QWidget>,编译器将会抱怨道 QWidget 的复制构造函数和赋值运算符被禁用了。如果需要在容器中保存这些类型的元素,可以保存其指针,如 QList<QWidget *>。

下面是一个满足可赋值数据类型要求的自定义数据类型的例子:

%%% Local Variables:
%%% mode: latex
%%% TeX-master: "../../master"
%%% End:


\chapter{QCoreApplication}

QCoreApplication 类

QCoreApplication类为没有UI的Qt程序提供了一个事件循环。

\begin{tabular}{|r|l|}
\hline
属性&	方法\\
\hline
头文件&	\#include<QCoreApplication>\\
\hline
qmake&	QT+=core\\
\hline
自从&	Qt 4.6\\
\hline
继承&	QObject\\
\hline
派生&	QGuiApplication\\
\hline
\end{tabular}

\section{属性}

\begin{tabular}{|l|l|l|l|}
\hline
属性&	类型&	属性&	类型\\
\hline
applicationName&	QString&	organizationName&	QString\\
\hline
applicationVersion&	QString&	quitLockEnabled	&QString\\
\hline
organizationDomain&	QString&		&\\
\hline
\end{tabular}

\splitLine

\section{公共成员函数}

\subsection{详细说明}

此类为那些没有GUI的应用程序提供消息循环。对于使用Qt但无GUI的程序,它们必须有且仅有一个QCoreApplication对象。对于GUI应用程序,请参见QGuiApplication。对于使用了Qt Widgets的模块,请参见QApplication。

QCoreApplication包含主事件循环,这些来自于操作系统(如定时器、网络事件等)及其它来源的事件将由它来处理和派发。它同时也处理应用程序的初始化和析构,以及全系统和全程序的设置。

\subsection{事件循环及事件处理}

消息循环从调用exec()开始。长时间运行的一些操作也可以通过调用processEvents()让程序保持响应。

一般地,我们建议您尽可能早地在您的main()函数中创建QCoreApplication、QGuiApplication或QApplication。当消息循环退出时,例如当quit()被调用时,exec()才会返回。

我们也提供了一些便捷的静态函数。QCoreApplication对象能够通过instance()来获取。您可以通过sendEvent()来发送事件,以及通过postEvent()来投送事件。待处理的事件能通过removePostedEvents()来移除,亦可通过sendPostedEvents()来派发。

此类提供了一个槽函数quit()及一个信号aboutToQuit()。

\subsection{程序和库路径}

一个应用程序有一个applicationDirPath()和一个applicationFilePath()。库路径(参见QLibrary)能通过libraryPaths()被获取,且能通过setLibraryPaths()、addLibraryPath()和removeLibraryPath()来对它进行操作。

\subsection{国际化和翻译}

翻译文件能分别通过installTranslator()和removeTranslator()被加载和移除。您可以通过translate()来翻译应用中的字符串。QObject::tr()和QObject::trUtf8()这两个函数根据translate()来进行了实现。

\subsection{访问命令行参数}

您应该通过arguments()来获取传递给QCoreApplication构造函数的命令行参数。

\begin{notice}
	QCoreApplication将移除 \hl{-qmljsdebugger="..."} 选项。它会解析
\hl{qmljsdebugger} 参数,然后删除此选项及其参数。
\end{notice}


对于一些更加高级的命令行参数的处理,请创建一个QCommandLineParser。

\subsection{区域设置}

运行在Unix/Linux的Qt程序,将会默认使用系统的区域设置。这可能会导致在使用POSIX函数时发生冲突,例如,数据类型转换时由转换浮点数转换为字符串,由于不同区域符号的差异可能会导致一些冲突。为了解决这个问题,在初始化QGuiApplication、QApplication或QCoreApplication之后,需要马上调用POSIX函数setlocale(LC\_NUMERIC, "C"),以重新将数字的格式设置为"C"-locale。

\begin{notice}[另请参阅]
QGuiApplication, QAbstractEventDispatcher, QEventLoop, Semaphores Example, 以及Wait Conditions Example。
\end{notice}



\splitLine

\section{属性文档}

applicationName : QString

此属性保存应用程序的名字。

当使用空的构造函数初始化QSettings类的实例时,此属性被使用。这样一来,每次创建QSettings对象时,都不必重复此信息。

如果未设置,则应用程序名称默认为可执行文件名称(自5.0开始)。

访问函数:

\begin{tabular}{|l|l|}
\hline
QString	&applicationName()\\
\hline
void	&setApplicationName(const QString \&application)\\
\hline
\end{tabular}

通知信号:

\begin{tabular}{|l|l|}
\hline
void	&applicationNameChanged()\\
\hline
\end{tabular}

另请参阅 organizationName, organizationDomain, applicationVersion, 以及 applicationFilePath()。

applicationVersion: QString

此属性保存应用程序的版本。

如果没有设置此属性,那么此属性将会被默认设置为平台相关的值,该值由主应
用程序的可执行文件或程序包确定(自Qt 5.9起):

\begin{tabular}{|l|l|}
\hline
平台	&源\\
\hline
Windows (经典桌面)&	VERSIONINFO 资源中的 PRODUCTVERSION 参数\\
\hline
Windows通用应用平台(UWP)&	应用程序包中清单文件的版本属性\\
\hline
macOS, iOS, tvOS, watchOS&	信息属性列表中的CFBundleVersion属性\\
\hline
Android	&AndroidManifest.xml清单中的android:versionName属性\\
\hline
\end{tabular}

在其他平台上,此属性默认值为空字符串。

此属性自Qt 4.4引入。

访问函数:

\begin{tabular}{|l|l|}
\hline
QString&	applicationVersion()\\
\hline
void&	setApplicationVersion(const QString \&version)\\
\hline
\end{tabular}

通知信号:

\begin{tabular}{|l|l|}
\hline
void	&applicationVersionChanged()\\
\hline
\end{tabular}

\begin{notice}[另请参阅]
applicationName, organizationName, 以及organizationDomain。
\end{notice}

organizationDomain: QString

此属性保存编写此应用程序的组织的Internet域。

当使用空的构造函数初始化QSettings类的实例时,此属性被使用。这样一来,每次创建QSettings对象时,都不必重复此信息。

在Mac上,如果organizationDomain()不是一个空值,那么QSettings将会使用它;否则它将会使用organizatioName()。在其他平台上,QSettings将organizationName()作为组织名来使用。

访问函数:

\begin{tabular}{|l|l|}
\hline
QString&	organizationName()\\
\hline
void&	setOrganizationName(const QString \&orgName)\\
\hline
\end{tabular}

通知信号:

\begin{tabular}{|l|l|}
\hline
void	&organizationNameChanged()\\
\hline
\end{tabular}



\begin{notice}[另请参阅]
 applicationDomain和applicationName。
\end{notice}

quitLockEnabled: bool

此属性保存使用QEventLoopLocker是否能退出的特性。

默认值是true。

访问函数:

\begin{tabular}{|l|l|}
\hline
bool&	isQuitLockEnabled()\\
hline
void&	setQuitLockEnabled(bool enabled)\\
\hline
\end{tabular}


\begin{notice}[另请参阅]
QEventLoopLocker。
\end{notice}

\splitLine

成员函数文档

QCoreApplication::QCoreApplication(int \&argc, char **argv)

构造一个Qt内核程序。所谓内核程序,就是没有图形用户界面的程序。这样的程序使用控制台,或者是作为服务进程运行着。

argc和argv参数将会被应用程序处理,将其转换为一种更加便捷的形式,并可以通过arguments()来获取。


\begin{notice}[ 警告]
argc和argv这两个值所指向的内存,必须在整个QCoreApplication生命周期内有效。另外,agrc必须要大于0,且argv必须至少包含一个合法的字符串。
\end{notice}

void QCoreApplication::aboutToQuit() [signal]

当程序即将退出主消息循环时,如当消息循环嵌套层数降为0时,此事件被发射。它可能发生在应用程序中调用quit()之后,亦发生在关闭整个桌面会话时。



\begin{notice}
这是一个私有信号。它能够被连接,但是用户无法发射它。
\end{notice}

\begin{notice}[另请参阅]
quit()。
\end{notice}

void QCoreApplication::quit() [static slot]

告知程序以返回值0来退出。等效于调用 QCoreApplication::exit(0)。

一般我们将quit()槽连接到QGuiApplication::lastWindowClosed()信号,您同样可以将此槽连接到QAction、QMenu或QMenuBar的QAbstractButton::clicked()信号上。

将信号以QueuedConnection参数连接此槽是一个不错的实践。如果连接了此槽的一个信号(未在队列中的)在控制流程进入主消息循环前(如在int main中调用exec()之前)被发射,那么这个槽不会有任何效果,应用程序也不会退出。使用队列连接方式能保证控制路程进入主消息循环后,此槽才会被触发。

例如:

\begin{lstlisting}[language=C++]
QPushButton *quitButton = new QPushButton("Quit");
connect(quitButton, &QPushButton::clicked, &app, &QCoreApplication::quit, Qt::QueuedConnection);
\end{lstlisting}

\begin{notice}[另请参阅]
exit(), aboutToQuit(), and QGuiApplication::lastWindowClosed()。
\end{notice}


QCoreApplication::$\sim$QCoreApplication() [virtual]

销毁QCoreApplication对象。

void QCoreApplication::addLibraryPath(const QString \&path) [static]

将path添加到库路径开头,保证它先会被库搜索到。如果path为空或者已经存在于路径列表,那么路径列表保持不变。

默认的路径列表只包含一个条目,即插件安装路径。默认的插件安装文件夹是INSTALL/plugins,其中INSTALL是Qt安装文件夹。

当QCoreApplication被销毁后,这些库路径将会被重设为默认值。

\begin{notice}[另请参阅]
removeLibraryPath(), libraryPaths(), and setLibraryPaths().
\end{notice}

QString QCoreApplication::applicationDirPath() [static]
返回包含此可执行文件的文件夹路径。

例如,您已经在C:/Qt安装了Qt,然后运行regexp示例,那么这个函数会返回"C:/Qt:/examples/tools/regexp"。

在macOS和iOS上,它会指向实际包含可执行文件的目录,该目录可能在一个应用程序包内(如果是以应用程序包形式存在)。

\begin{notice}[警告]
在Linux上,这个函数会尝试从/proc获取文件路径。如果失败了,那么它假设argv[0]包含了可执行文件的绝对路径。此函数同样也假设了应用程序不会改变当前路径。
\end{notice}


\begin{notice}[另请参阅]
applicationFilePath()。
\end{notice}


QString QCoreApplication::applicationFilePath() [static]

返回包含此可执行文件的文件路径。

例如,您已经在/usr/local/qt目录安装了Qt,然后运行regexp示例,那么这个函数会返回"/usr/local/qt/examples/tools/regexp/regexp"。


\begin{notice}[警告]
在Linux上,这个函数会尝试从/proc获取文件路径。如果失败了,那么它假设argv[0]包含了可执行文件的绝对路径。此函数同样也假设了应用程序不会改变当前路径。
\end{notice}




\begin{notice}[另请参阅]
applicationDirPath()。
\end{notice}

qint64 QCoreApplication::applicationPid() [static]

返回当前应用程序的进程ID。

此函数自Qt 4.4引入。

QStringList QCoreApplication::arguments() [static]

返回命令行参数列表。

一般情况下,arguments().at(0)表示可执行文件名,arguments().at(1)是第一个参数,arguments().last()是最后一个参数。请见下面关于Windows的注释。

调用这个函数需要花费很多时间——您应该在解析命令行时,将结果缓存起来。



\begin{notice}[警告]
在Unix下,这个参数列表由main()函数中的argc和argv参数生成。argv中的字符串数据将会通过QString::fromLocal8Bit()来解析,因此,在Latin1的区域环境下,是不可能来传递日语的命令行的,其他情况以此类推。大部分现代Unix系统没有此限制,因为它们是基于Unicode的。
\end{notice}

在Windows下,这个参数列表,只有在构造函数中传入了修改的argc和argv时,才会从这个argc和argv中解析。这种情况下,就会出现编码问题。

如果不是上述情况,那么arguments()将会从GetCommandLine()中构造。此时arguments().at(0)在Windows下未必是可执行文件名,而是取决于程序是如何被启动的。

此函数自Qt 4.1引入。



\begin{notice}[另请参阅]
applicationFilePath()和QCommandLineParser。
\end{notice}

bool QCoreApplication::closingDown() [static]

如果application对象正在被销毁中,则返回true,否则返回false。


\begin{notice}[另请参阅]
startingUp()。
\end{notice}

bool QCoreApplication::event(QEvent *e) [override virtual protected]

重写了:QObject::event(QEvent* e)。

QAbstractEventDispatcher *QCoreApplication::eventDispatcher() [static]

返回指向主线程事件派发器的指针。如果线程中没有事件派发器,则返回nullptr。



\begin{notice}[另请参阅]
 setEventDispatcher()。
\end{notice}

int QCoreApplication::exec() [static]

进入主消息循环,直到exit()被调用。其返回值是为exit()传入的那个参数(如果是调用quit(),等效于调用exit(0))。

通过此函数来开始事件循环是很有必要的。主线程事件循环将从窗口系统接收事件,并派发给应用程序下的窗体。

为了能让您的程序在空闲时来处理事件(在没有待处理的事件时,通过调用一个特殊的函数),可以使用一个超时为0的QTimer。可以使用processEvents()来跟进一步处理空闲事件。

我们建议您连接aboutToQuit()信号来做一些清理工作,而不是将它们放在main函数中。因为在某些平台下,exec()可能不会返回。例如,在Windows下,当用户注销时,系统将在Qt关闭所有顶层窗口后才终止进程。因此,不能保证程序有时间退出其消息循环来执行main函数中exec()之后的代码。


\begin{notice}[另请参阅]
quit(),exit(), processEvent()和QApplication::exec()。

\end{notice}

void QCoreApplication::exit(int returnCode = 0) [static]

告诉程序需要退出了,并带上一个返回值。

在此函数被调用后,程序将离开主消息循环,并且从exec()中返回。其返回值就是returnCode。如果消息循环没有运行,那么此方法什么都不做。

一般我们约定0表示成功,非0表示产生了一个错误。

将信号以QueuedConnection参数连接此槽是一个不错的实践。如果连接了此槽的一个信号(未在队列中的)在控制流程进入主消息循环前(如在int main中调用exec()之前)被发射,那么这个槽不会有任何效果,应用程序也不会退出。使用队列连接方式能保证控制路程进入主消息循环后,此槽才会被触发。


\begin{notice}[另请参阅]
quit()和exec()。
\end{notice}

void QCoreApplication::installNativeEventFilter(QAbstractNativeEventFilter *filterObj)

在主线程为应用程序所能接收到的原生事件安装一个事件过滤器。

事件过滤器filterObj通过nativeEventFilter()来接收事件。它可以接收到主线程所有的原生事件。

如果某个原生事件需要被过滤或被屏蔽,那么QAbstractNativeEventFilter::nativeEventFilter需要返回true。如果它需要使用Qt默认处理流程,则返回false:那么接下来这个原生事件则会被翻译为一个QEvent,并且由Qt的标准事件过滤器来处理。

如果有多个事件过滤器被安装了,那么最后安装的过滤器将会被最先调用。


\begin{notice}
 此处设置的过滤器功能接收原生事件,即MSG或XCB事件结构。
\end{notice}


\begin{notice}
如果设置了Qt::AA\_PluginApplication属性,那么原生事件过滤器将会被屏蔽。
\end{notice}


为了最大可能保持可移植性,您应该总是尽可能使用QEvent和QObject::installEventFilter()。



\begin{notice}[另请参阅]
QObject::installEventFilter()。
\end{notice}

bool QCoreApplication::installTranslator(QTranslator *translationFile) [static]

将translationFile添加到翻译文件列表,它将会被用于翻译。

您可以安装多个翻译文件。这些翻译文件将会按照安装顺序的逆序被搜索到,因此最近添加的翻译文件会首先被搜索到,第一个安装的搜索文件会最后被搜索。一旦翻译文件中匹配了一个字符串,那么搜索就会终止。

安装、移除一个QTranslator,或者更改一个已经安装的QTranslator将会为QCoreApplication实例产生一个LanguageChange事件。一个QApplication会将这个事件派发到所有的顶层窗体,使用tr()来传递用户可见的字符串到对应的属性设置器,通过这种方式来重新实现changeEvent则可以重新翻译用户的界面。通过Qt设计师(Qt Designer)生成的窗体类提供了一个retranslateUi()可以实现上述效果。

函数若执行成功则返回true,失败则返回false。



\begin{notice}[另请参阅]
removeTranslator(),translate(),QTranslator::load()和动态翻译。
\end{notice}

QCoreApplication* QCoreApplication::instance() [static]

返回程序的QCoreApplication (或QGuiApplication/QApplication)实例的指针。

bool QCoreApplication::isSetuidAllowed() [static]

如果在UNIX平台中,允许应用程序使用setuid,则返回true。

此函数自Qt 5.3引入。


\begin{notice}[另请参阅]
QCoreApplication::setSetuidAllowed()。
\end{notice}

QStringList QCoreApplication::libraryPaths() [static]

返回一个路径列表,其中的路径表示动态加载链接库时的搜索路径。

此函数的返回值也许会在QCoreApplication创建之后改变,因此不建议在QCoreApplication创建之前调用。应用程序所在的路径(非工作路径),如果是已知的,那么它会被放入列表中。为了能知道这个路径,QCoreApplication必须要在创建时使用argv[0]来表示此路径。

Qt提供默认的库搜索路径,但是它们同样也可以通过qt.conf文件配置。在此文件中所指定的路径会覆盖默认路径。注意如果qt.conf文件存在于应用程序所在的文件夹目录下,那么直到QCoreApplication被创建时它才可以被发现。如果它没有被发现,那么调用此函数仍然返回默认搜索路径。

如果插件存在,那么这个列表会包含插件安装目录(默认的插件安装目录是
\hl{INSTALL/plugins},其中 \hl{INSTALL} 是Qt所安装的目录。用分号分隔的\hl{QT\_PLUGIN\_PATH}环境变量中的条目一定会被添加到列表。插件安装目录(以及它存在)在应用程序目录已知时可能会被更改。

如果您想遍历列表,可以使用foreach伪关键字:

\begin{lstlisting}[language=C++]
foreach (const QString &path, app.libraryPaths())
    do_something(path);
\end{lstlisting}



\begin{notice}[另请参阅]
 setLibraryPaths(), addLibraryPath(), removeLibraryPath(), QLibrary , 以及 如何创建Qt插件。
\end{notice}

bool QCoreApplication::notify(QObject receiver, QEvent event) [virtual]

将事件发送给接收者:receiver->event(event)。其返回值为接受者的事件处理器的返回值。注意这个函数将会在任意线程中调用,并将事件转发给任意对象。

对于一些特定的事件(如鼠标、键盘事件),如果事件处理器不处理此事件(也就是它返回false),那么事件会被逐级派发到对象的父亲,一直派发到顶层对象。

处理事件有5种不同的方式:重写虚函数只是其中一种。所有的五种途径如下所
示:

\begin{compactenum}
\item 重写paintEvent(),mousePressEvent()等。这个是最通用、最简单但最不强大的一种方法。
\item 重写此函数。这非常强大,提供了完全控制,但是一次只能激活一个子类。
\item 将一个事件过滤器安装到QCoreApplication。这样的一个事件过滤器可以处理所有窗体的所有事件,就像是重写了notify()函数这样强大。此外,您还可以提供多个应用级别全局的事件过滤器。全局事件过滤器甚至可以接收到那些不可用窗体的鼠标事件。注意程序的事件过滤器仅能响应主线程中的对象。
\item 重写QObject::event()(就像QWidget那样)。如果您是这样做的,您可以接收到Tab按键,及您可以在任何特定窗体的事件过滤器被调用之前接收到事件。
\item 在对象上安装事件过滤器。这样的事件过滤器将可以收到所有事件,包括Tab和Shift+Tab事件——只要它们不更改窗体的焦点。
\end{compactenum}


未来规划:在Qt 6中,这个函数不会响应主线程之外的对象。需要该功能的应用程序应同时为其事件检查需求找到其他解决方案。该更改可能会扩展到主线程,因此不建议使用此功能。

\begin{notice}
如果您重写此函数,在您的应用程序开始析构之前,您必须保证所有正在处理事件的线程停止处理事件。这包括了您可能在用的其他库所创建的线程,但是不适用于Qt自己的线程。
\end{notice}


\begin{notice}[另请参阅]
QObject::event()和installNativeEventFilter()。
\end{notice}


void QCoreApplication::postEvent(QObject* receiver, QEvent* event, int priority = Qt::NormalEventPriority) [static]

添加一个event事件,其中receiver表示事件的接收方。事件被添加到消息队列,并立即返回。

被添加的事件必须被分配在堆上,这样消息队列才能接管此事件,并在它被投送之后删除它。当它被投递之后,再来访问此事件是不安全的。

当程序流程返回到了主事件循环时,所有的队列中的事件会通过notify()来发送。

队列中的事件按照priority降序排列,这意味着高优先级的事件将排列于低优先
级之前。优先级priority可以是任何整数,只要它们在\hl{INT\_MAX} 和
\hl{INT\_MIN} 之闭区间内。相同优先级的事件会按照投送顺序被处理。


\begin{notice}
此函数是线程安全的。
\end{notice}


此函数自Qt 4.3引入。

\begin{notice}[另请参阅]
sendEvent(),notify(),sendPostedEvents(),和Qt::EventPriority。

\end{notice}


void QCoreApplication::processEvents(QEventLoop::ProcessEventsFlags flags = QEventLoop::AllEvents) [static]

根据flags处理调用线程的所有待处理事件,直到没有事件需要处理。

您可以在您程序进行一项长时间操作的时候偶尔调用此函数(例如拷贝一个文件时)。

如果您在一个本地循环中持续调用这个函数,而不是在消息循环中,那么DeferredDelete事件不会被处理。这会影响到一些窗体的行为,例如QToolTip,它依赖DeferredDelete事件。以使其正常运行。一种替代方法是从该本地循环中调用sendPostedEvents()。

此函数只处理调用线程的事件,当所有可处理事件处理完毕之后返回。可用事件是在函数调用之前排队的事件。这意味着在函数运行时投送的事件将会排队到下一轮事件处理为止。


\begin{notice}
此函数是线程安全的。
\end{notice}

\begin{notice}[另请参阅]
exec(),QTimer,QEventLoop::processEvents(),flush()和sendPostedEvents()。
\end{notice}



void QCoreApplication::processEvents(QEventLoop::ProcessEventsFlags
flags = QEventLoop::AllEvents, int ms) [static]

此函数重写了processEvents()。

此函数将用ms毫秒为调用线程处理待处理的事件,或者直到没有更多事件需要处理。

您可以在您程序进行一项长时间操作的时候偶尔调用此函数(例如拷贝一个文件时)。

此函数只处理调用线程的事件。


\begin{notice}
不像processEvents()的重写,这个函数同样也处理当函数正在运行中时被投送的事件。),QTimer,QEventLoop::processEvents()。
\end{notice}




\begin{notice}[另请参阅]
exec(),QTimer,QEventLoop::processEvents()。
\end{notice}

void QCoreApplication::removeLibraryPath(const QString \&path) [static]

从库的搜索路径列表中移除path。如果path是空的,或者不存在于列表,则列表不会改变。

当QCoreApplication被析构时,此列表会被还原。


\begin{notice}[另请参阅]
exec(),QTimer和QEventLoop::processEvents()。
\end{notice}


void QCoreApplication::removeNativeEventFilter(QAbstractNativeEventFilter *filterObject)

从此实例中移除filterObject事件过滤器。如果这个事件过滤器没有被安装,则什么也不做。

当此实例被销毁时,所有的事件过滤器都会自动被移除。

任何时候——甚至正在事件过滤被激活时(通过nativeEventFilter()函数)——移除一个事件过滤器都是安全的。

此函数自Qt 5.0引入。



\begin{notice}[另请参阅]
installNativeEventFilter()。
\end{notice}


void QCoreApplication::removePostedEvents(QObject *receiver, int eventType = 0) [static]

移除所指定的 eventType 类型且由postEvent()所添加的事件。

这些事件不会被派发,而是直接从队列中移除。您从来都不需要调用此方法。如果您确实调用了它,那么请注意杀掉事件可能会影响 receiver 的不变性(invariant)。

如果接收者为 \hl{nullptr},那么所有对象将会移除 eventType 所指定的所有事件。如果 eventType 为0,那么receiver的所有事件将会被移除。自始自终,您都不应将 0 传递给 eventType 。


\begin{notice}
此函数是线程安全的。
\end{notice}

此函数自Qt 4.3引入。

bool QCoreApplication::removeTranslator(QTranslator *translationFile) [static]

从此应用程序使用的翻译文件列表中删除翻译文件 translationFile 。 (不会在文件系统中删除此翻译文件。)

该函数成功时返回 true ,失败时返回 false 。


\begin{notice}[另请参阅]
installTranslator(),translate(),与 QObject::tr()。
\end{notice}

bool QCoreApplication::sendEvent(QObject *receiver, QEvent *event) [static]

通过函数 notify() 将事件 event 直接发送至接收对象 receiver 。返回从事件处理对象得到的返回值。

事件 不会 在发送之后删除掉。通常的方法是在栈上创建事件,例如:

\begin{lstlisting}[language=C++]
QMouseEvent event(QEvent::MouseButtonPress, pos, 0, 0, 0);
QApplication::sendEvent(mainWindow, &event);
\end{lstlisting}



\begin{notice}[另请参阅]
postEvent() 与 notify()。
\end{notice}

void QCoreApplication::sendPostedEvents(QObject *receiver = nullptr, int event\_type = 0) [static]

立刻分派所有先前通过 QCoreApplication::postEvent() 进入队列的事件。这些事件是针对接收对象的,且事件类型为 event\_type 。

该函数不会对来自窗口系统的事件进行分派,如有需求请参考 processEvents()。

如果接收对象指针为 nullptr ,event\_type 的所有事件将分派到所有对象。如果 event\_type 为 0 ,所有事件将发送到接收对象 receiver 。


\begin{notice}
该方法必须由其 QObject 参数 receiver 所在的线程调用。

\end{notice}


\begin{notice}[另请参阅]
flush() 与 postEvent()。
\end{notice}


void QCoreApplication::setAttribute(Qt::ApplicationAttribute attribute, bool on = true) [static]

如果 on 为 true ,则设置属性 attribute;否则清除该属性。



\begin{notice}
在创建 QCoreApplication 实例之前,一些应用程序的属性必须要设置。 有关更多信息,请参考 Qt::ApplicationAttribute 文档。
\end{notice}


\begin{notice}[另请参阅]
testAttribute()。
\end{notice}


void QCoreApplication::setEventDispatcher(QAbstractEventDispatcher *eventDispatcher) [static]

将主线程的事件分派器设置为 eventDispatcher 。只有在还没有安装事件分派器的情况下,也就是在实例化 QCoreApplication 之前,才可以进行设置。此方法获取对象的所有权。

\begin{notice}[另请参阅]
eventDispatcher()。
\end{notice}

void QCoreApplication::setLibraryPaths(const QStringList \&paths) [static]

将加载库时要搜索的目录列表设置为 paths。现有的所有路径将被删除,路径列表将从参数 paths 中获取。

当实例 QCoreApplication 被析构时,库路径将设置为默认值。

\begin{notice}[另请参阅]
libraryPaths()、addLibraryPath()、removeLibraryPath() 与 QLibrary。
\end{notice}

void QCoreApplication::setSetuidAllowed(bool allow) [static]

若 allow 为 true,允许程序在 UNIX 平台上运行 setuid 。

若 allow 为 false (默认值),且 Qt 检测到程序使用与实际用户id不同的有效用户id运行,那么在创建 QCoreApplication 实例时程序将中止。

Qt 受攻击面较大,因此它并不是一个恰当 setuid 程序解决方案。 不过,出于历史原因,可能某些程序仍需要这种方式运行。 当检测到此标志时,可防止 Qt 中止应用程序。须在创建 QCoreApplication 实例之前将其进行设置。



\begin{notice}
强烈建议不要打开此选项,它会带来安全风险。
\end{notice}

此函数自Qt 5.3引入。

\begin{notice}[另请参阅]
isSetuidAllowed()。
\end{notice}

bool QCoreApplication::startingUp()

如果应用程序对象尚未创建,则返回 true ;否则返回 false 。



\begin{notice}[另请参阅]
closingDown()。
\end{notice}

bool QCoreApplication::testAttribute(Qt::ApplicationAttribute attribute) [static]

如果设置了属性 attribute ,返回 true ;否则返回 false 。


\begin{notice}[另请参阅]
setAttribute()。
\end{notice}

QString QCoreApplication::translate(const char *context, const char *sourceText, const char *disambiguation = nullptr, int n = -1) [static]

搜索最近到首次安装的翻译文件,查询翻译文件中 sourceText 对应的翻译内容,返回其结果。

QObject::tr() 作用相同且使用更便捷。

context 通常为类名(如 "MyDialog"),sourceText 则是英文文本或简短的识别文本。

disambiguation 为一段识别字符串,用于同一上下文中不同角色使用相同的 sourceText 。它的默认值为 nullptr。

有关上下文、消除歧义和注释的更多信息,请参阅 QTranslator 和 QObject::tr() 文档。

n 与 \%n 一并使用以便支持复数形式。详见 QObject::tr() 。

如果没有任何翻译文件包含 context 中 sourceText 的翻译内容,该函数则返回 QString 包裹的 sourceText 。

此函数非虚函数。您可以子类化 QTranslator 来使用其他翻译技术。


\begin{notice}
此函数线程安全。
\end{notice}


\begin{notice}[另请参阅]
QObject::tr(),installTranslator(),removeTranslator(),与 translate()。
\end{notice}

\splitLine

相关非成员函数

void qAddPostRoutine(QtCleanUpFunction ptr)

添加一个全局例程,它将在 QCoreApplication 析构函数中调用。这个函数通常用来作为在程序范围功能内,添加程序清除例程的函数。

清除例程的调用顺序与添加例程的顺序相反。

参数 \emph{ptr} 指定的函数应既没有参数,也没有返回值,例如:


\begin{lstlisting}[language=C++]
static int *global_ptr = nullptr;

static void cleanup_ptr()
{
    delete [] global_ptr;
    global_ptr = nullptr;
}

void init_ptr()
{
    global_ptr = new int[100];      // allocate data
    qAddPostRoutine(cleanup_ptr);   // delete later
}
\end{lstlisting}



\begin{notice}
对于应用程序或模块范围的清理,qAddPostRoutine() 通常不太合适。例如,若程序被分割成动态加载的模块,那么相关的模块可能在 QCoreApplication 的析构函数调用之前就卸载了。在这种情况下,如果仍想使用 qAddPostRoutine() ,可以使用 qRemovePostRoutine() 来防止 QCoreApplication 的析构函数调用一个例程。举个例子,在模块卸载前调用 qRemovePostRoutine() 。
\end{notice}

对于模块或库,使用引用计数初始化管理器,或者 Qt 对象树删除机制可能会更
好。下面是一个私有类使用对象树机制正确调用清除函数的一个例子:

\begin{lstlisting}[language=C++]
class MyPrivateInitStuff : public QObject
{
public:
    static MyPrivateInitStuff *initStuff(QObject *parent)
    {
        if (!p)
            p = new MyPrivateInitStuff(parent);
        return p;
    }

    ~MyPrivateInitStuff()
    {
        // cleanup goes here
    }

private:
    MyPrivateInitStuff(QObject *parent)
        : QObject(parent)
    {
        // initialization goes here
    }

    MyPrivateInitStuff *p;
};
\end{lstlisting}

通过选择正确的父对象,正常情况下可以在正确的时机清理模块的数据。

\begin{notice}
 函数自 Qt 5.10 已线程安全
\end{notice}

\begin{notice}
该函数线程安全
\end{notice}

\begin{notice}[另请参阅]
qRemovePostRoutine()。
\end{notice}


void qRemovePostRoutine(QtCleanUpFunction ptr)

\begin{notice}
函数自 Qt 5.10 已线程安全
\end{notice}

 

\begin{notice}
该函数线程安全
\end{notice}


此函数自 Qt 5.3 引入。



\begin{notice}[另请参阅]
qAddPostRoutine()。
\end{notice}

\splitLine

\section{宏文档}

Q\_COREAPP\_STARTUP\_FUNCTION(QtStartUpFunction ptr)

添加一个全局函数,它将在 QCoreApplication 的构造函数中调用。这个宏通常用于为程序范围内的功能初始化库,无需应用程序调用库进行初始化。

参数 \emph{ptr} 指定的函数应既没有参数,也没有返回值,例如:


\begin{lstlisting}[language=C++]
// Called once QCoreApplication exists
static void preRoutineMyDebugTool()
{
    MyDebugTool* tool = new MyDebugTool(QCoreApplication::instance());
    QCoreApplication::instance()->installEventFilter(tool);
}

Q_COREAPP_STARTUP_FUNCTION(preRoutineMyDebugTool)
\end{lstlisting}



 

如果删除了 QCoreApplication 并创建了另一个 QCoreApplication ,将再次调用启动函数。

 
\begin{notice}
此宏不适用于静态链接到应用程序中的库代码中使用,因为链接器可能会删除该函数,使其根本不会被调用。
\end{notice}


\begin{notice}
 此函数是可重入的。
\end{notice}


此函数自 Qt 5.1 引入。

\splitLine

Q\_DECLARE\_TR\_FUNCTIONS(context)

Q\_DECLARE\_TR\_FUNCTIONS() 宏使用以下签名声明并实现了两个转换函数 tr() 和 trUtf8():

\begin{lstlisting}[language=C++]
static inline QString tr(const char *sourceText,
                         const char *comment = nullptr);
static inline QString trUtf8(const char *sourceText,
                             const char *comment = nullptr);
\end{lstlisting}

如果您想在不继承 QObject 的类使用 QObject::tr() 或者 QObject::Utf8() ,这个宏就非常适用。

\textbf{Q\_DECLARE\_TR\_FUNCTIONS()} 宏必须写在类的第一行(在第一个 public: 或
protected: 之前),例如:

\begin{lstlisting}[language=C++]
class MyMfcView : public CView
{
    Q_DECLARE_TR_FUNCTIONS(MyMfcView)

public:
    MyMfcView();
    ...
};
\end{lstlisting}

通常 \emph{context} 参数为类名,不过也可以是任何文本。

\begin{notice}[另请参阅]
	Q\_OBJECT,QObject::tr(),与 QObject::trUtf8()。
\end{notice}

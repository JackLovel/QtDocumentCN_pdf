\chapter{QPluginLoader}

QPluginLoader 在运行时加载插件。

\begin{tabular}{|l|l|}
\hline
属性 &	方法\\
\hline
头文件:& 	\#include <QPluginLoader>\\
\hline
qmake:& 	QT += core\\
\hline
继承:	&  \href{https://gitee.com/wcc210/QtDocumentCN/blob/master/Src/O/QObject/QObject.md}{QObject} \\
\hline
\end{tabular}

\begin{notice}
该类提供的所有函数都是可重入的。
\end{notice}


\section{属性}


\begin{tabular}{|l|l|}
\hline
属性 &	类型\\
\hline
fileName &	QString \\ 
\hline
loadHints	& QLibrary::LoadHints \\ 
\hline
\end{tabular}

\section{公共成员函数}

\begin{longtable}[l]{|l|m{25em}|}
\hline
 类型& 	函数名\\
\hline
& QPluginLoader(const QString \emph{\&*fileName}, QObject \emph{*parent} = nullptr) \\
\hline
&QPluginLoader(QObject *parent = nullptr) \\
\hline
virtual	&$\sim$QPluginLoader() \\
\hline
QString	&errorString() const \\
\hline
QString	&fileName() const \\
\hline
QObject *	&instance() \\
\hline
bool	&isLoaded() const \\
\hline
bool	&load() \\
\hline
QLibrary::LoadHints	&loadHints() const \\
\hline
QJsonObject	&metaData() const \\
\hline
void	&setFileName(const QString \emph{\&fileName}) \\
\hline
void&	setLoadHints(QLibrary::LoadHints \emph{loadHints}) \\
\hline
bool&	unload() \\
\hline
\end{longtable}

\section{静态公共成员函数}

\begin{tabular}{|l|l|}
\hline
返回类型& 	函数名\\
\hline
QObjectList	&staticInstances() \\
\hline
QVector<QStaticPlugin> &	staticPlugins() \\
\hline
\end{tabular}


\section{相关的非成员函数}

\begin{tabular}{|l|l|}
\hline
返回类型& 	函数名\\
\hline
void	& qRegisterStaticPluginFunction(QStaticPlugin \emph{plugin}) \\
\hline
\end{tabular}


\section{详细介绍}

QPluginLoader 提供对 Qt 插件的访问。
Qt 插件存储在共享库(DLL)中,而相比使用 QLibrary 访问的共享库,它具有以下优点:

\begin{compactitem}
\item QPluginLoader 检查插件是否链接到与应用程序相同的 Qt 版本。
\item QPluginLoader 提供对根组件对象的直接访问(instance()),而无需手动解析C函数。
\end{compactitem}

QPluginLoader对象的实例在被称为插件的单个共享库文件上运行。
它以独立于平台的方式提供对插件中功能的访问。
要指定加载的插件,可以在构造函数中传递文件名,或者通过 setFileName() 进行设置。

最重要的函数有:用来动态加载插件文件的 load(),
用来检查加载是否成功的 isLoaded() , 以及用来访问插件根组件的 instance()。
如果尚未加载插件,则 instance() 函数会隐式尝试加载该插件。
 可以使用 QPluginLoader 的多个实例来访问同一个实际的插件。

加载后,插件将保留在内存中,直到所有 QPluginLoader 实例都已卸载,或者应用程序终止为止。
您可以使用 unload() 来尝试卸载插件,但如果有其它 QPluginLoader 实例正在使用同一个库,
那么这一函数调用会失败,而当所有实例都调用了 unload() 后插件才会真正被卸载。
在卸载发生之前,根组件也将被删除。

有关如何使应用程序可通过插件扩展的更多信息,请参见如何创建 Qt 插件。

请注意,如果您的应用程序与 Qt 静态链接,则无法使用 QPluginLoader。
在这种情况下,您还必须静态链接到插件。
如果需要在静态链接的应用程序中加载动态库,则可以使用 QLibrary。

\begin{seeAlso}
QLibrary 和插件与绘制示例。
\end{seeAlso}

\section{属性文档}

fileName : QString

该属性记录插件的文件名。

我们建议在文件名中省略文件的后缀,因为 QPluginLoader 将自动查找具有适当后缀的文件(请参阅 QLibrary::isLibrary())。

加载插件时,除非文件名具有绝对路径,
否则 QPluginLoader 会搜索 QCoreApplication::libraryPaths() 指定的所有插件位置。
成功加载插件后,fileName() 返回插件的完全限定文件名,
如果在构造函数中已指定或传递给 setFileName(),则包括插件的完整路径。

如果文件名不存在,改属性将不会设置,并包含一个空字符串。

默认情况下,该属性包含一个空字符串。

存取函数


\begin{tabular}{|l|l|}
\hline
返回类型& 	函数名\\
\hline
QString	& fileName() const \\ 
\hline
void	& setFileName(const QString \emph{\&fileName}) \\ 
\hline
\end{tabular}

\begin{seeAlso}
load()。
\end{seeAlso}

loadHints : QLibrary::LoadHints

为 load() 函数提供一些有关其行为方式的提示。

您可以提供有关如何解析插件中符号的提示。
从 Qt 5.7 起,默认设置为 QLibrary::PreventUnloadHint。

有关该属性如何工作的完整说明,请参阅 QLibrary::loadHints 的文档。

该属性在 Qt 4.4 中引入。

存取函数

\begin{tabular}{|l|l|}
\hline
返回类型& 	函数名\\
\hline
QLibrary::LoadHints	& loadHints() const \\ 
\hline
void & setLoadHints(QLibrary::LoadHints \emph{loadHints}) \\
\hline
\end{tabular}

\begin{seeAlso}
QLibrary::loadHints。
\end{seeAlso}

\section{成员函数文档}

%%%%%%%%%%%%%%%%%%%%%%%%%%%%

QPluginLoader::QPluginLoader(const QString \&fileName, QObject *parent = nullptr)

使用给定的 parent 构造一个插件加载器,并加载 fileName 指定的插件。

为了可加载,文件的后缀必须是可加载库的有效后缀,具体取决于平台,例如,Unix 上的 .so,macOS 和 iOS .dylib,以及 Windows 上的 .dll。后缀可以通过 QLibrary::isLibrary() 验证。

\begin{seeAlso}
setFileName()。
\end{seeAlso}

QPluginLoader::QPluginLoader(QObject \emph{*parent} = nullptr)

使用给定的 parent 构造一个插件加载器。

[virtual]QPluginLoader::$\sim$QPluginLoader()

销毁 QPluginLoader 对象。

除非 unload() 被显式调用,插件会一直留在内存中直到程序结束。

\begin{seeAlso}
isLoaded() 和 unload()。
\end{seeAlso}

QString QPluginLoader::errorString() const

返回带有最后发生的错误描述文本的字符串。

该函数在 Qt 4.2 中引入。

QObject *QPluginLoader::instance()

返回插件的根组件对象。必要时会加载插件。
如果无法加载插件或者根组件对象无法实例化时,该函数将返回 \hl{nullptr}。

如果根组件对象已经被销毁了,该函数在调用时会创建一个新的实例。

该函数返回的根组件不会随着 QPluginLoader 的销毁而被删除。
如果您希望保证根组件会被删除,可以在您不再需要访问核心组件是立即调用 unload()。
当库最终卸载时,对应根组件也会自动删除。

组件对象是一个 QObject。使用 qobject\_cast() 来访问你想要的接口。

\begin{seeAlso}
load()。
\end{seeAlso}

bool QPluginLoader::isLoaded() const

如果已经成功加载插件则返回 \hl{true},否则返回 \hl{false}。

\begin{seeAlso}
load()。
\end{seeAlso}

bool QPluginLoader::load()

加载插件,并在插件成功加载时返回 \hl{true},否则返回 \hl{false}。
由于 instance() 始终在解析任何符号之前调用此函数,因此无需显式调用它。
在某些情况下,您可能需要预先加载插件,这时您才要使用该函数。

\begin{seeAlso}
unload()。
\end{seeAlso}

QJsonObject QPluginLoader::metaData() const

返回该插件的元数据。元数据是在编译插件时使用 Q\_PLUGIN\_METADATA() 宏以json格式指定的数据。

无需实际加载插件即可以快速又经济的方式查询元数据。
这使得例如可以在其中储存插件的功能,并根据该元数据来决定是否加载插件。

[static]QObjectList QPluginLoader::staticInstances()

返回由插件加载器保存的静态插件实例(根组件)的列表。

另请参阅 staticPlugins()。

[static]QVector<QStaticPlugin> QPluginLoader::staticPlugins()

返回由插件加载器保存的 QStaticPlugins 列表。 
该函数类似于 staticInstances(),
除了 QStaticPlugin 还包含元数据信息。

\begin{seeAlso}
staticInstances()。
\end{seeAlso}

bool QPluginLoader::unload()

卸载插件,并在插件卸载成功时返回 \hl{true},否则返回 \hl{false}。
这会在应用程序终止时自动发生,因此您通常不需要调用此函数。
如果存在其它 QPluginLoader 实例正在使用同一个插件,调用会失败,
卸载只会发生在所有实例都调用了 \hl{unload()} 时。
不要试图删除根组件。
相反,凭借 \hl{unload()} ,它会在必要时自动将其删除。

\begin{seeAlso}
instance() 和 load()。
\end{seeAlso}

\section{相关的非成员函数}

void qRegisterStaticPluginFunction(QStaticPlugin \emph{plugin})

注册由插件加载器指定的 \emph{plugin},并由 Q\_IMPORT\_PLUGIN() 使用。

该函数在 Qt 5.0 中引入。

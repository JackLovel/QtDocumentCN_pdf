\chapter{QWaitCondition}

QWaitCondition 提供一个用于同步线程的条件变量。\href{https://github.com/QtDocumentCN/QtDocumentCN/blob/master/Src/W/QWaitCondition/QWaitCondition.md#%E8%AF%A6%E7%BB%86%E6%8F%8F%E8%BF%B0}{更多...} 
	
\begin{tabular}{|r|l|}
	\hline
	属性 & 内容 \\
	\hline
	头文件 & \#include<QWaitCondition>\\      
	\hline
	qmake & QT += core\\      
	\hline
\end{tabular}

\begin{notice}
	此类中所有函数都是\href{https://github.com/QtDocumentCN/QtDocumentCN/blob/master/Src/R/Reentrancy_and_Thread-Safety/Reentrancy_and_Thread-Safety.md}{线程安全}的。
\end{notice}

 	
\section{公共成员函数}

\begin{tabular}{|r|l|}
	\hline
	返回类型 & 函数 \\
	\hline
	int	& appDpiX(int screen = -1)\\
	\hline
	&QWaitCondition()\\
	\hline
	& $\sim$QWaitCondition()\\
	\hline
	void &	notify\_all()\\
	\hline
	void	&notify\_one()\\
		\hline
	bool&	wait(QMutex *lockedMutex, QDeadlineTimer deadline = QDeadlineTimer(QDeadlineTimer::Forever)) \\
	\hline
	bool &	wait(QMutex *lockedMutex, unsigned long time) \\
	\hline
	bool&	wait(QReadWriteLock *lockedReadWriteLock, QDeadlineTimer deadline = QDeadlineTimer(QDeadlineTimer::Forever)) \\
	\hline
	bool &	wait(QReadWriteLock *lockedReadWriteLock, unsigned long time) \\
	void &	wakeAll()\\
	\hline
	void &	wakeOne()\\
	\hline
\end{tabular}

\section{详细描述}

QWaitCondition允许线程告诉其他线程某种条件已经满足。一个或多个线程可以被阻塞并等待 QWaitCondition 用 wakeOne() 或 wakeAll() 设置条件。使用 wakeOne() 唤醒一个随机选择的线程,或使用 wakeAll() 唤醒所有线程。

例,假设我们有三个任务,当用户按下一个键时,应该执行某些任务。每个任务可以分成一个线程,每个线程都有一个 run() 主体, 如下所示:

\begin{lstlisting}[language=C++]
forever {
	mutex.lock();
	keyPressed.wait(&mutex);
	do_something();
	mutex.unlock();
}
\end{lstlisting}

这里,keyPressed 变量是 QWaitCondition 类型的全局变量。
第四个线程将读取按键,并在每次收到按键时唤醒其他三个线程,如下所示:

\begin{lstlisting}[language=C++]
forever {
	getchar();
	keyPressed.wakeAll();
}
\end{lstlisting}

唤醒三个线程的顺序是未知的。另外,如果某些线程在按下键时仍在 do\_something() 中,它们将不会被唤醒(因为它们没有等待条件变量),因此该按键不会执行任务。这个问题可以通过使用计数器和 QMutex() 来解决。例如,下面是工作线程的新代码:

\begin{lstlisting}[language=C++]
forever {
	mutex.lock();
	keyPressed.wait(&mutex);
	++count;
	mutex.unlock();
	
	do_something();
	
	mutex.lock();
	--count;
	mutex.unlock();
}
\end{lstlisting}

下面是第四个线程的代码:

\begin{lstlisting}[language=C++]
forever {
	getchar();
	
	mutex.lock();
	// Sleep until there are no busy worker threads
	while (count > 0) {
		mutex.unlock();
		sleep(1);
		mutex.lock();
	}
	keyPressed.wakeAll();
	mutex.unlock();
}
\end{lstlisting}

互斥量是必需的,因为当两个线程同时更改同一变量的值时,结果是不可预测的。
等待条件是一个强大的线程同步原语。Wait Conditions示例演示了如何使用 QWaitCondition 作为 QSemaphore() 的替代品,来控制生产者消费者的共享循环缓冲区的访问。

\begin{seeAlso}
QMutex、QSemaphore、QThread() 和 Wait Conditions示例。
\end{seeAlso}


\section{成员函数文档}

QWaitCondition::QWaitCondition()

构造。

QWaitCondition::~QWaitCondition()

析构。

void QWaitCondition::notify\_all()

用于STL兼容。它相当于 wakeAll()。

在 Qt 5.8 引入该函数。

void QWaitCondition::notify\_one()

用于STL兼容。它相当于 wakeOne()。
在 Qt 5.8 引入该函数。

bool QWaitCondition::wait(QMutex *lockedMutex, QDeadlineTimer deadline = QDeadlineTimer(QDeadlineTimer::Forever))

释放 lockedMutex 并等待条件。lockedMutex 最初必须由调用线程锁定。如果 lockedMutex 未处于锁定状态,则行为未定义。如果 lockedMutex 是递归互斥体,则此函数将立即返回。lockedMutex 将被解锁,调用线程将阻塞,直到满足以下任一条件:

另一个线程调用 wakeOne() 或 wakeAll() 发出信号。在这种情况下,此函数将返回true。
截止日期已到。如果 deadline 是默认值 QDeadlineTimer::Forever,则永远不会超时(必须用信号通知事件)。如果等待超时,此函数将返回false。
lockedMutex 将返回到相同的锁定状态。提供此函数是为了允许原子从锁定状态转换到等待状态。   在 Qt 5.12 引入该函数。

\begin{seeAlso}
 wakeOne()、wakeAll()。
\end{seeAlso}

bool QWaitCondition::wait(QMutex \emph{*lockedMutex}, unsigned long \emph{time})
重载。

bool QWaitCondition::wait(QReadWriteLock **lockedReadWriteLock, QDeadlineTimer deadline = QDeadlineTimer(QDeadlineTimer::Forever))

释放 lockedReadWriteLock 并等待条件。lockedReadWriteLock 最初必须由调用线程锁定。如果 lockedReadWriteLock 未处于锁定状态,则此函数将立即返回。 lockedReadWriteLock 不能递归锁定,否则此函数将无法正确释放锁。lockedReadWriteLock 将被解锁,调用线程将阻塞,直到满足以下任一条件:

另一个线程调用 wakeOne() 或 wakeAll() 发出信号。在这种情况下,此函数将返回true。
截止日期已到。如果 deadline 是默认值 QDeadlineTimer::Forever,则永远不会超时(必须用信号通知事件)。如果等待超时,此函数将返回false。
lockedReadWriteLock 将返回到相同的锁定状态。提供此函数是为了允许原子从锁定状态转换到等待状态。
在 Qt 5.12 引入该函数。


\begin{seeAlso}
wakeOne()、wakeAll()。
\end{seeAlso}


bool QWaitCondition::wait(QReadWriteLock \emph{*lockedReadWriteLock}, unsigned long \emph{time})

重载。

void QWaitCondition::wakeAll()

唤醒等待条件的所有线程。线程的唤醒顺序取决于操作系统的调度策略,无法控制或预测。

\begin{seeAlso}
wakeOne()。
\end{seeAlso}

void QWaitCondition::wakeOne()

唤醒一个等待条件的线程。线程的唤醒顺序取决于操作系统的调度策略,无法控制或预测。
如果要唤醒特定线程,解决方案通常是使用不同的等待条件,并让线程在不同的条件下等待。

\begin{seeAlso}
wakeAll()。
\end{seeAlso}

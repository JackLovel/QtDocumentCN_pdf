\chapter{QSqlDatabase}

QSqlDatabase 类 用于处理数据库的连接

\begin{tabular}{|c|c|}
	\hline
	属性 & 方法 \\
	\hline
	头文件 & \#include <QSqlDatabase>\\      
	\hline
	qmake & QT += sql\\      
	\hline
\end{tabular}

\href{https://doc.qt.io/qt-5/qsqldatabase-members.html}{列出所有的成员,包括继承成员}


公共类型

\resizebox{\textwidth}{!}{ % Latex表格宽度超出文本宽度
\begin{tabular}{|r|l|}
\hline
返回值 & 函数名 \\
\hline
 & QSqlDatabase(const QSqlDatabase \&other) \\ 
\hline
 & QSqlDatabase()\\
\hline
QSqlDatabase \&	&operator=(const QSqlDatabase \&other)\\
\hline
 & ~QSqlDatabase()\\
\hline
void& close()\\
\hline
bool&commit()\\
\hline
QString	&connectOptions() const\\
\hline
QString	&connectionName() const\\
\hline
QString	&databaseName() const\\
\hline
QSqlDriver *&	driver() const\\
\hline
QString	&driverName() const\\
\hline
QSqlQuery&	exec(const QString \&query = QString()) const\\
\hline
QString	&hostName() const\\
\hline
bool	&isOpen() const\\
\hline
bool	&isOpenError() const\\
\hline
bool	&isValid() const\\
\hline
QSqlError&	lastError() const\\
\hline
QSql::NumericalPrecisionPolicy & numericalPrecisionPolicy() const\\
\hline
bool	&open()\\
\hline
bool&	open(const QString \&user, const QString \&password)\\
\hline
QString&	password() const\\
\hline
int	&port() const\\
\hline
QSqlIndex&	primaryIndex(const QString \&tablename) const\\
\hline
QSqlRecord&	record(const QString \&tablename) const\\
\hline
bool	&rollback()\\
\hline
void	&setConnectOptions(const QString \&options = QString())\\
\hline
void	&setDatabaseName(const QString \&name)\\
\hline
void	&setHostName(const QString \&host)\\
\hline
void	& setNumericalPrecisionPolicy(QSql::NumericalPrecisionPolicy    precisionPolicy)\\
\hline
void	&setPassword(const QString \&password)\\
\hline
void	&setPort(int port)\\
\hline
void	&setUserName(const QString \&name)\\
\hline
QStringList	&tables(QSql::TableType type = QSql::Tables) const\\
\hline
bool&	transaction()\\
\hline
QString	&userName() const\\
\hline
\end{tabular}
}

静态公共成员

\resizebox{\textwidth}{!}{ % Latex表格宽度超出文本宽度
\begin{tabular}{|r|l|}
	\hline
	返回值 & 函数名 \\
	\hline
	QSqlDatabase&	addDatabase(const QString \&type, const QString \&connectionName = QLatin1String(defaultConnection))\\
		\hline
	QSqlDatabase&	addDatabase(QSqlDriver *driver, const QString \&connectionName = QLatin1String(defaultConnection))\\
		\hline
	QSqlDatabase&	cloneDatabase(const QSqlDatabase \&other, const QString \&connectionName)\\
		\hline
	QSqlDatabase&	cloneDatabase(const QString \&other, const QString \&connectionName)\\
		\hline
	QStringList&	connectionNames()\\
		\hline
	bool&	contains(const QString \&connectionName = QLatin1String(defaultConnection))\\
		\hline
	QSqlDatabase&	database(const QString \&connectionName = QLatin1String(defaultConnection), bool open = true)\\
		\hline
	QStringList&	drivers()\\
		\hline
	bool&	isDriverAvailable(const QString \&name)\\
		\hline
	void&	registerSqlDriver(const QString \&name, QSqlDriverCreatorBase *creator)\\
		\hline
	void&	removeDatabase(const QString \&connectionName)\\
	\hline
\end{tabular}
}

受保护的成员函数

\begin{tabular}{|r|l|}
	\hline
	返回值 & 函数名 \\
	\hline
	&QSqlDatabase(QSqlDriver *driver)\\
	\hline
	&QSqlDatabase(const QString \&type)\\
	\hline
\end{tabular}


详细的介绍

QSqlDatabase 类提供接口用于数据库的连接 。一个 QSqlDatabase 实例对象表示连接。 这个连接提供 数据库 所需要的 驱动,这个驱动来自于 QSqlDriver。 换而言之,您可以实现自己的数据库驱动,通过继承 QSqlDriver。查看\href{https://doc.qt.io/qt-5/sql-driver.html#how-to-write-your-own-database-driver}{如何实现自己的数据库驱动}来获取更多的信息。

通过调用一个静态的 addDatabase()函数,来创建一个连接(即:实例化一个QSqlDatabase类),并且可以指定驱动或者驱动类型去使用(依赖于数据库的类型 )和 一个连接的名称。 一个连接是通过它自已的名称,而不是通过数据库的名称去连接的。对于一个数据库您可以有多个连接。QSqlDatabase 也支持默认连接,您可以不 传递连接名参数给 addDatabase() 来创建 它。随后,这个默认连接假定您 在调用任何静态函数情况下,而不去指定连接名称。 下面的一段代码片段展示了 如何去创建 和打开一个默认连接,去连接 PostgreSQL 数据库:

\begin{lstlisting}
QSqlDatabase db = QSqlDatabase::database();
\end{lstlisting}

QSqlDatabase是一个值类。通过一个 QSqlDatabase 实例对数据库连接所做的操作将影响表示相同连接的其他 QSqlDatabase 实例。 使用 cloneDatabase() 在基于已存在数据库的连接 来 创建 独立的数据库的连接。

警告:强烈建议不要将QSqlDatabase的拷贝作为类成员,因为这将阻止关闭时正确清理实例。 如果需要访问已经存在QSqlDatabase,应该使用database()访问。如果您选择使用作为成员变量的QSqlDatabase,则需要在删除QCoreApplication实例之前删除它,否则可能会导致未定义的行为。

如果您想创建多个数据库连接,可以调用 addDatabase(), 并且给一个独一无二的参数(即:连接名称)。使用 带有连接名的database() 函数,来获取该连接。使用 带有连接名的removeDatabase() 函数,来删除 一个连接。如果尝试删除由其他QSqlDatabase对象引用的连接,QSqlDatabase将输出警告。可以使用 \href{https://github.com/QtDocumentCN/QtDocumentCN/blob/master/Src/S/QSqlDatabase/QSqlDatabase.md#static-bool-qsqldatabasecontainsconst-qstring-connectionname--qlatin1stringdefaultconnection}{contains()}查看给定的连接名是否在连接列表中。


\begin{tabular}{|c|c|}
	\hline	
		& 一些实用的方法\\
	\hline
	\href{https://github.com/QtDocumentCN/QtDocumentCN/blob/master/Src/S/QSqlDatabase/QSqlDatabase.md#qstringlist-qsqldatabasetablesqsqltabletype-type--qsqltables-const}{tables()} &	返回 数据表的列表\\
		\hline
	\href{https://github.com/QtDocumentCN/QtDocumentCN/blob/master/Src/S/QSqlDatabase/QSqlDatabase.md#qsqlindex-qsqldatabaseprimaryindexconst-qstring-tablename-const}{primaryIndex()} &返回数据表的主索引\\
		\hline
	\href{URL}{record()} &	返回数据表字段的元信息\\
		\hline
	\href{URL}{transaction()} &开始一个事务\\
		\hline
	\href{URL}{commit()}&	保存并完成一个事务\\
		\hline
	\href{URL}{rollback()}&	取消一个事务\\
		\hline
	\href{URL}{hasFeature()}&	检查驱动程序是否支持事务\\
		\hline
	\href{URL}{lastError()}	&返回有关上一个错误的信息\\
		\hline
	\href{URL}{drivers()}	&返回可用的数据库驱动名称\\
		\hline
	\href{URL}{isDriverAvailable()}	&检查特定驱动程序是否可用\\
		\hline
	\href{URL}{registerSqlDriver()}	&注册自定义驱动程序\\
		\hline
\end{tabular}

注意: QSqlDatabase::exec() 方法已经被弃用。请使用 QSqlQuery::exec()

注意: 使用事务时,必须在创建查询之前启动事务。


成员函数文档

[protected] QSqlDatabase::QSqlDatabase(QSqlDriver *driver)

这是一个重载函数

使用给定驱动程序来创建连接

[protected] QSqlDatabase::QSqlDatabase(const \href{https://github.com/QtDocumentCN/QtDocumentCN/blob/master/Src/S/QString/QString.md}{QString} \&type)

这是一个重载函数

通过引用所给的数据库驱动类型来创建一个连接。如果不给定 数据库驱动类型 ,那么这个数据库连接将会没有什么作用。

当前可用的驱动类型:

\begin{tabular}{|r|l|}
	\hline	
	驱动类别& 介绍\\
	\hline
	QDB2&	IBM DB2\\
		\hline
	QIBASE	&Borland InterBase 驱动\\
		\hline
	QMYSQL	&MySQL 驱动\\
		\hline
	QOCI	&Oracle 调用的接口驱动\\
		\hline
	QODBC	&ODBC 驱动 (包含 Microsoft SQL Server)\\
		\hline
	QPSQL	&PostgreSQL 驱动\\
		\hline
	QSQLITE	&SQLite 第三版本 或者 以上\\
		\hline
	QSQLITE2&	SQLite 第二版本\\
		\hline
	QTDS	&Sybase Adaptive Server\\
	\hline
\end{tabular}

其他第三方驱动程序,包括自己自定义的驱动程序,都可以动态加载。

请参阅 \href{https://doc.qt.io/qt-5/sql-driver.html}{SQL Database Drivers}, \href{https://github.com/QtDocumentCN/QtDocumentCN/blob/master/Src/S/QSqlDatabase/QSqlDatabase.md#static-void-qsqldatabaseregistersqldriverconst-qstring-name-qsqldrivercreatorbase-creator}{registerSqlDriver()} 和 \href{https://github.com/QtDocumentCN/QtDocumentCN/blob/master/Src/S/QSqlDatabase/QSqlDatabase.md#static-qstringlist-qsqldatabasedrivers}{drivers()}。

QSqlDatabase::QSqlDatabase(const QSqlDatabase \&other)

创建一个其它的副本

QSqlDatabase::QSqlDatabase()
创建一个 无效的 QSqlDatabase 空对象。
使用\href{https://github.com/QtDocumentCN/QtDocumentCN/blob/master/Src/S/QSqlDatabase/QSqlDatabase.md#static-qsqldatabase-qsqldatabaseadddatabaseconst-qstring-type-const-qstring-connectionname--qlatin1stringdefaultconnection}{addDatabase()},  \href{https://github.com/QtDocumentCN/QtDocumentCN/blob/master/Src/S/QSqlDatabase/QSqlDatabase.md#static-void-qsqldatabaseremovedatabaseconst-qstring-connectionname}{removeDatabase()} 和\href{https://github.com/QtDocumentCN/QtDocumentCN/blob/master/Src/S/QSqlDatabase/QSqlDatabase.md#static-qsqldatabase-qsqldatabasedatabaseconst-qstring-connectionname--qlatin1stringdefaultconnection-bool-open--true}{database()} 来获得一个有效的 QSqlDatabase 对象。

QSqlDatabase \&QSqlDatabase::operator=(const QSqlDatabase \&other)

给这个对象赋一个其他其他对象的值

QSqlDatabase::~QSqlDatabase()

销毁这个对象,并且释放所有配置的资源 注意: 当最后的连接被销毁,这个折构函数就会暗中的调用 close()函数,去删除这个数据库的其他连接。

另请参阅 \href{https://github.com/QtDocumentCN/QtDocumentCN/blob/master/Src/S/QSqlDatabase/QSqlDatabase.md#void-qsqldatabaseclose}{close()}。




	

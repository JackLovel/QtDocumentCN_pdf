\chapter{QSqlDatabase}

QSqlDatabase 类 用于处理数据库的连接

\begin{tabular}{|c|c|}
	\hline
	属性 & 方法 \\
	\hline
	头文件 & \#include <QSqlDatabase>\\      
	\hline
	qmake & QT += sql\\      
	\hline
\end{tabular}

\href{https://doc.qt.io/qt-5/qsqldatabase-members.html}{列出所有的成员,包括继承成员}


公共类型

\resizebox{\textwidth}{!}{ % Latex表格宽度超出文本宽度
\begin{tabular}{|r|l|}
\hline
返回值 & 函数名 \\
\hline
 & QSqlDatabase(const QSqlDatabase \&other) \\ 
\hline
 & QSqlDatabase()\\
\hline
QSqlDatabase \&	&operator=(const QSqlDatabase \&other)\\
\hline
 & ~QSqlDatabase()\\
\hline
void& close()\\
\hline
bool&commit()\\
\hline
QString	&connectOptions() const\\
\hline
QString	&connectionName() const\\
\hline
QString	&databaseName() const\\
\hline
QSqlDriver *&	driver() const\\
\hline
QString	&driverName() const\\
\hline
QSqlQuery&	exec(const QString \&query = QString()) const\\
\hline
QString	&hostName() const\\
\hline
bool	&isOpen() const\\
\hline
bool	&isOpenError() const\\
\hline
bool	&isValid() const\\
\hline
QSqlError&	lastError() const\\
\hline
QSql::NumericalPrecisionPolicy & numericalPrecisionPolicy() const\\
\hline
bool	&open()\\
\hline
bool&	open(const QString \&user, const QString \&password)\\
\hline
QString&	password() const\\
\hline
int	&port() const\\
\hline
QSqlIndex&	primaryIndex(const QString \&tablename) const\\
\hline
QSqlRecord&	record(const QString \&tablename) const\\
\hline
bool	&rollback()\\
\hline
void	&setConnectOptions(const QString \&options = QString())\\
\hline
void	&setDatabaseName(const QString \&name)\\
\hline
void	&setHostName(const QString \&host)\\
\hline
void	& setNumericalPrecisionPolicy(QSql::NumericalPrecisionPolicy    precisionPolicy)\\
\hline
void	&setPassword(const QString \&password)\\
\hline
void	&setPort(int port)\\
\hline
void	&setUserName(const QString \&name)\\
\hline
QStringList	&tables(QSql::TableType type = QSql::Tables) const\\
\hline
bool&	transaction()\\
\hline
QString	&userName() const\\
\hline
\end{tabular}
}

静态公共成员

\resizebox{\textwidth}{!}{ % Latex表格宽度超出文本宽度
\begin{tabular}{|r|l|}
	\hline
	返回值 & 函数名 \\
	\hline
	QSqlDatabase&	addDatabase(const QString \&type, const QString \&connectionName = QLatin1String(defaultConnection))\\
		\hline
	QSqlDatabase&	addDatabase(QSqlDriver *driver, const QString \&connectionName = QLatin1String(defaultConnection))\\
		\hline
	QSqlDatabase&	cloneDatabase(const QSqlDatabase \&other, const QString \&connectionName)\\
		\hline
	QSqlDatabase&	cloneDatabase(const QString \&other, const QString \&connectionName)\\
		\hline
	QStringList&	connectionNames()\\
		\hline
	bool&	contains(const QString \&connectionName = QLatin1String(defaultConnection))\\
		\hline
	QSqlDatabase&	database(const QString \&connectionName = QLatin1String(defaultConnection), bool open = true)\\
		\hline
	QStringList&	drivers()\\
		\hline
	bool&	isDriverAvailable(const QString \&name)\\
		\hline
	void&	registerSqlDriver(const QString \&name, QSqlDriverCreatorBase *creator)\\
		\hline
	void&	removeDatabase(const QString \&connectionName)\\
	\hline
\end{tabular}
}

受保护的成员函数

\begin{tabular}{|r|l|}
	\hline
	返回值 & 函数名 \\
	\hline
	&QSqlDatabase(QSqlDriver *driver)\\
	\hline
	&QSqlDatabase(const QString \&type)\\
	\hline
\end{tabular}


详细的介绍

QSqlDatabase 类提供接口用于数据库的连接 。一个 QSqlDatabase 实例对象表示连接。 这个连接提供 数据库 所需要的 驱动,这个驱动来自于 QSqlDriver。 换而言之,您可以实现自己的数据库驱动,通过继承 QSqlDriver。查看\href{https://doc.qt.io/qt-5/sql-driver.html#how-to-write-your-own-database-driver}{如何实现自己的数据库驱动}来获取更多的信息。

通过调用一个静态的 addDatabase()函数,来创建一个连接(即:实例化一个QSqlDatabase类),并且可以指定驱动或者驱动类型去使用(依赖于数据库的类型 )和 一个连接的名称。 一个连接是通过它自已的名称,而不是通过数据库的名称去连接的。对于一个数据库您可以有多个连接。QSqlDatabase 也支持默认连接,您可以不 传递连接名参数给 addDatabase() 来创建 它。随后,这个默认连接假定您 在调用任何静态函数情况下,而不去指定连接名称。 下面的一段代码片段展示了 如何去创建 和打开一个默认连接,去连接 PostgreSQL 数据库:

\begin{lstlisting}
QSqlDatabase db = QSqlDatabase::database();
\end{lstlisting}

QSqlDatabase是一个值类。通过一个 QSqlDatabase 实例对数据库连接所做的操作将影响表示相同连接的其他 QSqlDatabase 实例。 使用 cloneDatabase() 在基于已存在数据库的连接 来 创建 独立的数据库的连接。

警告:强烈建议不要将QSqlDatabase的拷贝作为类成员,因为这将阻止关闭时正确清理实例。 如果需要访问已经存在QSqlDatabase,应该使用database()访问。如果您选择使用作为成员变量的QSqlDatabase,则需要在删除QCoreApplication实例之前删除它,否则可能会导致未定义的行为。

如果您想创建多个数据库连接,可以调用 addDatabase(), 并且给一个独一无二的参数(即:连接名称)。使用 带有连接名的database() 函数,来获取该连接。使用 带有连接名的removeDatabase() 函数,来删除 一个连接。如果尝试删除由其他QSqlDatabase对象引用的连接,QSqlDatabase将输出警告。可以使用 \href{https://github.com/QtDocumentCN/QtDocumentCN/blob/master/Src/S/QSqlDatabase/QSqlDatabase.md#static-bool-qsqldatabasecontainsconst-qstring-connectionname--qlatin1stringdefaultconnection}{contains()}查看给定的连接名是否在连接列表中。


\begin{tabular}{|c|c|}
	\hline	
		& 一些实用的方法\\
	\hline
	\href{https://github.com/QtDocumentCN/QtDocumentCN/blob/master/Src/S/QSqlDatabase/QSqlDatabase.md#qstringlist-qsqldatabasetablesqsqltabletype-type--qsqltables-const}{tables()} &	返回 数据表的列表\\
		\hline
	\href{https://github.com/QtDocumentCN/QtDocumentCN/blob/master/Src/S/QSqlDatabase/QSqlDatabase.md#qsqlindex-qsqldatabaseprimaryindexconst-qstring-tablename-const}{primaryIndex()} &返回数据表的主索引\\
		\hline
	\href{URL}{record()} &	返回数据表字段的元信息\\
		\hline
	\href{URL}{transaction()} &开始一个事务\\
		\hline
	\href{URL}{commit()}&	保存并完成一个事务\\
		\hline
	\href{URL}{rollback()}&	取消一个事务\\
		\hline
	\href{URL}{hasFeature()}&	检查驱动程序是否支持事务\\
		\hline
	\href{URL}{lastError()}	&返回有关上一个错误的信息\\
		\hline
	\href{URL}{drivers()}	&返回可用的数据库驱动名称\\
		\hline
	\href{URL}{isDriverAvailable()}	&检查特定驱动程序是否可用\\
		\hline
	\href{URL}{registerSqlDriver()}	&注册自定义驱动程序\\
		\hline
\end{tabular}

注意: QSqlDatabase::exec() 方法已经被弃用。请使用 QSqlQuery::exec()

注意: 使用事务时,必须在创建查询之前启动事务。


成员函数文档

[protected] QSqlDatabase::QSqlDatabase(QSqlDriver *driver)

这是一个重载函数

使用给定驱动程序来创建连接

[protected] QSqlDatabase::QSqlDatabase(const \href{https://github.com/QtDocumentCN/QtDocumentCN/blob/master/Src/S/QString/QString.md}{QString} \&type)

这是一个重载函数

通过引用所给的数据库驱动类型来创建一个连接。如果不给定 数据库驱动类型 ,那么这个数据库连接将会没有什么作用。

当前可用的驱动类型:

\begin{tabular}{|r|l|}
	\hline	
	驱动类别& 介绍\\
	\hline
	QDB2&	IBM DB2\\
		\hline
	QIBASE	&Borland InterBase 驱动\\
		\hline
	QMYSQL	&MySQL 驱动\\
		\hline
	QOCI	&Oracle 调用的接口驱动\\
		\hline
	QODBC	&ODBC 驱动 (包含 Microsoft SQL Server)\\
		\hline
	QPSQL	&PostgreSQL 驱动\\
		\hline
	QSQLITE	&SQLite 第三版本 或者 以上\\
		\hline
	QSQLITE2&	SQLite 第二版本\\
		\hline
	QTDS	&Sybase Adaptive Server\\
	\hline
\end{tabular}

其他第三方驱动程序,包括自己自定义的驱动程序,都可以动态加载。

请参阅 \href{https://doc.qt.io/qt-5/sql-driver.html}{SQL Database Drivers}, \href{https://github.com/QtDocumentCN/QtDocumentCN/blob/master/Src/S/QSqlDatabase/QSqlDatabase.md#static-void-qsqldatabaseregistersqldriverconst-qstring-name-qsqldrivercreatorbase-creator}{registerSqlDriver()} 和 \href{https://github.com/QtDocumentCN/QtDocumentCN/blob/master/Src/S/QSqlDatabase/QSqlDatabase.md#static-qstringlist-qsqldatabasedrivers}{drivers()}。

QSqlDatabase::QSqlDatabase(const QSqlDatabase \&other)

创建一个其它的副本

QSqlDatabase::QSqlDatabase()
创建一个 无效的 QSqlDatabase 空对象。
使用\href{https://github.com/QtDocumentCN/QtDocumentCN/blob/master/Src/S/QSqlDatabase/QSqlDatabase.md#static-qsqldatabase-qsqldatabaseadddatabaseconst-qstring-type-const-qstring-connectionname--qlatin1stringdefaultconnection}{addDatabase()},  \href{https://github.com/QtDocumentCN/QtDocumentCN/blob/master/Src/S/QSqlDatabase/QSqlDatabase.md#static-void-qsqldatabaseremovedatabaseconst-qstring-connectionname}{removeDatabase()} 和\href{https://github.com/QtDocumentCN/QtDocumentCN/blob/master/Src/S/QSqlDatabase/QSqlDatabase.md#static-qsqldatabase-qsqldatabasedatabaseconst-qstring-connectionname--qlatin1stringdefaultconnection-bool-open--true}{database()} 来获得一个有效的 QSqlDatabase 对象。

QSqlDatabase \&QSqlDatabase::operator=(const QSqlDatabase \&other)

给这个对象赋一个其他其他对象的值

QSqlDatabase::~QSqlDatabase()

销毁这个对象,并且释放所有配置的资源 注意: 当最后的连接被销毁,这个折构函数就会暗中的调用 close()函数,去删除这个数据库的其他连接。

另请参阅 \href{https://github.com/QtDocumentCN/QtDocumentCN/blob/master/Src/S/QSqlDatabase/QSqlDatabase.md#void-qsqldatabaseclose}{close()}。

[static] QSqlDatabase QSqlDatabase::addDatabase(const QString \&type,const QString \&connectionName = QLatin1String(defaultConnection))

使用驱动程序类型和连接名称,将数据库添加到数据库连接列表中。如果存在相同的连接名,那么这个连接将会被删除。

通过引用连接名,来返回一个新的连接。

如果数据库的类别不存在或者没有被加载,那么 isValid()函数将会返回 false

如果我们没有指定连接名参数,那么应用程序就会返回默认连接。 如果我们提供了连接名参数,那么可以使用database(connectionName) 函数来获取该连接。

警告: 如果您指定了 相同的连接名参数,那么就会替换之前的那个相同的连接。如果您多次调用这个函数而不指定 连接名参数,则默认连接将被替换。

在使用连接之前,它必须经过初始化。比如: 调用下面一些或者全部 \href{https://github.com/QtDocumentCN/QtDocumentCN/blob/master/Src/S/QSqlDatabase/QSqlDatabase.md#void-qsqldatabasesetdatabasenameconst-qstring-name}{setDatabaseName()} 、 \href{https://github.com/QtDocumentCN/QtDocumentCN/blob/master/Src/S/QSqlDatabase/QSqlDatabase.md#void-qsqldatabasesetusernameconst-qstring-name}{setUserName()}、 \href{https://github.com/QtDocumentCN/QtDocumentCN/blob/master/Src/S/QSqlDatabase/QSqlDatabase.md#void-qsqldatabasesetpasswordconst-qstring-password}{setPassword()} 、 \href{https://github.com/QtDocumentCN/QtDocumentCN/blob/master/Src/S/QSqlDatabase/QSqlDatabase.md#void-qsqldatabasesethostnameconst-qstring-host}{setHostName()}、 \href{https://github.com/QtDocumentCN/QtDocumentCN/blob/master/Src/S/QSqlDatabase/QSqlDatabase.md#void-qsqldatabasesetportint-port}{setPort()} 和 \href{https://github.com/QtDocumentCN/QtDocumentCN/blob/master/Src/S/QSqlDatabase/QSqlDatabase.md#void-qsqldatabasesetconnectoptionsconst-qstring-options--qstring}{setConnectOptions()},并最终调用 \href{https://github.com/QtDocumentCN/QtDocumentCN/blob/master/Src/S/QSqlDatabase/QSqlDatabase.md#bool-qsqldatabaseopen}{open()}

注意: 这个函数是线程安全的

请查看 \href{https://github.com/QtDocumentCN/QtDocumentCN/blob/master/Src/S/QSqlDatabase/QSqlDatabase.md#static-qsqldatabase-qsqldatabasedatabaseconst-qstring-connectionname--qlatin1stringdefaultconnection-bool-open--true}{database()}, \href{https://github.com/QtDocumentCN/QtDocumentCN/blob/master/Src/S/QSqlDatabase/QSqlDatabase.md#static-void-qsqldatabaseremovedatabaseconst-qstring-connectionname}{removeDatabase()} 以及 \href{https://doc.qt.io/qt-5/threads-modules.html#threads-and-the-sql-module}{ 线程和SQL 单元}。

[static] QSqlDatabase QSqlDatabase::addDatabase(QSqlDriver *driver, const QString \&connectionName = QLatin1String(defaultConnection))

这个重载函数是非常有用的,当您想创建一个带有\href{https://doc.qt.io/qt-5/qsqldriver.html}{驱动} 连接时,您可以实例化它。有可能您想拥有自己的数据库驱动,或者去实例化 Qt自带的驱动。如果您真的想这样做,我非常建议您把驱动的代码导入到您的应用程序中。例如,您可用自已的 QPSQL 驱动来创建一个 PostgreSQL 连接,像下面这样:

\begin{lstlisting}
PGconn *con = PQconnectdb("host=server user=bart password=simpson dbname=springfield");
QPSQLDriver *drv = new QPSQLDriver(con);
QSqlDatabase db = QSqlDatabase::addDatabase(drv); // 产生成新的默认连接
QSqlQuery query;
query.exec("SELECT NAME, ID FROM STAFF");	
\end{lstlisting}

上面的代码用于设置一个 PostgreSQL 连接和实例化一个 QPSQLDriver 对象。接下来,addDatabase() 被调用产生一个已知的连接,以便于它可以使用 Qt SQL 相关的类。Qt假定您已经打开了数据库连接,当使用连接句柄(或一组句柄)实例化驱动程序时。

注意: 我们假设qtdir是安装Qt的目录。假定您的PostgreSQL头文件己经包含在搜索路径中,然后这里才能引用所需要的PostgreSQL客户端库和去实例化QPSQLDriver对象。

请记住,必须将数据库客户端库到您的程序里。确保客户端库在您的链接器的搜索路径中,并且像下面这样添加到您的 .pro 文件里:

\begin{lstlisting}
unix:LIBS += -lpq
win32:LIBS += libpqdll.lib
\end{lstlisting}

这里介绍了所有驱动支持的方法。只有驱动的构造参数有所不同。列举了一个关于 Qt附带的程序,以及它们的源代码文件,和它们的构造函数参数的列表:

\resizebox{\textwidth}{!}{ % Latex表格宽度超出文本宽度
\begin{tabular}{|l|l|l|l|}
	\hline
	驱动	&类名	& 构造器参数	& 用于导入的文件\\
	\hline
QPSQL	&QPSQLDriver	&PGconn *connection	&qsql\_psql.cpp \\
\hline
QMYSQL	&QMYSQLDriver	&MYSQL *connection	&qsql\_mysql.cpp\\
\hline
QOCI	&QOCIDriver	&OCIEnv *environment, OCISvcCtx *serviceContext	&qsql\_oci.cpp\\
\hline
QODBC	&QODBCDriver	&SQLHANDLE environment, SQLHANDLE connection	&qsql\_odbc.cpp\\
\hline
QDB2	&QDB2	&SQLHANDLE environment, SQLHANDLE connection	&qsql\_db2.cpp\\
\hline
QTDS	&QTDSDriver	&LOGINREC *loginRecord, DBPROCESS *dbProcess, const QString \&hostName	&qsql\_tds.cpp\\
\hline
QSQLITE	&QSQLiteDriver	&sqlite *connection	&qsql\_sqlite.cpp\\
\hline
QIBASE	&QIBaseDriver	&isc\_db\_handle connection	&qsql\_ibase.cpp\\
 
	\hline
\end{tabular}}

当构造用于为内部查询创建新连接的QTDSDriver时,需要主机名(或服务名)。这是为了防止在同时使用多个QSqlQuery对象时发生阻塞。

警告: 添加一个存在连接名的连接时,这个新添加的连接将会替换另一个。 警告: SQL框架拥有驱动程序的所有权。它不能被删除。可以使用removeDatabase(),去删除这个连接。 另请参阅drivers()

[protected] QSqlDatabase QSqlDatabase::cloneDatabase(const QString \&other, const QString \&connectionName)

克隆其他数据库连接并将其存储为connectionName。原始数据库中的所有设置,例如databaseName()、hostName()等,都会被复制。如果其他数据库无效,则不执行任何操作。返回最新被创建的数据库连接。

注意: 这个新的连接不能被打开。您必须调用 open(),才能使用这个新的连接。

[static] QSqlDatabase QSqlDatabase::cloneDatabase(const QString \&other, const QString \&connectionName)
这是个重载函数。

克隆其他数据库连接并将其存储为connectionName。原始数据库中的所有设置,例如databaseName()、hostName()等,都会被复制。如果其他数据库无效,则不执行任何操作。返回最新被创建的数据库连接。

注意: 这个新的连接不能被打开。您必须调用 open(),才能使用这个新的连接。

当我们在另一个线程克隆这个数据库,这个重载是非常有用的。

qt5.13中引入了这个函数。

void QSqlDatabase::close()
关闭数据库连接,释放获取的所有资源,并使与数据库一起使用的任何现有QSqlQuery对象无效

这个函数也会影响它的QSqlDatabase对象副本。

另请参阅 removeDatabase()

bool QSqlDatabase::commit()
如果驱动支持事务和一个transaction()已经被启动,那就可以提交一个事务到这个数据库中。如果这个操作成功,就会返回 true。否则返回 false。

注意: 对于一些数据库,如果对数据库使用SELECT进行查询操作,将会提交失败并且返回false。在执行提交之前,使查询处于非活动状态。

调用 lastError() 函数获取错误信息。

另请参阅 QSqlQuery::isActive(), QSqlDriver::hasFeature(),和 rollback()。

QString QSqlDatabase::connectOptions() const
返回用于此连接的连接选项字符串。这个字符串可能是空。

另请参阅 setConnectOptions()

QString QSqlDatabase::connectionName() const
返回连接名,它有可能为空。

注意: 这个连接名不是 数据库名

qt4.4 中引入了这个函数。

另请参阅 addDatabase()

[static] QStringList QSqlDatabase::connectionNames()
返回包含所有连接名称的列表。

注意: 这个函数是线程安全的。

另请参阅 contains(),database(), 和 线程和SQL模块

[static] bool QSqlDatabase::contains(const QString \&connectionName = QLatin1String(defaultConnection))


如果所给的连接名,包含在所给的数据库连接列表里,那么就返回 true;否则返回 false。

注意: 这个函数是 线程安全的

另请参阅 connectionNames(), database() 和 线程和SQL模块。

[static] QSqlDatabase QSqlDatabase::database(const QString \&connectionName = QLatin1String(defaultConnection), bool open = true)


返回一个调用 connectionName 参数的数据库连接。这个数据库连接使用之前,必须已经通过 addDatabase() 函数进行添加。如果open为true(默认值),并且数据库连接尚未打开,则现在打开它。如果未指定连接名参数,则使用默认连接。如果连接名不存在数据库列表中,那么将会返回一个非法的连接。

注意: 这个函数是 线程安全的

另请参阅 isOpen() 和 线程和SQL模块。

QString QSqlDatabase::databaseName() const


返回连接的连接数据库名称,当然它也可能是空的。

注意: 这个数据库名不是连接名

另请参阅 setDatabaseName()。

QSqlDriver *QSqlDatabase::driver() const


返回被使用的数据库连接的所使用的数据库驱动。

另请参阅 addDatabase() 和 drivers()

QString QSqlDatabase::driverName() const


返回连接的驱动名称

另请参阅 addDatabase() 和 driver()

[static] QStringList QSqlDatabase::drivers()


返回一个可使用的数据库驱动列表 另请参阅 registerSqlDriver()

QSqlQuery QSqlDatabase::exec(const QString \&query = QString()) const


在这个数据库里执行 SQL 表达式和 返回一个 QSqhttps://doc.qt.io/qt-5/qsqlquery.html 对象。使用 lastError() 来获取错误的信息。

如果查询为空,则返回一个空的、无效的查询。并且 lastError()。

另请参阅 QSqlQuery 和 lastError()。

QString QSqlDatabase::hostName() const


返回连接的主机名;它有可能为空。

另请参阅 setHostName()

[static] bool QSqlDatabase::isDriverAvailable(const QString \&name)


如果调用一个叫 name 的驱动,是可以使用的,那么就返回 true;反之返回 false。

另请参阅 drivers()。

bool QSqlDatabase::isOpen() const


如果当前数据库连接是打开的,那么就返回 true,否则返回 false。

bool QSqlDatabase::isOpenError() const


如果打开数据库的连接有错误,那么就返回 true,否则返回 false。可以调用 lastError() 函数去获取相关的错误信息。

bool QSqlDatabase::isValid() const


如果 QSqlDatabase() 有一个有效的驱动,那么就返回 true。

例子:

\begin{lstlisting}
QSqlDatabase db;
qDebug() << db.isValid();    // 返回 false

db = QSqlDatabase::database("sales");
qDebug() << db.isValid();    // 如果 "sales" 连接存在,就返回 true

QSqlDatabase::removeDatabase("sales");
qDebug() << db.isValid();    // 返回 false
\end{lstlisting}

\href{https://doc.qt.io/qt-5/qsqlerror.html}{QSqlError} QSqlDatabase::lastError() const

返回这个数据库出现的最新错误信息。

使用 QSqlQuery::lastError() 函数来获取一个单个查询上的错误。

另请参阅 QSqlError and QSqlQuery::lastError()。

QSql::NumericalPrecisionPolicyQSqlDatabase::numericalPrecisionPolicy() const
返回数据库连接的当前默认精度策略。

qt4.6中引入了这个函数。

另请参阅 QSql::NumericalPrecisionPolicy, setNumericalPrecisionPolicy()、

QSqlQuery::numericalPrecisionPolicy() 和 QSqlQuery::setNumericalPrecisionPolicy()。

bool QSqlDatabase::open()


使用当前连接值打开数据库连接。如果操作成功就返回 true; 反之返回 false。可以调用 lastError()来获取错误的信息。

另请参阅 lastError()、 setDatabaseName()、setUserName()、

setPassword()、setHostName()、setPort()和 setConnectOptions()。

bool QSqlDatabase::open(const QString \&user, const QString \&password)


这是一个重载函数。

使用所给的 username 和 password 两个参数,打开数据连接,如果成功就返回 true; 反之返回 false。使用 [lastError()]QSqlDatabase.md\#qsqlerror-qsqldatabaselasterror-const) 来获取错误的信息。

这个函数不存放所给的 password 参数,相反的它会把 password 参数直接传给驱动用于打连接,然后销毁这个参数。

另请参阅 lastError()。

QString QSqlDatabase::password() const


返回连接的密码。如果没有使用setPassword() 函数进行密码的设置 并且 没有调用 open() 以及 没有使用密码,那么就返回空的字符串。

另请参阅 setPassword()

int QSqlDatabase::port() const


返回连接的端口号。如果端口号没有设置,那么这个值就是未定义的。

另请参阅 setPort()


QSqlIndex QSqlDatabase::primaryIndex(const QString \&tablename) const


返回所给表格名的索引。如果索引不存在,那么就返回一个空的 QSqlIndex

注意: 如果表在创建时没有被引用,一些驱动程序(如 QPSQL 驱动程序)可能要求您以小写的形式传递表格名。

查看关于 Qt SQL driver 文档的更多信息。

另请参阅 tables() 和 record()。

QSqlRecord QSqlDatabase::record(const QString \&tablename) const


返回一个QSqlRecord,其中填充了名为tablename的表(或视图)中所有字段的名称。字段在记录中出现的顺序未定义。如果没有这样的表格(或者视图)存在,将会返回一个空的记录。

注意: 如果表在创建时没有被引用,一些驱动程序(如 QPSQL 驱动程序)可能要求您以小写的形式传递表格名。

查看 Qt SQL driver 文档的更多信息。

[static] void QSqlDatabase::registerSqlDriver(const QString &name, QSqlDriverCreatorBase *creator)


这个函数在 SQL框架中注册一个名为 name 的新 SQL 驱动程序。这个是非常有用的,如果您有一个自定义的驱动,并且您并不想把它编译作为一个插件。

例如:


\begin{lstlisting}
QSqlDatabase::registerSqlDriver("MYDRIVER", new QSqlDriverCreator<QSqlDriver>);
QVERIFY(QSqlDatabase::drivers().contains("MYDRIVER"));
QSqlDatabase db = QSqlDatabase::addDatabase("MYDRIVER");
QVERIFY(db.isValid());
\end{lstlisting}

QSqlDatabase拥有 creator 指针的所有权,因此您不能自己删除它。

另请参阅 drivers()。

[static] void QSqlDatabase::removeDatabase(const QString \&connectionName)


从数据库列表中,删除一个叫 connectionName 数据库连接。

警告: 不应在数据库连接上打开查询的情况下调用此函数,否则将发生资源泄漏。

\clearpage

例子:

\begin{lstlisting}
// 错误
QSqlDatabase db = QSqlDatabase::database("sales");
QSqlQuery query("SELECT NAME, DOB FROM EMPLOYEES", db);
QSqlDatabase::removeDatabase("sales"); // 将会输出警告
// “db”现在是一个悬而未决的无效数据库连接,
// “查询”包含无效的结果集
\end{lstlisting}


正确的做法:

\begin{lstlisting}
{
    QSqlDatabase db = QSqlDatabase::database("sales");
    QSqlQuery query("SELECT NAME, DOB FROM EMPLOYEES", db);
}
// “db”和“query”都被销毁,因为它们超出了范围
QSqlDatabase::removeDatabase("sales"); // 正确的
\end{lstlisting}

如果要删除默认连接,这个连接可能是通过调用 addDatabase() 函数而创建的,但未指定连接名称,可以通过对database()返回的数据库调用connectionName() 来检索默认连接名称。注意,如果没有创建默认数据库,将返回一个无效的数据库。

注意: 这个函数是线程安全的

另请查阅 database(),connectionName(),和 线程和SQL模块。

bool QSqlDatabase::rollback()


在数据库里回滚一个事务,如果驱动支持一个事务以及一个 transaction() 已经被启动。如果操作成功返回 true。否则返回 false。

注意: 对于某些数据库,如果存在使用数据库进行选择的活动查询,则回滚将失败并返回false。确保在执行回滚操作之前,查询是 非活动 的状态。

调用 lastError() 操作获得错误的相关信息。

另请查阅 QSqlQuery::isActive(),QSqlDriver::hasFeature() 和 commit()。

void QSqlDatabase::setConnectOptions(const QString \&options = QString())


设置一组数据库的具体的可选项。它必须在打之这个连接之前执行这个操作,否则是无效的。另一个可能的原因是调用 QSqlDatabase::setConnectOptions() 去关闭这个连接,并且调用 open() 再次关闭这个连接。

选项字符串的格式是以分号分隔的选项名称,或选项=值对的列表。这个选项依赖于所使用的客户端:

\begin{tabular}{|c|c|c|}
	\hline
    ODBC & MySQL & PostgreSQL \\
	\hline
    SQL\_ATTR\_ACCESS\_MODE & CLIENT\_COMPRESS	& connect\_timeout \\
    \hline
    SQL\_ATTR\_LOGIN\_TIMEOUT & CLIENT\_FOUND\_ROWS &	options \\
    \hline
    SQL\_ATTR\_CONNECTION\_TIMEOUT	& CLIENT\_IGNORE\_SPACE &	tty \\
    \hline
    SQL\_ATTR\_CURRENT\_CATALOG &	CLIENT\_ODBC &	requiressl \\
\hline
    SQL\_ATTR\_METADATA\_ID &	CLIENT\_NO\_SCHEMA&	service \\
\hline
    SQL\_ATTR\_PACKET\_SIZE&	CLIENT\_INTERACTIVE&\\	
\hline
    SQL\_ATTR\_TRACEFILE&	UNIX\_SOCKET&	\\
\hline
    SQL\_ATTR\_TRACE&	MYSQL\_OPT\_RECONNECT& \\	
\hline
    SQL\_ATTR\_CONNECTION\_POOLING	&MYSQL\_OPT\_CONNECT\_TIMEOUT& \\	
\hline
    SQL\_ATTR\_ODBC\_VERSION&	MYSQL\_OPT\_READ\_TIMEOUT&\\	
\hline
                         &MYSQL\_OPT\_WRITE\_TIMEOUT&	\\
\hline
                         &SSL\_KEY&	 \\
\hline
                         &SSL\_CERT&\\	
\hline
                         &SSL\_CA&	\\
\hline
                         &SSL\_CAPATH&\\
\hline
                         &SSL\_CIPHER&\\
	\hline
\end{tabular}

\begin{tabular}{|c|c|c|}
	\hline
	DB2	&OCI	&TDS \\ 
	\hline
	SQL\_ATTR\_ACCESS\_MODE & OCI\_ATTR\_PREFETCH\_ROWS	& 无 \\
	\hline
	SQL\_ATTR\_LOGIN\_TIMEOUT & OCI\_ATTR\_PREFETCH\_MEMORY& \\	
	\hline
\end{tabular}

\begin{tabular}{|c|c|}
	\hline
	\href{https://doc.qt.io/qt-5/qtsql-attribution-sqlite.html#sqlite}{SQLite} & Interbase \\ 
	\hline
	QSQLITE\_BUSY\_TIMEOUT & ISC\_DPB\_LC\_CTYPE \\ 
	\hline
	QSQLITE\_OPEN\_READONLY	& ISC\_DPB\_SQL\_ROLE\_NAME \\ 
	\hline
	QSQLITE\_OPEN\_URI&	\\ 
	\hline
	QSQLITE\_ENABLE\_SHARED\_CACHE	& \\
	\hline
	QSQLITE\_ENABLE\_REGEXP &  \\
	\hline
\end{tabular}

\clearpage

例子:

\begin{lstlisting}
db.setConnectOptions("SSL_KEY=client-key.pem;SSL_CERT=client-cert.pem;SSL_CA=ca-cert.pem;CLIENT_IGNORE_SPACE=1"); // 使用安全套连字进行连接
if (!db.open()) {
	db.setConnectOptions(); // 清除连接的字符串
	// ...
}
// ...
// PostgreSQL 连接
db.setConnectOptions("requiressl=1"); // 确保 PostgreSQL 安全套接字连接
if (!db.open()) {
	db.setConnectOptions(); // 清除可选
	// ...
}
// ...
// ODBC 连接
db.setConnectOptions("SQL_ATTR_ACCESS_MODE=SQL_MODE_READ_ONLY;SQL_ATTR_TRACE=SQL_OPT_TRACE_ON"); // 设置 ODBC 连项
if (!db.open()) {
	db.setConnectOptions(); // 不要尝试去设置这个选项。
	// ...
}
}
\end{lstlisting}

查阅这个客户端库文档,获得更多关于不同可选项的更多信息。

另请查阅 connectOptions()。

void QSqlDatabase::setDatabaseName(const QString \&name)

通过所给的 name 参数来设置所连接的数据库名称。必须在打开连接之前设置数据库名称。 或者,可以调用close()函数关闭连接,设置数据库名称,然后再次调用open() 。

注意: 这个数据库名不是连接名。必须在创建连接对象时将连接名称传递给 addDatabase()。

对于 QSQLITE 驱动,如果数据库名指定的名字不存在,然后它将会创建这个文件,除非您设置了 QSQLITE\_OPEN\_READONLY

此外,可以把 name 参数设置为 :memory:, 可以创建一个临时数据库,该数据库仅在应用程序的生命周期内可用。

对于 QOCI (Oracle) 驱动,这个数据库名是 TNS Service Name。

对于 QODBC 驱动程序,名称可以是DSN,DSN文件名(在这种情况下,文件扩展名必须为.DSN)或者是一个连接字符串。

例如,Microsoft Access 可以使用下面的连接方式来直接打开 .mdb 文件,而不是在 ODBC 管理工具里创建一个 DSN 对象:

\clearpage

\begin{lstlisting}
// ...
QSqlDatabase db = QSqlDatabase::addDatabase("QODBC");
db.setDatabaseName("DRIVER={Microsoft Access Driver (*.mdb, *.accdb)};
FIL={MS Access};DBQ=myaccessfile.mdb");
if (db.open()) {
	// 成功!
}
// ...
\end{lstlisting}

这个没有默认的值

另请查阅 databaseName()、setUserName()、 setPassword()、 setHostName()、 

setPort()、setConnectOptions() 和 open()。


void QSqlDatabase::setHostName(const QString \&host)

通过 host 参数来设置连接的主机名。为了生效,必须在打开连接之前,设置主机名。或者,可以调用close()关闭连接,然后设置主机名,再次调用open()函数。

这个没有默认值。

另请查阅 hostName(), setUserName(), setPassword(),setDatabaseName(),setPort(), setConnectOptions() 和 open()。

void QSqlDatabase::setNumericalPrecisionPolicy(QSql::NumericalPrecisionPolicy precisionPolicy)


设置在此数据库连接上创建的查询使用的默认数值精度策略。

注意: 驱动程序不支持以低精度获取数值,将忽略精度策略。您可以使用 QSqlDriver::hasFeature()来查找一个驱动是否支持这个功能。

注意: 通过 precisionPolicy 来设置这个默认的精度策略,将不会响影任何当前的活动查询。

qt4.6中引入了这个函数。

另请查阅 QSql::NumericalPrecisionPolicy,
numericalPrecisionPolicy(),QSqlQuery::setNumericalPrecisionPolicy 和 QSqlQuery::numericalPrecisionPolicy.

void QSqlDatabase::setPassword(const QString \&password)
通过 password 参数来设置连接的密码。为了生效,必须在打开连接之前来设置密码。或者,您可以调用close()关闭连接,然后设置密码,再次调用open()函数。

这个没有默认值。

警告: 这个函数以明文的形式把密码存放到qt里。 将密码作为参数来避免这个行为,然后使用 open()进行调用。

另请查阅 password(),setUserName(),setDatabaseName(),setHostName(),

setPort(), setConnectOptions()和 open()。

void QSqlDatabase::setPort(int port)

通过port参数设置连接的端口号。为了生效,您必须在打开连接之前,进行端口号的设置。或者,您可以调用close()关闭连接,然后设置端口号,再次调用open()函数

这个没有默认的值。

另请查阅 port(), setUserName(), setPassword(),

setHostName(),setDatabaseName(), setConnectOptions() 和 open()。

void QSqlDatabase::setUserName(const QString \&name)

通过 name 参数来设置连接的用户名。为了生效,必须在打开连接之前设置用户名。或者,您可以调用 close()函数来关闭连接,设置用户,然后再次调用 open()

这个没有默认值。

另请查阅 userName(),setDatabaseName(),setPassword(),

setHostName(),setPort(),setConnectOptions() 和 open()。

QStringList QSqlDatabase::tables(QSql::TableType type = QSql::Tables) const
返回由 parameter type 参数 指定的数据库的表格、系统表和视图的列表。

另请查阅 primaryIndex() 和 record()。

bool QSqlDatabase::transaction()
如果驱动程序支持事务,则在数据库上开始事务。如果操作成功的话,返回 true, 否则返回 false。

另请查阅 QSqlDriver::hasFeature(), commit() 和 rollback()。

QString QSqlDatabase::userName() const
返回连接的用户名; 它也许为空。

另请查阅 setUserName()。


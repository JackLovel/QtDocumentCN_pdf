\chapter{QSqlDatabase}
QSqlDatabase 类 用于处理数据库的连接

\begin{tabular}{|c|c|p{1.5cm}|}
	\hline
	属性 & 方法 \\
	\hline
	头文件 & \#include <QSqlDatabase>\\      
	\hline
	qmake & QT += sql\\      
	\hline
\end{tabular}\\

\href{https://doc.qt.io/qt-5/qsqldatabase-members.html}{列出所有的成员,包括继承成员}

公共类型

\begin{tabular}{|c|c|}
\hline
返回值 & 函数名 \\
\hline
 & QSqlDatabase(const QSqlDatabase \&other) \\ 
\hline
 & QSqlDatabase()\\
\hline
QSqlDatabase \&	&operator=(const QSqlDatabase \&other)\\
\hline
 & ~QSqlDatabase()\\
\hline
void& close()\\
\hline
bool&commit()\\
\hline
QString	&connectOptions() const\\
\hline
QString	&connectionName() const\\
\hline
QString	&databaseName() const\\
\hline
QSqlDriver *&	driver() const\\
\hline
QString	&driverName() const\\
\hline
QSqlQuery&	exec(const QString \&query = QString()) const\\
\hline
QString	&hostName() const\\
\hline
bool	&isOpen() const\\
\hline
bool	&isOpenError() const\\
\hline
bool	&isValid() const\\
\hline
QSqlError&	lastError() const\\
\hline
QSql::NumericalPrecisionPolicy & numericalPrecisionPolicy() const\\
\hline
bool	&open()\\
\hline
bool&	open(const QString \&user, const QString \&password)\\
\hline
QString&	password() const\\
\hline
int	&port() const\\
\hline
QSqlIndex&	primaryIndex(const QString \&tablename) const\\
\hline
QSqlRecord&	record(const QString \&tablename) const\\
\hline
bool	&rollback()\\
\hline
void	&setConnectOptions(const QString \&options = QString())\\
\hline
void	&setDatabaseName(const QString \&name)\\
\hline
void	&setHostName(const QString \&host)\\
\hline
void	& setNumericalPrecisionPolicy(QSql::NumericalPrecisionPolicy    precisionPolicy)\\
\hline
void	&setPassword(const QString \&password)\\
\hline
void	&setPort(int port)\\
\hline
void	&setUserName(const QString \&name)\\
\hline
QStringList	&tables(QSql::TableType type = QSql::Tables) const\\
\hline
bool&	transaction()\\
\hline
QString	&userName() const\\
\hline
\end{tabular}\\

细节的介绍 \\

查看 \href{https://doc.qt.io/qt-5/qtsql-index.html}{Qt SQL}

类型 文档\\ 

enum QSql::Location


此枚举类型描述特殊的sql导航位置


\begin{tabular}{|c|c|c|}
	\hline
	常量	& 值 & 介绍 \\
	\hline
	QSql::BeforeFirstRow&-1&在第一个记录之前\\
	\hline
	QSql::AfterLastRow&-2&在最后一个记录之后\\
	\hline
\end{tabular}\\

另请参阅 \href{https://doc.qt.io/qt-5/qsqlquery.html#at}{QSqlQuery::at()}

enum QSql::NumericalPrecisionPolicy


数据库中的数值可以比它们对应的C++类型更精确。此枚举列出在应用程序中表示此类值的策略。


\begin{tabular}{|c|c|c|}
	\hline
	常量	& 值 & 介绍 \\
	\hline
	QSql::LowPrecisionInt32	&0x01 &对于32位的整形数值。在浮点数的情况下,小数部分将会被舍去。\\
	\hline
	QSql::LowPrecisionInt64	&0x02 &对于64位的整形数值。在浮点数的情况下,小数部分将会被舍去。\\
	\hline
	QSql::LowPrecisionDouble&0x04 &强制双精度值。这个默认的规则\\
	\hline
	QSql::HighPrecision	&0&字符串将会维技精度\\
	\hline
\end{tabular}\\

注意: 如果特定的驱动发生溢出,这是一个真实行为。像 Oracle数据库在这种情形下,就会返回一个错误。\\


enum QSql::ParamTypeFlag


flags QSql::ParamType


这个枚举用于指定绑定参数的类型

\begin{tabular}{|l|l|l|}
	\hline
	常量	& 值 & 介绍 \\
	\hline
	QSql::In&0x00000001&这个参数被用于向数据库里写入数据\\
	\hline
	QSql::Out&0x00000002&这个参数被用于向数据库里获得数据\\
	\hline
	QSql::InOut&In | Out&这个参数被用于向数据库里写入数据;使用 查询 来向数据库里,重写数据\\
	\hline
	QSql::Binary&0x00000004&如果您想 显示数据为 原始的二进制数据,那么必须是 OR'd 和其他的标志一 起使用\\
	\hline
\end{tabular}

类型参数 类型定义为 \href{https://doc.qt.io/qt-5/qflags.html}{QFlags}. 它被存放在 一个 OR与 类型参数标志的值 的组合。






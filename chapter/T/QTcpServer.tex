\chapter{QTcpServer}

QTcpServer 类提供了一个基于 TCP 协议的服务器。

\begin{tabular}{|r|l|}
	\hline
	属性 & 方法 \\
	\hline
	头文件 & \#include <QTcpServer>\\      
	\hline
	qmake & QT += network\\      
	\hline
	父类	  & QObject \\ 
	\hline
    子类	 & QSctpServer \\ 
	\hline
\end{tabular}


\begin{notice}
该类的所有成员函数都是可重入的。
\end{notice}

\section{公共成员函数}

\begin{longtable}{|r|l|}
	\hline
		& QTcpServer(QObject *parent = nullptr) \\
	\hline
	 virtual	& $\sim$QTcpServer() \\ 
	\hline
void	 & close() \\ 
\hline
QString	& errorString() const \\
\hline
virtual bool	& hasPendingConnections() const \\ 
\hline
bool	& isListening() const \\ 
\hline
bool	& listen(const QHostAddress \&address = QHostAddress::Any, quint16 port = 0) \\ 
\hline
int	& maxPendingConnections() const \\
\hline
virtual QTcpSocket *	& nextPendingConnection()\\
\hline
void	 & pauseAccepting() \\ 
\hline
QNetworkProxy	& proxy() const\\
\hline
void	 & resumeAccepting()\\
\hline
QHostAddress & 	serverAddress() const\\
\hline
QAbstractSocket::SocketError	 & serverError() const \\
\hline
quint16	 & serverPort() const\\
\hline
void	 & setMaxPendingConnections(int numConnections)\\
\hline
void	 & setProxy(const QNetworkProxy \&networkProxy)\\
\hline
bool	& setSocketDescriptor(qintptr socketDescriptor)\\
\hline
qintptr	& socketDescriptor() const\\
\hline
bool	& waitForNewConnection(int msec = 0, bool *timedOut = nullptr)\\
	\hline
\end{longtable}

\section{信号}

\begin{tabular}{|r|l|}
	\hline
	类型	 & 函数名 \\
	\hline
void	v & acceptError(QAbstractSocket::SocketError \emph{socketError}) \\
\hline
void	 & newConnection() \\
	\hline 
\end{tabular}

\section{保护成员函数}

\begin{tabular}{|r|l|}
	\hline
	类型	 & 函数名 \\
	\hline
void	 &  addPendingConnection(QTcpSocket \emph{*socket}) \\ 
\hline
virtual void	 & incomingConnection(qintptr \emph{socketDescriptor}) \\ 
	\hline 
\end{tabular}

\section{详细介绍}

这个类使“用Qt编写一个接受 TCP 链接的服务器”的构想成为可能。您可以为 QTcpServer 指定一个端口或者让它自动选一个。您也可以令 QTcpServer 仅监听一个指定的地址或者让它监听这台机器上的所有地址。

调用 listen() 函数来为到达的连接建立服务器监听。每次当有新的客户端连接到这个服务端时,QTcpServer 都会发出 newConnection() 信号。

调用 nextPendingConnection() 函数接受一个连接到 QTcpServer 的待处理连接。 该函数会返回一个处于已连接( QAbstractSocket::ConnectedState )状态的 QTcpSocket ,您可以使用这个 QTcpSocket 与客户端进行通信。

如果发生错误,serverError() 函数会返回该错误的类型。 您可以调用 errorString() 函数来获取一个能让人看懂的错误描述。

当服务器正在监听连接时,服务器所监听的地址和端口可以通过 serverAddress() 和 serverPort() 函数来获取。

调用 close() 函数后 QTcpServer 会停止监听到达的连接。

另外您也可以在 QTcpSocket 类文档和Qt官方给出的 Fortune Server Example ,Threaded Fortune Server Example ,Loopback Example 和 Torrent Example 示例文档中找到相关信息。

\section{成员函数文档}

QTcpServer::QTcpServer(QObject  \emph{*parent}= nullptr)

构造函数。构造一个 QTcpServer 对象。

\emph{parent} 参数将传递到 QObject 的构造函数中。

\begin{notice}[另请参阅]
listen() 和 setSocketDescriptor()。
\end{notice}


% gog 

[SIGNAL] void QTcpServer::acceptError(QAbstractSocket::SocketError \emph{socketError})

当接受一个新的连接时出错,QTcpServer 会发送此信号。 socketError 参数描述了该错误的类型。

该函数最早在Qt5.0版本引入。

\begin{notice}[另请参阅]
pauseAccepting() 和 resumeAccepting()。
\end{notice}

[SIGNAL] void QTcpServer::newConnection()

每当有新的连接可用时,QTcpServer 都会发送该消息。

\begin{notice}[另请参阅]
hasPendingConnections() 和 nextPendingConnection()。
\end{notice}

[virtual] QTcpServer::$\sim$QTcpServer()

析构函数。销毁 QTcpServer 对象。如果服务器正在监听连接,套接字会自动关闭。

在服务端删除之前,所有仍处于连接状态的客户端 QTcpSocket 都必须断开连接或者重新指定父类。

\begin{notice}[另请参阅]
 close() 。
\end{notice}


[protected] void QTcpServer::addPendingConnection(QTcpSocket  \emph{*socket})

该函数由 QTcpServer::incomingConnection() 函数调用,作用是添加 socket 到待处理连接列表中。

\begin{notice}
如果您不想破坏连接处理机制,请不要忘记从重新实现的 incomingConnection() 中调用此成员。
\end{notice}

该函数最初在Qt4.7版本引入。

\begin{notice}[另请参阅]
incomingConnection() 
\end{notice}

void QTcpServer::close()

关闭服务器。服务器将不再监听到达的连接。

\begin{notice}[另请参阅]
 listen() 
\end{notice}

QString QTcpServer::errorString() const

将最后一个出现的错误的相关信息按照适合人阅读的形式返回。


\begin{notice}[另请参阅]
serverError() 
\end{notice}


[virtual] bool QTcpServer::hasPendingConnections() const

当服务端有待处理的连接时该函数返回 true ,否则返回 false 。

\begin{notice}[另请参阅]
nextPendingConnection() 和 setMaxPendingConnections()。
\end{notice}








\chapter{QTimeLine}

QTimeLine类提供用于控制动画的时间线。

\begin{tabular}{|r|l|}
	\hline
	属性 & 方法 \\
	\hline
	头文件 & \#include <QTimeLine>\\      
	\hline
	qmake & QT += core\\      
	\hline
	Since: 	& Qt 4.2\\ 
	\hline
	父类: &	QObject \\ 
	\hline
\end{tabular}

\section{公共成员类型}

\begin{tabular}{|r|m{32em}|}
\hline
类型 &	方法 \\
\hline
enum 	& TransformationType \{ TxNone, TxTranslate, TxScale, TxRotate, TxShear, TxProject \}\\
\hline
\end{tabular}


\section{公共类型}

\begin{tabular}{|l|l|l|l|}
\hline
类型 	& 方法 \\ 
\hline
enum 	& Direction \{ Forward, Backward \} \\ 
\hline
enum  &	State \{ NotRunning, Paused, Running \} \\ 
\hline
\end{tabular}

\section{属性}

\begin{tabular}{|l|l|}
\hline
currentTime:int &	easingCurve:QEasingCurve \\ 
\hline
direction:Direction &	loopCount:int \\ 
\hline
duration:int 	& updateInterval:int \\ 
\hline
\end{tabular}

\section{公共成员函数}

\begin{longtable}{|l|m{25em}|}
\hline
返回类型 	&函数名\\
\hline
& QTimeLine(int duration = 1000, QObject *parent = nullptr) \\ 
\hline
virtual &	$\sim$QTimeLine() \\ 
\hline
int &	currentFrame() const \\ 
\hline 
int &	currentTime() const \\ 
\hline
qreal &	currentValue() const\\
\hline
QTimeLine::Direction &	direction() const \\ 
\hline
int 	& duration() const \\ 
\hline
QEasingCurve &	easingCurve() const \\ 
\hline
int &	endFrame() const \\ 
\hline
int &	frameForTime(int \emph{msec}) const \\ 
\hline
int &	loopCount() const \\ 
\hline
void &	setDirection(QTimeLine::Direction \emph{direction}) \\ 
\hline
void &	setDuration(int \emph{duration}) \\ 
\hline
void &	setEasingCurve(const QEasingCurve \emph{\&curve}) \\
\hline
void 	& setEndFrame(int \emph{frame}) \\ 
\hline
void &	setFrameRange(int \emph{startFrame}, int \emph{endFrame})  \\ 
\hline
void &	setLoopCount(int \emph{count}) \\ 
\hline
void &	setStartFrame(int \emph{frame}) \\
\hline
void &	setUpdateInterval(int \emph{interval}) \\ 
\hline
int &	startFrame() const \\ 
\hline
QTimeLine::State &	state() const \\ 
\hline
int  &	updateInterval() const \\ 
\hline
virtual qreal &	valueForTime(int \emph{msec}) const \\ 
\hline	
\end{longtable}

\section{公共槽函数}

\begin{tabular}{|r|l|}
	\hline
	返回类型 &	函数名 \\ 
	\hline
	void  &	resume() \\ 
	\hline
	void &	setCurrentTime(int msec) \\ 
	\hline
	void &	setPaused(bool paused) \\ 
	\hline
	void 	&start() \\ 
	\hline
	void &	stop() \\ 
	\hline
	void &	toggleDirection() \\ 
	\hline
\end{tabular}


% ss


\section{信号}

\begin{tabular}{|r|l|}
	\hline
	返回类型 &	函数名 \\ 
	\hline
	void  &	finished() \\ 
	\hline
	void 	&frameChanged(int \emph{frame}) \\ 
	\hline
	void 	&stateChanged(QTimeLine::State \emph{newState}) \\ 
	\hline 
	void 	&valueChanged(qreal \emph{value}) \\
	\hline
\end{tabular}

\section{重写保护成员函数}

\begin{tabular}{|r|l|}
	\hline
	返回类型 &	函数名 \\ 
	\hline
	virtual void &	timerEvent(QTimerEvent \emph{*event}) override\\ 
	\hline
\end{tabular}

\section{详细介绍}

它通常用于周期性地调用槽函数来使 GUI 控件设置动画效果。您可以通过以毫秒为单位为QTimeline的构造函数来构建时间线。
时间轴的持续时间描述了动画将运行多长时间。然后通过调用setframerange()设置合适的帧范围。
最后将FrameChanged()信号连接到您希望动画的窗口小部件中的合适插槽(例如,QProgressBar中的SetValue())。
当您继续调用 start ()时,QTimeLine 将进入 Running 状态,并开始定期发送 framecchanged ()信号 ,使您的小部件的连接属性的值从较低端增长到较高端,并以一个稳定的速率增长到帧范围的较高端。可以通过调用 setUpdateInterval ()来指定更新间隔。完成后,QTimeLine 进入 NotRunning 状态,并发出 finished ()。

例子:

\begin{lstlisting}[language=C++]
...
progressBar = new QProgressBar(this);
progressBar->setRange(0, 100);

// Construct a 1-second timeline with a frame range of 0 - 100
QTimeLine *timeLine = new QTimeLine(1000, this);
timeLine->setFrameRange(0, 100);
connect(timeLine, &QTimeLine::frameChanged, progressBar, &QProgressBar::setValue);

// Clicking the push button will start the progress bar animation
pushButton = new QPushButton(tr("Start animation"), this);
connect(pushButton, &QPushButton::clicked, timeLine, &QTimeLine::start);
...
\end{lstlisting}

默认情况下,时间线从开始到结束运行一次,您必须在此时间线上再次调用 start ()以重新启动。要使时间线循环,可以调用 setLoopCount () ,传递在完成之前时间线应该运行的次数。还可以通过调用 setDirection ()更改方向,从而使时间线向后运行。您还可以在时间轴运行时调用 setestated ()来暂停和取消暂停。对于交互控制,提供了 setCurrentTime ()函数,它直接设置时间线的时间位置。尽管在 NotRunning 状态下最有用(例如,连接到 QSlider 中的 valueChanged ()信号) ,但是这个函数可以在任何时候调用。

框架界面对于标准部件很有用,但 QTimeLine 可以用来控制任何类型的动画。QTimeLine 的核心在 valueForTime ()函数中,该函数在给定时间内生成介于0和1之间的值。此值通常用于描述动画的步骤,其中0是动画的第一步,1是最后一步。在运行时,QTimeLine 通过调用 valueForTime ()和发出 valueChanged ()生成0到1之间的值。默认情况下,valueForTime ()应用插值算法来生成这些值。可以通过调用 setEasingCurve ()从一组预定义的时间线算法中进行选择。

请注意,默认情况下,QTimeLine 使用 QEasingCurve: : InOutSine,它提供一个缓慢增长的值,然后稳定增长,最后缓慢增长。对于自定义时间线,可以重新实现 valueForTime () ,在这种情况下,忽略 QTimeLine 的 easingCurve 属性。

\begin{seeAlso}
 QProgressBar 和 QProgressDialog.
\end{seeAlso}

\section{成员类型文档}

enum QTimeLine::Direction

此枚举描述处于 Running状态时的时间线方向。

\begin{tabular}{|c|c|c|}
\hline
函数 &	值 & 	描述 \\
\hline
QTimeLine::Forward 	&0& 	“当前时间”随时间递增(即从0向终点/duration 移动)\\
\hline
QTimeLine::Backward &	1& 	”当前时间“随时间递减(即从终点/duration 向0移动)\\
\hline
\end{tabular}

\begin{seeAlso}
setDirection().
\end{seeAlso}

enum QTimeLine::State

此枚举描述时间线的状态。

%%%%%%%%%


\begin{tabular}{|c|c|c|}
\hline
函数 &	值 & 	描述 \\
\hline
QTimeLine::NotRunning &	0 &	时间线没有运行。这是 QTimeLine 的初始状态,状态 QTimeLine 在完成时重新进入。当前时间、帧和值保持不变,直到调用 setCurrentTime ()或调用 start ()启动时间线。\\
\hline
QTimeLine::Paused &	1 &	时间线暂停(即暂时暂停)。调用setPaused(false)将恢复时间线活动。 \\ 
\hline
QTimeLine::Running &	2 	&时间轴正在运行。虽然控件在事件循环中,QTimeline将以规则的间隔更新其当前时间,并在适当时发出ValueChanged()和FrameChanged()。\\
\hline
\end{tabular}

\begin{seeAlso}
state() and stateChanged().
\end{seeAlso}

\section{属性文档}

currentTime : int

此属性保存时间线的当前时间。

当 QTimeLine 处于 Running 状态时,此值将作为时间轴持续时间和方向的函数不断更新。否则,它是上次调用 stop ()时的当前值,或者是由 setCurrentTime ()设置的值。

默认情况下,此属性包含值0。

访问函数:


\begin{tabular}{|l|c|}
\hline
返回类型	& 函数名 \\ 
\hline
int &	currentTime() const \\ 
\hline
void& 	setCurrentTime(int \emph{msec}) \\ 
\hline
\end{tabular}

direction : Direction

该属性持有时间线的方向,当QTimeLine处于Running状态时。

这个方向表示时间是从0向时间线持续时间移动,还是从持续时间的值开始,在调用start()后向0移动。

默认情况下,这个属性被设置为Forward。

访问函数:




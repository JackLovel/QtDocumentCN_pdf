\chapter{QTransform}

QTransform为2D坐标系提供坐标转化

\begin{tabular}{|r|l|}
	\hline
	属性 & 方法 \\
	\hline
	头文件 & \#include <QTransform>\\      
	\hline
	qmake & QT+=gui\\      
	\hline
	自从 	&Qt 4.3\\ 
	\hline
\end{tabular}

\section{公共成员类型}

\begin{tabular}{|r|m{32em}|}
\hline
类型 &	方法 \\
\hline
enum 	& TransformationType \{ TxNone, TxTranslate, TxScale, TxRotate, TxShear, TxProject \}\\
\hline
\end{tabular}


\section{公共成员类型}

\begin{tabular}{|l|l|l|l|}
\hline
属性 	&类型 	&属性 	&类型 \\ 
\hline
currentLoop &	const int &	duration &	const int \\ 
\hline
currentTime &	int &	loopCount &	int \\ 
\hline
direction &	Direction &	state &	const  State \\ 
\hline
\end{tabular}

%%%%%%%%%%%%


\section{公共成员类型}

\begin{longtable}{|l|m{25em}|}
\hline
返回类型 	&函数名\\
\hline
	&QTransform(qreal m11, qreal m12, qreal m21, qreal m22, qreal dx, qreal dy)\\
	\hline
	&QTransform(qreal m11, qreal m12, qreal m13, qreal m21, qreal m22, qreal m23, qreal m31, qreal m32, qreal m33 = 1.0)\\
	\hline
	&QTransform()\\
	\hline
QTransform \& &	operator=(const QTransform \&matrix)\\
\hline
qreal &	m11() const\\
\hline
qreal &	m12() const\\
\hline
qreal &	m13() const\\
\hline
qreal &	m21() const\\
\hline
qreal &	m22() const\\
\hline
qreal &	m23() const\\
\hline
qreal &	m31() const\\
\hline
qreal &	m32() const\\
\hline
qreal &	m33() const\\
\hline
QTransform &	adjoint() const\\
\hline
qreal &	determinant() const\\
\hline
qreal &	dx() const\\
\hline
qreal &	dy() const\\
\hline
QTransform &	inverted(bool *invertible = nullptr) const\\
\hline
bool 	&isAffine() const\\
\hline
bool 	&isIdentity() const\\
\hline
bool &	isInvertible() const\\
\hline
bool &	isRotating() const\\
\hline
bool &	isScaling() const\\
\hline
bool 	&isTranslating() const\\
\hline
void &	map(qreal x, qreal y, qreal *tx, qreal *ty) const\\
\hline
QPoint &	map(const QPoint \&point) const\\
\hline
QPointF &	map(const QPointF \&p) const\\
\hline
QLine &	map(const QLine \&l) const\\
\hline
QLineF 	&map(const QLineF \&line) const\\
\hline
QPolygonF &	map(const QPolygonF \&polygon) const\\
\hline
QPolygon &	map(const QPolygon \&polygon) const\\
\hline
QRegion &	map(const QRegion \&region) const\\
\hline
QPainterPath 	&map(const QPainterPath \&path) const\\
\hline
void 	&map(int x, int y, int *tx, int *ty) const\\
\hline
QRectF &	mapRect(const QRectF \&rectangle) const\\
\hline
QRect &	mapRect(const QRect \&rectangle) const\\
\hline
QPolygon 	&mapToPolygon(const QRect \&rectangle) const\\
\hline
void &	reset()\\
\hline
QTransform \& 	&rotate(qreal angle, Qt::Axis axis = Qt::ZAxis)\\
\hline
QTransform \& 	&rotateRadians(qreal angle, Qt::Axis axis = Qt::ZAxis)\\
\hline
QTransform \& 	&scale(qreal sx, qreal sy)\\
\hline
void 	&setMatrix(qreal m11, qreal m12, qreal m13, qreal m21, qreal m22, qreal m23, qreal m31, qreal m32, qreal m33)\\
\hline
QTransform \& &	shear(qreal sh, qreal sv)\\
\hline
QTransform \& &	translate(qreal dx, qreal dy)\\
\hline
QTransform &	transposed() const\\
\hline
QTransform::TransformationType &	type() const\\
\hline
QVariant &	operator QVariant() const\\
\hline
bool 	&operator!=(const QTransform \&matrix) const\\
\hline
QTransform &	operator*(const QTransform \&matrix) const\\
\hline
QTransform \& &	operator*=(const QTransform \&matrix)\\
\hline
QTransform \& &	operator*=(qreal scalar) \\
\hline
QTransform \& &	operator+=(qreal scalar)\\
\hline
QTransform \& &	operator-=(qreal scalar)\\
\hline
QTransform \& &	operator/=(qreal scalar) \\
\hline
bool &	operator==(const QTransform \&matrix) const \\
\hline
\end{longtable}

%%%%%%%%%%%%


\section{静态公共成员函数}

\begin{tabular}{|l|m{32em}|}
\hline
返回类型 &	函数名 \\ 
\hline
QTransform &	fromScale(qreal sx, qreal sy) \\ 
\hline
QTransform &	fromTranslate(qreal dx, qreal dy) \\
\hline
bool& 	quadToQuad(const QPolygonF \&one, const QPolygonF \&two, QTransform \&trans) \\ 
\hline 
bool 	&quadToSquare(const QPolygonF \&quad, QTransform \&trans) \\ 
\hline
bool &	squareToQuad(const QPolygonF \&quad, QTransform \&trans) \\ 
\hline
\end{tabular}

\section{相关非成员函数}

\begin{tabular}{|c|l|}
\hline
返回类型 &	函数名 \\ 
\hline
bool &	qFuzzyCompare(const QTransform \&t1, const QTransform \&t2) \\ 
\hline
uint &	qHash(const QTransform \&key, uint seed = 0) \\ 
\hline
QPainterPath &	operator*(const QPainterPath \&path, const QTransform \&matrix) \\ 
\hline
QPoint &	operator*(const QPoint \&point, const QTransform \&matrix) \\ 
\hline
QPointF &	operator*(const QPointF  \&point, const QTransform \&matrix) \\ 
\hline
QLineF 	&operator*(const QLineF \&line, const QTransform \&matrix) \\ 
\hline
QLine &	operator*(const QLine \&line, const QTransform \&matrix) \\ 
\hline
QPolygon &	operator*(const QPolygon \&polygon, const QTransform \&matrix) \\ 
\hline
QPolygonF &	operator*(const QPolygonF \&polygon, const QTransform \&matrix) \\ 
\hline
QRegion &	operator*(const QRegion \&region, const QTransform \&matrix) \\ 
\hline
QDataStream \& &	operator<<(QDataStream \&stream, const QTransform \&matrix) \\ 
\hline
QDataStream \& &	operator>>(QDataStream \&stream, QTransform \&matrix)\\
\hline
\end{tabular}


%%%%%%%%%%%%%%%%%%%%

\section{详细介绍}

这里填一些详细介绍。

一个转化是指如何平移,缩放,剪切,旋转或投影坐标系,通常在渲染图形时使用。

QTransform与QMatrix的不同之处在于,它是真正的3x3矩阵,允许映射变换。QTransform的toAffine()方法允许将QTransform强制转换为QMatrix。如果在矩阵上指定了映射变换,则该变换将导致其数据丢失。

QTransform是Qt中推荐的坐标转换类。

QTransform 可以使用函数 setMatrix(), scale(), rotate(), translate() , shear() 来构建 ,或者,可以通过应用基本矩阵运算来构建它。
也可以在构造矩阵时对其进行定义,并可以使用reset()函数将其重置为恒等矩阵(默认值)

QTransform类支持基本图元的映射:可以使用map()函数将给定的点,线,多边形,区域或绘画路径映射到此矩阵定义的坐标系。如果是矩形,可以使用mapRect()函数转换其坐标。
也可以使用mapToPolygon()函数将矩形转换为多边形(映射到由此矩阵定义的坐标系)。

QTransform提供isIdentity()函数,来判断是否为单位矩阵(在矩阵的乘法中,有一种矩阵起着特殊的作用,如同数的乘法中的1,这种矩阵被称为单位矩阵。
它是个方阵,从左上角到右下角的对角线(称为主对角线)上的元素均为1。除此以外全都为0。via百度百科), isInvertible() 函数来判断矩阵是否可逆(i.e. AB = BA = I).inverted()函数提供一个翻转的拷贝矩阵(否则返回单位矩阵),adjoint() 函数来判断矩阵是否为共轭矩阵。另外,QTransform提供了determinant()函数,该函数返回矩阵的行列式。

最后,QTransform类支持矩阵乘法,加法和减法,并且可以比较该类的其它对象。

\section{渲染图形}

\section{成员变量文档}

enum QAbstractAnimation::DeletionPolicy


\begin{tabular}{|c|c|l|}
\hline
函数 	&值 &	描述 \\ 
\hline
QAbstractAnimation::KeepWhenStopped &	0 &	动画停止时不会被删除 \\
\hline 
QAbstractAnimation::DeleteWhenStopped &	1 & 	动画停止时会被自动删除 \\ 
\hline
\end{tabular}

enum QAbstractAnimation::Direction	

\begin{tabular}{|c|c|l|}
\hline
函数 	&值 &	描述 \\ 
\hline
QAbstractAnimation::Forward &	0 &	“当前时间”随时间递增(即从0向终点/duration 移动)\\
\hline
QAbstractAnimation::Backward &	1 &	”当前时间“随时间递减(即从终点/duration 向0移动)\\
\hline
\end{tabular}

\section{属性文档}	

currentLoop : const int


存取函数

\begin{tabular}{|c|l|}
\hline
返回类型 	&函数名 \\ 
\hline
int &	currentLoop() const \\ 
\hline
\end{tabular}


通知信号

\begin{tabular}{|c|l|}
\hline
返回类型 	&函数名 \\ 
\hline
void &	currentLoopChanged(int currentLoop) \\
\hline
\end{tabular}


currentTime : int

存取函数

\begin{tabular}{|c|l|}
\hline
返回类型 	&函数名 \\ 
\hline
int &	currentTime() const \\ 
\hline
void 	&setCurrentTime(int msecs) \\
\hline
\end{tabular}

\section{成员类型文档}

enum QAbstractSocket::BindFlag | flags QAbstractSocket::BindMode

这里填该类型详细信息。

\section{成员函数文档}

QAbstractAnimation::QAbstractAnimation(QObject \emph{*parent} = Q\_NULLPTR)

构造 QAbstractAnimation 基类,并将 parent 参数传递给 QObject 的构造函数。

\begin{seeAlso}
QVariantAnimation 和 QAnimationGroup
\end{seeAlso}

[signal] void QAbstractAnimation::currentLoopChanged(int \emph{currentLoop})

每当当前循环发生变化时,QAbstractAnimation 会发射该信号。currentLoop 为当前循环。

\begin{notice}
属性 currentLoop 的通知信号。
\end{notice}


\begin{seeAlso}
currentLoop() 和 loopCount()。
\end{seeAlso}

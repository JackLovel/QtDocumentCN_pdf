\chapter{QTimer}

QTimer提供了重复和信号槽的定时器。

\begin{tabular}{|r|l|}
	\hline
	属性 & 方法 \\
	\hline
	头文件 & \#include <QTimer>\\      
	\hline
	qmake & QT += core\\      
	\hline
	继承	  & QObject \\ 
	\hline
\end{tabular}

\section{属性}

\begin{tabular}{|r|l|}
	\hline
属性名	 & 类型 \\ 
\hline
active	& const bool\\
\hline
singleShot	& bool\\
\hline
interval &	int\\
\hline
timeType	& Qt::TimerType\\
\hline
remainingTime &	const int\\
	\hline
\end{tabular}


\section{公共成员函数}

\begin{longtable}{|r|l|}
\hline
类型 & 	函数名 \\
\hline
 & QTimer(QObject *parent = nullptr) \\
 \hline
virtual	& $\sim$QTimer() \\
\hline
QMetaObject::Connection	& callOnTimeout(Functor slot, Qt::ConnectionType connectionType = ...) \\
\hline
QMetaObject::Connection	& callOnTimeout(const QObject *context, Functor slot, Qt::ConnectionType connectionType = ...) \\ 
\hline
QMetaObject::Connection	& callOnTimeout(const QObject *receiver, PointerToMemberFunction slot, Qt::ConnectionType connectionType = ...) \\ 
\hline
int	& interval() const \\ 
\hline
std::chrono::milliseconds	 & intervalAsDuration() const \\ 
\hline
bool	& isActive() const \\
\hline
bool &	isSingleShot() const \\
\hline
int	& remainingTime() const \\
\hline
std::chrono::milliseconds	 & remainingTimeAsDuration() const \\
\hline
void	 & setInterval(int msec) \\ 
\hline
void & 	setInterval(std::chrono::milliseconds value) \\ 
\hline
void	 & setSingleShot(bool singleShot) \\ 
\hline
void	  & setTimerType(Qt::TimerType atype) \\
\hline
void	 & start(std::chrono::milliseconds msec) \\
\hline
int	& timerId() const \\
\hline
Qt::TimerType	 & timerType() const \\
\hline
\end{longtable}

\begin{compactitem}
\item 32个共有成员函数继承自QObject
\end{compactitem}

\section{公有槽函数}

\begin{tabular}{|r|l|}
	\hline
	类型	 & 函数名 \\
	\hline
void &	start(int msec) \\
\hline
void &	start() \\
\hline
void &	stop()  \\
	\hline 
\end{tabular}

\begin{compactitem}
\item 一个公有槽函数继承自QObject
\end{compactitem}

\section{信号}

\begin{tabular}{|r|l|}
	\hline
	类型	 & 函数名 \\
	\hline
void	 & timeout() \\
	\hline 
\end{tabular}

\begin{compactitem}
\item 2个信号继承自QObject
\end{compactitem}

\section{静态公有成员函数}

\begin{tabular}{|r|l|}
	\hline
	类型	 & 函数名 \\
	\hline
void	 & singleShot(int msec, const QObject *receiver, const char *member) \\
\hline
void &	singleShot(int msec, Qt::TimerType timerType, const QObject *receiver, const char *member) \\
\hline
void	 & singleShot(int msec, const QObject *receiver, PointerToMemberFunction method) \\
\hline
void	 & singleShot(int msec, Qt::TimerType timerType, const QObject *receiver, PointerToMemberFunction method) \\ 
\hline
void	 & singleShot(int msec, Functor functor) \\
\hline
void	 & singleShot(int msec, Qt::TimerType timerType, Functor functor) \\
\hline
void	 & singleShot(int msec, const QObject *context, Functor functor) \\
\hline
void	 & singleShot(int msec, Qt::TimerType timerType, const QObject *context, Functor functor) \\
\hline
void &	singleShot(std::chrono::milliseconds msec, const QObject *receiver, const char *member) \\
\hline
void	 & singleShot(std::chrono::milliseconds msec, Qt::TimerType timerType, const QObject *receiver, const char *member) \\
\hline
const QMetaObject	 & staticMetaObject \\
	\hline 
\end{tabular}


\begin{compactitem}
\item 10个静态公有成员函数继承自QObject
\end{compactitem}

\section{重新实现保护成员函数}


\begin{tabular}{|r|l|}
	\hline
	类型	 & 函数名 \\
	\hline
virtual void	& timerEvent(QTimerEvent *e) override \\ 
\hline
\end{tabular}


\begin{compactitem}
\item 9个保护成员函数继承自QObject
\end{compactitem}








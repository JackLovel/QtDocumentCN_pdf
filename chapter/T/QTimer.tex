\chapter{QTimer}

QTimer提供了重复和信号槽的定时器。

\begin{tabular}{|r|l|}
	\hline
	属性 & 方法 \\
	\hline
	头文件 & \#include <QTimer>\\      
	\hline
	qmake & QT += core\\      
	\hline
	继承	  & QObject \\ 
	\hline
\end{tabular}

\section{属性}

\begin{tabular}{|r|l|}
	\hline
属性名	 & 类型 \\ 
\hline
active	& const bool\\
\hline
singleShot	& bool\\
\hline
interval &	int\\
\hline
timeType	& Qt::TimerType\\
\hline
remainingTime &	const int\\
	\hline
\end{tabular}


\section{公共成员函数}

\begin{longtable}{|r|l|}
\hline
类型 & 	函数名 \\
\hline
 & QTimer(QObject *parent = nullptr) \\
 \hline
virtual	& $\sim$QTimer() \\
\hline
QMetaObject::Connection	& callOnTimeout(Functor slot, Qt::ConnectionType connectionType = ...) \\
\hline
QMetaObject::Connection	& callOnTimeout(const QObject *context, Functor slot, Qt::ConnectionType connectionType = ...) \\ 
\hline
QMetaObject::Connection	& callOnTimeout(const QObject *receiver, PointerToMemberFunction slot, Qt::ConnectionType connectionType = ...) \\ 
\hline
int	& interval() const \\ 
\hline
std::chrono::milliseconds	 & intervalAsDuration() const \\ 
\hline
bool	& isActive() const \\
\hline
bool &	isSingleShot() const \\
\hline
int	& remainingTime() const \\
\hline
std::chrono::milliseconds	 & remainingTimeAsDuration() const \\
\hline
void	 & setInterval(int msec) \\ 
\hline
void & 	setInterval(std::chrono::milliseconds value) \\ 
\hline
void	 & setSingleShot(bool singleShot) \\ 
\hline
void	  & setTimerType(Qt::TimerType atype) \\
\hline
void	 & start(std::chrono::milliseconds msec) \\
\hline
int	& timerId() const \\
\hline
Qt::TimerType	 & timerType() const \\
\hline
\end{longtable}

\begin{compactitem}
\item 32个共有成员函数继承自QObject
\end{compactitem}


\section{公有槽函数}

\begin{tabular}{|r|l|}
	\hline
	类型	 & 函数名 \\
	\hline
void &	start(int msec) \\
\hline
void &	start() \\
\hline
void &	stop()  \\
	\hline 
\end{tabular}

\begin{compactitem}
\item 一个公有槽函数继承自QObject
\end{compactitem}

\section{信号}

\begin{tabular}{|r|l|}
	\hline
	类型	 & 函数名 \\
	\hline
void	 & timeout() \\
	\hline 
\end{tabular}

\begin{compactitem}
\item 2个信号继承自QObject
\end{compactitem}

\section{静态公有成员函数}

\begin{tabular}{|r|l|}
	\hline
	类型	 & 函数名 \\
	\hline
void	 & singleShot(int msec, const QObject *receiver, const char *member) \\
\hline
void &	singleShot(int msec, Qt::TimerType timerType, const QObject *receiver, const char *member) \\
\hline
void	 & singleShot(int msec, const QObject *receiver, PointerToMemberFunction method) \\
\hline
void	 & singleShot(int msec, Qt::TimerType timerType, const QObject *receiver, PointerToMemberFunction method) \\ 
\hline
void	 & singleShot(int msec, Functor functor) \\
\hline
void	 & singleShot(int msec, Qt::TimerType timerType, Functor functor) \\
\hline
void	 & singleShot(int msec, const QObject *context, Functor functor) \\
\hline
void	 & singleShot(int msec, Qt::TimerType timerType, const QObject *context, Functor functor) \\
\hline
void &	singleShot(std::chrono::milliseconds msec, const QObject *receiver, const char *member) \\
\hline
void	 & singleShot(std::chrono::milliseconds msec, Qt::TimerType timerType, const QObject *receiver, const char *member) \\
\hline
const QMetaObject	 & staticMetaObject \\
	\hline 
\end{tabular}


\begin{compactitem}
\item 10个静态公有成员函数继承自QObject
\end{compactitem}

\section{重新实现保护成员函数}


\begin{tabular}{|r|l|}
	\hline
	类型	 & 函数名 \\
	\hline
virtual void	& timerEvent(QTimerEvent *e) override \\ 
\hline
\end{tabular}


\begin{compactitem}
\item 9个保护成员函数继承自QObject
\end{compactitem}

\section{详细描述}

QTimer 类提供重复和单次定时器。

The QTimer 类为定时器提供了高级编程接口。 要使用它,请创建一个QTimer, 将 timeout() 信号连接到相应的插槽, 然后调用 start(). 从那时起,它将以固定时间间隔发出 timeout() 信号。

一秒 (1000 毫秒) 定时器的示例 (来自 Analog Clock 例子):

\begin{lstlisting}[language=C++]
QTimer *timer = new QTimer(this);
connect(timer, SIGNAL(timeout()), this, SLOT(update()));
timer->start(1000);
\end{lstlisting}


从那时起,update()槽函数每秒都被调用。

你可以调用setSingleShot(true)仅触发一次定时器。你也可以使用静态QTimer::singleShot() 函数在特定的时间内调用一个槽函数

\begin{lstlisting}[language=C++]
QTimer::singleShot(200, this, SLOT(updateCaption()));
\end{lstlisting}

在多线程应用,你可以使用QTimer在任何一个有事件循环的线程。使用QThread::exec(),开启一个非GUI线程的事件循环。Qt使用线程亲和性定义哪一个线程会发出timeout()信号。因此,你必须在指定线程内开启和关闭定时器, 它不可能开启一个定时器从另一个线程。

在特殊案例里,窗口系统的事件队列中的所有事件都已处理后,超时为0的定时器将立即超时。 在提供快速的用户界面时,这可以用来完成繁重的工作:

\begin{lstlisting}[language=C++]
 QTimer *timer = new QTimer(this);
 connect(timer, SIGNAL(timeout()), this, SLOT(processOneThing()));
 timer->start();
\end{lstlisting}

从那以后,processOneThing()将会被重复调用。应该以始终快速返回(通常在处理一个数据项之后)的方式编写。 因此,Qt一处理完定时器上所有工作后,就能够分发事件提供给用户接口和停止定时器。对于GUI应用来说,这是典型的方法实现繁重工作。但如今越来越多应用了多线程,我们期待0毫秒的定时器对象会逐渐被QThread取代。

\section{精度和时间分辨率}

定时器的精度依赖底层操作系统和硬件。大多数定时器支持1毫秒的分辨率,尽管在许多实际情况下定时器的精度将不等于该分辨率。

定时器的精度依靠timer type。

对于Qt::PreciseTimer, QTimer会尽量保持一毫秒的精度。精确的定时器也永远不会比预期的超时。对于Qt::CoarseTimer 和 Qt::VeryCoarseTimer 类型, QTimer 可能会早于我们预期中唤醒, 在这些类型的范围内: 5% 的间隔 Qt::CoarseTimer 和 500 毫秒 Qt::VeryCoarseTimer.

如果操作系统繁忙或是不能提供所需精度,所有的时间类型都可能会晚于我们期待的。在这种超时溢出的情况下,即使多个超时已过期,Qt也会仅发出一次,然后将恢复原始间隔。

\section{精度和定时器分辨率}

另一周使用 QTimer 的方法是调用 QObject::startTimer() 为你的对象 和更新实现QObject::timerEvent() 事件 在类内处理 (必须继承 QObject). timerEvent() 的缺点是 不支持单次定时器或信号的高级功能。另一种选择是QBasicTimer。通常,与直接使用 QObject::startTimer() 相比,它不那么麻烦。 有关这三种方法的概述,请参见 Timers 。

有些操作系统可能会限制定时器的使用数量,Qt尽量工作在这些范围之内。

\begin{seeAlso}
BasicTimer, QTimerEvent, QObject::timerEvent(), Timers, Analog Clock Example, 和 Wiggly Example。
\end{seeAlso}

\section{属性文档}

\begin{tabular}{|r|l|}
	\hline
	属性名&	类型 \\ 
	\hline
active&	const bool \\ 
	\hline 
\end{tabular}

如果定时器正在执行,这个属性是true, 否则是false。

访问函数

\begin{tabular}{|r|l|}
	\hline
类型 &	函数名 \\ 
\hline
bool	& isActive() const \\ 
	\hline 
\end{tabular}

\begin{tabular}{|r|l|}
	\hline
属性名	 & 类型 \\
\hline
interval & 	int \\
	\hline 
\end{tabular}

此属性保存超时间隔(以毫秒为单位)

一旦处理完窗口系统事件队列中的所有事件,超时间隔为0的QTimer 就会超时。

此属性的默认值为0。 设置活动定时器的间隔会改变 timerId()。

访问函数:

\begin{tabular}{|r|l|}
	\hline
类型	 & 函数名 \\
\hline
int & 	interval() const \\
\hline
void	 & setInterval(int \emph{msec}) \\
\hline
void &	setInterval(std::chrono::milliseconds \emph{value}) \\
	\hline 
\end{tabular}


\begin{seeAlso}
singleShot。
\end{seeAlso}

\begin{tabular}{|l|r|}
\hline
属性名	 & 类型 \\ 
\hline
remainingTime	& const int \\ 
\hline
\end{tabular}

此属性保留剩余时间(以毫秒为单位)

返回定时器的剩余值(以毫秒为单位),直到超时为止。 如果定时器处于非活动状态,则返回值为-1。 如果定时器过期,则返回值为0。

此属性在Qt 5.0中引入。

访问函数

\begin{tabular}{|l|r|}
\hline
类型 & 函数名 \\ 
\hline
int	& remainingTime() const \\
\hline
\end{tabular}


\begin{seeAlso}
\href{https://github.com/JackLovel/QtDocumentCN/blob/master/Src/T/QTimer/qtimer.html#interval-prop}{interval}。
\end{seeAlso}


\begin{tabular}{|l|r|}
\hline
属性名 & 类型 \\ 
\hline
singleShot &	bool \\
\hline
\end{tabular}


此属性保持定时器是否为单次定时器。

单触发定时器仅触发一次,非单触发定时器每interval毫秒触发一次。

此属性的默认值是false。

访问函数:

\begin{tabular}{|r|l|}
\hline
类型 & 	函数名 \\ 
\hline
bool &	isSingleShot() const \\ 
\hline
void	 & setSingleShot(bool singleShot) \\
\hline
\end{tabular}


\begin{seeAlso}
interval 和 singleShot()。
\end{seeAlso}

\begin{tabular}{|r|l|}
\hline
类型	 & 属性名 \\ 
\hline
timerType	 & Qt::TimerType \\ 
\hline
\end{tabular}

控制定时器的精度。

默认的属性是 Qt::CoarseTimer。

访问函数:

\begin{tabular}{|r|l|}
\hline
类型	 & 函数名 \\ 
\hline
Qt::TimerType & 	timerType() const \\ 
\hline
void	 & setTimerType(Qt::TimerType \emph{atype}) \\ 
\hline
\end{tabular}

\begin{seeAlso}
Qt::TimerType。
\end{seeAlso}

\section{成员函数文档}

QTimer::QTimer(QObject \emph{*parent} = nullptr)

使用给定的 \emph{parent} 构造一个定时器。

[virtual] QTimer::$\sim$QTimer()

销毁定时器。

QMetaObject::Connection QTimer::callOnTimeout(Functor \emph{slot}, Qt::ConnectionType connectionType = ...)

这是一个重载函数。

创建一个从timeout() 信号到slot的连接,并返回一个连接句柄。

为了方便提供了这个方法。它等价于调用:

\begin{lstlisting}
It's equivalent to calling QObject::connect(timer, &QTimer::timeout, timer, slot, connectionType).
\end{lstlisting}

在Qt5.12引入该函数。

\begin{seeAlso}
QObject::connect() 和 timeout().
\end{seeAlso}

QMetaObject::Connection QTimer::callOnTimeout(const QObject \emph{*context}, Functor \emph{slot}, Qt::ConnectionType \emph{connectionType} = ...)

此函数重载 callOnTimeout().

创建一个从 timeout() 信号到slot的连接,将其放置在context的特定事件循环中,并返回该连接的句柄。

为了方便提供了这个方法。它等价于调用:


\begin{lstlisting}
QObject::connect(timer, &QTimer::timeout, context, slot, connectionType).
\end{lstlisting}
	
在Qt5.12引入该函数。
	
\begin{seeAlso}
QObject::connect() 和 timeout().
\end{seeAlso}

QMetaObject::Connection QTimer::callOnTimeout(const QObject \emph{*receiver}, PointerToMemberFunction \emph{slot}, Qt::ConnectionType \emph{connectionType} = ...)

此函数重载 callOnTimeout().

创建一个从 timeout() 信号到slot的连接,将其放置在context的特定事件循环中,并返回该连接的句柄。

为了方便提供了这个方法。它等价于调用:


\begin{lstlisting}
QObject::connect(timer, &QTimer::timeout, receiver, slot, connectionType).
\end{lstlisting}
	
在Qt5.12引入该函数。
	
\begin{seeAlso}
QObject::connect() 和 timeout().
\end{seeAlso}

% 
std::chrono::milliseconds QTimer::intervalAsDuration() const

以std::chrono::milliseconds 对象的形式返回此定时器的间隔。

在Qt5.8引入该函数。

\begin{seeAlso}
interval。
\end{seeAlso}

bool QTimer::isActive() const

如果定时器正在运行(正在等待),则返回true;否则,返回false。 否则返回false。

\begin{notice}
获取函数来自属性active。
\end{notice}

std::chrono::milliseconds QTimer::remainingTimeAsDuration() const

以 std::chrono::milliseconds 对象的形式返回此定时器对象中剩余的时间。 
如果此定时器到期或过期,则返回的值为std::chrono::milliseconds::zero()。 
如果找不到剩余时间或定时器未激活,则此函数返回负值持续时间。

在Qt5.8引入该函数。

[static] void QTimer::singleShot(int msec, const QObject \emph{*receiver}, const char \emph{*member})

在给定的时间间隔后,此静态函数调用slot。

非常方便去使用这个函数。因为你不需要关心timerEvent或创造一个局部的QTimer对象。

\begin{lstlisting}[language=C++]
Example:

#include <QApplication>
#include <QTimer>

int main(int argc, char *argv[])
{
	QApplication app(argc, argv);
	QTimer::singleShot(600000, &app, SLOT(quit()));
	...
	return app.exec();
}
\end{lstlisting}

这个示例程序自动在10分钟(600,000毫秒)触发。

receiver 是接收对象,而 member 是插槽。 时间间隔是 msec 毫秒。

\begin{notice}
该函数是可重入。
\end{notice}

[static] void QTimer::singleShot(int \emph{msec}, Qt::TimerType \emph{timerType}, const QObject \emph{*receiver}, const char \emph{*member})

这是一个重载函数。

在给定的时间间隔后,此静态函数调用slot。

非常方便去使用这个函数。因为你不需要关心timerEvent或创造一个局部的QTimer对象。

receiver 是接收对象,而 member 是插槽。 时间间隔是 msec 毫秒。 timerType 影响了定时器的精度。

\begin{notice}
该函数是可重入。
\end{notice}

\begin{seeAlso}
start()。
\end{seeAlso}


[static] void QTimer::singleShot(int \emph{msec}, const QObject \emph{*receiver}, PointerToMemberFunction \emph{method})

这是一个重载函数。

在给定的时间间隔后,此静态函数调用slot。

非常方便去使用这个函数。因为你不需要关心timerEvent或创造一个局部的QTimer对象。

receiver 是接收对象,而 member 是插槽。 时间间隔是 msec 毫秒。

如果在间隔发生之前销毁了receiver,则不会调用该方法。 该函数将在receiver的线程中运行。 接收者的线程必须具有正在运行的Qt事件循环。


\begin{notice}
该函数是可重入。
\end{notice}

在Qt5.4引入该函数。

\begin{seeAlso}
start()。
\end{seeAlso}

[static] void QTimer::singleShot(int \emph{msec}, Qt::TimerType \emph{timerType}, const QObject \emph{*receiver}, PointerToMemberFunction \emph{method})

这是一个重载函数。

在给定的时间间隔后,此静态函数调用slot。

非常方便去使用这个函数。因为你不需要关心timerEvent或创造一个局部的QTimer对象。

receiver 是接收对象,而 member 是插槽。 时间间隔是 msec 毫秒。 timerType 影响了定时器的精度。

如果在间隔发生之前销毁了receiver,则不会调用该方法。 该函数将在receiver的线程中运行。 接收者的线程必须具有正在运行的Qt事件循环。

\begin{notice}
该函数是可重入。
\end{notice}

在Qt5.4引入该函数。

\begin{seeAlso}
start()。
\end{seeAlso}

[static] void QTimer::singleShot(int \emph{msec}, Functor \emph{functor})

这是一个重载函数。

在给定的时间间隔后,此静态函数将调用functor。

非常方便去使用这个函数。因为你不需要关心timerEvent或创造一个局部的QTimer对象。

时间间隔是msec 毫秒。

\begin{notice}
该函数是可重入。
\end{notice}

在Qt5.4引入该函数。

\begin{seeAlso}
start()。
\end{seeAlso}

[static] void QTimer::singleShot(int \emph{msec}, Qt::TimerType \emph{timerType}, Functor \emph{functor})

这是一个重载函数。

在给定的时间间隔后,此静态函数将调用 functor。

非常方便去使用这个函数。因为你不需要关心timerEvent或创造一个局部的 QTimer 对象。

时间间隔是msec 毫秒。 timerType 影响了定时器的精度。

\begin{notice}
该函数是可重入。
\end{notice}
	

在Qt5.4引入该函数。

\begin{seeAlso}
start()。
\end{seeAlso}

[static] void QTimer::singleShot(int \emph{msec}, const QObject \emph{*context}, Functor \emph{functor})

这是一个重载函数。

在给定的时间间隔后,此静态函数将调用functor。

非常方便去使用这个函数。因为你不需要关心timerEvent或创造一个局部的QTimer对象。

时间间隔是 msec 毫秒。

如果 context 在间隔发生之前被破坏,则不会调用该方法。 该函数将在 context 线程中运行。 上下文的线程必须具有正在运行的Qt事件循环。

\begin{notice}
该函数是可重入。
\end{notice}

在Qt5.4引入该函数。

\begin{seeAlso}
start()。
\end{seeAlso}

[static] void QTimer::singleShot(int \emph{msec}, Qt::TimerType \emph{timerType}, const QObject \emph{*context}, Functor \emph{functor})

这是一个重载函数。

在给定的时间间隔后,此静态函数将调用functor。

非常方便去使用这个函数。因为你不需要关心timerEvent或创造一个局部的QTimer对象。

时间间隔是msec 毫秒。 timerType 影响了定时器的精度。

如果context在间隔发生之前被破坏,则不会调用该方法。 该函数将在context线程中运行。 上下文的线程必须具有正在运行的Qt事件循环。

\begin{notice}
该函数是可重入。
\end{notice}

在Qt5.4引入该函数。

\begin{seeAlso}
start()。
\end{seeAlso}

[static] void QTimer::singleShot(std::chrono::milliseconds \emph{msec}, const QObject \emph{*receiver}, const char \emph{*member})

这是一个重载函数。

时间间隔是 msec 毫秒。 timerType 影响了定时器的精度。非常方便去使用这个函数。因为你不需要关心timerEvent或创造一个局部的QTimer对象。

The receiver是接收对象,而member 是插槽。时间间隔是 msec。

\begin{notice}
该函数是可重入。
\end{notice}

在Qt5.8引入该函数。

\begin{seeAlso}
start()。
\end{seeAlso}

[static] void QTimer::singleShot(std::chrono::milliseconds \emph{msec}, Qt::TimerType \emph{timerType}, const QObject \emph{*receiver}, const char \emph{*member})

这是一个重载函数。

在给定的时间间隔后,此静态函数将调用 \emph{slot}。

非常方便去使用这个函数。因为你不需要关心timerEvent或创造一个局部的QTimer对象。

\emph{receiver}是接收对象,\emph{member}是插槽。 \emph{msec}在持续时间对象毫秒中给出。 \emph{timerType}影响定时器的准确性。

\begin{notice}
该函数是可重入。
\end{notice}

在Qt5.8引入该函数。

\begin{seeAlso}
start()。
\end{seeAlso}

[slot] void QTimer::start(int \emph{msec})

以 \hl{msec} 毫秒的超时间隔启动或重新启动定时器。

如果定时器已经在运行,它将被stopped并重新启动。如果singleShot 是 true, 则定时器将仅激活一次。

[slot] void QTimer::start()

重载start()函数。

以interval.中指定的超时时间启动或重新启动定时器。

如果定时器已经运行,它将stopped 和重新启动。

如果 singleShot 是 true, 则定时器将仅激活一次。

void QTimer::start(std::chrono::milliseconds \emph{msec})

这是重载函数。

以msec毫秒的持续时间启动或重启定时器。

如果定时器已经运行,它将stopped 和重新启动。

如果 singleShot 是 true, 则定时器将仅激活一次。

在Qt5.8引入该函数。

[slot] void QTimer::stop()

停止定时器。

\begin{seeAlso}
start()。
\end{seeAlso}

[signal] void QTimer::timeout()

定时器超时时发出此信号。

\begin{notice}
这是一个私人信号。 它可以用于信号连接,但不能由用户发射。
\end{notice}

\begin{seeAlso}
interval, start(), 和 stop().
\end{seeAlso}

[override virtual protected] void QTimer::timerEvent(QTimerEvent \emph{*e})

重新实现自 QObject::timerEvent()。

int QTimer::timerId() const

如果定时器正在运行,则返回定时器的ID。 否则返回-1。





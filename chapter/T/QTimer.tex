\chapter{QTimer}

QTimer提供了重复和信号槽的定时器。

\begin{tabular}{|r|l|}
	\hline
	属性 & 方法 \\
	\hline
	头文件 & \#include <QTimer>\\      
	\hline
	qmake & QT += core\\      
	\hline
	继承	  & QObject \\ 
	\hline
\end{tabular}

\section{属性}

\begin{tabular}{|r|l|}
	\hline
属性名	 & 类型 \\ 
\hline
active	& const bool\\
\hline
singleShot	& bool\\
\hline
interval &	int\\
\hline
timeType	& Qt::TimerType\\
\hline
remainingTime &	const int\\
	\hline
\end{tabular}


\section{公共成员函数}

\begin{longtable}{|r|l|}
\hline
类型 & 	函数名 \\
\hline
 & QTimer(QObject *parent = nullptr) \\
 \hline
virtual	& $\sim$QTimer() \\
\hline
QMetaObject::Connection	& callOnTimeout(Functor slot, Qt::ConnectionType connectionType = ...) \\
\hline
QMetaObject::Connection	& callOnTimeout(const QObject *context, Functor slot, Qt::ConnectionType connectionType = ...) \\ 
\hline
QMetaObject::Connection	& callOnTimeout(const QObject *receiver, PointerToMemberFunction slot, Qt::ConnectionType connectionType = ...) \\ 
\hline
int	& interval() const \\ 
\hline
std::chrono::milliseconds	 & intervalAsDuration() const \\ 
\hline
bool	& isActive() const \\
\hline
bool &	isSingleShot() const \\
\hline
int	& remainingTime() const \\
\hline
std::chrono::milliseconds	 & remainingTimeAsDuration() const \\
\hline
void	 & setInterval(int msec) \\ 
\hline
void & 	setInterval(std::chrono::milliseconds value) \\ 
\hline
void	 & setSingleShot(bool singleShot) \\ 
\hline
void	  & setTimerType(Qt::TimerType atype) \\
\hline
void	 & start(std::chrono::milliseconds msec) \\
\hline
int	& timerId() const \\
\hline
Qt::TimerType	 & timerType() const \\
\hline
\end{longtable}

\begin{compactitem}
\item 32个共有成员函数继承自QObject
\end{compactitem}


\section{公有槽函数}

\begin{tabular}{|r|l|}
	\hline
	类型	 & 函数名 \\
	\hline
void &	start(int msec) \\
\hline
void &	start() \\
\hline
void &	stop()  \\
	\hline 
\end{tabular}

\begin{compactitem}
\item 一个公有槽函数继承自QObject
\end{compactitem}

\section{信号}

\begin{tabular}{|r|l|}
	\hline
	类型	 & 函数名 \\
	\hline
void	 & timeout() \\
	\hline 
\end{tabular}

\begin{compactitem}
\item 2个信号继承自QObject
\end{compactitem}

\section{静态公有成员函数}

\begin{tabular}{|r|l|}
	\hline
	类型	 & 函数名 \\
	\hline
void	 & singleShot(int msec, const QObject *receiver, const char *member) \\
\hline
void &	singleShot(int msec, Qt::TimerType timerType, const QObject *receiver, const char *member) \\
\hline
void	 & singleShot(int msec, const QObject *receiver, PointerToMemberFunction method) \\
\hline
void	 & singleShot(int msec, Qt::TimerType timerType, const QObject *receiver, PointerToMemberFunction method) \\ 
\hline
void	 & singleShot(int msec, Functor functor) \\
\hline
void	 & singleShot(int msec, Qt::TimerType timerType, Functor functor) \\
\hline
void	 & singleShot(int msec, const QObject *context, Functor functor) \\
\hline
void	 & singleShot(int msec, Qt::TimerType timerType, const QObject *context, Functor functor) \\
\hline
void &	singleShot(std::chrono::milliseconds msec, const QObject *receiver, const char *member) \\
\hline
void	 & singleShot(std::chrono::milliseconds msec, Qt::TimerType timerType, const QObject *receiver, const char *member) \\
\hline
const QMetaObject	 & staticMetaObject \\
	\hline 
\end{tabular}


\begin{compactitem}
\item 10个静态公有成员函数继承自QObject
\end{compactitem}

\section{重新实现保护成员函数}


\begin{tabular}{|r|l|}
	\hline
	类型	 & 函数名 \\
	\hline
virtual void	& timerEvent(QTimerEvent *e) override \\ 
\hline
\end{tabular}


\begin{compactitem}
\item 9个保护成员函数继承自QObject
\end{compactitem}

\section{详细描述}

QTimer 类提供重复和单次定时器。

The QTimer 类为定时器提供了高级编程接口。 要使用它,请创建一个QTimer, 将 timeout() 信号连接到相应的插槽, 然后调用 start(). 从那时起,它将以固定时间间隔发出 timeout() 信号。

一秒 (1000 毫秒) 定时器的示例 (来自 Analog Clock 例子):

\begin{lstlisting}[language=C++]
QTimer *timer = new QTimer(this);
 connect(timer, SIGNAL(timeout()), this, SLOT(update()));
  timer->start(1000);
\end{lstlisting}


从那时起,update()槽函数每秒都被调用。

你可以调用setSingleShot(true)仅触发一次定时器。你也可以使用静态QTimer::singleShot() 函数在特定的时间内调用一个槽函数

\begin{lstlisting}[language=C++]
QTimer::singleShot(200, this, SLOT(updateCaption()));
\end{lstlisting}

在多线程应用,你可以使用QTimer在任何一个有事件循环的线程。使用QThread::exec(),开启一个非GUI线程的事件循环。Qt使用线程亲和性定义哪一个线程会发出timeout()信号。因此,你必须在指定线程内开启和关闭定时器, 它不可能开启一个定时器从另一个线程。

在特殊案例里,窗口系统的事件队列中的所有事件都已处理后,超时为0的定时器将立即超时。 在提供快速的用户界面时,这可以用来完成繁重的工作:

\begin{lstlisting}[language=C++]
 QTimer *timer = new QTimer(this);
 connect(timer, SIGNAL(timeout()), this, SLOT(processOneThing()));
 timer->start();
\end{lstlisting}

从那以后,processOneThing()将会被重复调用。应该以始终快速返回(通常在处理一个数据项之后)的方式编写。 因此,Qt一处理完定时器上所有工作后,就能够分发事件提供给用户接口和停止定时器。对于GUI应用来说,这是典型的方法实现繁重工作。但如今越来越多应用了多线程,我们期待0毫秒的定时器对象会逐渐被QThread取代。

\section{精度和时间分辨率}

定时器的精度依赖底层操作系统和硬件。大多数定时器支持1毫秒的分辨率,尽管在许多实际情况下定时器的精度将不等于该分辨率。

定时器的精度依靠timer type。

对于Qt::PreciseTimer, QTimer会尽量保持一毫秒的精度。精确的定时器也永远不会比预期的超时。对于Qt::CoarseTimer 和 Qt::VeryCoarseTimer 类型, QTimer 可能会早于我们预期中唤醒, 在这些类型的范围内: 5% 的间隔 Qt::CoarseTimer 和 500 毫秒 Qt::VeryCoarseTimer.

如果操作系统繁忙或是不能提供所需精度,所有的时间类型都可能会晚于我们期待的。在这种超时溢出的情况下,即使多个超时已过期,Qt也会仅发出一次,然后将恢复原始间隔。

\section{精度和定时器分辨率}

另一周使用 QTimer 的方法是调用 QObject::startTimer() 为你的对象 和更新实现QObject::timerEvent() 事件 在类内处理 (必须继承 QObject). timerEvent() 的缺点是 不支持单次定时器或信号的高级功能。另一种选择是QBasicTimer。通常,与直接使用 QObject::startTimer() 相比,它不那么麻烦。 有关这三种方法的概述,请参见 Timers 。

有些操作系统可能会限制定时器的使用数量,Qt尽量工作在这些范围之内。

\begin{notice}[另请参阅]
BasicTimer, QTimerEvent, QObject::timerEvent(), Timers, Analog Clock Example, 和 Wiggly Example。
\end{notice}

\section{属性文档}

\begin{tabular}{|r|l|}
	\hline
	属性名&	类型 \\ 
	\hline
active&	const bool \\ 
	\hline 
\end{tabular}

如果定时器正在执行,这个属性是true, 否则是false。

访问函数

\begin{tabular}{|r|l|}
	\hline
类型 &	函数名 \\ 
\hline
bool	& isActive() const \\ 
	\hline 
\end{tabular}

\begin{tabular}{|r|l|}
	\hline
属性名	 & 类型 \\
\hline
interval & 	int \\
	\hline 
\end{tabular}

此属性保存超时间隔(以毫秒为单位)

一旦处理完窗口系统事件队列中的所有事件,超时间隔为0的QTimer 就会超时。

此属性的默认值为0。 设置活动定时器的间隔会改变 timerId()。

访问函数:

\begin{tabular}{|r|l|}
	\hline
类型	 & 函数名 \\
\hline
int & 	interval() const \\
\hline
void	 & setInterval(int \emph{msec}) \\
\hline
void &	setInterval(std::chrono::milliseconds \emph{value}) \\
	\hline 
\end{tabular}


\begin{notice}[另请参阅]
singleShot。
\end{notice}

\begin{tabular}{|l|r|}
\hline
属性名	 & 类型 \\ 
\hline
remainingTime	& const int \\ 
\hline
\end{tabular}

此属性保留剩余时间(以毫秒为单位)

返回定时器的剩余值(以毫秒为单位),直到超时为止。 如果定时器处于非活动状态,则返回值为-1。 如果定时器过期,则返回值为0。

此属性在Qt 5.0中引入。

访问函数

\begin{tabular}{|l|r|}
\hline
类型 & 函数名 \\ 
\hline
int	& remainingTime() const \\
\hline
\end{tabular}


\begin{notice}[另请参阅]
\href{https://github.com/JackLovel/QtDocumentCN/blob/master/Src/T/QTimer/qtimer.html#interval-prop}{interval}。
\end{notice}


\begin{tabular}{|l|r|}
\hline
属性名 & 类型 \\ 
\hline
singleShot &	bool \\
\hline
\end{tabular}


此属性保持定时器是否为单次定时器。

单触发定时器仅触发一次,非单触发定时器每interval毫秒触发一次。

此属性的默认值是false。

访问函数:

\begin{tabular}{|r|l|}
\hline
类型 & 	函数名 \\ 
\hline
bool &	isSingleShot() const \\ 
\hline
void	 & setSingleShot(bool singleShot) \\
\hline
\end{tabular}


\begin{notice}[另请参阅]
interval 和 singleShot()。
\end{notice}

\begin{tabular}{|r|l|}
\hline
类型	 & 属性名 \\ 
\hline
timerType	 & Qt::TimerType \\ 
\hline
\end{tabular}

控制定时器的精度。

默认的属性是 Qt::CoarseTimer。

访问函数:

\begin{tabular}{|r|l|}
\hline
类型	 & 函数名 \\ 
Qt::TimerType & 	timerType() const \\ 
void	 & setTimerType(Qt::TimerType \emph{atype}) \\ 
\hline
\end{tabular}

\begin{notice}[另请参阅]
 Qt::TimerType。
\end{notice}

\section{成员函数文档}















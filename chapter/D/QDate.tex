\chapter{QDate}

QDate Class

\begin{tabular}{|l|l|}
\hline
属性&	方法\\
\hline
头文件&	\#include <QDate>\\
\hline
qmake&	QT += core\\
\hline
\end{tabular}

注意: 该类提供的所有函数都是可重入的。

\splitLine

公共成员类型

\begin{tabular}{|l|l|}
\hline
类型&	名称\\
\hline
enum&	MonthNameType{ DateFormat, StandaloneFormat }\\
\hline
\end{tabular}

\splitLine

公共成员函数

\begin{longtable}{|l|l|}
\hline
类型	&函数名\\
\hline
&QDate(int y, int m, int d)\\
\hline
&QDate()\\
\hline
QDate&	addDays(qint64 ndays) const\\
\hline
QDate&	addMonths(int nmonths, QCalendar cal) const\\
\hline
QDate&	addMonths(int nmonths) const\\
\hline
QDate&	addYears(int nyears, QCalendar cal) const\\
\hline
QDate&	addYears(int nyears) const\\
\hline
int&	day(QCalendar cal) const\\
\hline
int&	day() const\\
\hline
int&	dayOfWeek(QCalendar cal) const\\
\hline
int&	dayOfWeek() const\\
\hline
int&	dayOfYear(QCalendar cal) const\\
\hline
int&	dayOfYear() const\\
\hline
int&	daysInMonth(QCalendar cal) const\\
\hline
int&	daysInMonth() const\\
\hline
int&	daysInYear(QCalendar cal) const\\
\hline
int&	daysInYear() const\\
\hline
qint64&	daysTo(const QDate \&d) const\\
\hline
QDateTime&	endOfDay(Qt::TimeSpec spec = Qt::LocalTime, int offsetSeconds = 0) const\\
\hline
QDateTime&	endOfDay(const QTimeZone \&zone) const\\
\hline
void&	getDate(int *year, int *month, int *day) const\\
\hline
bool&	isNull() const\\
\hline
bool&	isValid() const\\
\hline
int&	month(QCalendar cal) const\\
\hline
int&	month() const\\
\hline
bool&	setDate(int year, int month, int day)\\
\hline
bool&	setDate(int year, int month, int day, QCalendar cal)\\
\hline
QDateTime&	startOfDay(Qt::TimeSpec spec = Qt::LocalTime, int offsetSeconds = 0) const\\
\hline
QDateTime&	startOfDay(const QTimeZone \&zone) const\\
\hline
qint64&	toJulianDay() const\\
\hline
QString&	toString(Qt::DateFormat format = Qt::TextDate) const\\
\hline
QString&	toString(const QString \&format) const\\
\hline
QString&	toString(const QString \&format, QCalendar cal) const\\
\hline
QString&	toString(QStringView format) const\\
\hline
QString&	toString(QStringView format, QCalendar cal) const\\
\hline
int&	weekNumber(int *yearNumber = nullptr) const\\
\hline
int&	year(QCalendar cal) const\\
\hline
int&	year() const\\
\hline
bool&	operator!=(const QDate \&d) const\\
\hline
bool&	operator<(const QDate \&d) const\\
\hline
bool&	operator<=(const QDate \&d) const\\
\hline
bool&	operator==(const QDate \&d) const\\
\hline
bool&	operator>(const QDate \&d) const\\
\hline
bool&	operator>=(const QDate \&d) const\\
\hline
\end{longtable}

静态公共成员

\begin{tabular}{|l|l|}
\hline
类型&	函数名\\
\hline
QDate&	currentDate()\\
\hline
QDate&	fromJulianDay(qint64 jd)\\
\hline
QDate&	fromString(const QString \&string, Qt::DateFormat format = Qt::TextDate)\\
\hline
QDate&	fromString(const QString \&string, const QString \&format)\\
\hline
QDate&	fromString(const QString \&string, const QString \&format, QCalendar cal)\\
\hline
bool&	isLeapYear(int year)\\
\hline
bool&	isValid(int year, int month, int day)\\
\hline
\end{tabular}

\splitLine

相关非成员函数

\begin{tabular}{|l|l|}
\hline
类型&	函数名\\
\hline
QDataStream \& &	operator<<(QDataStream \&out, const QDate \&date) \\
\hline
QDataStream \&	& operator>>(QDataStream \&in, QDate \&date)\\
\hline
\end{tabular}

详细描述

无论系统的日历和地域设置如何,一个\hl{QDate}对象代表特定的一天。它可以告诉您某一天的年、月、日,其对应着格里高利历或者您提供的\hl{QCalendar}。

一个\hl{QDate} 对象一般由显式给定的年月日创建。注意 \hl{QDate} 将1~99的年数保持,不做任何偏移。静态函数currentDate()会创建一个从系统时间读取的\hl{QDate}对象。显式的日期设定也可以使用 setDate()。fromString() 函数返回一个由日期字符串和日期格式确定的日期。

year()、month()、day() 函数可以访问年月日。另外,还有 dayOfWeek()、dayOfYear() 返回一周或一年中的天数。文字形式的信息可以通过 toString()获得。天数和月数可以由 QLocale 类映射成文字。

\hl{QDate} 提供比较两个对象的全套操作,小于意味着日期靠前。

您可以通过 addDays() 给日期增减几天,同样的还有 addMonths()、addYears() 增减几个月、几年。daysTo() 函数返回两个日期相距的天数。

daysInMonth() 和 daysInYear() 函数返回一个月、一年里有几天。isLeapYear() 用于判断闰年。

特别注意

\begin{itemize}
\item 年数没有0 第0年的日期是非法的。公元后是正年份,公元前是负年份。QDate(1, 1, 1) 的前一天是 QDate(-1, 12, 31)。
\item 合法的日期范围 日期内部存储为儒略日天号,使用连续的整数记录天数,公元前4714年11月24日是格里高利历第0天(儒略历的公元前4713年1月1日)。除了可以准确有效地表示绝对日期,此类也可以用来做不同日历系统的转换。儒略历天数可以通过 QDate::toJulianDay() 和 QDate::fromJulianDay()读写。 由于技术原因,储存的儒略历天号在 -784350574879~784354017364 之间,大概是公元前2亿年到公元后2亿年之间。
\end{itemize}

另请参阅:QTime、QDateTime、QCalendar、QDateTime::YearRange、QDateEdit、QDateTimeEdit 和 QCalendarWidget。

\splitLine

成员类型文档

enum QDate::MonthNameType

此枚举描述字符串中月份名称的表示类型

%%% Local Variables:
%%% mode: latex
%%% TeX-master: "../../master"
%%% End:
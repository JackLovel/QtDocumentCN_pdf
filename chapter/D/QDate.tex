\chapter{QDate}

\begin{tabular}{|l|l|}
\hline
属性&	方法\\
\hline
头文件&	\#include <QDate>\\
\hline
qmake&	QT += core\\
\hline
\end{tabular}


\begin{notice}
该类提供的所有函数都是可重入的。
\end{notice}


\splitLine

\section{公共成员类型}

\begin{tabular}{|l|l|}
\hline
类型&	名称\\
\hline
enum&	MonthNameType{ DateFormat, StandaloneFormat }\\
\hline
\end{tabular}

\splitLine

\section{公共成员函数}

\begin{longtable}[l]{|l|m{30em}|}
\hline
类型	&函数名\\
\hline
&QDate(int \emph{y}, int \emph{m}, int \emph{d})\\
\hline
&QDate()\\
\hline
QDate&	addDays(qint64 \emph{ndays}) const\\
\hline
QDate&	addMonths(int \emph{nmonths}, QCalendar \emph{cal}) const\\
\hline
QDate&	addMonths(int \emph{nmonths}) const\\
\hline
QDate&	addYears(int \emph{nyears}, QCalendar \emph{cal}) const\\
\hline
QDate&	addYears(int \emph{nyears}) const\\
\hline
int&	day(QCalendar \emph{cal}) const\\
\hline
int&	day() const\\
\hline
int&	dayOfWeek(QCalendar \emph{cal}) const\\
\hline
int&	dayOfWeek() const\\
\hline
int&	dayOfYear(QCalendar \emph{cal}) const\\
\hline
int&	dayOfYear() const\\
\hline
int&	daysInMonth(QCalendar \emph{cal}) const\\
\hline
int&	daysInMonth() const\\
\hline
int&	daysInYear(QCalendar \emph{cal}) const\\
\hline
int&	daysInYear() const\\
\hline
qint64&	daysTo(const QDate \emph{\&d}) const\\
\hline
QDateTime&	endOfDay(Qt::TimeSpec \emph{spec} = Qt::LocalTime, int \emph{offsetSeconds} = 0) const\\
\hline
QDateTime&	endOfDay(const QTimeZone \emph{\&zone}) const\\
\hline
void&	getDate(int \emph{*year}, int \emph{*month}, int \emph{*day}) const\\
\hline
bool&	isNull() const\\
\hline
bool&	isValid() const\\
\hline
int&	month(QCalendar \emph{cal}) const\\
\hline
int&	month() const\\
\hline
bool&	setDate(int \emph{year}, int \emph{month}, int \emph{day})\\
\hline
bool&	setDate(int \emph{year}, int \emph{month}, int \emph{day}, QCalendar \emph{cal})\\
\hline
QDateTime&	startOfDay(Qt::TimeSpec \emph{spec} = Qt::LocalTime, int \emph{offsetSeconds} = 0) const\\
\hline
QDateTime&	startOfDay(const QTimeZone \emph{\&zone}) const\\
\hline
qint64&	toJulianDay() const\\
\hline
QString&	toString(Qt::DateFormat \emph{format} = Qt::TextDate) const\\
\hline
QString&	toString(const QString \emph{\&format}) const\\
\hline
QString&	toString(const QString \emph{\&format}, QCalendar \emph{cal}) const\\
\hline
QString&	toString(QStringView \emph{format}) const\\
\hline
QString&	toString(QStringView \emph{format}, QCalendar \emph{cal}) const\\
\hline
int&	weekNumber(int \emph{*yearNumber} = nullptr) const\\
\hline
int&	year(QCalendar \emph{cal}) const\\
\hline
int&	year() const\\
\hline
bool&	operator!=(const QDate \emph{\&d}) const\\
\hline
bool&	operator<(const QDate \emph{\&d}) const\\
\hline
bool&	operator<=(const QDate \emph{\&d}) const\\
\hline
bool&	operator==(const QDate \emph{\&d}) const\\
\hline
bool&	operator>(const QDate \emph{\&d}) const\\
\hline
bool&	operator>=(const QDate \emph{\&d}) const\\
\hline
\end{longtable}

\section{静态公共成员}

\begin{tabular}{|l|l|}
\hline
类型&	函数名\\
\hline
QDate&	currentDate()\\
\hline
QDate&	fromJulianDay(qint64 \emph{jd})\\
\hline
QDate&	fromString(const QString \emph{\&string}, Qt::DateFormat \emph{format} = Qt::TextDate)\\
\hline
QDate&	fromString(const QString \emph{\&string}, const QString \emph{\&format})\\
\hline
QDate&	fromString(const QString \emph{\&string}, const QString \emph{\&format}, QCalendar \emph{cal})\\
\hline
bool&	isLeapYear(int \emph{year})\\
\hline
bool&	isValid(int \emph{year}, int \emph{month}, int \emph{day})\\
\hline
\end{tabular}

\splitLine

\section{相关非成员函数}

\begin{tabular}{|l|l|}
\hline
类型&	函数名\\
\hline
QDataStream \& &	operator<<(QDataStream \emph{\&out}, const QDate \emph{\&date}) \\
\hline
QDataStream \&	& operator>>(QDataStream \emph{\&in}, QDate \emph{\&date})\\
\hline
\end{tabular}

\section{详细描述}

无论系统的日历和地域设置如何,一个\hl{QDate}对象代表特定的一天。它可以告诉您某一天的年、月、日,其对应着格里高利历或者您提供的\hl{QCalendar}。

一个\hl{QDate} 对象一般由显式给定的年月日创建。注意 \hl{QDate} 将1~99的年数保持,不做任何偏移。静态函数currentDate()会创建一个从系统时间读取的\hl{QDate}对象。显式的日期设定也可以使用 setDate()。fromString() 函数返回一个由日期字符串和日期格式确定的日期。

year()、month()、day() 函数可以访问年月日。另外,还有 dayOfWeek()、dayOfYear() 返回一周或一年中的天数。文字形式的信息可以通过 toString()获得。天数和月数可以由 QLocale 类映射成文字。

\hl{QDate} 提供比较两个对象的全套操作,小于意味着日期靠前。

您可以通过 addDays() 给日期增减几天,同样的还有 addMonths()、addYears() 增减几个月、几年。daysTo() 函数返回两个日期相距的天数。

daysInMonth() 和 daysInYear() 函数返回一个月、一年里有几天。isLeapYear() 用于判断闰年。

\section{特别注意}

\begin{compactitem}[\arr]
\item 年数没有0 第0年的日期是非法的。公元后是正年份,公元前是负年份。\hl{QDate(1, 1, 1)} 的前一天是 \hl{QDate(-1, 12, 31)}。
\item 合法的日期范围 日期内部存储为儒略日天号,使用连续的整数记录天数,公元前4714年11月24日是格里高利历第0天(儒略历的公元前4713年1月1日)。除了可以准确有效地表示绝对日期,此类也可以用来做不同日历系统的转换。儒略历天数可以通过 QDate::toJulianDay() 和 QDate::fromJulianDay()读写。 由于技术原因,储存的儒略历天号在 -784350574879~784354017364 之间,大概是公元前2亿年到公元后2亿年之间。
\end{compactitem}


\begin{seeAlso}
QTime、QDateTime、QCalendar、QDateTime::YearRange、QDateEdit、QDateTimeEdit 和 QCalendarWidget。
\end{seeAlso}


\splitLine

\section{成员类型文档}

enum QDate::MonthNameType

此枚举描述字符串中月份名称的表示类型

\begin{tabular}{|l|l|m{20em}|}
\hline
常量	&数值&	描述\\
\hline
\hl{QDate::DateFormat}&	\hl{0}&	用于日期到字符串格式化。\\
\hline
\hl{QDate::StandaloneFormat}	&\hl{1}&	用于枚举月份和一周七天。通常单独的名字用首
  字母大写的单数形式书写。\\
\hline
\end{tabular}

此枚举在Qt 4.5中引入或修改。

\splitLine

成员函数文档

QString QDate::toString(QStringView \emph{format}) const

QString QDate::toString(QStringView \emph{format}, QCalendar \emph{cal}) const

QString QDate::toString(const QString \emph{\&format}) const

QString QDate::toString(const QString \emph{\&format}, QCalendar \emph{cal}) const

返回字符串形式的日期。 参数format控制输出的格式。如果传入参数cal, 它决定了使用的日历系统,默认是格里高利历。

日期格式的表示如下:

\begin{tabular}{|l|m{30em}|}
\hline
占位符 &	输出格式\\
\hline
d&	无占位0的日期 (1 to 31)\\
\hline
dd&	有占位0的日期 (01 to 31)\\
\hline
ddd&	简写的一周七天名称(例如 'Mon' to 'Sun')。使用系统地域设置来格
     式化, 也就是 QLocale::system()\\
\hline
dddd&	长版的一周七天名称 (例如 'Monday' to 'Sunday')。使用系统地域设
      置来格式化, 也就是 QLocale::system()\\
\hline
M&	无占位0的月份 (1 to 12)\\
\hline
MM&	有占位0的月份 (01 to 12)\\
\hline
MMM&	缩写的月份名称 (例如 'Jan' to 'Dec')。使用系统地域设置来格式化,
     也就是 QLocale::system()\\
\hline
MMMM&	长版的月份名称 (例如 'January' to 'December')。使用系统地域设
      置来格式化, 也就是 QLocale::system()\\
\hline
yy&	两位数年份 (00 to 99)\\
\hline
yyyy&	四位数年份。 如果是负数,加上符号是五个字符\\
\hline
\end{tabular}

单引号包裹的内容将会一字不差地放进输出到字符串中(不带外面的单引号),尽管包含上述格式占位符。连续两个单引号('')会被转义成一个单引号输出。所有其他字符不会被转义,将原封不动输出。

没有分隔符的多个占位符(例如 "ddMM")受支持,但要慎用。因为结果的可读性不好,容易引发歧义(例如难以区分“dM”的输出"212"是清2.12还是21,2)

假设今天是1969年7月20日,下面是一些格式字符串的例子

\begin{tabular}{|l|l|}
\hline
dd.MM.yyyy	&20.07.1969\\
\hline
ddd MMMM d yy&	Sun July 20 69\\
\hline
'The day is' dddd&	The day is Sunday\\
\hline
\end{tabular}

如果日期非法,会返回空字符串。


\begin{notice}
如果需要本地化的日期表达, 请使用 QLocale::system().toString(),因为 QDate 将在 Qt 6 使用英文表达(C语言风格)。
\end{notice}


\begin{seeAlso}
fromString()、QDateTime::toString()、QTime::toString() 和
QLocale::toString()。
\end{seeAlso}


\splitLine

QDateTime QDate::endOfDay(Qt::TimeSpec \emph{spec} = Qt::LocalTime, int \emph{offsetSeconds} = 0) const

QDateTime QDate::endOfDay(const QTimeZone \emph{\&zone}) const

返回这一天的最后一刻的时间。通常来说,是午夜前1ms:然而,如果时区转换让这一天跨过午夜(如夏令时),返回的是真实的最后一刻时间。这种情况只可能在使用时区参数QTimeZone zone或者本地时间参数 Qt::LocalTime spec时发生。

参数 offsetSeconds 会被忽略,除非参数 spec 为 Qt::OffsetFromUTC,其表示出了时区信息。如果UTC和那个时区间没有过渡,一天的结束是 QTime(23, 59, 59, 999)。

在罕见的日期被整体跳过(只在从东向西跨越国际日期变更线时),返回可能是非法的。将 Qt::TimeZone 作为 spec 参数传递 (而不是一个 QTimeZone) 也会造成非法结果,如果那一时刻超过了 QDateTime 的表示范围。

函数在 Qt 5.14 中引入。


\begin{seeAlso}
startOfDay()。
\end{seeAlso}

\splitLine

QDateTime QDate::startOfDay(Qt::TimeSpec \emph{spec} = Qt::LocalTime, int \emph{offsetSeconds} = 0) const

QDateTime QDate::startOfDay(const QTimeZone \emph{\&zone}) const

返回一天的开始时刻。通常来说应该是午夜那一时刻:然而,如果时区转换让这一天跨过午夜(如夏令时),返回的是真实的最早的一刻时间。这种情况只可能在使用时区参数QTimeZone zone或者本地时间参数 Qt::LocalTime spec时发生。

参数 offsetSeconds 会被忽略,除非参数 spec 为 Qt::OffsetFromUTC,其表示出了时区信息。如果UTC和那个时区间没有过渡,一天的结束是 QTime(0, 0)。

在罕见的日期被整体跳过(只在从东向西跨越国际日期变更线时),返回可能是非法的。

将 Qt::TimeZone 作为 spec 参数传递 (而不是一个 QTimeZone) 也会造成非法结果,如果那一时刻超过了 QDateTime 的表示范围。

函数在 Qt 5.14 中引入。

\begin{seeAlso}
endOfDay()。
\end{seeAlso}

\splitLine

QDate::QDate(int \emph{y}, int \emph{m}, int \emph{d})

从年月日构造对象,使用格里高利历。如果日期非法,对象不会修改,且 isValid() 返回 \hl{false}。


\begin{notice}
1~99年就对应于它本身,不会偏移。第0年是非法的。
\end{notice}


\begin{seeAlso}
isValid() 和 QCalendar::dateFromParts()。
\end{seeAlso}



\splitLine

QDate::QDate()
构造一个空的日期,也是不可用的。


\begin{seeAlso}
isNull() 和 isValid()。
\end{seeAlso}

\splitLine

QDate QDate::addDays(qint64 \emph{ndays}) const

返回一个 ndays 天之后的新日期对象(负数意味着往前减日期)。

当前日期或新日期是非法日期时,返回非法日期。



\begin{seeAlso}
addMonths()、addYears() 和 daysTo()。
\end{seeAlso}

\splitLine

QDate QDate::addMonths(int \emph{nmonths}) const

QDate QDate::addMonths(int \emph{nmonths}, QCalendar \emph{cal}) const

返回一个 nmonths 月之后的新日期对象(负数意味着往前减日期)。

如果传入 cal 参数,会使用日历系统使用,否则使用格里高利历。


\begin{notice}
如果新的日期超出年、月的合理范围,函数讲返回那个月中最接近的日期。
\end{notice}

\begin{seeAlso}
addDays() 和 addYears()。
\end{seeAlso}

\splitLine

QDate QDate::addYears(int \emph{nyears}) const

QDate QDate::addYears(int \emph{nyears}, QCalendar \emph{cal}) const

返回一个 nyears 年之后的新日期对象(负数意味着往前减日期)。

如果传入 cal 参数,会使用日历系统使用,否则使用格里高利历。



\begin{notice}
如果新的日期超出年、月的合理范围,函数讲返回那个月中最接近的日期。
\end{notice}

\begin{seeAlso}
addDays() 和 addMonths()。
\end{seeAlso}


\splitLine

[static] QDate QDate::currentDate()

返回系统时钟所示的今天的日期对象。



\begin{seeAlso}
QTime::currentTime() 和 QDateTime::currentDateTime()。
\end{seeAlso}

\splitLine

int QDate::day() const

int QDate::day(QCalendar \emph{cal}) const

返回日期是当前月份的第几天。

如果传入 cal 参数,会使用此日历系统,否则使用格里高利历。非法日期则返回0。

一些日历系统中,返回值可能大于7。

\begin{seeAlso}
year()、month() 和 dayOfWeek()。
\end{seeAlso}

\splitLine

int QDate::dayOfWeek() const

int QDate::dayOfWeek(QCalendar \emph{cal}) const

返回日期是当前周的第几天(1=周一,7=周日)。

如果传入 cal 参数,会使用此日历系统,否则使用格里高利历。非法日期则返回0。

一些日历系统中,返回值可能大于7。

\begin{seeAlso}
day()、dayOfYear() 和 Qt::DayOfWeek。
\end{seeAlso}

\splitLine

int QDate::dayOfYear() const

int QDate::dayOfYear(QCalendar \emph{cal}) const

返回日期是当年的第几天(第一天返回1)。

如果传入 cal 参数,会使用此日历系统,否则使用格里高利历。非法日期则返回0。

\begin{seeAlso}
day() 和 dayOfWeek()。
\end{seeAlso}

\splitLine

int QDate::daysInMonth() const

int QDate::daysInMonth(QCalendar \emph{cal}) const

返回日期对应的月份中总天数。

如果传入 cal 参数,会使用此日历系统,否则使用格里高利历 (返回值是 28~31)。非法日期则返回0。



\begin{seeAlso}
day() 和 daysInYear()。
\end{seeAlso}

\splitLine

int QDate::daysInYear() const

int QDate::daysInYear(QCalendar \emph{cal}) const

返回日期对应的一年中的总天数。

如果传入 cal 参数,会使用此日历系统,否则使用格里高利历 (返回值是 365 或 366)。非法日期则返回0。



\begin{seeAlso}
day() 和 daysInMonth()。
\end{seeAlso}

\splitLine

qint64 QDate::daysTo(const QDate \&\emph{d}) const

返回两个日期相差的天数 (\emph{d} 日期靠前则返回为负)。

如果比较的二者中有非法日期,返回0。

示例:

\begin{lstlisting}[language=C++]
QDate d1(1995, 5, 17);  // May 17, 1995
QDate d2(1995, 5, 20);  // May 20, 1995
d1.daysTo(d2);          // returns 3
d2.daysTo(d1);          // returns -3
\end{lstlisting}


\begin{seeAlso}
addDays()。
\end{seeAlso}

\splitLine

\hl{[static]}QDate QDate::fromJulianDay(qint64 \emph{jd})

将jd表示的儒略日解析为日期并返回。

\begin{seeAlso}
toJulianDay()。
\end{seeAlso}

\splitLine

\hl{[static]}QDate QDate::fromString(const QString \emph{\&string},
Qt::DateFormat \emph{format} = Qt::TextDate)

返回 string 表示的 QDate 对象,非法字符串不会被解析。

\begin{notice}
Qt::TextDate:建议使用英文的简写月份名称(例如 "Jan")。
尽管本地化的月份名称在 Qt 5 中也可使用,但它们会依赖于用户的区域设置。
\end{notice}

\begin{notice}
对本地化的日期的支持,包括 Qt::SystemLocaleDate、Qt::SystemLocaleShortDate、Qt::SystemLocaleLongDate、Qt::LocaleDate、Qt::DefaultLocaleShortDate 和 Qt::DefaultLocaleLongDate,
将在 Qt 6 被移除。使用 QLocale::toDate() 代替。
\end{notice} 

\begin{seeAlso}
toString() 和 QLocale::toDate()。
\end{seeAlso}

\splitLine

\hl{[static]}QDate QDate::fromString(const QString \emph{\&string}, const QString \emph{\&format})

\hl{[static]}QDate QDate::fromString(const QString \emph{\&string}, const QString \emph{\&format}, QCalendar \emph{cal})

返回 \emph{string}  表示的 QDate 对象,非法字符串不会被解析。

如果传入 \emph{cal} 参数,会使用此日历系统,否则使用格里高利历。下面的格式描述针对于格里高利历,其它日历可能不同。

日期格式的表示如下:

\begin{tabular}{|l|l|}
\hline
占位符&	输出格式\\
\hline
d&	无占位0的日期 (1 to 31)\\
\hline
dd&	有占位0的日期 (01 to 31)\\
\hline
ddd&	简写的一周七天名称(例如 'Mon' to 'Sun')。使用系统地域设置来格式化, 也就是 QLocale::system()\\
\hline
dddd&	长版的一周七天名称 (例如 'Monday' to 'Sunday')。使用系统地域设置来格式化, 也就是 QLocale::system()\\
\hline
M&	无占位0的月份 (1 to 12)\\
\hline
MM&	有占位0的月份 (01 to 12)\\
\hline
MMM&	缩写的月份名称 (例如 'Jan' to 'Dec')。使用系统地域设置来格式化, 也就是 QLocale::system()\\
\hline
MMMM&	长版的月份名称 (例如 'January' to 'December')。使用系统地域设置来格式化, 也就是 QLocale::system()\\
\hline
yy&	两位数年份 (00 to 99)\\
\hline
yyyy&	四位数年份。 如果是负数,加上符号是五个字符\\
\hline
\end{tabular}

\begin{notice}
 不行此函数的其他版, 日期和月份名必须使用用户本地语言。只有用户语言是英语时,英文名称才适用。
\end{notice}

所有其他字符将会被舍弃,单引号('')包裹的内容,无论是否包含格式占位符也会被舍弃。例如:

\begin{lstlisting}[language=C++]
QDate date = QDate::fromString("1MM12car2003", "d'MM'MMcaryyyy");
// date is 1 December 2003
\end{lstlisting}

如果字符串不符合格式,一个将返回一个非法的 QDate。不需要前面补0的格式
是贪婪的,这意味着他们会读取两位数,尽管数字大小超出日期合法范围,并且
会占据给后续格式读取的数字位置。例如,下面的日期可以表示1月30日,但M格
式会捕获两位数字,导致日期返回非法:

\begin{lstlisting}[language=C++]
QDate date = QDate::fromString("130", "Md"); // invalid
\end{lstlisting}

年月日中,没有出现在格式中的,将会使用如下默认值

\begin{tabular}{|l|l|}
\hline
Year&	1900\\
\hline
Month&	1\\
\hline
Day&	1\\
\hline
\end{tabular}

下面是默认值的示例:

\begin{lstlisting}[language=C++]
QDate::fromString("1.30", "M.d");           // January 30 1900
QDate::fromString("20000110", "yyyyMMdd");  // January 10, 2000
QDate::fromString("20000110", "yyyyMd");    // January 10, 2000
\end{lstlisting}



\begin{notice}
如果使用本地化的日期表达, 请使用 QLocale::system().toDate(), 因为 QDate 将在 Qt 6使用英文表达 (C语言风格) 。
\end{notice}


\begin{seeAlso}
toString()、QDateTime::fromString()、QTime::fromString() 和 QLocale::toDate()。
\end{seeAlso}


\splitLine

void QDate::getDate(int  \emph{*year}, int \emph{*month}, int \emph{*day}) const
读取日期,储存到 \emph{*year},  \emph{*month}和 \emph{*day}。指针可以是空指针。

如果日期非法,返回的是0。

\begin{notice}
 在 Qt 5.7 之前, 这个函数不是常函数。
\end{notice}

此函数在 Qt 4.5 中引入。

\begin{seeAlso}
year(),month(),day(),isValid() 和
QCalendar::partsFromDate()。
\end{seeAlso}

\splitLine

\hl{[static]}bool QDate::isLeapYear(int \emph{year})

判断是否是格里高利历的闰年,是则返回 \hl{true}。

\begin{seeAlso}
QCalendar::isLeapYear()。
\end{seeAlso}

\splitLine

bool QDate::isNull() const

判断日期是否为空,日期为空则返回 \hl{true}。 空日期是非法的。

\begin{notice}
行为与 isValid() 等价。
\end{notice}

\begin{seeAlso}
isValid()。
\end{seeAlso}

\splitLine

bool QDate::isValid() const

判断日期是否合法,合法返回 \hl{true}。



\begin{seeAlso}
isNull() 和 QCalendar::isDateValid()。
\end{seeAlso}

\splitLine

[static]bool QDate::isValid(int \emph{year}, int \emph{month}, int \emph{day})

是上述方法的重载。

判断日期(以格里高利历解析)是否合法,合法返回 \hl{true}。

例如:

\begin{lstlisting}[language=C++]
QDate::isValid(2002, 5, 17);  // true
QDate::isValid(2002, 2, 30);  // false (2月没有20日)
QDate::isValid(2004, 2, 29);  // true (是闰年)
QDate::isValid(2000, 2, 29);  // true (是闰年)
QDate::isValid(2006, 2, 29);  // false (不是闰年)
QDate::isValid(2100, 2, 29);  // false (不是闰年)
QDate::isValid(1202, 6, 6);   // true (即使这一年在格里高利历之前)
\end{lstlisting}


\begin{seeAlso}
isNull(),setDate() 和 QCalendar::isDateValid()。
\end{seeAlso}

\splitLine

int QDate::month() const

int QDate::month(QCalendar \emph{cal}) const

返回月份编号,第一个月返回1。

如果传入 \emph{cal} 参数,会使用日历系统使用,否则使用格里高利历。

对于格里高利历,1月就是咱中文说的阳历一月。

对于非法日期返回0。注意有一些日历中,月份可能多于12个。



\begin{seeAlso}
 year() 和 day()。
\end{seeAlso}

\splitLine

bool QDate::setDate(int \emph{year}, int  \emph{month}, int \emph{day})

设置对应的日期,使用的是格里高利历。 如果设置的日期合法,将返回
\hl{true},否则日期标记为非法并返回 \hl{false}。

此函数在 Qt 4.2 中引入。


\begin{seeAlso}
isValid() 和 QCalendar::dateFromParts()。
\end{seeAlso}

\splitLine

bool QDate::setDate(int \emph{year}, int \emph{month}, int \emph{day}, QCalendar \emph{cal})

设置对应的日期,使用的是cal对应的日历。如果设置的日期合法,将返回 true,否则日期标记为非法并返回 false。

函数在 Qt 5.14 中引入。

\begin{seeAlso}
isValid() 和 QCalendar::dateFromParts()。
\end{seeAlso}

\splitLine

qint64 QDate::toJulianDay() const

将日期转换为儒略日。

\begin{seeAlso}
fromJulianDay()。
\end{seeAlso}

\splitLine

QString QDate::toString(Qt::DateFormat \emph{format} = Qt::TextDate) const

这是一个重载函数,返回日期的字符串。 format 参数决定字符串格式。

如果 format 是 Qt::TextDate,日期使用默认格式。
日期月份将使用系统地域设置,也就是 QLocale::system()。
一个例子是 "Sat May 20 1995"。

如果 format 是 Qt::ISODate,字符串按照 ISO 8601 格式展开,格式形如 yyyy-MM-dd。例如2002-01-05

format 中的 Qt::SystemLocaleDate,Qt::SystemLocaleShortDate 和Qt::SystemLocaleLongDate 将在 Qt 6 中删除。这些应当由 QLocale::system().toString(date, QLocale::ShortFormat) 或 QLocale::system().toString(date, QLocale::LongFormat) 替代。

format 中的 Qt::LocaleDate,Qt::DefaultLocaleShortDate 和Qt::DefaultLocaleLongDate 将在 Qt 6 中删除。这些应当由 QLocale().toString(date, QLocale::ShortFormat) 或 QLocale().toString(date, QLocale::LongFormat) 替代。

如果 format 是 Qt::RFC2822Date, 字符串会转换成 RFC 2822 兼容的格式。示例其一是 "20 May 1995"。

如果日期非法,返回空字符串。

\begin{warning}
Qt::ISODate 格式只在0~9999年可用。
\end{warning}

\begin{seeAlso}
fromString() 和 QLocale::toString()。
\end{seeAlso}


\splitLine

int QDate::weekNumber(int \emph{*yearNumber} = nullptr) const

返回 ISO 8601 周序号 (1 到 53)。对于非法日期返回0。

如果 yearNumber 不是 nullptr(默认参数), 年号返回值存于 *yearNumber。

根据 ISO 8601, 格里高利历中,跨年的周属于天数更多的那年。 由于 ISO 8601 规定一周始于周一,周三在哪一年这周就属于哪一年。 大多数年有52周,但也有53周的年份。

\begin{notice}
 *yearNumber 不总是与 year() 相等。例如, 2000.1.1是1999年第52周, 2002.12.31是2003年第1周。
\end{notice}

\begin{seeAlso}
isValid()。
\end{seeAlso}

\splitLine

int QDate::year() const

int QDate::year(QCalendar \emph{cal}) const

返回整数型的年份。

如果传入参数 cal , 它决定了使用的日历系统,默认是格里高利历。

如果日期不合法,返回0。在一些日历系统中,比第一年早的年份非法。

如果使用包含第0年的日历系统,在返回0时通过 isValid() 判断是否合法。这些日历正常的使用负年份,-1年的下一年是0年,再下一年是1年。

一些日历中,没有0年份但是有负数年份。例如格里高利历,公元前x年就是年份-x。

\begin{seeAlso}
month(),day(),QCalendar::hasYearZero() 和QCalendar::isProleptic()。
\end{seeAlso}

\splitLine

bool QDate::operator!=(const QDate \emph{\&d}) const

bool QDate::operator<(const QDate \emph{\&d}) const

bool QDate::operator<=(const QDate \emph{\&d}) const

bool QDate::operator==(const QDate \emph{\&d}) const

bool QDate::operator>(const QDate \emph{\&d}) const

bool QDate::operator>=(const QDate \emph{\&d}) const

对于日期A和B,大于意味着日期靠后,小于意味着日期靠前,相等就是同一天。

\splitLine

\textbf{相关非成员函数}

QDataStream \&operator<<(QDataStream \emph{\&out}, const QDate \emph{\&date})

向数据流写入日期

\begin{seeAlso}
Serializing Qt Data Types。
\end{seeAlso}

QDataStream \&operator>>(QDataStream \emph{\&in}, QDate \emph{\&date})

从数据流读出日期

\begin{seeAlso}
Serializing Qt Data Types。
\end{seeAlso}
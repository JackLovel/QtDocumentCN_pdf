\chapter{QLibrary}

Qlibrary用于运行时加载库。

\begin{tabular}{|l|l|}
\hline
属性 &	内容\\
\hline
头文件:& 	\#include <QLibrary>\\
\hline
qmake:& 	QT += core\\
\hline
继承于:& 	QObject\\
\hline
\end{tabular}

\begin{notice}
此类中全部函数可重入。
\end{notice}

\section{公共成员类型}

\begin{tabular}{|l|m{23em}|}
\hline
类型 &	名称\\
\hline
enum &	LoadHint { ResolveAllSymbolsHint, ExportExternalSymbolsHint,
       LoadArchiveMemberHint, PreventUnloadHint, DeepBindHint }\\
\hline
flags &	LoadHints \\ 
\hline
\end{tabular}

\section{属性}

\begin{itemize}
\item fileName : QString
\item loadHints : LoadHints
\end{itemize}

\section{公共成员函数}

\begin{longtable}{|l|m{25em}|}
\hline
 类型& 	函数名\\
\hline
	&QLibrary(const QString \&fileName, const QString \&version,
   QObject *parent = nullptr)\\
\hline
	&QLibrary(const QString \&fileName, int verNum, QObject *parent =
   nullptr)\\
\hline
	&QLibrary(const QString \&fileName, QObject *parent = nullptr)\\
\hline
	&QLibrary(QObject *parent = nullptr)\\
\hline
virtual& 	$\sim$QLibrary()\\
\hline
QString& 	errorString() const\\
\hline
QString& 	fileName() const\\
\hline
bool& 	isLoaded() const\\
\hline
bool& 	load()\\
\hline
QLibrary::LoadHints& 	loadHints() const\\
\hline
QFunctionPointer& 	resolve(const char *symbol)\\
\hline
void& 	setFileName(const QString \&fileName)\\
\hline
void& 	setFileNameAndVersion(const QString \&fileName, int versionNumber)\\
\hline
void& 	setFileNameAndVersion(const QString \&fileName, const QString
      \&version)\\
\hline
void& 	setLoadHints(QLibrary::LoadHints hints)\\
\hline
bool& 	unload()\\
\hline
\end{longtable}

\section{静态公共成员}

\begin{tabular}{|l|l|}
\hline
类型& 	函数名\\
\hline
bool& 	isLibrary(const QString \&fileName)\\
\hline
QFunctionPointer& 	resolve(const QString \&fileName, const char
                  *symbol)\\
\hline
QFunctionPointer& 	resolve(const QString \&fileName, int verNum, const char *symbol)\\
\hline
QFunctionPointer& 	resolve(const QString \&fileName, const QString \&version, const char *symbol)\\
\hline
\end{tabular}

\section{详细描述}

QLibrary的实例用于操作一个动态链接库文件(文中称为库,也就是DLL)。QLibrary提供访问库中函数的一种平台无关方式。您可以在构造时传递库文件名,也可以通过 setFileName() 给对象显式设置。加载库时,QLibrary在所有系统指定的位置搜索 (例如: Unix上的 LD\_LIBRARY\_PATH), 除非文件名是绝对路径。

如果文件路径是绝对路径,则会首先尝试在这个位置加载。如果找不到,QLibrary尝试不同系统相关的前后缀的文件名,比如Unix系的前缀“lib”,后缀“.so”,Mac及IOS的后缀".dylib",Windows的后缀".dll"。

如果文件路径不是绝对路径,Qlibrary改变搜索顺序,首先尝试系统特定的前后缀,之后是特定文件路径。

这让使用除去前后缀的库基本名称来指定库文件变得可能。因此代码可以在不同操作系统里执行,但不用太多代码尝试各种文件名称。

最重要的函数是 load() 用于动态加载库,isLoaded() 用于检查是否加载成功,以及 resolve() 来解析库中的符号。如果库还没加载,resolve() 函数隐式地加载这个库。多个QLibrary实例访问同一个物理库文件是可行的。一旦被加载,库在内存中一直保留到程序结束。您可以通过 unload() 尝试卸载一个库,但如果有其他QLibrary实例在使用同一个库文件,调用会失败。只有在每一个实例都调用过 unload() 后,库才会真正卸载。

Qlibrary 的一种典型用法是解析库中的导出符号,并调用其对应的C语言函数。这叫做显式链接,对应于隐式链接。隐式链接是构建中的链接可执行文件和静态库的步骤。

下面的代码片段加载了个库,解析"mysymbol"符号,并在一切就绪的情况下调用
这个函数。如果出现了问题, 例如库文件不存在或者符号未定义,函数指针将
会是nullptr,且不会调用。

\begin{lstlisting}[language=C++]
QLibrary myLib("mylib");
typedef void (*MyPrototype)();
MyPrototype myFunction = (MyPrototype) myLib.resolve("mysymbol");
if (myFunction)
    myFunction();
\end{lstlisting}

符号必须作为C函数导出,resolve()才能工作。这意味着用C++编译器编译的函
数必须由 \hl{extern "C"}块包裹。在Windows上,还要求导出函数要使用dllexport宏;
实现详情见 resolve()。方便起见,resolve() 函数有静态形式,您可以在不现
实加载库的情况下使用:


\begin{lstlisting}[language=C++]
typedef void (*MyPrototype)();
MyPrototype myFunction =
        (MyPrototype) QLibrary::resolve("mylib", "mysymbol");
if (myFunction)
    myFunction();
\end{lstlisting}

\begin{notice}[另请参阅]
QPluginLoader.
\end{notice}

\section{成员类型介绍}

enum QLibrary::LoadHint flags QLibrary::LoadHints

这个枚举描述了可能的可以用来改变库的加载行为的指示。这些取值指示在库加
载后如何解析符号,通过 setLoadHints() 指定。

\begin{tabular}{|l|l|m{20em}|}
\hline
常量 	&值& 	描述\\
\hline
QLibrary::ResolveAllSymbolsHint& 	0x01 &	在加载库的时候解析符号,而
                                           不是简单等到 resolve() 调用。
  \\
\hline
QLibrary::ExportExternalSymbolsHint& 	0x02& 	导出库中未解析的符号和
                                              外部符号,这些符号可以在
                                              后续动态加载的库中解析。
  \\
\hline
QLibrary::LoadArchiveMemberHint& 	0x04 &	运行库的文件名指定压缩包中
                                           的特定对象。如果设置了这个
                                           指示,文件名包含一个路径,
                                           其指向归档文件,接着是其中
                                           的成员名称。\\
\hline
QLibrary::PreventUnloadHint& 	0x08 	&阻止库从地址空间通过close()卸
                                        载。如果之后再有open()调用,库
                                        中的静态变量不会重新初始化。\\
\hline
QLibrary::DeepBindHint& 	0x10 	&Instructs the linker to prefer definitions in the loaded library over exported definitions in the loading application when resolving external symbols in the loaded library. This option is only supported on Linux.
命令链接器在解析加载过的库中的外部符号时,优先使用加载了的库中的定义,
  而不是在应用程序加载中的定义。【译者注:翻译存疑,故保留原文参考,详
  情参考globc--dlopen()--RTLD\_DEEPBIND】\\
\hline
\end{tabular}
%%% Local Variables:
%%% mode: latex
%%% TeX-master: "../../master"
%%% End:
\chapter{QLibrary}

Qlibrary用于运行时加载库。

\begin{tabular}{|l|l|}
\hline
属性 &	内容\\
\hline
头文件:& 	\#include <QLibrary>\\
\hline
qmake:& 	QT += core\\
\hline
继承于:& 	QObject\\
\hline
\end{tabular}

\begin{notice}
此类中全部函数可重入。
\end{notice}

\section{公共成员类型}

\begin{tabular}{|l|l|}
\hline
类型 &	名称\\
\hline
enum &	LoadHint { ResolveAllSymbolsHint, ExportExternalSymbolsHint,
       LoadArchiveMemberHint, PreventUnloadHint, DeepBindHint }\\
\hline
flags &	LoadHints \\ 
\hline
\end{tabular}

\section{属性}

\begin{itemize}
\item fileName : QString
\item loadHints : LoadHints
\end{itemize}

\section{公共成员函数}

\begin{longtable}{|l|l|}
\hline
 类型& 	函数名\\
\hline
	&QLibrary(const QString \&fileName, const QString \&version,
   QObject *parent = nullptr)\\
\hline
	&QLibrary(const QString \&fileName, int verNum, QObject *parent =
   nullptr)\\
\hline
	&QLibrary(const QString \&fileName, QObject *parent = nullptr)\\
\hline
	&QLibrary(QObject *parent = nullptr)\\
\hline
virtual& 	$\sim$QLibrary()\\
\hline
QString& 	errorString() const\\
\hline
QString& 	fileName() const\\
\hline
bool& 	isLoaded() const\\
\hline
bool& 	load()\\
\hline
QLibrary::LoadHints& 	loadHints() const\\
\hline
QFunctionPointer& 	resolve(const char *symbol)\\
\hline
void& 	setFileName(const QString \&fileName)\\
\hline
void& 	setFileNameAndVersion(const QString \&fileName, int versionNumber)\\
\hline
void& 	setFileNameAndVersion(const QString \&fileName, const QString
      \&version)\\
\hline
void& 	setLoadHints(QLibrary::LoadHints hints)\\
\hline
bool& 	unload()\\
\hline
\end{longtable}

\section{静态公共成员}

\begin{tabular}{|l|l|}
\hline
类型& 	函数名\\
\hline
bool& 	isLibrary(const QString \&fileName)\\
\hline
QFunctionPointer& 	resolve(const QString \&fileName, const char
                  *symbol)\\
\hline
QFunctionPointer& 	resolve(const QString \&fileName, int verNum, const char *symbol)\\
\hline
QFunctionPointer& 	resolve(const QString \&fileName, const QString \&version, const char *symbol)\\
\hline
\end{tabular}

\section{详细描述}

QLibrary的实例用于操作一个动态链接库文件(文中称为库,也就是DLL)。QLibrary提供访问库中函数的一种平台无关方式。您可以在构造时传递库文件名,也可以通过 setFileName() 给对象显式设置。加载库时,QLibrary在所有系统指定的位置搜索 (例如: Unix上的 LD\_LIBRARY\_PATH), 除非文件名是绝对路径。

如果文件路径是绝对路径,则会首先尝试在这个位置加载。如果找不到,QLibrary尝试不同系统相关的前后缀的文件名,比如Unix系的前缀“lib”,后缀“.so”,Mac及IOS的后缀".dylib",Windows的后缀".dll"。

如果文件路径不是绝对路径,Qlibrary改变搜索顺序,首先尝试系统特定的前后缀,之后是特定文件路径。

这让使用除去前后缀的库基本名称来指定库文件变得可能。因此代码可以在不同操作系统里执行,但不用太多代码尝试各种文件名称。

最重要的函数是 load() 用于动态加载库,isLoaded() 用于检查是否加载成功,以及 resolve() 来解析库中的符号。如果库还没加载,resolve() 函数隐式地加载这个库。多个QLibrary实例访问同一个物理库文件是可行的。一旦被加载,库在内存中一直保留到程序结束。您可以通过 unload() 尝试卸载一个库,但如果有其他QLibrary实例在使用同一个库文件,调用会失败。只有在每一个实例都调用过 unload() 后,库才会真正卸载。

Qlibrary 的一种典型用法是解析库中的导出符号,并调用其对应的C语言函数。这叫做显式链接,对应于隐式链接。隐式链接是构建中的链接可执行文件和静态库的步骤。

下面的代码片段加载了个库,解析"mysymbol"符号,并在一切就绪的情况下调用
这个函数。如果出现了问题, 例如库文件不存在或者符号未定义,函数指针将
会是nullptr,且不会调用。

\begin{lstlisting}[language=C++]

QLibrary myLib("mylib");
typedef void (*MyPrototype)();
MyPrototype myFunction = (MyPrototype) myLib.resolve("mysymbol");
if (myFunction)
    myFunction();

\end{lstlisting}

%%% Local Variables:
%%% mode: latex
%%% TeX-master: "../../master"
%%% End:
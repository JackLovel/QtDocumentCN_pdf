\chapter{QX11Info}

提供有关X11相关的相关配置信息(就是linux下的x11相关的配置信息


\begin{tabular}{|r|l|}
	\hline
	属性     & 方法                   \\
	\hline
	头文件    & \#include <QX11Info> \\
	\hline
	qmake  & QT += x11extras      \\
	\hline
	Since: & Qt5.1                \\
	\hline
\end{tabular}


\section{简述}

\begin{tabular}{|r|l|}
	\hline
	类型                   & 函数名                                          \\
	\hline
	int                  & appDpiX(int screen = -1)                     \\
	\hline
	int                  & appDpiY(int screen = -1)                     \\
	\hline
	unsigned long        & appRootWindow(int screen = -1)               \\
	\hline
	int                  & appScreen()                                  \\
	\hline
	unsigned long        & appTime()                                    \\
	\hline
	unsigned long        & appUserTime()                                \\
	\hline
	xcb\_connection\_t * & connection()                                 \\
	\hline
	Display *            & display()                                    \\
	\hline
	unsigned long        & getTimestamp()                               \\
	\hline
	bool                 & isCompositingManagerRunning(int screen = -1) \\
	\hline
	bool                 & isPlatformX11()                              \\
	\hline
	QByteArray           & nextStartupId()                              \\
	\hline
	void                 & setAppTime(unsigned long time)               \\
	\hline
	void                 & setAppUserTime(unsigned long time)           \\
	\hline
	void                 & setNextStartupId(const QByteArray \&id)      \\
	\hline
\end{tabular}

\section{详细说明}

该类提供了关于 x window相关的显式配置信息

该类提供了两类API:一种是提供特定的widget或者特定的pixmap相关的非静态函数,一种是为应用程序提供默认信息的静态函数。(这个分类简直了!!!)

\section{成员函数}

\begin{itemSpace}
int QX11Info::appDpiX(int screen = -1) static

返回指定屏幕的水平分辨率。

参数screen是指哪个x屏幕(比如两个的话,第一个就是0,第二个就是1)。请注意,如果用户使用的系统是指Xinerama(而不是传统的x11多屏幕),则只有一个x屏幕。请使用QDesktopWidget来查询有关于Xinerama屏幕的信息。

\begin{seeAlso}
	apDipY();
\end{seeAlso}
\end{itemSpace}

\begin{itemSpace}
int QX11Info::appDpiY(int screen = -1) static

返回指定屏幕的垂直分辨率。

参数screen是指哪个x屏幕(比如两个的话,第一个就是0,第二个就是1)。

请注意,如果用户使用的系统是指Xinerama(而不是传统的x11多屏幕),则只有一个x屏幕。

请使用QDesktopWidget来查询有关于Xinerama屏幕的信息。

\begin{seeAlso}
	apDipX();
\end{seeAlso}
\end{itemSpace}


\begin{itemSpace}
	unsigned long QX11Info::appRootWindow(int \emph{screen} = -1) static

	返回指定屏幕应用程序窗口的句柄

	参数screen是指哪个x屏幕(比如两个的话,第一个就是0,第二个就是1)。

	请注意,如果用户使用的系统是指Xinerama(而不是传统的x11多屏幕),则只有一个x屏幕。请使用QDesktopWidget来查询有关于Xinerama屏幕的信息。
\end{itemSpace}


\begin{itemSpace}
	int QX11Info::appScreen() static

	返回应用程序正在显示的屏幕编号。 此方法是指每个原始的X11屏幕使用不同的DISPLAY环境变量。只有当您的应用程序需要知道它在哪个X屏幕上运行时,这个信息才有用。 在典型的多个物理机连接到一个X11屏幕中时。意味着这个方法对于每台物理机来讲都是相同的编号。在这样的设置中,如果您对X11的RandR拓展程序感兴趣,可以通过QDesktopWidget和QScreen获得。
\end{itemSpace}

\begin{itemSpace}
	unsigned long QX11Info::appTime() static

	返回X11的时间
\end{itemSpace}

\begin{itemSpace}
	unsigned long QX11Info::appUserTime() static

	返回X11的用户时间
\end{itemSpace}

\begin{itemSpace}
	xcb\_connection\_t *QX11Info::connection() static

	返回应用程序默认的XCB信息。
\end{itemSpace}

\begin{itemSpace}
	Display *QX11Info::display() static

	返回应用程序默认的显式屏幕
\end{itemSpace}

\begin{itemSpace}
	unsigned long QX11Info::getTimestamp() static

	从X服务器上获取当前X11的时间戳。 此方法创建一个事件来阻塞住X11服务器,直到它从X服务器接受回来。 这个函数是从Qt5.2中引入的。
\end{itemSpace}

\begin{itemSpace}
	bool QX11Info::isCompositingManagerRunning(int \emph{screen} = -1) static

	如果屏幕的合成管理器在运行时,则返回 true (ps,合成管理器运行会有一些特殊的效果,比如一些透明色的绘制,可以用这个函数判断下。),否则则返回 false。 这个函数是从Qt5.7中引入的。
\end{itemSpace}

\begin{itemSpace}
	bool QX11Info::isPlatformX11() static

	如果应用程序运行在X11上则返回true。 这个函数是从Qt5.2开始引入的。

\end{itemSpace}
\begin{itemSpace}
	QByteArray QX11Info::nextStartupId()

	返回此进程显式的下一个窗口的启动ID。 显式下一个窗口后,下一个启动ID则为空。

	(Qt官网很少给这种链接啊)
	http://standards.freedesktop.org/startup-notification-spec/startup-notification-latest.txt

	这个函数在Qt5.4引入。
\end{itemSpace}

\begin{itemSpace}
	void QX11Info::setAppTime(unsigned long \emph{time}) static

	将X11时间设置成指定的值。
\end{itemSpace}

\begin{itemSpace}
	void QX11Info::setAppUserTime(unsigned long \emph{time}) static

	设置X11用户的时间
\end{itemSpace}

\begin{itemSpace}
	void QX11Info::setNextStartupId(const QByteArray \emph{\&id}) static

	设置下一个启动程序的ID。 第一个窗口的启动ID来自环境变量DESKTOP\_STARTUP\_ID。当请求来自另一个进程(比如通过QDus)时,此方法对于后续窗口很有用。

	这个函数是从Qt5.4中引用的。
\end{itemSpace}
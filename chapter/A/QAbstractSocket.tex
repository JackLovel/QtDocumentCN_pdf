\chapter{QAbstractSocket}

QAbstractSocket 类是Qt中 Socket 通信类的基类,被 QTcpSocket 和
QUdpSocket等类继承。QAbstractSocket 类为所有的socket通信类提供了最基本
的功能。

\begin{tabular}{|r|l|}
	\hline
	属性 & 方法 \\
  \hline
	头文件 & \#include<QAbstractSocket>\\      
	\hline
	qmake & QT+=network\\      
	\hline
	父类 & QIODevice\\
	\hline
	子类 & QTcpSocket、QUdpSocket \\
	\hline
\end{tabular}

公共成员类型 

\begin{tabular}{|m{5em}|m{35em}|}
\hline
类型&方法 \\ 
\hline
enum&	BindFlag { ShareAddress, DontShareAddress, ReuseAddressHint,
      DefaultForPlatform }\\
\hline
flags&	BindMode\\
\hline
enum&	NetworkLayerProtocol{ IPv4Protocol, IPv6Protocol, AnyIPProtocol, UnknownNetworkLayerProtocol }\\
\hline
enum&	PauseMode { PauseNever, PauseOnSslErrors }\\
\hline
flags&	PauseModes\\
\hline
enum&	SocketError { ConnectionRefusedError, RemoteHostClosedError, HostNotFoundError, SocketAccessError, SocketResourceError, …, UnknownSocketError }\\
\hline
enum&	SocketOption{ LowDelayOption, KeepAliveOption, MulticastTtlOption, MulticastLoopbackOption, TypeOfServiceOption, …, PathMtuSocketOption }\\
\hline
enum&	SocketState { UnconnectedState, HostLookupState, ConnectingState, ConnectedState, BoundState, …, ListeningState }\\
\hline
enum&	SocketType { TcpSocket, UdpSocket, SctpSocket, UnknownSocketType }\\
  \hline
\end{tabular}

公共成员函数

\begin{longtable}{|m{15em}|m{27em}|}
\hline
类型&函数名 \\
\hline
	&QAbstractSocket(QAbstractSocket::SocketType socketType, QObject
   *parent)\\
\hline
virtual&	$\sim$QAbstractSocket()\\
\hline
void&	abort()\\
\hline
bool&	bind(const QHostAddress \&address, quint16 port = 0,
QAbstractSocket::BindMode mode = DefaultForPlatform)\\
\hline
bool&	bind(quint16 port = 0, QAbstractSocket::BindMode mode = DefaultForPlatform)\\
\hline
virtual void&	connectToHost(const QString \&hostName, quint16 port,QIODevice::OpenMode
 openMode = ReadWrite, QAbstractSocket::NetworkLayerProtocol protocol = AnyIPProtocol)\\
\hline
virtual void &connectToHost(const QHostAddress \&address, quint16 port, QIODevice::Ope
nMode openMode = ReadWrite)\\
\hline
virtual void	&disconnectFromHost()\\
\hline
QAbstractSocket::SocketError & error() const\\
\hline
bool&	flush()\\
\hline
bool&	isValid() const\\
\hline
QHostAddress&	localAddress() const\\
\hline
quint16&	localPort() const\\
\hline
QAbstractSocket::PauseModes	& pauseMode() const\\
\hline
QHostAddress&	peerAddress() const\\
\hline
QString&	peerName() const\\
\hline
quint16&	peerPort() const\\
\hline
QString&	protocolTag() const\\
\hline
QNetworkProxy&	proxy() const\\
\hline
qint64	&readBufferSize() const\\
\hline
virtual void&	resume()\\
\hline
void&	setPauseMode(QAbstractSocket::PauseModes pauseMode)\\
\hline
void&	setProtocolTag(const QString \&tag)\\
\hline
void&	setProxy(const QNetworkProxy \&networkProxy)\\
\hline
virtual void&	setReadBufferSize(qint64 size)\\
\hline
virtual bool&	setSocketDescriptor(qintptr socketDescriptor, QAbstractSocket::SocketStat
e socketState = ConnectedState, QIODevice::OpenMode openMode = ReadWrite)\\
\hline
virtual void	&setSocketOption(QAbstractSocket::SocketOption option,
const QVariant \&value)\\
\hline
virtual qintptr&	socketDescriptor() const\\
\hline
virtual QVariant&	socketOption(QAbstractSocket::SocketOption option)\\
\hline
QAbstractSocket::SocketType	&socketType() const\\
\hline
QAbstractSocket::SocketState&	state() const\\
\hline
virtual bool&	waitForConnected(int msecs = 30000)\\
\hline
virtual bool&	waitForDisconnected(int msecs = 30000)\\
\hline
\end{longtable}

重载公共成员函数

\begin{tabular}{|l|l|}
\hline
类型& 函数名\\ 
\hline
virtual bool&atEnd() const override\\
\hline
virtual qint64&	bytesAvailable() const override\\
\hline
virtual qint64&		bytesToWrite() const override\\
\hline
virtual bool&	canReadLine() const override\\
\hline
virtual void&	close() override\\
\hline
virtual bool&	isSequential() const override\\
\hline
virtual bool&	waitForBytesWritten(int msecs = 30000) override\\
\hline
virtual bool&	waitForReadyRead(int msecs = 30000) override\\
\hline
\end{tabular}

信号

\begin{tabular}{|m{5em}|m{35em}|}
\hline
类型&函数名 \\
\hline
void&	connected()\\
\hline
void&	disconnected()\\
\hline
void&	errorOccurred(QAbstractSocket::SocketError socketError)\\
\hline
void&	hostFound()\\
\hline
void&	proxyAuthenticationRequired(const QNetworkProxy \&proxy,
                                                          QAuthenticator
                                                          *authenticator)\\
\hline
void&	stateChanged(QAbstractSocket::SocketState socketState)\\
\hline
\end{tabular}

保护成员函数

\begin{tabular}{|m{5em}|m{35em}|}
\hline
类型&函数名 \\
\hline
void&	setLocalAddress(const QHostAddress \&address)\\
\hline
void&	setLocalPort(quint16 port)\\
\hline
void&	setPeerAddress(const QHostAddress \&address)\\
\hline
void&	setPeerName(const QString \&name)\\
\hline
void&	setPeerPort(quint16 port)\\
\hline
void&	setSocketError(QAbstractSocket::SocketError socketError)\\
\hline
void&	setSocketState(QAbstractSocket::SocketState state)\\
\hline
\end{tabular}

重载保护成员函数

\begin{tabular}{|m{5em}|m{35em}|}
\hline
类型&函数名 \\
\hline
virtual qint64&	readData(char *data, qint64 maxSize) override\\
\hline
virtual qint64&	readLineData(char *data, qint64 maxlen) override\\
\hline
virtual qint64&	writeData(const char *data, qint64 size) override\\
\hline
\end{tabular}

详细介绍

QAbstractSocket 类是 QTcpSocket 类和 QUdpSocket 类的基类,包含了这两个
类所有的常规功能。您可以通过以下两种方法使用一个套接字( Socket ):

\begin{itemize}
\item 实例化一个 QTcpSocket 或者 QUdpSocket 对象
\item 声明一个自定义套接字描述符,实例化 QAbstractSocket ,然后调用 setSocketDescriptor() 函数包装该自定义套接字描述符。
\end{itemize}

TCP(传输控制协议)是一种可靠的,面向流,面向连接的传输协议。 UDP(用户数据报协议)是一种不可靠的,面向数据报的无连接协议。 实际上,这意味着TCP更适合于连续数据传输,而当可靠性不重要时,可以使用更轻量的UDP。

QAbstractSocket 的 API 统一了这两种协议之间的大部分差异。 例如,尽管 UDP 是无连接的,但 connectToHost() 为 UDP 套接字建立了虚拟连接,使您可以忽略底层协议,以几乎相同的方式使用 QAbstractSocket 类。 在 QAbstractSocket 类的内部实现中,QAbstractSocket 记录了传递给 connectToHost() 的地址和端口,并且能在调用 read() 和 write() 之类的成员函数时使用这些值。

任何情况下,QAbstractSocket 类都有一个状态( state ,该值可以由 state() 成员函数的返回值获得)。 初始状态为未连接( QAbstractSocket :: UnconnectedState )状态。 调用 connectToHost() 成员函数连接主机后,套接字会首先进入寻找主机( QAbstractSocket :: HostLookupState )状态。 如果找到了主机,则 QAbstractSocket 会进入连接中( QAbstractSocket :: ConnectingState )状态,并发送 hostFound() 信号。 建立连接后,它将进入已连接( QAbstractSocket :: ConnectedState )状态并发送 connected() 信号。 如果在以上列出任何阶段发生了错误,则会发出 errorOccurred() 信号。 每当状态发生更改时,QAbstractSocket 都会发出 stateChanged() 信号。 为方便起见,当套接字已准备好进行读取和写入数据操作时,isValid() 成员函数的返回值为 true 。但是要注意一下,在进行读写操作之前,套接字的状态必须为已连接( QAbstractSocket :: ConnectedState )状态。

您可以通过调用 read() 或 write() 来进行数据读写操作,同时为了方便进行特殊的数据读入操作,您还可以使用 QAbstractSocket 的父类 QIODevice 提供的成员函数 readLine() 和 readAll() 。当我们需要以字节为单位进行数据读写操作时,可以使用 QAbstractSocket 的父类 QIODevice 提供的成员函数 getChar(),putChar() 和 ungetChar() 。 待数据写入套接字后,QAbstractSocket 会发出继承自父类 QIODevice 的信号 bytesWritten() 。请特别注意一下,Qt并不限制写缓冲区的大小。 您可以通过监听 bytesWritten() 信号来监视其大小。

每当有新的数据块到达时,QAbstractSocket 都会发出继承自父类 QIODevice 的信号 readyRead() 。 您可以通过 bytesAvailable() 成员函数的返回值来获得当前读取缓冲区中可读取的字节数。 通常来讲,您可以将 readyRead() 信号与一个槽函数相连接,然后在该槽函数中读取所有可用的数据。 如果您不一次性读取所有数据,则其余数据以后仍可以读取,并且任何新的传入数据都将追加到 QAbstractSocket 的内部读取缓冲区中。您可以通过调用 setReadBufferSize() 成员函数来限制读取缓冲区的大小。

您可以通过调用 disconnectFromHost() 成员函数关闭套接字。 调用 disconnectFromHost() 成员函数后,QAbstractSocket 会进入关闭中( QAbstractSocket :: ClosingState )状态。 待所有未处理数据写入套接字后,QAbstractSocket 将关闭套接字,进入未连接( QAbstractSocket :: UnconnectedState )状态,并发送 disconnected() 信号。 如果要立即中止连接,并丢弃所有未处理数据,请调用 abort() 成员函数。 如果远程主机关闭了该连接,QAbstractSocket 将发出 errorOccurred(QAbstractSocket :: RemoteHostClosedError) 错误信号,在此期间套接字状态仍为已连接( QAbstractSocket :: ConnectedState )状态,此后将发送 disconnected() 信号。

您可以通过调用 peerPort() 和 peerAddress() 成员函数来获取已连接的对等方的端口和地址。 peerName() 成员函数则会返回传递给 connectToHost() 的对等方的主机名。 另外,localPort() 和 localAddress() 成员函数可返回本地套接字的端口和地址。

QAbstractSocket 提供了一组函数,这些函数可以挂起调用线程,直到发出某些
信号为止。 这些函数可用于实现阻塞套接字:

\begin{itemize}
\item waitForConnected() 阻塞套接字直到一个新的连接建立
\item waitForReadyRead() 阻塞套接字直到有新的数据可以读取
\item waitForBytesWritten() 阻塞套接字直到一个有效的荷载数据写入到了套接字
\item waitForDisconnected() 阻塞套接字直到连接已经关闭
\end{itemize}

Qt官方提供了如下示例:

\begin{lstlisting}[language=C++]
int numRead = 0, numReadTotal = 0;
char buffer[50];

forever {
	numRead  = socket.read(buffer, 50);

	// do whatever with array

	numReadTotal += numRead;
	if (numRead == 0 && !socket.waitForReadyRead())
		break;
}

\end{lstlisting}

waitForReadyRead() 成员函数返回值为 false,则说明连接已关闭或发生了错误。

使用阻塞套接字进行编程与使用非阻塞套接字进行编程完全不同。 阻塞套接字不需要有一个事件循环,这通常可以简化代码。 但是,在GUI应用程序中,阻塞套接字只能在非GUI线程中使用,以避免冻结用户界面。 有关这两种方法的概述,请参见 fortuneclien 和 blockingfortuneclient 示例。

注意: Qt官方并不推荐将阻塞函数与信号一起使用。

QAbstractSocket 可以与 QTextStream 和 QDataStream 的流运算符(operator<<() 和operator>>())一起使用。 但是,有一个问题需要注意:在尝试使用operator>>() 读取数据之前,必须确保有足够的数据可用。

您也可以在 QNetworkAccessManager 类和 QTcpServer 类的文档中找到一部分相关的信息。

\splitLine

成员类型文档

enum QAbstractSocket::BindFlag | flags QAbstractSocket::BindMode
该枚举描述了一些不同的标志,这些标志可以传递为 bind() 成员函数的参数,指定了不同的主机绑定模式。

\begin{tabular}{|m{10em}|m{5em}|m{30em}|}
\hline
常量&值&描述 \\
\hline
QAbstractSocket::ShareAddress&	0x1	&允许其他服务绑定到相同的地址和端口。
                               在多个进程通过侦听相同的地址和端口来分
                               担单个服务的负载的情况下(例如,具有多
                               个预分支侦听器的Web服务器可以大大缩短响
                               应时间),该模式十分有效。 但是,由于该
                               模式下允许任何服务对主机进行重新绑定,
                               因此此选项存在着安全隐患。 请注意,您还
                               可以允许您的服务重新绑定现有的共享地址
                               通过将此选项与
                               QAbstractSocket::ReuseAddressHint 结合
                               使用。 在 Unix 上,这等效于
                               SO\_REUSEADDR 套接字选项。 在 Windows 上,
                               这是默认行为,因此将忽略此选项。\\
\hline
QAbstractSocket::DontShareAddress&	0x2&	绑定主机时独占地址和端口,不允
                                   许其他服务重新绑定。 将此选项作为
                                   QAbstractSocket :: bind() 参数,可
                                   以确保在绑定主机成功后,您当前的服
                                   务是唯一侦听指定地址和端口的服务。
                                   即使其他服务通过指定
                                   QAbstractSocket::ReuseAddressHint
                                   模式来绑定服务,这个操作也是不允许
                                   的。 该模式相比
                                   QAbstractSocket::ShareAddress 提供
                                   了更高的安全性,但是在某些操作系统
                                   上,它要求您以管理员权限运行该服务。
                                   在 Unix 和 macOS 上,独占地址和端口
                                   是绑定主机的默认行为,因此该选项将
                                   被忽略。 在 Windows 上,此选项使用
                                   SO\_EXCLUSIVEADDRUSE 套接字选项。\\
\hline
QAbstractSocket::ReuseAddressHint&	0x4	&示意 QAbstractSocket 即使地址
                                   和端口已被另一个套接字绑定,它也应
                                   尝试重新绑定服务。 在 Windows 和
                                   Unix 上,这等效于 SO\_REUSEADDR 套接
                                   字选项。\\
\hline
QAbstractSocket::DefaultForPlatform&	0x0&	使用当前平台的默认模式。 在 Unix 和 macOS 上,这等效于( QAbstractSocket::DontShareAddress + QAbstractSocket::ReuseAddressHint ),在 Windows 上,它等效于 QAbstractSocket::ShareAddress 。\\
\hline
\end{tabular}

该枚举最初在Qt5.0版本引入。

BindMode 类型是 typedef QFlags <BindFlag> 生成的用户自定义类型。 它存储着 BindFlag 值的 OR 组合。

enum QAbstractSocket::NetworkLayerProtocol

该枚举描述了Qt中可以使用的网络层协议

\begin{tabular}{|m{26em}|m{2em}|m{8em}|}
\hline
常量&值&描述 \\
\hline
QAbstractSocket::IPv4Protocol&	0	&IPv4\\
\hline
QAbstractSocket::IPv6Protocol&	1	&IPv6\\
\hline
QAbstractSocket::AnyIPProtocol&	2	&IPv4 或 IPv6\\
\hline
QAbstractSocket::UnknownNetworkLayerProtocol&	-1	&除IPv4和IPv6之外的协议\\
\hline
\end{tabular}

您也可以在 QHostAddress::protocol() 中找到有关网络层协议的相关知识。

enum QAbstractSocket::PauseMode | flags QAbstractSocket::PauseModes

该枚举描述了套接字在什么情况下该停止传输中的数据。 

当前Qt支持的唯一通知是 QSslSocket::sslErrors()。

\begin{tabular}{|m{19em}|m{2em}|m{18em}|}
\hline
常量	&值&	描述\\
\hline
QAbstractSocket::PauseNever&	0x0&	“永远”不暂停套接字上的数据传输。 该模
  式是默认设置,符合Qt 4的标准。\\
\hline
QAbstractSocket::PauseOnSslErrors&	0x1&	收到 SSL 错误(即 QSslSocket::sslErrors() )的通知后,暂停套接字上的数据传输。\\
\hline
\end{tabular}

该枚举最初在Qt5.0版本引入。

PauseModes 类型是 typedef QFlags<PauseMode> 生成的用户自定义类型。它储存着 PauseMode 值的 OR 组合。

enum QAbstractSocket::SocketError

该枚举描述了常见的套接字错误。

\begin{longtable}{|m{20em}|m{2em}|m{19em}|}
\hline
常量&	值&	描述 \\ 
\hline
QAbstractSocket::ConnectionRefusedError&	0&	连接被对方拒绝(或连接超时)。\\
\hline
QAbstractSocket::RemoteHostClosedError&	1&	远程主机关闭了连接。 请注意,在发送远程
主机连接关闭的通知后,客户端套接字(即此套接字)将被关闭。\\
\hline
QAbstractSocket::HostNotFoundError&	2&	未能找到主机地址。\\
\hline
QAbstractSocket::SocketAccessError&	3&	程序没有权限来进行特定的套接字操作\\
\hline
QAbstractSocket::SocketResourceError&	4&	本地系统资源不足(例如本地系统使用了过多的套接字)。\\
\hline
QAbstractSocket::SocketTimeoutError&	5&	套接字操作超时。\\
\hline
QAbstractSocket::DatagramTooLargeError&	6&	数据报大于操作系统的限制(该限制可以低至
8192字节)。\\
\hline
QAbstractSocket::NetworkError&	7&	网络发生错误(例如网线断开)。\\
\hline
QAbstractSocket::AddressInUseError&	8&	QAbstractSocket::bind() 指定的主机地址已
被另一个套接字使用独占模式绑定。\\
\hline
QAbstractSocket::SocketAddressNotAvailableError&	9& QAbstractSocket::bind() 指定的主机地址不属于主机。\\
\hline
QAbstractSocket::UnsupportedSocketOperationError&	10&	本地操作系统不支持请求的套
接字操作(例如,缺少IPv6支持)。\\
\hline
QAbstractSocket::ProxyAuthenticationRequiredError&	12&	套接字正在使用代理,并且代
理需要身份验证。\\
\hline
QAbstractSocket::SslHandshakeFailedError&	13&	SSL / TLS 握手失败,因此连接已关
闭(仅在 QSslSocket 中使用)。\\
\hline
QAbstractSocket::UnfinishedSocketOperationError&	11&	仅由QAbstractSocketEngine 使用,上一次尝试的操作尚未完成(仍在后台进行)。\\
\hline
QAbstractSocket::ProxyConnectionRefusedError&	14&	代理服务器拒绝了连接。\\
\hline
QAbstractSocket::ProxyConnectionClosedError&	15&	与代理服务器的连接意外关闭(在
建立与对方的最终连接之前)。\\
\hline
QAbstractSocket::ProxyConnectionTimeoutError&	16&	与代理服务器的连接超时或代理服
务器在身份验证阶段停止响应。\\
\hline
QAbstractSocket::ProxyNotFoundError&	17&	找不到使用 setProxy() (或应用程序代理
)设置的代理地址。\\
\hline
QAbstractSocket::ProxyProtocolError&	18&	与代理服务器通信失败,因为无法理解来自代
理服务器的响应,这通常是代理服务协议错误的锅。\\
\hline
QAbstractSocket::OperationError&	19&	套接字所处的状态不允许该操作\\
\hline
QAbstractSocket::SslInternalError&	20&	使用的 SSL 库出现内部错误。 这可能是由于 SS
L 库安装错误或配置错误造成的。\\
\hline
QAbstractSocket::SslInvalidUserDataError&	21&	提供了无效的数据(证书,密钥,密码
等),其使用导致 SSL 库出现错误。\\
\hline
QAbstractSocket::TemporaryError&	22&	发生临时错误(例如,操作将阻塞套接字,而套接
字未阻塞)。\\
\hline
QAbstractSocket::UnknownSocketError&	-1&	神了,出现了一个未知错误。\\
\hline
\end{longtable}

您也可以在 QAbstractSocket::error() 和 QAbstractSocket::errorOccurred() 成员函数的详细介绍中找到一部分套接字错误的介绍。

enum QAbstractSocket::SocketOption

该枚举表示可以在套接字上设置的选项。 如果需要设置这些选项,您可以在套接字接收到 connectd() 信号之后,或者在 QTcpServer 接收到新的套接字之后,对它们进行设置。

%%% Local Variables:
%%% mode: latex
%%% TeX-master: "../../master"
%%% End:

\chapter{QAbstractItemModel}

QAbstractItemModel类为项模型类提供了抽象接口。\href{https://github.com/JackLovel/QtDocumentCN/blob/master/Src/A/QAbstractItemModel}{更多...} 

\begin{tabular}{|r|l|}
	\hline
	属性 & 方法 \\
	\hline
	头文件 & \#include <QAbstractItemModel>\\      
	\hline
	qmake & QT+=core\\      
	\hline
	自从 & Qt 4.6\\
	\hline
	继承&QObject \\
	\hline
	派生 & QAbstractListModel、QAbstractProxyModel、QAbstractTableModel、QConcatenateTablesProxyModel、QDirModel、QFileSystemModel 和 QStandardItemModel \\
	\hline
\end{tabular}

\splitLine

公有成员类型

\begin{tabular}{|r|l|}
	\hline
	类型 & 类型名称 \\
	\hline
enum class&	CheckIndexOption { NoOption, IndexIsValid, DoNotUseParent,
            ParentIsInvalid }\\
\hline
flags&	CheckIndexOptions\\
\hline
enum&	LayoutChangeHint { NoLayoutChangeHint, VerticalSortHint, HorizontalSortHint }\\
\hline
\end{tabular}

公共成员函数

%gog 

\begin{tabular}{|r|l|}
	\hline
	类型 & 函数名称 \\
	\hline
	QAbstractItemModel(QObject *parent = nullptr)
virtual	~QAbstractItemModel()
virtual QModelIndex	buddy(const QModelIndex \&index) const
virtual bool	canDropMimeData(const QMimeData *data, Qt::DropAction action, int row, int column, const QModelIndex \&parent) const s
virtual bool	canFetchMore(const QModelIndex \&parent) const
bool	checkIndex(const QModelIndex \&index, QAbstractItemModel::CheckIndexOptions options = CheckIndexOption::NoOption) const
virtual int	columnCount(const QModelIndex \&parent = QModelIndex()) const = 0
virtual QVariant	data(const QModelIndex \&index, int role = Qt::DisplayRole) const = 0
virtual bool	dropMimeData(const QMimeData *data, Qt::DropAction action, int row, int column, const QModelIndex &parent)
virtual void	fetchMore(const QModelIndex &parent)
virtual Qt::ItemFlags	flags(const QModelIndex &index) const
virtual bool	hasChildren QModelIndex &parent = QModelIndex()) const
bool	hasIndex row, int column, const QModelIndex &parent = QModelIndex()) const
virtual QVariant	headerData section, Qt::Orientation orientation, int role = Qt::DisplayRole) const
virtual QModelIndex	index row, int column, const QModelIndex &parent = QModelIndex()) const = 0
bool	insertColumn column, const QModelIndex &parent = QModelIndex())
virtual bool	insertColumns column, int count, const QModelIndex &parent = QModelIndex())
bool	insertRow(int row, const QModelIndex &parent = QModelIndex())
virtual bool	insertRows row, int count, const QModelIndex &parent = QModelIndex())
virtual QMap<int, QVariant>	itemData QModelIndex &index) const
virtual QModelIndexList	match QModelIndex &start, int role, const QVariant &value, int hits = 1, Qt::MatchFlags flags = Qt::MatchFlags (Qt::MatchStartsWith
virtual QMimeData *	mimeData QModelIndexList &indexes) const
virtual QStringList	mimeTypes const
bool	moveColumn QModelIndex &sourceParent, int sourceColumn, const QModelIndex &destinationParent, int destinationChild)
virtual bool	moveColumns QModelIndex &sourceParent, int sourceColumn, int count, const QModelIndex &destinationParent, int destinationChild)
bool	moveRow QModelIndex &sourceParent, int sourceRow, const QModelIndex &destinationParent, int destinationChild)
virtual bool	moveRows QModelIndex &sourceParent, int sourceRow, int count, const QModelIndex &destinationParent, int destinationChild)
virtual QModelIndex	parent QModelIndex \&index) const = 0
bool	removeColumn column, const QModelIndex \&parent = QModelIndex())
virtual bool	removeColumns column, int count, const QModelIndex \&parent = QModelIndex())
bool	removeRow row, const QModelIndex \&parent = QModelIndex())
virtual bool	removeRows row, int count, const QModelIndex \&parent = QModelIndex())
virtual QHash<int, QByteArray>	roleNames const
virtual int	rowCount QModelIndex \&parent = QModelIndex()) const = 0
virtual bool	setData QModelIndex \&index, const QVariant \&value, int role = Qt::EditRole)
virtual bool	setHeaderData section, Qt::Orientation orientation, const QVariant \&value, int role = Qt::EditRole)
virtual bool	setItemData QModelIndex \&index, const QMap<int, QVariant> \&roles)
virtual QModelIndex	sibling row, int column, const QModelIndex \&index) const
virtual void	sort column, Qt::SortOrder order = Qt::AscendingOrder)
virtual QSize	span QModelIndex \&index) const
virtual Qt::DropActions	supportedDragActions const
virtual Qt::DropActions	supportedDropActions const

\hline
\end{tabular}

%%% Local Variables:
%%% mode: latex
%%% TeX-master: "../../master"
%%% End:

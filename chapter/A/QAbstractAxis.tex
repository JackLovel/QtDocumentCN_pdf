\chapter{QAbstractAxis}

QAbstractAxis类是用于专门处理坐标轴的类

\begin{tabular}{|r|l|}
	\hline
	属性 & 方法 \\
	\hline
	头文件 & \#include <QAbstractAxis>\\      
	\hline
	实例化 & AbstractAxis\\      
	\hline
	继承&QObject \\
	\hline
	派生 & QBarCategoryAxis, QDateTimeAxis, QLogValueAxis, and QValueAxis \\
	\hline
\end{tabular}

简述

公共类型

\begin{tabular}{|r|l|}
\hline
类型 & 方法 \\
\hline
enum&	AxisType { AxisTypeNoAxis, AxisTypeValue, AxisTypeBarCategory,
      AxisTypeCategory, AxisTypeDateTime, AxisTypeLogValue }\\
\hline
flags&	AxisTypes\\
\hline
\end{tabular}

属性

\begin{tabular}{|r|l|}
\hline
函数名 & 类型 \\
\hline
alignment :&	const Qt::Alignment\\
\hline
color :&	QColor\\
\hline
gridLineColor :&	QColor\\
\hline
gridLinePen :&	QPen\\
\hline
gridVisible :&	bool\\
\hline
labelsAngle :&	int\\
\hline
labelsBrush :&	QBrush\\
\hline
labelsColor :&	QColor\\
\hline
labelsFont :&	QFont\\
\hline
labelsVisible :&	bool\\
\hline
linePen :&	QPen\\
\hline
lineVisible :&	bool\\
\hline
minorGridLineColor :&	QColor\\
\hline
minorGridLinePen :&	QPen\\
\hline
minorGridVisible :&	bool\\
\hline
orientation :&	const Qt::Orientation\\
\hline
reverse :&	bool\\
\hline
shadesBorderColor :&	QColor\\
\hline
shadesBrush :&	QBrush\\
\hline
shadesColor :&	QColor\\
\hline
shadesPen :&	QPen\\
\hline
shadesVisible :&	bool\\
\hline
titleBrush :&	QBrush\\
\hline
titleFont :&	QFont\\
\hline
titleText :&	QString\\
\hline
titleVisible :&	bool\\
\hline
visible :&	bool\\
\hline
\end{tabular}

公共函数


\begin{longtable}{|r|l|}
\hline
类型  & 函数名\\
\hline
& $\sim$QAbstractAxis() \\
\hline
Qt::Alignment&	alignment() const \\
\hline
QColor&	gridLineColor()\\
\hline
QPen&	gridLinePen() const\\
\hline
void&	hide()\\
\hline
bool&	isGridLineVisible() const\\
\hline
bool&	isLineVisible() const\\
\hline
bool&	isMinorGridLineVisible() const\\
\hline
bool&	isReverse() const\\
\hline
bool&	isTitleVisible() const\\
\hline
bool&	isVisible() const\\
\hline
int&	labelsAngle() const\\
\hline
QBrush&	labelsBrush() const\\
\hline
QColor&	labelsColor() const\\
\hline
QFont&	labelsFont() const\\
\hline
bool&	labelsVisible() const\\
\hline
QPen&	linePen() const\\
\hline
QColor&	linePenColor() const\\
\hline
QColor&	minorGridLineColor()\\
\hline
QPen&	minorGridLinePen() const\\
\hline
Qt::Orientation &orientation() const\\
\hline
void&	setGridLineColor(const QColor \&color)\\
\hline
void&	setGridLinePen(const QPen \&pen)\\
\hline
void&	setGridLineVisible(bool visible = true)\\
\hline
void&	setLabelsAngle(int angle)\\
\hline
void&	setLabelsBrush(const QBrush \&brush)\\
\hline
void&	setLabelsColor(QColor color)\\
\hline
void&	setLabelsFont(const QFont \&font)\\
\hline
void	&setLabelsVisible(bool visible = true)\\
\hline
void	&setLinePen(const QPen \&pen)\\
\hline
void	&setLinePenColor(QColor color)\\
\hline
void	&setLineVisible(bool visible = true)\\
\hline
void	&setMax(const QVariant \&max)\\
\hline
void	&setMin(const QVariant \&min)\\
\hline
void	&setMinorGridLineColor(const QColor \&color)\\
\hline
void	&setMinorGridLinePen(const QPen \&pen)\\
\hline
void	&setMinorGridLineVisible(bool visible = true)\\
\hline
void	&setRange(const QVariant \&min, const QVariant \&max)\\
\hline
void	&setReverse(bool reverse = true)\\
\hline
void	&setShadesBorderColor(QColor color)\\
\hline
void	&setShadesBrush(const QBrush \&brush)\\
\hline
void	&setShadesColor(QColor color)\\
\hline
void	&setShadesPen(const QPen \&pen)\\
\hline
void&	setShadesVisible(bool visible = true)\\
\hline
void&	setTitleBrush(const QBrush \&brush)\\
\hline
void&	setTitleFont(const QFont \&font)\\
\hline
void&	setTitleText(const QString \&title)\\
\hline
void&	setTitleVisible(bool visible = true)\\
\hline
void&	setVisible(bool visible = true)\\
\hline
QColor&	shadesBorderColor() const\\
\hline
QBrush&	shadesBrush() const\\
\hline
QColor&	shadesColor() const\\
\hline
QPen&	shadesPen() const\\
\hline
bool&	shadesVisible() const\\
\hline
void&	show()\\
\hline
QBrush&	titleBrush() const\\
\hline
QFont&	titleFont() const\\
\hline
QString&	titleText() const\\
\hline
virtual AxisType &type() const = 0\\
\hline
\end{longtable}

信号

\begin{tabular}{|r|l|}
\hline 
类型 & 函数名\\
\hline
void&	colorChanged(QColor color)\\
\hline
void&	gridLineColorChanged(const QColor \&color)\\
\hline
void&	gridLinePenChanged(const QPen \&pen)\\
\hline
void&	gridVisibleChanged(bool visible)\\
\hline
void&	labelsAngleChanged(int angle)\\
\hline
void&	labelsBrushChanged(const QBrush \&brush)\\
\hline
void&	labelsColorChanged(QColor color)\\
\hline
void&	labelsFontChanged(const QFont \&font)\\
\hline
void&	labelsVisibleChanged(bool visible)\\
\hline
void&	linePenChanged(const QPen \&pen)\\
\hline
void&	lineVisibleChanged(bool visible)\\
\hline
void&	minorGridLineColorChanged(const QColor \&color)\\
\hline
void&	minorGridLinePenChanged(const QPen \&pen)\\
\hline
void&	minorGridVisibleChanged(bool visible)\\
\hline
void&	reverseChanged(bool reverse)\\
\hline
void&	shadesBorderColorChanged(QColor color)\\
\hline
void&	shadesBrushChanged(const QBrush \&brush)\\
\hline
void&	shadesColorChanged(QColor color)\\
\hline
void&	shadesPenChanged(const QPen \&pen)\\
\hline
void&	shadesVisibleChanged(bool visible)\\
\hline
void&	titleBrushChanged(const QBrush \&brush)\\
\hline
void&	titleFontChanged(const QFont \&font)\\
\hline
void&	titleTextChanged(const QString \&text)\\
\hline
void&	titleVisibleChanged(bool visible)\\
\hline
void&	visibleChanged(bool visible)\\
\hline
\end{tabular}

详细说明

QAbstractAxis是专门用于处理坐标轴的基类。

每一个连续的序列可以绑定到一个或者多个水平轴和垂直轴,但是不同域的混合轴的类型是不支持的。比如在同一个方向指定QValueAxis和QLogValueAxis。

每个轴的元素(比如轴线,标题,标签,网格线,阴影,可见性)都是可以控制的。

成员变量
enum QAbstractAxis::AxisType flags QAbstractAxis::AxisTypes

这个枚举类型指定了轴对象的类型

\begin{tabular}{|r|l|}
\hline
变量&值\\
\hline
QAbstractAxis::AxisTypeNoAxis&	0x0\\
\hline
QAbstractAxis::AxisTypeValue&	0x1\\
\hline
QAbstractAxis::AxisTypeBarCategory&	0x2\\
\hline
QAbstractAxis::AxisTypeCategory&	0x4\\
\hline
QAbstractAxis::AxisTypeDateTime&	0x8\\
\hline
QAbstractAxis::AxisTypeLogValue&	0x10\\
\hline
\end{tabular}

AxisTypes是QFlags<AxisType>的typedef。它是AxisType类型的组合。 (也就是个宏呗)

alignment : const Qt::Alignment 该属性是轴的对齐属性 其值可以为
Qt::AlignLeft, Qt::AlignRight, Qt::AlignBottom, or Qt::AlignTop. 相关
函数

\begin{tabular}{|r|l|}
\hline
类型&函数名\\
\hline
Qt::Alignment&	alignment() const\\
\hline
\end{tabular}

\splitLine

color : QColor 该属性是指坐标轴与刻度的颜色

相关函数

\begin{tabular}{|r|l|}
\hline
类型 & 函数名\\
\hline
QColor&	linePenColor() const\\
\hline
void&	setLinePenColor(QColor color)\\
\hline
\end{tabular}

通知信号

\begin{tabular}{|r|l|}
\hline
类型&函数名\\ 
\hline
void	&colorChanged(QColor color)\\
\hline
\end{tabular}

\splitLine

gridLinePen : QPen 该属性是指绘制网格线的笔

相关函数

\begin{tabular}{|r|l|}
\hline
类型&函数名\\ 
\hline
QPen&	gridLinePen() const\\
\hline
void&	setGridLinePen(const QPen \&pen)\\
\hline
\end{tabular}

通知信号

\begin{tabular}{|r|l|}
\hline
类型&函数名\\ 
\hline
void	&gridLinePenChanged(const QPen \&pen)\\
\hline
\end{tabular}

\splitLine

labelsAngle : int 该属性以度数保存轴坐标的角度

相关函数

\begin{tabular}{|r|l|}
\hline
类型&函数名\\ 
\hline
int	& 	labelsAngle() const\\
\hline
void & setLabelsAngle(int angle)\\
\hline
\end{tabular}


通知信号

\begin{tabular}{|r|l|}
\hline
类型&函数名\\ 
\hline
void & labelsAngleChanged(int angle)\\
\hline
\end{tabular}

\splitLine

labelsBrush : QBrush 该属性表示用于绘制标签的画笔 只有画刷的颜色是相关的(这句话其实我不太理解Only the color of the brush is relevant.)

相关函数

\begin{tabular}{|r|l|}
\hline
类型&函数名\\ 
\hline
QBrush	&labelsBrush() const\\
\hline
void & setLabelsBrush(const QBrush \&brush)\\
\hline
\end{tabular}

通知信号

\begin{tabular}{|r|l|}
\hline
类型&函数名\\ 
\hline
void &	labelsBrushChanged(const QBrush \&brush)\\
\hline
\end{tabular}

\splitLine

labelsColor : QColor 该属性表示轴标签的颜色

相关函数

\begin{tabular}{|r|l|}
\hline
类型&函数名\\ 
\hline
QColor	&labelsColor() const\\
\hline
void &setLabelsColor(QColor color)\\
\hline
\end{tabular}

通知信号

\begin{tabular}{|r|l|}
\hline
类型&函数名\\ 
\hline
void &	labelsColorChanged(QColor color)\\
\hline
\end{tabular}

\splitLine

labelsFont : QFont 该属性表示轴标签的字体信息

相关函数

\begin{tabular}{|r|l|}
\hline
类型&函数名\\ 
\hline
QFont	&labelsFont() const\\
\hline
void	&setLabelsFont(const QFont \&font)\\
\hline
\end{tabular}

通知信号

\begin{tabular}{|r|l|}
\hline
类型&函数名\\ 
\hline
void	&labelsFontChanged(const QFont \&font)\\
\hline
\end{tabular}

\splitLine

labelsVisible : bool 该属性表示轴标签是否可见

相关函数

\begin{tabular}{|r|l|}
\hline
类型&函数名\\ 
\hline
bool&	labelsVisible() const\\
\hline
void	&setLabelsVisible(bool visible = true)\\
\hline
\end{tabular}

通知信号

\begin{tabular}{|r|l|}
\hline
类型&函数名\\ 
\hline
void	&labelsVisibleChanged(bool visible)\\
\hline
\end{tabular}

\splitLine

linePen : QPen 该属性表示绘制轴线的笔相关

相关函数

\begin{tabular}{|r|l|}
\hline
类型&函数名\\ 
\hline
QPen	&linePen() const\\
\hline
void	& setLinePen(const QPen \&pen)\\
\hline
\end{tabular}

通知信号

\begin{tabular}{|r|l|}
\hline
类型&函数名\\ 
\hline
void &	linePenChanged(const QPen \&pen)\\
\hline
\end{tabular}

\splitLine

lineVisible : bool 该属性表示轴线是否可见

相关函数

\begin{tabular}{|r|l|}
\hline
类型&函数名\\ 
\hline
bool&	isLineVisible() const\\
\hline
void &	setLineVisible(bool visible = true)\\
\hline
\end{tabular}

通知信号

\begin{tabular}{|r|l|}
\hline
类型&函数名\\ 
\hline
void &	lineVisibleChanged(bool visible)\\
\hline
\end{tabular}

\splitLine

minorGridLineColor : QColor 该属性表示副格线的颜色 仅适用于支持副网格线的轴

相关函数

\begin{tabular}{|r|l|}
\hline
类型&函数名\\ 
\hline
QColor	&minorGridLineColor()\\
\hline
void &	setMinorGridLineColor(const QColor \&color)\\
\hline
\end{tabular}

通知信号

\begin{tabular}{|r|l|}
\hline
类型&函数名\\ 
\hline
void &	minorGridLineColorChanged(const QColor \&color)\\
\hline
\end{tabular}

\splitLine

minorGridVisible : bool 该属性表示副格线是否可见 仅适用于支持副网格线的轴

相关函数

\begin{tabular}{|r|l|}
\hline
类型&函数名\\ 
\hline
bool&	isMinorGridLineVisible() const\\
\hline
void & setMinorGridLineVisible(bool visible = true)\\
\hline
\end{tabular}

通知信号

\begin{tabular}{|r|l|}
\hline
类型&函数名\\ 
\hline
void& minorGridVisibleChanged(bool visible)\\
\hline
\end{tabular}

\splitLine

orientation : const Qt::Orientation 该属性表示坐标轴的方向。 当坐标轴被添加到图表时,该属性为Qt::Horizontal或者Qt::Vertical

相关函数

\begin{tabular}{|r|l|}
\hline
类型&函数名\\ 
\hline
Qt::Orientation	&orientation() const\\
\hline
\end{tabular}

\splitLine

reverse : bool 该属性表示是否使用反转轴。 该值默认为false。 
反转轴由直线,样条,散列图系列以及笛卡尔图表组成的区域支持。
如果一个方向相反,或者行为为定义,则所有相同方向的所有轴必须保持一致。

相关函数

\begin{tabular}{|r|l|}
\hline
类型&函数名\\ 
\hline
bool&	isReverse() const\\
\hline
void&	setReverse(bool reverse = true)\\
\hline
\end{tabular}

通知信号

\begin{tabular}{|r|l|}
\hline
类型&函数名\\ 
\hline
void&	reverseChanged(bool reverse)\\
\hline
\end{tabular}

%\hrulefill %实线
\splitLine % 虚线
 
shadesBorderColor : QColor 该属性表示坐标轴阴影的边框(笔)颜色

相关函数

\begin{tabular}{|r|l|}
\hline
类型&函数名\\ 
\hline
QColor	&shadesBorderColor() const\\
\hline
void&	setShadesBorderColor(QColor color)\\
\hline
\end{tabular}

通知信号

\begin{tabular}{|r|l|}
\hline
类型&函数名\\ 
\hline
void& shadesBorderColorChanged(QColor color)\\
\hline
\end{tabular}

\splitLine

% wcc 

shadesBrush : QBrush 该属性表示用于绘制轴阴影的画笔(网格线之间的区域)

相关函数

\begin{tabular}{|r|l|}
\hline
类型&函数名\\ 
\hline
QPen&	shadesPen() const\\
\hline
void& setShadesPen(const QPen \&pen)\\
\hline
\end{tabular}

通知信号

\begin{tabular}{|r|l|}
\hline
类型&函数名\\ 
\hline
void& 	shadesPenChanged(const QPen \&pen)\\
\hline
\end{tabular}

\splitLine

shadesVisible : bool 该属性表示轴阴影是否可见

相关函数

\begin{tabular}{|r|l|}
\hline
类型&函数名\\ 
\hline
bool& 	shadesVisible() const\\
\hline
void & setShadesVisible(bool visible = true)\\
\hline
\end{tabular}

通知信号

\begin{tabular}{|r|l|}
\hline
类型&函数名\\ 
\hline
void& 	shadesVisibleChanged(bool visible)\\
\hline
\end{tabular}

\splitLine 

titleBrush : QBrush 该属性表示用于绘制坐标轴标题文本的画笔。 只影响画刷的颜色。

相关函数

\begin{tabular}{|r|l|}
\hline
类型&函数名\\ 
\hline
QBrush	&titleBrush() const\\
\hline
void & setTitleBrush(const QBrush \&brush)\\
\hline
\end{tabular}

通知信号

\begin{tabular}{|r|l|}
\hline
类型&函数名\\ 
\hline
void & titleBrushChanged(const QBrush \&brush)\\
\hline
\end{tabular}

\splitLine 

titleFont : QFont 该属性表示坐标轴标题的字体。

相关函数

\begin{tabular}{|r|l|}
\hline
类型&函数名\\ 
\hline
QFont&	titleFont() const\\
\hline
void & setTitleFont(const QFont \&font)\\
\hline
\end{tabular}

通知信号

\begin{tabular}{|r|l|}
\hline
类型&函数名\\ 
\hline
void & titleFontChanged(const QFont \&font)\\
\hline
\end{tabular}

\splitLine

titleText : QString 该属性表示坐标轴的标题 默认为空,坐标轴的标题支持HTML的格式。

相关函数

\begin{tabular}{|r|l|}
\hline
类型&函数名\\ 
\hline
QString & titleText() const\\ 
\hline
void & setTitleText(const QString \&title)\\
\hline
\end{tabular}

通知信号

\begin{tabular}{|r|l|}
\hline
类型&函数名\\ 
\hline
void & titleTextChanged(const QString \&text)\\
\hline
\end{tabular}

\splitLine

titleVisible : bool 该属性表示坐标轴的可见性。 默认值为true。

相关函数

\begin{tabular}{|r|l|}
\hline
类型&函数名\\ 
\hline
bool & isTitleVisible() const\\
\hline
void & setTitleVisible(bool visible = true)\\
\hline
\end{tabular}

通知信号

\begin{tabular}{|r|l|}
\hline
类型&函数名\\ 
\hline
void & titleVisibleChanged(bool visible)\\
\hline
\end{tabular}

\splitLine

visible : bool 该属性表示坐标轴的可见性。

相关函数

\begin{tabular}{|r|l|}
\hline
类型&函数名\\ 
\hline
bool & isVisible() const\\
\hline
void & 	setVisible(bool visible = true)\\
\hline
\end{tabular}

通知信号

\begin{tabular}{|r|l|}
\hline
类型&函数名\\ 
\hline
bool & isVisible() const\\
\hline
void & 	visibleChanged(bool visible)\\
\hline
\end{tabular}

\splitLine

成员函数

QAbstractAxis::$\sim$QAbstractAxis() 

析构函数,销毁轴对象,当一个坐标轴添加到图表时,该图表获得该坐标轴的所有权。

void QAbstractAxis::colorChanged(QColor color) [信号] 

当坐标轴的颜色变化时,该信号被发射。 注意:属性color的通知信号。

void QAbstractAxis::gridLineColorChanged(const QColor \&color) [信号] 

当绘制网格线的笔的颜色改变时,该信号被发射。 注意:属性gridLineColor的通知信号。

QPen QAbstractAxis::gridLinePen() const

 返回用于绘制网格的笔。 注意:属性gridLinePen的Getter函数。 另参见函数setGridLinePen()。

void QAbstractAxis :: gridLinePenChanged(const QPen \&pen) [信号] 

当用于绘制网格线的笔变化时,会发出此信号。 注意:属性gridLinePen的通知信号。

void QAbstractAxis::gridVisibleChanged(bool visible) [信号] 

当坐标轴的网格线的可见性变化时,发出该信号。 注意:属性gridVisible的通知信号。

void QAbstractAxis::hide() 使坐标轴,阴影,标签,网格线不可见。

void QAbstractAxis::labelsAngleChanged(int angle) [信号]

 当坐标轴标签的角度变化时,发出该信号。 注意:属性标签角度的通知信号

QBrush QAbstractAxis::labelsBrush() const 

返回用于绘制标签的画笔。 注意:属性labelsBrush的Getter函数。 另参见setLabelsBrush().

void QAbstractAxis::labelsBrushChanged(const QBrush \&brush) [信号] 

当用于绘制坐标轴标签的画笔改变时,会发出此信号。 属性Brush的通知信号。

void QAbstractAxis :: labelsColorChanged(QColor color) [信号signal] 

当坐标轴标签的颜色改变时,会发出此信号。 属性labelsColor的通知信号。

QFont QAbstractAxis::labelsFont() const 

返回用于绘制标签的字体。 注意:属性labelsFont的Getter函数。 另参见函数setLabelsFont()。

void QAbstractAxis :: labelsFontChanged(const QFont \&font)[信号] 

当坐标轴的字体改变时,会发出此信号。 属性labelsFont的通知信号。

void QAbstractAxis::labelsVisibleChanged(bool visible) [信号] 

当坐标轴标签的可见性变化时,会发出此信号。 属性labelsVisible的通知信号。

QPen QAbstractAxis::linePen() const 

返回用于绘制轴线与刻度线的笔。 注意:属性linePen的Getter函数。 另参阅函数setLinePen()。

void QAbstractAxis::linePenChanged(const QPen \&pen) [信号] 

当绘制坐标轴的笔变化时,会发出此信号。 注意:属性linePen的通知信号。

void QAbstractAxis::lineVisibleChanged(bool visible) [信号] 

当坐标轴线的可见性变化时,会发出此信号。 注意:属性lineVisible的通知信号。

void QAbstractAxis::minorGridLineColorChanged(const QColor \&color)
[信号]

 当绘制副格线的笔的颜色变化时,该信号被发射。 注意:属性minorGridLineColor的通知信号。

void QAbstractAxis::minorGridLinePenChanged(const QPen \&pen) [信号] 

当绘制副格线的笔变化时,该信号被发射。 注意:属性minorGridLinePen的通知信号。

void QAbstractAxis::minorGridVisibleChanged(bool visible) [信号] 

当绘制副格线的可见性变化时,该信号被发射。 注意:属性minorGridVisible的通知信号。

Qt::Orientation QAbstractAxis::orientation() const 

返回坐标轴的方向(垂直或者水平) 注意:坐标轴方向的Getter函数。

void QAbstractAxis::setGridLinePen(const QPen \&pen) 

设置绘制网格线的笔。 注意:gridLinePen的Setter函数。 另参见函数gridLinePen()。

void QAbstractAxis::setLabelsBrush(const QBrush \&brush)

 设置用于绘制标签的画笔。 注意:属性LabelsBrush的Setter函数。 另参阅labelsBrush().

void QAbstractAxis::setLabelsFont(const QFont \&font) 

设置用于绘制标签的字体相关 注意:属性labelsFont的Setter函数。 另参阅函数labelsFont()。

void QAbstractAxis::setLinePen(const QPen \&pen) 

设置用于绘制坐标轴线和刻度线的笔。 注意:属性linePen的Setter函数。 另参见函数linePen()。

void QAbstractAxis::setLineVisible(bool visible = true) 

设置坐标轴线与刻度线是否可见。 注意:属性lineVisible的Setter函数 另参见函数isLineVisible()。

void QAbstractAxis::setMax(const QVariant \&max) 

设置坐标轴上显示的最大值。根据当前坐标轴的类型,最大值参数会被转换为适当的值。如果转化失败,该函数设置无效。

void QAbstractAxis::setMin(const QVariant \&min) 

设置坐标轴上显示的最小值。根据当前坐标轴的类型,最小值参数会被转换为适当的值。如果转化失败,该函数设置无效。

void QAbstractAxis::setRange(const QVariant \&min, const QVariant
\&max) 

设置坐标轴的范围。根据当前坐标轴的类型,最大值最小值会被转换为适当的值。如果转化失败,该函数设置无效。

void QAbstractAxis::setShadesBrush(const QBrush \&brush)

 设置用于绘制阴影的画笔。 注意:属性shadesBrush的Setter函数。 另参见函数shadesBrush().

void QAbstractAxis::setShadesPen(const QPen \&pen)

 设置用于绘制阴影的笔。 注意:属性shadesPen的Setter函数。 另参见函数shadesPen().

void QAbstractAxis::setTitleBrush(const QBrush \&brush) 

设置用于绘制标题的画笔。 注意:属性titleBrush的Setter函数。 另参见函数titleBrush().

void QAbstractAxis::setTitleFont(const QFont \&font) 

设置用于绘制标题的笔。 注意:属性titleFont的Setter函数。 另参见函数titleFont().

void QAbstractAxis::setVisible(bool visible = true) 

设置坐标轴,阴影,标签,网格线是否可见。 注意:属性visible的Setter函数。 另参见函数isVisible().

void QAbstractAxis::shadesBorderColorChanged(QColor color) [信号
signal] 

当绘制坐标轴边框的颜色变化时,会发出此信号。 注意:属性shadesBorderColor的通知信号。

QBrush QAbstractAxis::shadesBrush() const 

返回用于绘制阴影的画笔。 注意:属性shadesBrush()的Getter函数。 另参阅函数setShadesBrush().

void QAbstractAxis::shadesBrushChanged(const QBrush \&brush) [信号signal] 

当绘制坐标轴阴影的画刷变化时,会发出此信号。 注意:属性shadesBrush()的通知信号。

void QAbstractAxis::shadesColorChanged(QColor color) [信号signal] 

当绘制坐标轴阴影颜色发生变化时,会发出此信号。 注意:属性shadesColor()的通知信号。

QPen QAbstractAxis::shadesPen() const 

返回用于绘制阴影的笔。 注意:属性shadesPen()的Getter函数 另参阅函数setShadesPen().

void QAbstractAxis::shadesPenChanged(const QPen \&pen) [信号signal] 

当绘制坐标轴阴影的笔发生变化时,会发出此信号。 注意:属性shadesPen()的通知信号。

void QAbstractAxis::shadesVisibleChanged(bool visible) [信号signal] 

当坐标轴的阴影可见性变化时,会发出此信号。 注意:属性shadesVisible()的通知信号。

void QAbstractAxis::show() 

使坐标轴,阴影,标签,网格线可见。

QBrush QAbstractAxis::titleBrush() const 

返回用于绘制标题的画刷 注意:属性titleBrush的Getter函数 另参阅函数setTitleBrush()

void QAbstractAxis::titleBrushChanged(const QBrush \&brush) [信号
signal] 

当绘制坐标轴标题的画刷变化时,该信号被发射 注意:titleBrush的通知信号。

QFont QAbstractAxis::titleFont() const 

返回绘制标题的笔相关属性。 注意:titleFont的Getter函数 另参阅函数setTitleFont()

void QAbstractAxis::titleFontChanged(const QFont \&font) [信号 signal]

当坐标轴标题的字体属性更改时,会发出此信号。 注意:titleFont的通知信号。

void QAbstractAxis::titleTextChanged(const QString \&text) [信号signal] 

当坐标轴标题的文本内容改变时,会发出此信号。 注意:titleText的通知信号。

void QAbstractAxis::titleVisibleChanged(bool visible) [信号 signal] 

当坐标轴的标题文本的可见性变化时,会发出此信号。 注意:titleVisible的通知信号

AxisType QAbstractAxis::type() const 

返回坐标轴的类型

void QAbstractAxis::visibleChanged(bool visible) 

当坐标轴的可见性变化时,会发出此信号。 注意:坐标轴的visible的通知信号

%%% Local Variables:
%%% mode: latex
%%% TeX-master: "../../master"
%%% End:

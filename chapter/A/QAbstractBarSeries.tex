\chapter{QAbstractBarSeries}

QAbstractBarSeries是所有柱状图/条形图系列的基类

\begin{tabular}{|r|m{25em}|}
	\hline
	属性 & 方法 \\
	\hline
	头文件 & \#include<QAbstractBarSeries>\\      
	\hline
	实例化 & AbstractBarSeries\\      
	\hline
	继承&QAbstractSeries \\
	\hline
	派生 & QBarSeries, QHorizontalBarSeries,QHorizontalPercentBarSeries,

 QHorizontalStackedBarSeries, QPercentBarSeries, and QStackedBarSeries \\
	\hline
\end{tabular}

\splitLine

\section{简述}

\subsection{公共类型}

\begin{tabular}{|r|m{20em}|}
\hline
类型 & 方法 \\
\hline
enum&	LabelsPosition \{ LabelsCenter, LabelsInsideEnd, LabelsInsideBase, LabelsOutsideEnd \}\\
\hline
\end{tabular}

\subsection{属性}

\begin{tabular}{|r|l|}
\hline
函数名 & 类型 \\
\hline
barWidth :	&qreal\\
\hline
count :	&const int\\
\hline
labelsAngle :&qreal\\
\hline
5个属性继承自QAbstractSeries &	\\
\hline
1个属性继承自QObject	&\\
\hline
\end{tabular}

\splitLine

Public Functions

\begin{tabular}{|r|l|}
\hline
函数名 & 类型 \\
\hline
virtual&	$\sim$QAbstractBarSeries()\\
\hline
bool&	append(QBarSet *set)\\
\hline
bool&	append(QList<QBarSet *> sets)\\
\hline
QList<QBarSet *>&	barSets() const\\
\hline
qreal&	barWidth() const\\
\hline
void&	clear()\\
\hline
int&	count() const\\
\hline
bool&	insert(int index, QBarSet *set)\\
\hline
bool&	isLabelsVisible() const\\
\hline
qreal&	labelsAngle() const\\
\hline
QString&	labelsFormat() const\\
\hline
QAbstractBarSeries::LabelsPosition&	labelsPosition() const\\
\hline
bool&	remove(QBarSet *set)\\
\hline
void&	setBarWidth(qreal width)\\
\hline
void&	setLabelsAngle(qreal angle)\\
\hline
void&setLabelsFormat(const QString \&format)\\
\hline
void&	setLabelsPosition(QAbstractBarSeries::LabelsPosition position)\\
\hline
void&	setLabelsVisible(bool visible = true)\\
\hline
bool&	take(QBarSet *set)\\
\hline
15个公共函数继承自QAbstractSeries&	\\
\hline
32个公共函数继承自QObject&\\
\hline
\end{tabular}

\splitLine

\subsection{信号}

\begin{tabular}{|r|l|}
\hline
函数名 & 类型 \\
\hline
void&	barsetsAdded(QList<QBarSet *> sets)\\
\hline
void&	barsetsRemoved(QList<QBarSet *> sets)\\
\hline
void&	clicked(int index, QBarSet *barset)\\
\hline
void&	countChanged()\\
\hline
void&	doubleClicked(int index, QBarSet *barset)\\
\hline
void&	hovered(bool status, int index, QBarSet *barset)\\
\hline
void&	labelsAngleChanged(qreal angle)\\
\hline
void&	labelsFormatChanged(const QString \&format)\\
\hline
void&	labelsPositionChanged(QAbstractBarSeries::LabelsPosition position)\\
\hline
void&	labelsVisibleChanged()\\
\hline
void&	pressed(int index, QBarSet *barset)\\
\hline
void&	released(int index, QBarSet *barset)\\
\hline
\end{tabular}

\splitLine

额外继承的
1个公共槽继承自QObject 11个静态成员函数继承自QObject 9个保护函数继承自QObject

\splitLine

\subsection{详细说明}

QAbstractBarSeries类是所有条形柱的抽象类。

在条形图中,条形柱被定义为包含一种数据的集合。条形柱的位置由其类别与数值来决定。条形柱组合则是属于同一类别的条形柱。条形柱的显示则是由创建图表的时候决定的。

如果使用QValueAxis来代替QBarCategoryAxis当做图表的主轴。那么条形柱别按照索引值来分类。

可以参考Qt Example(example 这里我还没有来得及翻译)

\splitLine

\subsection{成员类型}

enum QAbstractBarSeries::LabelsPosition**

这个枚举值表示的是条形柱标签的位置:

\begin{tabular}{|r|l|c|}
\hline
枚举值 & 数值 & 描述\\
\hline
QAbstractBarSeries::LabelsCenter&	0&	中部\\
\hline
QAbstractBarSeries::LabelsInsideEnd&	1&	顶部\\
\hline
QAbstractBarSeries::LabelsInsideBase&	2&	底部\\
\hline
QAbstractBarSeries::LabelsOutsideEnd&	3&	外部\\
\hline
\end{tabular}


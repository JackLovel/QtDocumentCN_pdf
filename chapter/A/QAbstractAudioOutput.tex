\chapter{QAbstractAudioOutput}

QAbstractAudioOutput类是音频后端的基类

\begin{tabular}{|r|l|}
	\hline
	属性 & 方法 \\
	\hline
	头文件 & \#include <QAbstractAudioOutput>\\      
	\hline
	qmake &QT += multimedia\\      
	\hline
	继承&QObject \\
	\hline
\end{tabular}

\section{公有函数}

\begin{tabular}{|r|l|}
\hline
类型&方法\\
\hline
virtual int	&bufferSize() const = 0\\
\hline
virtual int	&bufferSize() const = 0\\
\hline
virtual int	&bytesFree() const = 0\\
\hline
virtual QString	&category() const\\
\hline
virtual qint64	&elapsedUSecs() const = 0\\
\hline
virtual QAudio::Error	&error() const = 0\\
\hline
virtual QAudioFormat	&format() const = 0\\
\hline
virtual int	& notifyInterval() const = 0\\
\hline
virtual int	& periodSize() const = 0\\
\hline
virtual qint64	&processedUSecs() const = 0\\
\hline
virtual void	&reset() = 0\\
\hline
virtual void	&resume() = 0\\
\hline
virtual void&	setBufferSize(int value) = 0\\
\hline
virtual void&	setCategory(const QString \&)\\
\hline
virtual void&	setFormat(const QAudioFormat \&fmt) = 0\\
\hline
virtual void&	setNotifyInterval(int ms) = 0\\
\hline
virtual void&	setVolume(qreal volume)\\
\hline
virtual void&	start(QIODevice *device) = 0\\
\hline
virtual QIODevice *	&start() = 0\\
\hline
virtual QAudio::State	&state() const = 0\\
\hline
virtual void	&stop() = 0\\
\hline
virtual void	&suspend() = 0 \\
\hline
virtual qreal	&volume() const\\
\hline
\end{tabular}

\section{信号}

\begin{tabular}{|r|l|}
\hline
类型&函数名\\
\hline
void&	errorChanged(QAudio::Error error)\\
\hline
void&	notify()\\
\hline
void&	stateChanged(QAudio::State state)\\
\hline
\end{tabular}

\section{详细描述}

QAbstractAudioOutput类是音频后端的基类。 QAbstractAudioOutput类是
QAudioOutput类的实现类。QAudioOutput的实现实际上是调用的
QAbstractAudioOutput类,有关实现相关的功能,请参考QAudioOutput()类中的
函数说明。

\section{成员函数}

int QAbstractAudioOutput::bufferSize() const [纯虚函数] 

以字节为单位,返回音频缓冲区的大小。

\begin{seeAlso}
setBufferSize()。
\end{seeAlso}

int QAbstractAudioOutput::bytesFree () const [纯虚函数] 

返回音频缓冲区的可用空间(以字节为单位)

QString QAbstractAudioOutput::category() const [虚函数 virtual] 

音频缓冲区的类别(官方文档没有,这是我个人经验,当然可能有误,望指正) 

\begin{seeAlso}
setCategory()。
\end{seeAlso}

qint64 QAbstractAudioOutput::elapsedUSecs() const [纯虚函数 pure virtual] 

返回调用start()函数之后的毫秒数,包括处于空闲状态的时间和挂起状态的时间。

QAudio::Error QAbstractAudioOutput::error() const [纯虚函数 pure
virtual] 

返回错误状态。

void QAbstractAudioOutput::errorChanged(QAudio::Error error) [信号 signal] 

当错误状态改变时,该信号被发射。

QAudioFormat QAbstractAudioOutput::format() const [纯虚函数 pure virtual]

返回正在使用的QAudioFormat()类 

\begin{seeAlso}
setFormat()。
\end{seeAlso}

void QAbstractAudioOutput::notify() [信号 signal] 

当函数setNotifyInterval(x)函数已经调用,即音频数据的时间间隔已经被设置时。该信号被发射。(就是调用setNotifyInterval(x)后,这个信号会被发射。官方文档讲的好详细啊=。=)

int QAbstractAudioOutput::notifyInterval() const [纯虚函数 pure virtual]

以毫秒为单位,返回时间间隔 

\begin{seeAlso}
setNotifyInterval()。
\end{seeAlso}

int QAbstractAudioOutput::periodSize() const [纯虚函数 pure virtual] 

以字节为单位返回周期大小。

qint64 QAbstractAudioOutput::processedUSecs() const [纯虚函数 pure
virtual] 

返回自调用start()函数后处理的音频数据量(单位为毫秒)

void QAbstractAudioOutput::reset() [纯虚函数 pure virtual] 

将所有音频数据放入缓冲区,并将缓冲区重置为零。

void QAbstractAudioOutput::resume() [纯虚函数 pure virtual] 

继续处理暂停后的音频数据 (也就是暂停后继续的意思呗)

void QAbstractAudioOutput::setBufferSize(int value) [纯虚函数 pure virtual]

重新设置音频缓冲区的大小(以字节为单位 即输入参数value) 

\begin{seeAlso}
bufferSize()()。
\end{seeAlso}

void QAbstractAudioOutput::setCategory(const QString \&) [虚函数 virtual] 

\begin{seeAlso}
category()。
\end{seeAlso}

void QAbstractAudioOutput::setFormat(const QAudioFormat \&fmt) [纯虚函数 pure virtual] 

QAbstractAudioOutput设置QAudioFormat类,只有当QAudio状态为QAudio::StoppedState时,音频格式才会被设置成功。

\begin{seeAlso}
format()。
\end{seeAlso}

void QAbstractAudioOutput::setNotifyInterval(int \emph{ms}) [纯虚函数 pure virtual] 

设置发送notify()信号的时间间隔。这个ms并不是实时处理的音频数据中的ms数。
这个时间间隔是平台相关的。

\begin{seeAlso}
notifyInterval()。
\end{seeAlso}

void QAbstractAudioOutput::setVolume(qreal volume) [虚函数 virtual] 

设置音量。音量的范围为[0.0 - 1.0]。 

\begin{seeAlso}
volume()。
\end{seeAlso}

void QAbstractAudioOutput::start(QIODevice *device) [纯虚函数 pure virtual] 

调用start()函数时,输入参数QIODevice*类型的变量device,用于音频后端处理数据传输。

QIODevice *QAbstractAudioOutput::start() [纯虚函数 pure virtual] 

返回一个指向正在处理数据传输的QIODevice类型的指针,这个指针是可以被写入的,用于处理音频数据。(参考上边的函数是咋写入的)

QAudio::State QAbstractAudioOutput::state() const [纯虚函数 pure
virtual] 

返回音频处理的状态。

void QAbstractAudioOutput::stateChanged(QAudio::State state) [信号
signal] 

当音频状态变化的时候,该信号被发射

void QAbstractAudioOutput::stop() [纯虚函数 pure virtual] 

停止音频输出

void QAbstractAudioOutput::suspend() [纯虚函数 pure virtual] 

停止处理音频数据,保存处理的音频数据。(就是暂停的意思啊=。=)

qreal QAbstractAudioOutput::volume() const [虚函数 virtual] 

返回音量。

音量范围为[0.0 - 1.0] 

\begin{seeAlso}
setVolume()。
\end{seeAlso}
\chapter{API Design Principles}

API 设计规范

\begin{quote}
译者注:

本文不来自于 Qt 文档,而是来自于 Qt Wiki:\href{https://wiki.qt.io/API_Design_Principles}{API\_Design\_Principles}

API(Application Programming Interface),应用开发接口,本文中也将 P 解释为 Programmer(开发者)。	
\end{quote}

Qt 最出名的特点之一是一致性强、易于学习、功能强大的 API。本文尝试对我们在设计 Qt 风格的 API 中积累的诀窍进行总结。其中许多准则都是通用的,其它的则是习惯性用法,我们主要为了保持接口一致性而继续遵循。

尽管这些准则主要面向公共接口,但也鼓励您在设计内部接口时使用相同的技术,这对与您合作开发的同僚会更加友好。

您可能也会有兴趣查阅 Jasmin Blanchette 的 
\href{https://people.mpi-inf.mpg.de/~jblanche/api-design.pdf}{Little Manual of API Design (PDF)},或它的前身,由 Matthias Ettrich 编写的 \href{https://doc.qt.io/archives/qq/qq13-apis.html}{Designing Qt-Style C++ APIs} 。

优秀接口的六大特点

API 是面向开发者的,而 GUI 则是面向终端用户。API 中的 P 代表开发者(Programmer),而非程序(Program),目的是指出 API 由开发者,即人类(而非计算机)所使用这一特点。

Matthias 在 Qt 季刊弟13期,关于 API 设计的文章中,声称他坚信 API 应该是最小化但完备的,具备清晰而简洁的语义,符合直觉,被开发者,易于记忆,能够引导开发者编写高可读性的代码。

最小化

最小化的 API 指包含尽可能少的公共成员和最少的类。这可让 API 更易于理解、记忆、调试和修改。

完备性
完备的 API 意味着具备所有应有的功能。这可能会与最小化产生冲突。此外,若一个成员函数位于错误的类中,则大多数潜在的用户回无法找到它。

清晰简洁的语义

与其它设计协作时,应该让您的设计做到最小例外。通用化会让事情更简单。特例可能存在,但不应成为关注的焦点。在处理特定问题时,不应让解决方案过度泛化。(例如,Qt 3 中的 QMineSourceFactory 本该被称作 QImageLoader,并且具备另一套 API。)

复合直觉

如同计算机上其它内容,API 应复合直觉。不同的开发经验和技术背景会导致对复合直觉与否有不同的感知。复合直觉的 API 应能让中等经验的开发者无需阅读文档并直接使用,并让不知道这个 API 的开发者可以理解使用它编写的代码。

易于记忆

为了让 API 易于记忆,请选择一组保持一致并足够精确的命名约定。使用可理解的模式和概念,并且避免缩写。

可读性导向

编写代码只需要一次,但阅读(以及调试和修改)则会非常频繁。高可读性的代码通常需要花更多时间编写,但可以在产品的生命周期中节约更多的时间。

最后需要谨记,不同类型的用户会使用 API 的不同部分。在单纯地创建一个 Qt 类实例能非常直观的同时,希望用户在尝试继承它之前先阅读文档则是很合理的。

静态多态

相似的代码类应具有相似的 API。可以使用继承来实现——当运行时多态支持时,这是很合理的。但多态同时也可以体现在设计截断。例如,若将代码中的 QProgressBar 换为 QSlider,或将 QString 换为 QByteArray,他们间相似的 API 会另替换操作变得非常容易。这便是为何我们称之为“静态多态”。

静态多态同样可让记忆 API 和开发模式变得更加简单。结果是,对于一组有关联的类,相似的 API 通常比为每个类独立设计的完美 API 更加好用。

在 Qt 中,当不具备有足够说服力的原因时,我们更倾向于使用静态多态,而非继承。这减少了 Qt 的公共类数量,并让新用户更容易在文档中找到需要的内容。

好的

​ QDialogBox 和 QMessageBox 具有相似的 APi,以用于处理按钮(addButton(), setStandardButtons(),但无需继承自某些 "QAbstractButtonBox" 类。

坏的

​ QAbstractSocket 被 QTcpSOcket 和 QUdpSocket 所继承,但这两个类的交互方式差异很大。看起来并没有人使用过(或能够使用) QAbstractSocket 指针来进行通用且有效的操作。

存疑

​ QBoxLayout 是 QHBoxLayout 和 QVBoxLayout 的基类。优点:可以在工具栏中使用 QBoxLayout,调用 setOrientation() 来令其水平/垂直排布。缺点:引入额外的类,用户可能会写出形如 ((QBoxLayout *)hbox)->setOrientation(Qt::Vertical) 的代码,而这是不合理的。

基于属性的 API
比较新的 Qt 类倾向于使用基于属性的 API,例如:

\begin{lstlisting}
QTimer timer;
timer.setInterval(1000);
timer.setSingleShot(true);
timer.start(); 
\end{lstlisting}

此处的属性,指的是作为对象状态一部分的任何概念性的特征——无论是否是实际的 Q\_PROPERTY。只要可行,用户都应该允许以任何顺序设置属性,也就是说,这些属性应该是正交的。例如,上文的代码也可以写为:

\begin{lstlisting}
QTimer timer;
timer.setSingleShot(true);
timer.setInterval(1000);
timer.start(); 
\end{lstlisting}

为了方便,我们也可以这样写:

\begin{lstlisting}
timer.start(1000);
\end{lstlisting}

类似的,对于 QRegExp,我们可以:

\begin{lstlisting}
QRegExp regExp;
regExp.setCaseSensitive(Qt::CaseInsensitive);
regExp.setPattern(".");
regExp.setPatternSyntax(Qt::WildcardSyntax);
\end{lstlisting}

为了实现此类 API,内部的对象需要被惰性构造。例如在 QRegExp 的案例中,不应该在还不知道表达式使用何种语法之前,就在 setPattern() 中过早地编译 . 表达式。

属性通常是级联的,此时我们应该谨慎地处理。仔细考虑下当前样式提供的“默认图标大小”与 QToolButton 的 iconSize 属性:

\begin{lstlisting}
toolButton->iconSize(); // 返回当前样式表的默认大小
toolButton->setStyle(otherStyle);
toolButton->iconSize(); // 返回 otherStyle 的默认大小
toolButton->setIconSize(QSize(52, 52));
toolButton->iconSize(); // 返回 (52, 52)
toolButton->setStyle(yetAnotherStyle);
toolButton->iconSize(); // 返回 (52, 52) 
\end{lstlisting}


注意,一旦 iconSize 被设置,它会被一直留存,此时修改当前样式不会影响它。这是 好的。有时,提供重置属性的渠道会很方便,这有两种实现方式:

\begin{itemize}
	\item 传递一个特定值(如 QSIze()、-1 或 Qt::Alignment(0))来指代“重置”;
	\item 提供显示的 resetFoo() 或 unsetFoo() 函数。
\end{itemize}

对于 iconSize,将 QSize()(即 QSize(-1, -1))设为“重置”便足够了。

某些场景中,取值方法会返回值会与设置的内容不同。例如,若调用 widget->setEnabled(true),可能通过 widget->isEnabled() 获得的依然是 false,因为父控件被禁用了。这并没问题,因为通常这正是我们要检查的状态(父控件被禁用时,子控件也应该变灰,表现为也被禁用,但同时在它内部,应该知道自己实际是“可用”的,并等待父控件可用后恢复状态),但必须在文档中正确地进行描述。

QProperty

\begin{quote}
译者注:该类型为 Qt 6.0 引入,需要参见 Qt 6 类文档。

本文原文中的内容与现有的 Qt 6.0 预览版存在出入,因此暂不翻译本节,待官方进一步维护更新原文。
\end{quote}

C++ 特性

值 与 对象

\begin{quote}
译者注:此条原文无内容,待官方更新
\end{quote}

指针 与 引用

作为输出参数,指针与引用哪个更好?

\begin{lstlisting}
void getHsv(int *h, int *s, int *v) const void getHsv(int &h, int &s, int &v) const
\end{lstlisting}

绝大多数 C++ 数据都推荐尽可能使用引用,因为引用在感知上比指针“更安全更漂亮”。事实上,我们在 Qt 软件中更倾向于使用指针,因为这会令代码更加已读。如对比:

\begin{lstlisting}
color.getHsv(&h, &s, &v);
color.getHsv(h, s, v);
\end{lstlisting}

第一行代码能很清晰地表示,h、s、v 对象有很大概率会被该函数调用所修改。

即便如此,由于编译器并不喜欢输出参数,在新 APi 中应该避免此用法,而是返回一个小结构体:

\begin{lstlisting}
struct Hsv { int hue, saturation, value }; Hsv getHsv() const;
\end{lstlisting}

\begin{quote}
	译者注:对于可能失败的带返回值的函数,Qt 倾向于返回数值,使用 bool* ok = 0 参数来存储调用结果,以便在不关心时忽略之。同样在 Qt 6 以后,该方式不再被建议使用,而是改用 std::optional<T> 返回类型。
\end{quote}

传递常引用 与 传递值
若类型大于16字节,传递常引用。

若类型具有非平凡的拷贝构造函数或非平凡的析构函数,传递常引用来避免执行这些方法。

所有其它类型都应使用值传递。

范例:

\begin{lstlisting}
void setAge(int age);
void setCategory(QChar cat);
void setName(QLatin1String name);
void setAlarm(const QSharedPointer<Alarm> &alarm); // 常引用远快于拷贝构造和析构
// QDate, QTime, QPoint, QPointF, QSize, QSizeF, QRect 都是其它应该值传递的好例子
\end{lstlisting}

虚函数

当 C++ 中的一个成员函数被声明为虚函数,这主要用于通过在子类中重写来自定义该函数的行为。将函数设为虚函数的目的是让对该函数的现有调用会被替代为访问自定义的代码分支。若在该类之外没人调用此函数,则在将其声明为虚函数之前需要小心斟酌:

\begin{lstlisting}
// Qt 3 中的 QTextEdit: 成员函数不需要作为虚函数的成员函数
virtual void resetFormat();
virtual void setUndoDepth( int d );
virtual void setFormat( QTextFormat &f, int flags );
virtual void ensureCursorVisible();
virtual void placeCursor( const QPoint &pos;, QTextCursorc = 0 );
virtual void moveCursor( CursorAction action, bool select );
virtual void doKeyboardAction( KeyboardAction action );
virtual void removeSelectedText( int selNum = 0 );
virtual void removeSelection( int selNum = 0 );
virtual void setCurrentFont( const QFont &f );
virtual void setOverwriteMode( bool b ) { overWrite = b; }
\end{lstlisting}




\chapter{QAbstractAudioDeviceInfo}

QAbstractAudioDeviceInfo 是音频后端的基类。

\begin{tabular}{|r|l|}
	\hline
	属性 & 方法 \\
	\hline
	头文件 & \#include<QAbstractAudioDeviceInfo>\\      
	\hline
	qmake & QT += multimedia\\      
	\hline
	继承&QObject \\
	\hline
\end{tabular}

\section{公共功能}

\resizebox{\textwidth}{!}{ % Latex表格宽度超出文本宽度
\begin{tabular}{|r|l|}
	\hline
	类型 & 方法 \\
	\hline
	virtual QString	 & deviceName() const = 0 \\
	\hline
	virtual bool &isFormatSupported(const QAudioFormat \&format) const = 0 \\
	\hline
	virtual QAudioFormat &	preferredFormat() const = 0\\
	\hline
	virtual QListQAudioFormat::Endian &	supportedByteOrders() = 0 \\
	\hline
	virtual QList &	supportedChannelCounts() = 0 \\
	\hline
	virtual QStringList&	supportedCodecs() = 0\\
	\hline
	virtual QList&	supportedSampleRates() = 0\\
	\hline
	virtual QList&	supportedSampleSizes() = 0\\
	\hline
	virtual QListQAudioFormat::SampleType&	supportedSampleTypes() = 0 \\
	\hline
\end{tabular}
}


\section{类型}

详细说明

QAbstractAudioDeviceInfo是音频后端的基类。

该类实现了QAudioDeviceInfo的音频功能,即QAudioDeviceInfo类中会保留一个QAbstractAudioDeviceInfo,并对其进行调用。关于QAbstractAudioDeviceInfo的实现的其它功能,您可以参考QAudioDeviceInfo的类与函数文档

\section{成员函数文档}

QString QAbstractAudioDeviceInfo::deviceName() const [纯虚函数] 

返回音频设备名称

bool QAbstractAudioDeviceInfo::isFormatSupported(const QAudioFormat \&format) const [纯虚函数] 

传入参数QAudioFormat(音频流)类,如果QAbstractAudioDeviceInfo支持的话,返回true(真是不好翻译)

QAudioFormat QAbstractAudioDeviceInfo::preferredFormat() const [纯虚函数]
 返回QAbstractAudioDeviceInfo更加倾向于使用的音频流。

QListQAudioFormat::Endian QAbstractAudioDeviceInfo::supportedByteOrders() 

[纯虚函数] 返回当前支持可用的字节顺序(QAudioFormat :: Endian)列表

QList QAbstractAudioDeviceInfo::supportedChannelCounts() [纯虚函数] 

返回当前可用的通道(应该是这样翻译)列表

QStringList QAbstractAudioDeviceInfo::supportedCodecs() [纯虚函数] 

返回当前可用编解码器的列表

QList QAbstractAudioDeviceInfo::supportedSampleRates() [纯虚函数] 

返回当前可用的采样率列表。(突然发现Google翻译真心吊啊)

QList QAbstractAudioDeviceInfo::supportedSampleSizes() [纯虚函数] 

返回当前可用的样本大小列表。

QListQAudioFormat::SampleType QAbstractAudioDeviceInfo::supportedSampleTypes() [纯虚函数] 

返回当前可用样本类型的列表。

%%% Local Variables:
%%% mode: latex
%%% TeX-master: "../../master"
%%% End:

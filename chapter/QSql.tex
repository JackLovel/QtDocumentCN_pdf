\chapter{QSql}
QSql 命名空间\\

QSql 命名空间 里的 各种名样的标识符,已经被运用在 Qt SQL 各个模块中。\href{https://doc.qt.io/qt-5/qsql.html#details}{更多} \\

\begin{tabular}{|c|c|p{1.5cm}|}
	\hline
	属性 & 方法 \\
	\hline
	头文件 & \#include <QSql>\\      
	\hline
	qmake & QT += sql\\      
	\hline
\end{tabular}\\

类型

\resizebox{\textwidth}{!}{ % Latex表格宽度超出文本宽度
\begin{tabular}{|c|l|}
	\hline
	 &  \\
	\hline
	enum & Location \{ BeforeFirstRow, AfterLastRow \}\\      
	\hline
	enum & NumericalPrecisionPolicy \{ LowPrecisionInt32, LowPrecisionInt64,LowPrecisionDouble,HighPrecision\}\\      
	\hline
	flags & ParamType \\
	\hline 
	enum & ParamTypeFlag \{ In, Out, InOut, Binary \} \\ 
	\hline
	enum & TableType \{ Tables, SystemTables, Views, AllTables \}\\
	\hline
\end{tabular}
}


细节的介绍 \\

查看 \href{https://doc.qt.io/qt-5/qtsql-index.html}{Qt SQL}

类型 文档\\ 

enum QSql::Location


此枚举类型描述特殊的sql导航位置


\begin{tabular}{|c|c|c|}
	\hline
	常量	& 值 & 介绍 \\
	\hline
	QSql::BeforeFirstRow&-1&在第一个记录之前\\
	\hline
	QSql::AfterLastRow&-2&在最后一个记录之后\\
	\hline
\end{tabular}\\

另请参阅 \href{https://doc.qt.io/qt-5/qsqlquery.html#at}{QSqlQuery::at()}

enum QSql::NumericalPrecisionPolicy


数据库中的数值可以比它们对应的C++类型更精确。此枚举列出在应用程序中表示此类值的策略。


\resizebox{\textwidth}{!}{ % Latex表格宽度超出文本宽度
\begin{tabular}{|c|c|c|}
	\hline
	常量	& 值 & 介绍 \\
	\hline
	QSql::LowPrecisionInt32	&0x01 &对于32位的整形数值。在浮点数的情况下,小数部分将会被舍去。\\
	\hline
	QSql::LowPrecisionInt64	&0x02 &对于64位的整形数值。在浮点数的情况下,小数部分将会被舍去。\\
	\hline
	QSql::LowPrecisionDouble&0x04 &强制双精度值。这个默认的规则\\
	\hline
	QSql::HighPrecision	&0&字符串将会维技精度\\
	\hline
\end{tabular}\\
}

注意: 如果特定的驱动发生溢出,这是一个真实行为。像 Oracle数据库在这种情形下,就会返回一个错误。\\


enum QSql::ParamTypeFlag


flags QSql::ParamType


这个枚举用于指定绑定参数的类型

\begin{tabular}{|l|l|l|}
	\hline
		常量	& 值 & 介绍 \\
	\hline
	QSql::In&0x00000001&这个参数被用于向数据库里写入数据\\
	\hline
	QSql::Out&0x00000002&这个参数被用于向数据库里获得数据\\
	\hline
	QSql::InOut&In | Out&这个参数被用于向数据库里写入数据;使用 查询 来向数据库里,重写数据\\
	\hline
	QSql::Binary&0x00000004&如果您想 显示数据为 原始的二进制数据,那么必须是 OR'd 和其他的标志一 起使用\\
	\hline
\end{tabular}

类型参数 类型定义为 \href{https://doc.qt.io/qt-5/qflags.html}{QFlags}. 它被存放在 一个 OR与 类型参数标志的值 的组合。






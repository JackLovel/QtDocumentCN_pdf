\chapter{QMultiMap}

template <typename Key, typename T> class QMultiMap

QMultiMap 类是一个便利的 QMap 派生类,提供多值映射功能。更多内容...

\begin{tabular}{|r|l|}
	\hline
	属性 & 方法 \\
	\hline
    头文件  &	\hl{\#include <QMultiMap>} \\
    \hline
    qmake: & QT += core    \\
    \hline
    基类: & QMap    \\
	\hline
\end{tabular}

\begin{compactitem}
\item 所有成员列表,包括继承的成员
\end{compactitem}

\begin{notice}
该类中的所有函数都是可重入的。
\end{notice}


\section{公共成员函数}


\begin{longtable}[l]{|r|m{28em}|}   
    \hline
    返回类型 	& 函数 \\
    \hline
    &QMultiMap(const QMap<Key, T> \emph{\&other}) \\
    \hline 
	&QMultiMap(std::initializer\_list<std::pair<Key, T>> \emph{list})\\
    \hline
	&QMultiMap() \\
    \hline
typename QMap<Key, T>::const\_iterator &	constFind(const Key \emph{\&key}, const T \emph{\&value}) const \\
\hline
bool 	& contains(const Key \&key, const T \emph{\&value}) const\\
\hline
int 	& count(const Key \&key, const T \emph{\&value}) const\\
\hline
typename QMap<Key, T>::iterator 	& find(const Key \emph{\&key}, const T \emph{\&value})\\
\hline
typename QMap<Key, T>::const\_iterator &	find(const Key \emph{\&key}, const T \emph{\&value}) const\\
\hline
typename QMap<Key, T>::iterator &insert(const Key \emph{\&key}, const T \emph{\&value})\\
\hline
typename QMap<Key, T>::iterator & insert(typename QMap<Key, T>::const\_iterator \emph{pos}, const Key \emph{\&key}, const T \emph{\&value}) \\
\hline
int &	remove(const Key \emph{\&key}, const T \emph{\&value}) \\
\hline
typename QMap<Key, T>::iterator &	replace(const Key \emph{\&key}, const T \emph{\&value}) \\
\hline
void 	&swap(QMultiMap<Key, T> \emph{\&other})\\
\hline
QList &	uniqueKeys() const \\
\hline
QMultiMap<K, V> \& &	unite(const QMultiMap<K, V> \emph{\&other}) \\
\hline
QList &	values(const Key \emph{\&key}) const \\
\hline
QMultiMap<K, V> 	&operator+(const QMultiMap<K, V> \emph{\&other}) const \\
\hline
QMultiMap<K, V> \& &	operator+=(const QMultiMap<K, V> \emph{\&other}) \\
    \hline
\end{longtable}


\section{详细描述}

QMultiMap<Key, T> 是一种 Qt 泛型容器类。
它继承 QMap 并扩展了一些功能,使之可以存储多值映射。
多值映射是一种允许将多个值关联到同一个键的映射;
QMap 不允许多值映射。

因为 QMultiMap 继承 QMap,所有 QMap 的功能也适用于 QMultiMap。
例如,可以使用 isEmpty() 测试 map 是否为空,可以使用 QMap 的迭代器类(例如 QMapIterator)遍历 QMultiMap。
除此之外,它还提供 insert() 函数来插入值,如果要插入的键已经存在,该函数不会覆盖已有的值,而 replace() 函数则不同,
如果 map 中已经存在要插入的键,该函数会覆盖已经存在的值。
此外,该类还提供方便的 operator+() 和 operator+=() 运算符。

例子:

\begin{lstlisting}[language=C++]
QMultiMap<QString, int> map1, map2, map3;

map1.insert("plenty", 100);
map1.insert("plenty", 2000);
// map1.size() == 2

map2.insert("plenty", 5000);
// map2.size() == 1

map3 = map1 + map2;
// map3.size() == 3
\end{lstlisting}

%%%%%%%

与 QMap 不同,QMultiMap 不提供 operator[] 运算符。
如果想用特定键访问最新插入的元素,使用 value() 或 replace()。

如果想取得单个键关联的所有值,可以使用 values(const Key \&key),
该函数返回一个 QList:

\begin{lstlisting}[language=C++]
QList<int> values = map.values("plenty");
for (int i = 0; i < values.size(); ++i)
    cout << values.at(i) << Qt::endl;
\end{lstlisting}

共享同一键的元素按照从最新到最早插入的顺序返回。

如果习惯用 STL 风格迭代器,可以传递键调用 find() 取得第一个元素的迭代器,从该元素开始遍历:

\begin{lstlisting}[language=C++]
QMultiMap<QString, int>::iterator i = map.find("plenty");
while (i != map.end() && i.key() == "plenty") {
    cout << i.value() << Qt::endl;
    ++i;
}
\end{lstlisting}

QMultiMap 键和值的数据类型必须是可赋值数据类型。
这涵盖了大多数可能会遇到的数据类型,但是编译器不会允许存储类似 QWidget 这样的对象作为值,
应该存储 QWidget *。另外,QMultiMap 的键类型必须提供 operator<() 运算符。 
具体请参考 QMap 文档。

\begin{seeAlso}
QMap,QMapIterator,QMutableMapIterator 和 QMultiHash。
\end{seeAlso}

\section{成员函数文档}

QMultiMap::QMultiMap(const QMap<Key, T> \emph{\&other})

构造一个 other 的副本(可能是一个 QMap 或 QMultiMap)。

\begin{seeAlso}
operator=()。
\end{seeAlso}

QMultiMap::QMultiMap(std::initializer\_list<std::pair<Key, T>> \emph{list})

用初始化列表 list 中每个元素的副本构造一个 multi-map。

只有当程序在 C++11 模式下编译时,该函数才可用。

Qt 5.1 中引入该函数。

QMultiMap::QMultiMap()

构造一个空 map。

typename QMap<Key, T>::const\_iterator QMultiMap::constFind(const Key \emph{\&key}, const T \emph{\&value}) const

返回迭代器,指向 map 中键为 key,值为 value 的元素。

如果 map 中不包含这样的元素,该函数返回 constEnd()。

Qt 4.3 中引入该函数。

\begin{seeAlso}
QMap::constFind()。
\end{seeAlso}

bool QMultiMap::contains(const Key \emph{\&key}, const T \emph{\&value}) const

如果该 map 包含键为 key,值为 value 的元素,返回 true;否则返回 false。

Qt 4.3 中引入该函数。

\begin{seeAlso}
QMap::contains()。
\end{seeAlso}

int QMultiMap::count(const Key \emph{\&key}, const T  \emph{\&value}) const

返回键为 key,值为 value 的元素个数。

Qt 4.3 中引入该函数。

\begin{seeAlso}
QMap::count()。
\end{seeAlso}

typename QMap<Key, T>::iterator QMultiMap::find(const Key \emph{\&key}, const T \emph{\&value})

返回迭代器,指向 map 中键为 key,值为 value 的元素。

如果 map 中不包含这样的元素,该函数返回 end()。

如果 map 包含多个键为 key (译者注:以及值为 value)的元素,函数返回指向最新插入的那个值的迭代器。

Qt 4.3 中引入该函数。

\begin{seeAlso}
QMap::find()。
\end{seeAlso}

typename QMap<Key, T>::const\_iterator QMultiMap::find(const Key \emph{\&key}, const T \emph{\&value}) const

这是一个重载函数。

返回常量迭代器,指向 map 中键为 key,值为 value 的元素。

如果 map 中不包含这样的元素,该函数返回 end()。

如果 map 包含多个键为 key(译者注:以及值为 value)的元素,函数返回指向最新插入的那个值的常量迭代器。

Qt 4.3 中引入该函数。

\begin{seeAlso}
QMap::find()。
\end{seeAlso}

typename QMap<Key, T>::iterator QMultiMap::insert(const Key \emph{\&key}, const T \emph{\&value})

用键 key 和值 value 插入一个新元素。

如果 map 中已经存在相同键的元素,该函数将创建一个新元素。
(这与 replace() 不同,replace() 是覆盖已经存在元素的值。)

\begin{seeAlso}
replace()。
\end{seeAlso}

typename QMap<Key, T>::iterator QMultiMap::insert(typename QMap<Key, T>::const\_iterator \emph{pos}, const Key \emph{\&key}, const T \emph{\&value})

用键 key 和值 value 插入一个新元素,pos 用来提示插入位置。

如果以 constBegin() 作为插入位置提示,表明 key 比 map 中的任何键都小,
而 constEnd() 则建议 key 大于 map 中的任何键。
否则提示应该满足条件 (pos - 1).key() < key <= pos.key()。
如果提示 pos 是错误的,其将被忽略,并以常规方式插入。

如果 map 中已经存在相同键的元素,该函数将创建一个新元素。

\begin{notice}
需小心对待提示。提供从旧的共享实例取得的迭代器可能引起崩溃,还会有默默污染 map 和 pos 的 map 的风险。
\end{notice}

Qt 5.1 中引入该函数。

int QMultiMap::remove(const Key \emph{\&key}, const T \emph{\&value})

从 map 中移除所有键为 key,值为 value 的元素。返回被移除元素的个数。

Qt 4.3 中引入该函数。

\begin{seeAlso}
QMap::remove()。
\end{seeAlso}

typename QMap<Key, T>::iterator QMultiMap::replace(const Key \emph{\&key}, const T \emph{\&value})

用键 key 和值 value 插入一个新元素。

如果已经存在键为 key 的元素,该元素的值将被 value 替换。

如果有多个键为 key 的元素,最新插入的元素的值将被 value 替换。

\begin{seeAlso}
insert()。
\end{seeAlso}

void QMultiMap::swap(QMultiMap<Key, T> \&other)

将 map other 与本 map 交换。该操作非常快,永远不失败。

Qt 4.8 中引入该函数。
QList QMultiMap::uniqueKeys() const

以升序返回 map 中所有键的列表。在 map 中多次出现的键在返回的列表中只出现一次。

Qt 4.2 中引入该函数。

QMultiMap<K, V> \&QMultiMap::unite(const QMultiMap<K, V> \emph{\&other})

将 other map 中的所有元素插入到本 map 中。
如果一个键在两个 map 中同时存在,结果 map 将多次包含这个键。

\lineHigh

QList QMultiMap::values(const Key \emph{\&key}) const

按照从最新到最早插入的顺序,返回所有与键 \emph{key} 相关联的值的列表。

\lineHigh

QMultiMap<K, V> QMultiMap::operator+(const QMultiMap<K, V> \emph{\&other}) const

返回一个 map,该 map 包含本 map 和 \emph{other} map 中的所有元素。
如果一个键在两个 map 中同时存在,结果 map 将多次包含这个键。

\begin{seeAlso}
operator+=()。
\end{seeAlso}

\lineHigh

QMultiMap<K, V> \&QMultiMap::operator+=(const QMultiMap<K, V> \emph{\&other})

将 \emph{other} map 中的所有元素插入到本 map 中,返回本 map 的引用。

\begin{seeAlso}
insert() 和 operator+()。
\end{seeAlso}
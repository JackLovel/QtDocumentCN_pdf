\chapter{QMetaType}

QMetaType 类管理元对象系统中的注名类型。更多内容...。

\begin{tabular}{|r|l|}
	\hline
	属性 & 方法 \\
	\hline
    头文件  &	\hl{\#include <QMetaType>} \\
    \hline
    qmake: & \hl{QT += core}    \\
	\hline
\end{tabular}

\begin{notice}
此类中所有函数都是线程安全的。
\end{notice}

\section{公共成员类型}

\begin{tabular}{|r|m{25em}|}   
\hline
类型 	& 名称 \\
\hline
enum &	Type \{ Void, Bool, Int, UInt, Double, ..., UnknownType \} \\
\hline
enum &	TypeFlag \{ NeedsConstruction, NeedsDestruction, MovableType, IsEnumeration, PointerToQObject \}\\
\hline
flags &	TypeFlags\\
\hline
\end{tabular}

\section{公共成员函数}

\begin{longtable}{|r|m{28em}|}   
\hline
返回类型 	& 函数 \\
\hline
& QMetaType(const int \emph{typeId} = QMetaType::UnknownType) \\ 
\hline
& $\sim$QMetaType() \\
\hline
void *	&construct(void \emph{*where}, const void \emph{*copy} = 0) const \\
\hline
void *	&create(const void \emph{*copy} = 0) const \\
\hline
void	&destroy(void \emph{*data}) const \\
\hline
void	&destruct(void \emph{*data}) const \\
\hline
QMetaType::TypeFlags &	flags() const \\
\hline
int	& id() const \\ 
\hline
bool	&isRegistered() const \\
\hline
bool	&isValid() const \\
\hline
const QMetaObject *	& metaObject() const \\
\hline
::QByteArray &	name() const \\
\hline
int	& sizeOf() const \\
\hline
\end{longtable}

\section{静态公共成员}

\begin{longtable}{|r|m{28em}|}   
\hline
返回类型 	& 函数 \\
\hline
bool	& compare(const void \emph{*lhs}, const void \emph{*rhs}, int \emph{typeId}, int \emph{*result})\\
\hline
void *	&construct(int \emph{type}, void \emph{*where}, const void \emph{*copy})\\
\hline
bool	& convert(const void \emph{*from}, int \emph{fromTypeId}, void \emph{*to}, int \emph{toTypeId})\\
\hline
void *	& create(int \emph{ype}, const void \emph{*copy} = nullptr)\\
\hline
bool	& debugStream(QDebug \emph{\&dbg}, const void \emph{*rhs}, int \emph{typeId})\\
\hline
void	& destroy(int \emph{type}, void \emph{*data})\\
\hline
void	& destruct(int \emph{type}, void \emph{*where})\\
\hline
bool	&equals(const void \emph{*lhs}, const void \emph{*rhs}, int \emph{typeId}, int \emph{*result})\\
\hline
QMetaType&	fromType()\\
\hline
bool	& hasRegisteredComparators()\\
\hline
bool	& hasRegisteredComparators(int \emph{typeId})\\
\hline
bool	&hasRegisteredConverterFunction(int \emph{fromTypeId}, int \emph{toTypeId})\\
\hline
bool	&hasRegisteredConverterFunction()\\
\hline
bool	&hasRegisteredDebugStreamOperator()\\
\hline
bool	&hasRegisteredDebugStreamOperator(int \emph{typeId})\\
\hline
bool	&load(QDataStream  \emph{\&stream}, int  \emph{type}, void \emph{*data})\\
\hline
const QMetaObject *&	metaObjectForType(int \emph{type})\\
\hline
bool&	registerComparators()\\
\hline
bool&	registerConverter()\\
\hline
bool&	registerConverter(MemberFunction \emph{function})\\
\hline
bool&	registerConverter(MemberFunctionOk \emph{function})\\
\hline
bool&	registerConverter(UnaryFunction \emph{function})\\
\hline
bool&	registerDebugStreamOperator()\\
\hline
bool&	registerEqualsComparator()\\
\hline
bool&	save(QDataStream \emph{\&stream}, int \emph{type}, const void \emph{*data}) \\
\hline
int	&sizeOf(int \emph{type}) \\
\hline
int	& type(const char  \emph{*typeName}) \\
\hline
int	 & type(const ::QByteArray \emph{\&typeName}) \\
\hline
QMetaType::TypeFlags &	typeFlags(int \emph{type}) \\
\hline
const char * &	typeName(int \emph{typeId}) \\
\hline
\end{longtable}


\section{相关非成员函数}

\begin{tabular}{|r|m{25em}|}   
\hline
返回类型 	& 函数 \\
\hline
int &	qMetaTypeId()  \\ 
\hline
int &	qRegisterMetaType(const char \emph{*typeName}) \\ 
\hline
int	 & qRegisterMetaType() \\ 
\hline
void &	qRegisterMetaTypeStreamOperators(const char \emph{*typeName}) \\ 
\hline
bool &	operator!=(const QMetaType \emph{\&a}, const QMetaType \emph{\&b}) \\ 
\hline
bool &	operator==(const QMetaType \emph{\&a}, const QMetaType \emph{\&b}) \\ 
\hline
\end{tabular}

\section{宏定义}

\begin{tabular}{|l|}   
\hline
宏定义 \\
\hline
Q\_DECLARE\_ASSOCIATIVE\_CONTAINER\_METATYPE(\emph{Container}) \\
\hline
Q\_DECLARE\_METATYPE(\emph{Type}) \\
\hline
Q\_DECLARE\_OPAQUE\_POINTER(\emph{PointerType}) \\
\hline
Q\_DECLARE\_SEQUENTIAL\_CONTAINER\_METATYPE(\emph{Container}) \\ 
\hline
Q\_DECLARE\_SMART\_POINTER\_METATYPE(\emph{SmartPointer})\\
\hline
\end{tabular}


\section{详细描述}

此类是一个辅助类,被用作序列化 QVariant 以及队列连接信号槽中的类型。
它将类型名称关联到对应类型,以支持运行时动态创建和销毁此类型。
通过 Q\_DECLARE\_METATYPE() 声明新类型,让它可以被 QVariant 和其它模板函数(qMetaTypeId() 等)使用。
调用 qRegisterMetaType() 来让其可以被非模板型函数使用,如信号槽的队列连接。

任何包含一个公共默认构造函数、一个公共拷贝构造函数、一个默认析构函数的类或结构体都可以被注册为元类型。

下述代码展示了如何分配和销毁一个 MyClass 的实例:

\begin{lstlisting}[language=C++]
int id = QMetaType::type("MyClass");
if (id != QMetaType::UnknownType) {
    void *myClassPtr = QMetaType::create(id);
    ...
    QMetaType::destroy(id, myClassPtr);
    myClassPtr = 0;
}
\end{lstlisting}

若我们想让流运算符 operator<<() 和 operator>>() 可被用于存储了自定义类型的 QVariant 对象,则这个自定义类型必须提供 operator<<() 和 operator>>() 运算符重载。

\begin{seeAlso}
Q\_DECLARE\_METATYPE(),QVariant::setValue(),QVariant::value() 和 QVariant::fromValue().
\end{seeAlso}

\section{成员类型文档}

enum QMetaType::Type

下表是 QMetaType 内置支持的类型:

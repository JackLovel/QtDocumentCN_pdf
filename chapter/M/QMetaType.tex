\chapter{QMetaType}

QMetaType 类管理元对象系统中的注名类型。更多内容...。

\begin{tabular}{|r|l|}
	\hline
	属性 & 方法 \\
	\hline
    头文件  &	\hl{\#include <QMetaType>} \\
    \hline
    qmake: & \hl{QT += core}    \\
	\hline
\end{tabular}

\begin{notice}
此类中所有函数都是线程安全的。
\end{notice}

\section{公共成员类型}

\begin{tabular}{|r|m{25em}|}   
\hline
类型 	& 名称 \\
\hline
enum &	Type \{ Void, Bool, Int, UInt, Double, ..., UnknownType \} \\
\hline
enum &	TypeFlag \{ NeedsConstruction, NeedsDestruction, MovableType, IsEnumeration, PointerToQObject \}\\
\hline
flags &	TypeFlags\\
\hline
\end{tabular}

\section{公共成员函数}

\begin{longtable}{|r|m{28em}|}   
\hline
返回类型 	& 函数 \\
\hline
& QMetaType(const int \emph{typeId} = QMetaType::UnknownType) \\ 
\hline
& $\sim$QMetaType() \\
\hline
void *	&construct(void \emph{*where}, const void \emph{*copy} = 0) const \\
\hline
void *	&create(const void \emph{*copy} = 0) const \\
\hline
void	&destroy(void \emph{*data}) const \\
\hline
void	&destruct(void \emph{*data}) const \\
\hline
QMetaType::TypeFlags &	flags() const \\
\hline
int	& id() const \\ 
\hline
bool	&isRegistered() const \\
\hline
bool	&isValid() const \\
\hline
const QMetaObject *	& metaObject() const \\
\hline
::QByteArray &	name() const \\
\hline
int	& sizeOf() const \\
\hline
\end{longtable}

\section{静态公共成员}

\begin{longtable}{|r|m{28em}|}   
\hline
返回类型 	& 函数 \\
\hline
bool	& compare(const void \emph{*lhs}, const void \emph{*rhs}, int \emph{typeId}, int \emph{*result})\\
\hline
void *	&construct(int \emph{type}, void \emph{*where}, const void \emph{*copy})\\
\hline
bool	& convert(const void \emph{*from}, int \emph{fromTypeId}, void \emph{*to}, int \emph{toTypeId})\\
\hline
void *	& create(int \emph{ype}, const void \emph{*copy} = nullptr)\\
\hline
bool	& debugStream(QDebug \emph{\&dbg}, const void \emph{*rhs}, int \emph{typeId})\\
\hline
void	& destroy(int \emph{type}, void \emph{*data})\\
\hline
void	& destruct(int \emph{type}, void \emph{*where})\\
\hline
bool	&equals(const void \emph{*lhs}, const void \emph{*rhs}, int \emph{typeId}, int \emph{*result})\\
\hline
QMetaType&	fromType()\\
\hline
bool	& hasRegisteredComparators()\\
\hline
bool	& hasRegisteredComparators(int \emph{typeId})\\
\hline
bool	&hasRegisteredConverterFunction(int \emph{fromTypeId}, int \emph{toTypeId})\\
\hline
bool	&hasRegisteredConverterFunction()\\
\hline
bool	&hasRegisteredDebugStreamOperator()\\
\hline
bool	&hasRegisteredDebugStreamOperator(int \emph{typeId})\\
\hline
bool	&load(QDataStream  \emph{\&stream}, int  \emph{type}, void \emph{*data})\\
\hline
const QMetaObject *&	metaObjectForType(int \emph{type})\\
\hline
bool&	registerComparators()\\
\hline
bool&	registerConverter()\\
\hline
bool&	registerConverter(MemberFunction \emph{function})\\
\hline
bool&	registerConverter(MemberFunctionOk \emph{function})\\
\hline
bool&	registerConverter(UnaryFunction \emph{function})\\
\hline
bool&	registerDebugStreamOperator()\\
\hline
bool&	registerEqualsComparator()\\
\hline
bool&	save(QDataStream \emph{\&stream}, int \emph{type}, const void \emph{*data}) \\
\hline
int	&sizeOf(int \emph{type}) \\
\hline
int	& type(const char  \emph{*typeName}) \\
\hline
int	 & type(const ::QByteArray \emph{\&typeName}) \\
\hline
QMetaType::TypeFlags &	typeFlags(int \emph{type}) \\
\hline
const char * &	typeName(int \emph{typeId}) \\
\hline
\end{longtable}


\section{相关非成员函数}

\begin{tabular}{|r|m{25em}|}   
\hline
返回类型 	& 函数 \\
\hline
int &	qMetaTypeId()  \\ 
\hline
int &	qRegisterMetaType(const char \emph{*typeName}) \\ 
\hline
int	 & qRegisterMetaType() \\ 
\hline
void &	qRegisterMetaTypeStreamOperators(const char \emph{*typeName}) \\ 
\hline
bool &	operator!=(const QMetaType \emph{\&a}, const QMetaType \emph{\&b}) \\ 
\hline
bool &	operator==(const QMetaType \emph{\&a}, const QMetaType \emph{\&b}) \\ 
\hline
\end{tabular}

\section{宏定义}

\begin{tabular}{|l|}   
\hline
宏定义 \\
\hline
Q\_DECLARE\_ASSOCIATIVE\_CONTAINER\_METATYPE(\emph{Container}) \\
\hline
Q\_DECLARE\_METATYPE(\emph{Type}) \\
\hline
Q\_DECLARE\_OPAQUE\_POINTER(\emph{PointerType}) \\
\hline
Q\_DECLARE\_SEQUENTIAL\_CONTAINER\_METATYPE(\emph{Container}) \\ 
\hline
Q\_DECLARE\_SMART\_POINTER\_METATYPE(\emph{SmartPointer})\\
\hline
\end{tabular}


\section{详细描述}

此类是一个辅助类,被用作序列化 QVariant 以及队列连接信号槽中的类型。
它将类型名称关联到对应类型,以支持运行时动态创建和销毁此类型。
通过 Q\_DECLARE\_METATYPE() 声明新类型,让它可以被 QVariant 和其它模板函数(qMetaTypeId() 等)使用。
调用 qRegisterMetaType() 来让其可以被非模板型函数使用,如信号槽的队列连接。

任何包含一个公共默认构造函数、一个公共拷贝构造函数、一个默认析构函数的类或结构体都可以被注册为元类型。

下述代码展示了如何分配和销毁一个 MyClass 的实例:

\begin{lstlisting}[language=C++]
int id = QMetaType::type("MyClass");
if (id != QMetaType::UnknownType) {
    void *myClassPtr = QMetaType::create(id);
    ...
    QMetaType::destroy(id, myClassPtr);
    myClassPtr = 0;
}
\end{lstlisting}

若我们想让流运算符 operator<<() 和 operator>>() 可被用于存储了自定义类型的 QVariant 对象,则这个自定义类型必须提供 operator<<() 和 operator>>() 运算符重载。

\begin{seeAlso}
Q\_DECLARE\_METATYPE(),QVariant::setValue(),QVariant::value() 和 QVariant::fromValue().
\end{seeAlso}

\section{成员类型文档}

enum QMetaType::Type

下表是 QMetaType 内置支持的类型:


\begin{longtable}{|l|l|m{20em}|}   
\hline
常量 &	数值 &	描述 \\ 
\hline
QMetaType::Void &	43&	void \\
\hline
QMetaType::Bool	&1&	bool \\
\hline
QMetaType::Int	&2&	int \\
\hline
QMetaType::UInt	&3	&unsigned int \\
\hline
QMetaType::Double&	6	&double \\
\hline
QMetaType::QChar&	7	&QChar \\
\hline
QMetaType::QString	&10&	QString \\
\hline
QMetaType::QByteArray&	12&	QByteArray \\
\hline
QMetaType::Nullptr&	51&	std::nullptr\_t \\
\hline
QMetaType::VoidStar	&31&	void * \\
\hline
QMetaType::Long	&32	&long \\
\hline
QMetaType::LongLong	&4	&long long \\
\hline
QMetaType::Short	&33&	short \\
\hline
QMetaType::Char&	34	&char \\
\hline
QMetaType::ULong	&35	&unsigned long \\
\hline
QMetaType::ULongLong&	5	&unsigned long long \\
\hline
QMetaType::UShort&	36&	unsigned short \\
\hline
QMetaType::SChar&	40&	signed char \\
\hline
QMetaType::UChar&	37&	unsigned char \\
\hline
QMetaType::Float	&38&	float \\
\hline
QMetaType::QObjectStar	&39&	QObject * \\
\hline
QMetaType::QVariant	&41	&QVariant \\
\hline
QMetaType::QCursor	&74&	QCursor \\
\hline
QMetaType::QDate&	14	&QDate \\
\hline
QMetaType::QSize&	21&	QSize \\
\hline
QMetaType::QTime&	15	&QTime\\
\hline
QMetaType::QVariantList&	9	&QVariantList\\
\hline
QMetaType::QPolygon&	71	&QPolygon\\
\hline
QMetaType::QPolygonF&	86&	QPolygonF\\
\hline
QMetaType::QColor	&67&	QColor\\
\hline
QMetaType::QColorSpace&	87&	QColorSpace(在 Qt 5.15 中被引入)\\
\hline
QMetaType::QSizeF	&22	&QSizeF\\
\hline
QMetaType::QRectF	&20	&QRectF\\
\hline
QMetaType::QLine	&23	&QLine\\
\hline
QMetaType::QTextLength	&77&	QTextLength\\
\hline
QMetaType::QStringList	&11&	QStringList\\
\hline
QMetaType::QVariantMap	&8&	QVariantMap\\
\hline
QMetaType::QVariantHash	&28&	QVariantHash\\
\hline
QMetaType::QIcon&	69&	QIcon\\
\hline
QMetaType::QPen	&76	&QPen \\
\hline
QMetaType::QLineF	&24&	QLineF\\
\hline
QMetaType::QTextFormat&	78&	QTextFormat \\
\hline
QMetaType::QRect&	19	&QRect \\
\hline
QMetaType::QPoint&	25	&QPoint \\
\hline
QMetaType::QUrl	&17	&QUrl \\
\hline
QMetaType::QRegExp	&27	&QRegExp \\
\hline
QMetaType::QRegularExpression	&44	&QRegularExpression \\
\hline
QMetaType::QDateTime	&16&	QDateTime\\
\hline
QMetaType::QPointF	&26&	QPointF\\
\hline
QMetaType::QPalette	&68	&QPalette\\
\hline
QMetaType::QFont	&64&	QFont\\
\hline
QMetaType::QBrush	&66	&QBrush\\
\hline
QMetaType::QRegion&	72&	QRegion\\
\hline
QMetaType::QBitArray&	13	&QBitArray\\
\hline
QMetaType::QImage	&70	&QImage\\
\hline
QMetaType::QKeySequence	&75	&QKeySequence\\
\hline
QMetaType::QSizePolicy&	121&	QSizePolicy\\
\hline
QMetaType::QPixmap&	65	&QPixmap\\
\hline
QMetaType::QLocale&	18	&QLocale\\
\hline
QMetaType::QBitmap&	73&	QBitmap\\
\hline
QMetaType::QMatrix&	79&	QMatrix\\
\hline
QMetaType::QTransform	&80	&QTransform\\
\hline
QMetaType::QMatrix4x4	&81	&QMatrix4x4\\
\hline
QMetaType::QVector2D	&82	&QVector2D\\
\hline
QMetaType::QVector3D	&83	&QVector3D\\
\hline
QMetaType::QVector4D	&84	&QVector4D\\
\hline
QMetaType::QQuaternion&	85	&QQuaternion\\
\hline
QMetaType::QEasingCurve	&29&	QEasingCurve\\
\hline
QMetaType::QJsonValue	&45&	QJsonValue\\
\hline
QMetaType::QJsonObject&	46&	QJsonObject\\
\hline
QMetaType::QJsonArray	&47	&QJsonArray\\
\hline
QMetaType::QJsonDocument	&48&	QJsonDocument\\
\hline
QMetaType::QCborValue	&53&	QCborValue\\
\hline
QMetaType::QCborArray	&54	&QCborArray \\
\hline
QMetaType::QCborMap	&55&	QCborMap \\
\hline
QMetaType::QCborSimpleType	&52&	QCborSimpleType \\
\hline
QMetaType::QModelIndex&	42	&QModelIndex \\
\hline
QMetaType::QPersistentModelIndex&	50	&QPersistentModelIndex(在 Qt 5.5 中被引入) \\
\hline
QMetaType::QUuid	&30	&QUuid \\
\hline
QMetaType::QByteArrayList&	49&	QByteArrayList \\
\hline
QMetaType::User	&1024&	用户类型的基础值(译者注:即起始值)\\
\hline
QMetaType::UnknownType&	0	&这是无效的类型编号,QMetaType 会在类型未注册时返回此值。\\
\hline
\end{longtable}
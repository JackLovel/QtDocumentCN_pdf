\chapter{QMetaType}

QMetaType 类管理元对象系统中的注名类型。更多内容...

\begin{tabular}{|r|l|}
	\hline
	属性 & 方法 \\
	\hline
    头文件  &	\hl{\#include <QMetaType>} \\
    \hline
    qmake: & \hl{QT += core}    \\
	\hline
\end{tabular}

\begin{notice}
此类中所有函数都是线程安全的。
\end{notice}

\section{公共成员类型}

\begin{tabular}{|r|m{25em}|}   
\hline
类型 	& 名称 \\
\hline
enum &	Type \{ Void, Bool, Int, UInt, Double, ..., UnknownType \} \\
\hline
enum &	TypeFlag \{ NeedsConstruction, NeedsDestruction, MovableType, IsEnumeration, PointerToQObject \}\\
\hline
flags &	TypeFlags\\
\hline
\end{tabular}

\section{公共成员函数}

\begin{longtable}[l]{|r|m{28em}|}   
\hline
返回类型 	& 函数 \\
\hline
& QMetaType(const int \emph{typeId} = QMetaType::UnknownType) \\ 
\hline
& $\sim$QMetaType() \\
\hline
void *	&construct(void \emph{*where}, const void \emph{*copy} = 0) const \\
\hline
void *	&create(const void \emph{*copy} = 0) const \\
\hline
void	&destroy(void \emph{*data}) const \\
\hline
void	&destruct(void \emph{*data}) const \\
\hline
QMetaType::TypeFlags &	flags() const \\
\hline
int	& id() const \\ 
\hline
bool	&isRegistered() const \\
\hline
bool	&isValid() const \\
\hline
const QMetaObject *	& metaObject() const \\
\hline
::QByteArray &	name() const \\
\hline
int	& sizeOf() const \\
\hline
\end{longtable}

\section{静态公共成员}

\begin{longtable}[l]{|r|m{28em}|}   
\hline
返回类型 	& 函数 \\
\hline
bool	& compare(const void \emph{*lhs}, const void \emph{*rhs}, int \emph{typeId}, int \emph{*result})\\
\hline
void *	&construct(int \emph{type}, void \emph{*where}, const void \emph{*copy})\\
\hline
bool	& convert(const void \emph{*from}, int \emph{fromTypeId}, void \emph{*to}, int \emph{toTypeId})\\
\hline
void *	& create(int \emph{ype}, const void \emph{*copy} = nullptr)\\
\hline
bool	& debugStream(QDebug \emph{\&dbg}, const void \emph{*rhs}, int \emph{typeId})\\
\hline
void	& destroy(int \emph{type}, void \emph{*data})\\
\hline
void	& destruct(int \emph{type}, void \emph{*where})\\
\hline
bool	&equals(const void \emph{*lhs}, const void \emph{*rhs}, int \emph{typeId}, int \emph{*result})\\
\hline
QMetaType&	fromType()\\
\hline
bool	& hasRegisteredComparators()\\
\hline
bool	& hasRegisteredComparators(int \emph{typeId})\\
\hline
bool	&hasRegisteredConverterFunction(int \emph{fromTypeId}, int \emph{toTypeId})\\
\hline
bool	&hasRegisteredConverterFunction()\\
\hline
bool	&hasRegisteredDebugStreamOperator()\\
\hline
bool	&hasRegisteredDebugStreamOperator(int \emph{typeId})\\
\hline
bool	&load(QDataStream  \emph{\&stream}, int  \emph{type}, void \emph{*data})\\
\hline
const QMetaObject *&	metaObjectForType(int \emph{type})\\
\hline
bool&	registerComparators()\\
\hline
bool&	registerConverter()\\
\hline
bool&	registerConverter(MemberFunction \emph{function})\\
\hline
bool&	registerConverter(MemberFunctionOk \emph{function})\\
\hline
bool&	registerConverter(UnaryFunction \emph{function})\\
\hline
bool&	registerDebugStreamOperator()\\
\hline
bool&	registerEqualsComparator()\\
\hline
bool&	save(QDataStream \emph{\&stream}, int \emph{type}, const void \emph{*data}) \\
\hline
int	&sizeOf(int \emph{type}) \\
\hline
int	& type(const char  \emph{*typeName}) \\
\hline
int	 & type(const ::QByteArray \emph{\&typeName}) \\
\hline
QMetaType::TypeFlags &	typeFlags(int \emph{type}) \\
\hline
const char * &	typeName(int \emph{typeId}) \\
\hline
\end{longtable}


\section{相关非成员函数}

\begin{tabular}{|r|m{25em}|}   
\hline
返回类型 	& 函数 \\
\hline
int &	qMetaTypeId()  \\ 
\hline
int &	qRegisterMetaType(const char \emph{*typeName}) \\ 
\hline
int	 & qRegisterMetaType() \\ 
\hline
void &	qRegisterMetaTypeStreamOperators(const char \emph{*typeName}) \\ 
\hline
bool &	operator!=(const QMetaType \emph{\&a}, const QMetaType \emph{\&b}) \\ 
\hline
bool &	operator==(const QMetaType \emph{\&a}, const QMetaType \emph{\&b}) \\ 
\hline
\end{tabular}

\section{宏定义}

\begin{tabular}{|l|}   
\hline
宏定义 \\
\hline
Q\_DECLARE\_ASSOCIATIVE\_CONTAINER\_METATYPE(\emph{Container}) \\
\hline
Q\_DECLARE\_METATYPE(\emph{Type}) \\
\hline
Q\_DECLARE\_OPAQUE\_POINTER(\emph{PointerType}) \\
\hline
Q\_DECLARE\_SEQUENTIAL\_CONTAINER\_METATYPE(\emph{Container}) \\ 
\hline
Q\_DECLARE\_SMART\_POINTER\_METATYPE(\emph{SmartPointer})\\
\hline
\end{tabular}


\section{详细描述}

此类是一个辅助类,被用作序列化 QVariant 以及队列连接信号槽中的类型。
它将类型名称关联到对应类型,以支持运行时动态创建和销毁此类型。
通过 Q\_DECLARE\_METATYPE() 声明新类型,让它可以被 QVariant 和其它模板函数(qMetaTypeId() 等)使用。
调用 qRegisterMetaType() 来让其可以被非模板型函数使用,如信号槽的队列连接。

任何包含一个公共默认构造函数、一个公共拷贝构造函数、一个默认析构函数的类或结构体都可以被注册为元类型。

下述代码展示了如何分配和销毁一个 MyClass 的实例:

\begin{lstlisting}[language=C++]
int id = QMetaType::type("MyClass");
if (id != QMetaType::UnknownType) {
    void *myClassPtr = QMetaType::create(id);
    ...
    QMetaType::destroy(id, myClassPtr);
    myClassPtr = 0;
}
\end{lstlisting}

若我们想让流运算符 operator<<() 和 operator>>() 可被用于存储了自定义类型的 QVariant 对象,则这个自定义类型必须提供 operator<<() 和 operator>>() 运算符重载。

\begin{seeAlso}
Q\_DECLARE\_METATYPE(),QVariant::setValue(),QVariant::value() 和 QVariant::fromValue().
\end{seeAlso}

\section{成员类型文档}

enum QMetaType::Type

下表是 QMetaType 内置支持的类型:


\begin{longtable}[l]{|l|l|m{20em}|}   
\hline
常量 &	数值 &	描述 \\ 
\hline
QMetaType::Void &	43&	void \\
\hline
QMetaType::Bool	&1&	bool \\
\hline
QMetaType::Int	&2&	int \\
\hline
QMetaType::UInt	&3	&unsigned int \\
\hline
QMetaType::Double&	6	&double \\
\hline
QMetaType::QChar&	7	&QChar \\
\hline
QMetaType::QString	&10&	QString \\
\hline
QMetaType::QByteArray&	12&	QByteArray \\
\hline
QMetaType::Nullptr&	51&	std::nullptr\_t \\
\hline
QMetaType::VoidStar	&31&	void * \\
\hline
QMetaType::Long	&32	&long \\
\hline
QMetaType::LongLong	&4	&long long \\
\hline
QMetaType::Short	&33&	short \\
\hline
QMetaType::Char&	34	&char \\
\hline
QMetaType::ULong	&35	&unsigned long \\
\hline
QMetaType::ULongLong&	5	&unsigned long long \\
\hline
QMetaType::UShort&	36&	unsigned short \\
\hline
QMetaType::SChar&	40&	signed char \\
\hline
QMetaType::UChar&	37&	unsigned char \\
\hline
QMetaType::Float	&38&	float \\
\hline
QMetaType::QObjectStar	&39&	QObject * \\
\hline
QMetaType::QVariant	&41	&QVariant \\
\hline
QMetaType::QCursor	&74&	QCursor \\
\hline
QMetaType::QDate&	14	&QDate \\
\hline
QMetaType::QSize&	21&	QSize \\
\hline
QMetaType::QTime&	15	&QTime\\
\hline
QMetaType::QVariantList&	9	&QVariantList\\
\hline
QMetaType::QPolygon&	71	&QPolygon\\
\hline
QMetaType::QPolygonF&	86&	QPolygonF\\
\hline
QMetaType::QColor	&67&	QColor\\
\hline
QMetaType::QColorSpace&	87&	QColorSpace(在 Qt 5.15 中被引入)\\
\hline
QMetaType::QSizeF	&22	&QSizeF\\
\hline
QMetaType::QRectF	&20	&QRectF\\
\hline
QMetaType::QLine	&23	&QLine\\
\hline
QMetaType::QTextLength	&77&	QTextLength\\
\hline
QMetaType::QStringList	&11&	QStringList\\
\hline
QMetaType::QVariantMap	&8&	QVariantMap\\
\hline
QMetaType::QVariantHash	&28&	QVariantHash\\
\hline
QMetaType::QIcon&	69&	QIcon\\
\hline
QMetaType::QPen	&76	&QPen \\
\hline
QMetaType::QLineF	&24&	QLineF\\
\hline
QMetaType::QTextFormat&	78&	QTextFormat \\
\hline
QMetaType::QRect&	19	&QRect \\
\hline
QMetaType::QPoint&	25	&QPoint \\
\hline
QMetaType::QUrl	&17	&QUrl \\
\hline
QMetaType::QRegExp	&27	&QRegExp \\
\hline
QMetaType::QRegularExpression	&44	&QRegularExpression \\
\hline
QMetaType::QDateTime	&16&	QDateTime\\
\hline
QMetaType::QPointF	&26&	QPointF\\
\hline
QMetaType::QPalette	&68	&QPalette\\
\hline
QMetaType::QFont	&64&	QFont\\
\hline
QMetaType::QBrush	&66	&QBrush\\
\hline
QMetaType::QRegion&	72&	QRegion\\
\hline
QMetaType::QBitArray&	13	&QBitArray\\
\hline
QMetaType::QImage	&70	&QImage\\
\hline
QMetaType::QKeySequence	&75	&QKeySequence\\
\hline
QMetaType::QSizePolicy&	121&	QSizePolicy\\
\hline
QMetaType::QPixmap&	65	&QPixmap\\
\hline
QMetaType::QLocale&	18	&QLocale\\
\hline
QMetaType::QBitmap&	73&	QBitmap\\
\hline
QMetaType::QMatrix&	79&	QMatrix\\
\hline
QMetaType::QTransform	&80	&QTransform\\
\hline
QMetaType::QMatrix4x4	&81	&QMatrix4x4\\
\hline
QMetaType::QVector2D	&82	&QVector2D\\
\hline
QMetaType::QVector3D	&83	&QVector3D\\
\hline
QMetaType::QVector4D	&84	&QVector4D\\
\hline
QMetaType::QQuaternion&	85	&QQuaternion\\
\hline
QMetaType::QEasingCurve	&29&	QEasingCurve\\
\hline
QMetaType::QJsonValue	&45&	QJsonValue\\
\hline
QMetaType::QJsonObject&	46&	QJsonObject\\
\hline
QMetaType::QJsonArray	&47	&QJsonArray\\
\hline
QMetaType::QJsonDocument	&48&	QJsonDocument\\
\hline
QMetaType::QCborValue	&53&	QCborValue\\
\hline
QMetaType::QCborArray	&54	&QCborArray \\
\hline
QMetaType::QCborMap	&55&	QCborMap \\
\hline
QMetaType::QCborSimpleType	&52&	QCborSimpleType \\
\hline
QMetaType::QModelIndex&	42	&QModelIndex \\
\hline
QMetaType::QPersistentModelIndex&	50	&QPersistentModelIndex(在 Qt 5.5 中被引入) \\
\hline
QMetaType::QUuid	&30	&QUuid \\
\hline
QMetaType::QByteArrayList&	49&	QByteArrayList \\
\hline
QMetaType::User	&1024&	用户类型的基础值(译者注:即起始值)\\
\hline
QMetaType::UnknownType&	0	&这是无效的类型编号,QMetaType 会在类型未注册时返回此值。\\
\hline
\end{longtable}

可以使用 Q\_DECLARE\_METATYPE() 注册额外的类型。

\begin{seeAlso}
type() 和 typeName()。
\end{seeAlso}

\splitLine

enum QMetaType::TypeFlag

flags QMetaType::TypeFlags

此枚举类型描述了被 QMetaType 支持的类型的属性

\begin{tabular}{|l|l|l|}
\hline
常量 	&数值 &	描述\\
\hline
QMetaType::NeedsConstruction &	0x1 &	此类型具有非平凡的构造函数。若某类型不具备此标志,则可通过 memset() 安全地清零。\\
\hline
QMetaType::NeedsDestruction &	0x2 	&此类型非平凡的析构函数。若某类型不具备此标志,则丢弃对象前不需要调用析构函数(译者注:即可以用 free() 释放对象)\\
\hline
QMetaType::MovableType &	0x4 &	具有此标志的类型实例可以通过 memcpy() 安全地移动。\\
\hline
QMetaType::IsEnumeration 	&0x10 &	此类型是枚举值。\\
\hline
QMetaType::PointerToQObject &	0x8 &	此类型是指向继承自 QObject 的类型的指针。\\
\hline
\end{tabular}

\hl{TypeFlags} 类型是 QFlags<TypeFlag> 的别名,支持通过或操作合并不同的 \hl{TypeFlag} 值。

\section{成员函数文档}

QMetaType::QMetaType(const int typeId = QMetaType::UnknownType)

构造一个包含 typeId 对应的类型信息的 QMetaType 对象。

注意: 默认参数在 Qt 5.15 中被引入。

此函数在 Qt 5.0 中被引入。

QMetaType::$\sim$QMetaType()

析构此对象。

[static] bool QMetaType::compare(const void \emph{*lhs}, const void \emph{*rhs}, int typeId, int \emph{*result})

比较 lhs 和 rhs 对象,双方都需要是 typeid 中的类型。result 会被设为小于、等于或大于零,表示 lhs 小于、等于或大于 rhs。若比较成功则返回 true,否则返回 false。

此函数在 Qt 5.2 中被引入。

[static] void *QMetaType::construct(int \emph{type}, void \emph{*where}, const void \emph{*copy})

在给定的内存地址 where 上构造对应 type 类型的对象,该对象是 copy 的副本,并返回 where。若 copy 是空指针,则执行默认构造。

这是用于显示管理存储 type 类型对象的内存的底层函数。若不需要此类底层控制,则考虑使用 create() 函数(也就是指,使用 new 而非 placement new)。

您必须确保 where 指向的内存区域大小足够存储 type 对应的数据,并且 where 地址需要对齐,对应类型的大小可通过 sizeOf() 获取。

内存对齐的规则是对齐至类型的自然边界,也就是大于等于类型大小的2的n次方值,直至平台有效对齐宽度上限为止。对于特定用途来说,超过 2 * sizeof(void*) 的对齐宽度只是某些特定硬件指令所必需的(例如,x86 平台中对齐后的 SSE 读取和存储)。

此函数在 Qt 5.0 中被引入。

\begin{seeAlso}
destruct() 和 sizeOf()。
\end{seeAlso}

void *QMetaType::construct(void \emph{*where}, const void \emph{*copy} = 0) const

在给定的内存地址 where 上构造此 QMetaType 类型的对象,
该对象是 copy 的副本,并返回 where。
若 copy 是空指针,则执行默认构造。

这是用于显示管理存储 type 类型对象的内存的底层函数。
若不需要此类底层控制,则考虑使用 create() 函数(也就是指,使用 new 而非 placement new)。

您必须确保 where 指向的内存区域大小足够存储 type 对应的数据,
并且 where 地址需要对齐,对应类型的大小可通过 sizeOf() 获取。

内存对齐的规则是对齐至类型的自然边界,也就是大于等于类型大小的2的n次方值
,直至平台有效对齐宽度上限为止。对于特定用途来说,超过 2 * sizeof(void*) 的对齐宽度只是某些特定硬件指令所必需的(例如,x86 平台中对齐后的 SSE 读取和存储)。

此函数在 Qt 5.0 中被引入。

[static] bool QMetaType::convert(const void \emph{*from}, int \emph{fromTypeId}, void \emph{*to}, int \emph{toTypeId})

将 from 对象从 fromTypeId 转换至 toTypeId 并存储到预分配空间 to 中。若转换成功则返回 true,否则返回 false。

此函数在 Qt 5.2 中被引入。

[static] void *QMetaType::create(int \emph{type}, const void \emph{*copy} = nullptr)

假设 copy 的类型是 type,返回它的的拷贝。若 copy 是空指针,则返回默认构造的实例。

\begin{seeAlso}
destroy(),isRegistered() 和 Type。
\end{seeAlso}

void *QMetaType::create(const void \emph{*copy} = 0) const

假设 copy 的类型是此 QMetaType ,返回它的的拷贝。若 copy 是空指针,则返回默认构造的实例。

此函数在 Qt 5.0 中被引入。

\begin{seeAlso}
QMetaType::destroy()。
\end{seeAlso}

[static] bool QMetaType::debugStream(QDebug \emph{\&dbg}, const void \emph{*rhs}, int \emph{typeId})

将 typeId 类型的 rhs 对象输出至调试流 debug,输出成功则返回 true,否则返回 false。

此函数在 Qt 5.2 中被引入。

[static] void QMetaType::destroy(int \emph{type}, void \emph{*data})

假设 data 的类型是 type,销毁该对象。

\begin{seeAlso}
create(),isRegistered() 和 Type。
\end{seeAlso}

void QMetaType::destroy(void \emph{*data}) const

假设 data 的类型是此 QMetaType ,销毁该对象。

此函数在 Qt 5.0 中被引入。

\begin{seeAlso}
QMetaType::create()。
\end{seeAlso}

[static] void QMetaType::destruct(int \emph{type}, void \emph{*where})

假设 where 地址中存储的对象类型是 type,销毁该对象。

与 destroy() 不同,此函数只会执行该类型的析构函数,但不会执行 delete 运算符(译者注:即不会释放内存,与 placement new 相同机制)。

此函数在 Qt 5.0 中被引入。

\begin{seeAlso}
construct()。
\end{seeAlso}

void QMetaType::destruct(void \emph{*data}) const

假设 data 地址中存储的对象类型是此 QMetaType ,销毁该对象。

与 destroy() 不同,此函数只会执行该类型的析构函数,但不会执行 delete 运算符(译者注:即不会释放内存,与 placement new 相同机制)。

此函数在 Qt 5.0 中被引入。

\begin{seeAlso}
QMetaType::construct()。
\end{seeAlso}

[static] bool QMetaType::equals(const void \emph{*lhs}, const void \emph{*rhs}, int \emph{typeId}, int \emph{*result})

比较 lhs 和 rhs 对象,双方都需要是 typeid 中的类型。若 lhs 等于 rhs,则 result 会被设为零。若比较成功则返回 true,否则返回 false。

此函数在 Qt 5.5 中被引入。

QMetaType::TypeFlags QMetaType::flags() const

返回此 QMetaType 实例的类型标志。

此函数在 Qt 5.0 中被引入。

\begin{seeAlso}
QMetaType::TypeFlags 和 QMetaType::typeFlags()。
\end{seeAlso}

[static] template <typename T> QMetaType QMetaType::fromType()

返回模板类型 T 对应的 QMetaType 实例。

此函数在 Qt 5.15 中被引入。

[static] template <typename T> bool QMetaType::hasRegisteredComparators()

若模板类型 T 已被注册至元对象系统则返回 true。

此函数在 Qt 5.2 中被引入。

[static] bool QMetaType::hasRegisteredComparators(int \emph{typeId})

若 typeId 的类型已被注册至元对象系统则返回 true。

此函数在 Qt 5.2 中被引入。

[static] bool QMetaType::hasRegisteredConverterFunction(int \emph{fromTypeId}, int \emph{toTypeId})

若自 fromTypeId 到 toTypeId 的类型转换已被注册至元对象系统则返回 true。

此函数在 Qt 5.2 中被引入。

[static] template <typename From, typename To> bool QMetaType::hasRegisteredConverterFunction()

若自模板类型 From 到 To 的类型转换已被注册至元对象系统则返回 true。

这是一个重载函数。

此函数在 Qt 5.2 中被引入。

[static] template <typename T> bool QMetaType::hasRegisteredDebugStreamOperator()

若自模板类型 T 的 QDebug 流运算符已被注册至元对象系统则返回 true。

此函数在 Qt 5.2 中被引入。

[static] bool QMetaType::hasRegisteredDebugStreamOperator(int \emph{typeId})

若自 typeId 对应类型的 QDebug 流运算符已被注册至元对象系统则返回 true。

此函数在 Qt 5.2 中被引入。

int QMetaType::id() const

返回此 QMetatype 实例的类型编号。

此函数在 Qt 5.13 中被引入。

[static] bool QMetaType::isRegistered(int \emph{type})

若 typeId 对应已被注册至元对象系统则返回 true,否则返回 false。

\begin{seeAlso}
type(),typeName() 和 Type。
\end{seeAlso}

bool QMetaType::isRegistered() const

若此 QMetaType 包含某类型的有效信息则返回 true,否则返回 false。

此函数在 Qt 5.0 中被引入。

bool QMetaType::isValid() const

若此 QMetaType 包含某类型的有效信息则返回 true,否则返回 false。

此函数在 Qt 5.0 中被引入。

[static] bool QMetaType::load(QDataStream \emph{\&stream}, int \emph{type}, void \emph{*data})

从数据流 stream 中读取对应 type 类型的对象至 data 中,若读取成功则返回 true,否则返回 false。

此类型必须在这之前通过 qRegisterMetaType() 和 qRegisterMetaTypeStreamOperators() 完成注册。

通常来说,您不需要显示调用此函数,而是应使用 QVariant 的 operator>>(),该运算符依赖 load() 来传递自定义类型。

\begin{seeAlso}
save() 和 qRegisterMetaTypeStreamOperators()。
\end{seeAlso}

const QMetaObject *QMetaType::metaObject() const

返回此类型对应的 QMetaObject。

若此类型是 QObject 子类的指针,即 flags() 包含 QMetaType::PointerToQObject,则此函数会返回对应类型的 QMetaObject。这可被用于结合 QMetaObject::construct(译者注:无此函数,请使用 QMetaObject::constructor 或 QMetaType::construct)来创建此类型的 QObject 实例。

若此类型是 Q\_GADGET,即 flags() 包含 QMetaType::IsGadget(译者注:文档中未给出,但 QMetaType::TypeFlag 中的确包含此枚举值),则此函数会返回对应类型的 QMetaObject。
这可以被用于获取 QMetaMethod 和 QMetaProperty,并将其用于此类型的对象指针上(例如通过 QVariant::data 获取指针 译者注:文档中无此函数,但此函数的确存在)。

若此类型是枚举,即 flags() 包含 QMetaType::IsEnumeration,
且该枚举值是通过 Q\_ENUM 注册的成员枚举类型,
则此函数会返回其所属的 QObject 对象的元对象,否则返回 nullptr。

此函数在 Qt 5.5 中被引入。

\begin{seeAlso}
QMetaType::metaObjectForType() 和 QMetaType::flags()。
\end{seeAlso}

[static] const QMetaObject *QMetaType::metaObjectForType(int \emph{type})

返回 type 类型对应的 QMetaType::metaObject。

此函数在 Qt 5.0 中被引入。

\begin{seeAlso}
metaObject()。
\end{seeAlso}

QByteArray QMetaType::name() const

返回此 QMetaType 对应的类型名称,若无有效类型则返回空指针。

此函数在 Qt 5.15 中被引入。

\begin{seeAlso}
typeName()。
\end{seeAlso}

[static] template <typename T> bool QMetaType::registerComparators()

将用户注册类型 T 的比较运算符注册至元对象系统。要求 T 具有 operator== 和 operator< 运算符。若注册成功则返回 true,否则返回 false。

此函数在 Qt 5.2 中被引入。

[static] template <typename From, typename To> bool QMetaType::registerConverter()

将类型 From 到 To 的可能的隐式转换注册到元对象系统,若注册成功则返回 true,否则返回 false。

此函数在 Qt 5.2 中被引入。

[static] template <typename MemberFunction, int> bool QMetaType::registerConverter(MemberFunction \emph{function})

这是一个重载函数。

将形如 To From::function() const 的成员方法 function 作为从 From 到 To 的转换函数注册至元对象系统,若注册成功则返回 true,否则返回 false。

此函数在 Qt 5.2 中被引入。

\begin{quote}
译者注:

第二个模板参数是官方使用 doxygen 生成文档时的变通写法,
实际代码中的函数签名是 \hl{template<typename From, typename To> static bool registerConverter(To(From::*function)() const)}。
使用时无需指定 \hl{int} 模板参数,在函数参数中直接填入用于转换的成员函数指针即可。
\end{quote}

[static] template <typename MemberFunctionOk, char> bool QMetaType::registerConverter(MemberFunctionOk \emph{function})

这是一个重载函数。

将形如 To From::function(bool *ok) const 的成员方法 function 作为从 From 到 To 的转换函数注册至元对象系统,若注册成功则返回 true,否则返回 false。

此函数在 Qt 5.2 中被引入。

\begin{quote}
译者注:

第二个模板参数是官方使用 doxygen 生成文档时的变通写法,
实际代码中的函数签名是 \hl{template<typename From, typename To> static bool registerConverter(To(From::*function)(bool*) const)}。
使用时无需指定 \hl{char} 模板参数,在函数参数中直接填入用于转换的成员函数指针即可。
\end{quote}


[static] template <typename UnaryFunction> bool QMetaType::registerConverter(UnaryFunction \emph{function})

这是一个重载函数。

将把类型 From 转换为类型 To 的一元函数 function 注册至元对象系统,若注册成功则返回 true,否则返回 false。

此函数在 Qt 5.2 中被引入。

\begin{quote}
译者注:

原文描述地非常晦涩,实际指的是任何可被 \hl{To dst = function(src)} 方式调用的函数对象,
包括全局函数、类静态函数、仿函数或 lamba 等,比上文另外两个 \hl{registerConverter} 的约束更为宽松。
\end{quote}


%%%%%%%%%%%%%%%%
[static] template <typename T> bool QMetaType::registerDebugStreamOperator()

将已注册类型 T 的 QDebug 流运算符注册至元对象系统,
要求类型 T 具备流运算符 operator<<(QDebug dbg, T)。
若注册成功则返回 true,否则返回 false。

[static] template <typename T> bool QMetaType::registerEqualsComparator()

将已注册类型 T 的等号运算符注册至元对象系统,要求类型 T 具备等号运算符 operator==。
若注册成功则返回 true,否则返回 false。

此函数在 Qt 5.5 中被引入。

[static] bool QMetaType::save(QDataStream \emph{\&stream}, int \emph{type}, const void \emph{*data})

从数据流 stream 中读取对应 type 类型的对象至 data 中,若读取成功则返回 true,否则返回 false。

此类型必须在这之前通过 qRegisterMetaType() 和 qRegisterMetaTypeStreamOperators() 完成注册。

通常来说,您不需要显示调用此函数,而是应使用 QVariant 的 operator>>(),
该运算符依赖 load() 来传递自定义类型。

\begin{seeAlso}
save() 和 qRegisterMetaTypeStreamOperators()。
\end{seeAlso}

将 type 类型对应的 data 对象输出至数据流 stream 中,若读取成功则返回 true,否则返回 false。

此类型必须在这之前通过 qRegisterMetaType() 和 qRegisterMetaTypeStreamOperators() 完成注册。

通常来说,您不需要显示调用此函数,而是应使用 QVariant 的 operator<<(),
该运算符依赖 save() 来传递自定义类型。

\begin{seeAlso}
load() 和 qRegisterMetaTypeStreamOperators()。
\end{seeAlso}

[static] int QMetaType::sizeOf(int \emph{type})

返回 type 对应类型的以字节为单位的大小(即 sizeof(T),其中 T 是 type 对应的实际类型)。

此函数通常结合 construct() 使用,来进行对此类型的更底层的内存管理。

此函数在 Qt 5.0 中被引入。

\begin{seeAlso}
construct()。
\end{seeAlso}

int QMetaType::sizeOf() const

返回此类型的以字节为单位的大小(即 sizeof(T),其中 T 是 QMetaType 对应的实际类型)。

此函数通常结合 construct() 使用,来进行对此类型的更底层的内存管理。

此函数在 Qt 5.0 中被引入。

\begin{seeAlso}
QMetaType::construct() 和 QMetaType::sizeOf()。
\end{seeAlso}

[static] int QMetaType::type(const char \emph{*typeName})

返回名为 typeName 的类型的元类型编号,若无此元类型则返回 QMetaType::UnknownType。

\begin{seeAlso}
isRegistered(),typeName() 和 Type。
\end{seeAlso}

[static] int QMetaType::type(const::QByteArray \emph{\&typeName})

这是一个重载函数。

返回名为 typeName 的类型的元类型编号,若无此元类型则返回 0(译者注:即QMetaType::UnknownType)。

此函数在 Qt 5.5 中被引入。

\begin{seeAlso}
isRegistered() 和 typeName()。
\end{seeAlso}

[static] QMetaType::TypeFlags QMetaType::typeFlags(int \emph{type})

返回 type 类型的类型标志。

此函数在 Qt 5.0 中被引入。

\begin{seeAlso}
QMetaType::TypeFlags。
\end{seeAlso}

[static] const char *QMetaType::typeName(int \emph{typeId})

返回 typeId 对应类型的类型名称,若该类型不存在则返回空指针。返回的指针不可被删除。

\begin{seeAlso}
type(),isRegistered(),Type 和 name()。
\end{seeAlso}

\section{相关非成员函数}

template <typename T> int qMetaTypeId()

返回类型 T 对应的元类型编号。
若该类型未通过 Q\_DECLARE\_METATYPE() 声明,则会引发编译错误。

典型用法:

\begin{lstlisting}[language=C++]
int id = qMetaTypeId<QString>();    // id 是 QMetaType::QString
id = qMetaTypeId<MyStruct>();       // 若 MyStruct 未被声明,则会产生编译错误
\end{lstlisting}

QMetaType::type() 返回值与 qMetaTypeId() 相同,
但会基于类型名称进行运行时检索。
QMetaType::type() 会稍慢一些,但即使类型未注册也能编译成功。

此函数在 Qt 4.1 中被引入。

\begin{seeAlso}
Q\_DECLARE\_METATYPE() 和 QMetaType::type()。
\end{seeAlso}

template <typename T> int qRegisterMetaType(const char \emph{*typeName})

将类型 T 通过类型名称 typeName 注册至元对象系统,
并返回 QMetaType 使用的类型编号。
任何包含一个公共默认构造函数、公共拷贝构造函数、公共析构函数的类或结构体均可被注册。

此函数要求类型 T 在此函数调用时被完整定义;对于指针类型,
同样要求被指向的类型被完整定义(\hl{译者注:即不可为前置声明类型})。
可以使用 Q\_DECLARE\_OPAQUE\_POINTER() 来注册前置声明类型的指针类型。

类型被注册后,可以在运行时动态地创建或销毁对象。

下述为注册 MyClass 类的示例:

\begin{lstlisting}[language=C++]
qRegisterMetaType<MyClass>("MyClass");
\end{lstlisting}

此函数可被用于注册类型别名,以便于将别名用于 QMetaProperty 或队列连接中。

\begin{lstlisting}[language=C++]
typedef QString CustomString;
qRegisterMetaType<CustomString>("CustomString");
\end{lstlisting}

\begin{warning}
此函数仅应被用于注册类型别名,其它场合请使用 Q\_DECLARE\_METATYPE 和 qMetaTypeId()。
\end{warning}

\begin{seeAlso}
qRegisterMetaTypeStreamOperators(),isRegistered() 和 Q\_DECLARE\_METATYPE()。
\end{seeAlso}

template <typename T> int qRegisterMetaType()

调用此函数来注册类型 T。
T 必须被 Q\_DECLARE\_METATYPE() 所声明。
返回此类型对应的元类型编号。

示例:

\begin{lstlisting}[language=C++]
int id = qRegisterMetaType<MyStruct>();
\end{lstlisting}

此函数要求类型 T 在此函数调用时被完整定义;
对于指针类型,同样要求被指向的类型被完整定义(\hl{译者注:即不可为前置声明类型})。
可以使用 Q\_DECLARE\_OPAQUE\_POINTER() 来注册前置声明类型的指针类型。

类型被注册后,可以在运行时动态地创建或销毁对象。

为了在 QVariant 中使用类型 T,使用 Q\_DECLARE\_METATYPE() 便已足够。
若要在信号槽的队列连接中使用 T,
则 qRegisterMetaType<T>() 必须在第一个连接建立前被调用。

同样地,若要在 QObject::property() 中使用 T,qRegisterMetaType<T>() 必须在这之前被调用。通常在使用到 T 的类的构造函数中,或在 main() 函数中调用。

此函数在 Qt 4.2 中被引入。

\begin{seeAlso}
Q\_DECLARE\_METATYPE()。
\end{seeAlso}

template <typename T> void qRegisterMetaTypeStreamOperators(const char \emph{*typeName})

通过类型名称 typeName 将 T 的流运算符注册至元对象系统。

在此之后,该类型可通过 QMetaType::load() 和 QMetaType::save() 进行序列化和反序列化。
这两个函数在将 QVariant 传递至数据流时被调用。

\begin{lstlisting}[language=C++]
qRegisterMetaTypeStreamOperators<MyClass>("MyClass");
\end{lstlisting}

流运算符需要具有下述的函数签名:

\begin{lstlisting}[language=C++]
QDataStream &operator<<(QDataStream &out, const MyClass &myObj);
QDataStream &operator>>(QDataStream &in, MyClass &myObj);
\end{lstlisting}

%%%%%%%%%%%%

\begin{seeAlso}
qRegisterMetaType(),QMetaType::isRegistered() 和 Q\_DECLARE\_METATYPE()。
\end{seeAlso}

bool operator!=(const QMetaType \emph{\&a}, const QMetaType \emph{\&b})

这是一个重载函数。

若 QMetaType a 的类型与 QMetaType b 不同则返回 true,否则返回false。

此函数在 Qt 5.15 中被引入。

bool operator==(const QMetaType \emph{\&a}, const QMetaType \emph{\&b})

这是一个重载函数。

若 QMetaType \emph{a} 的类型与 QMetaType \emph{b} 相同则返回 \hl{true},否则返回 \hl{false}。

此函数在 Qt 5.15 中被引入。

%%%%%%%%%%%%


\section{宏定义文档}

Q\_DECLARE\_ASSOCIATIVE\_CONTAINER\_METATYPE(Container)

此宏令容器类型 \hl{Container} 作为关联型容器被注册至 QMetaType,
即允许将 \hl{Container<T, U>} 实例存入 QVariant,
前提是 \hl{T} 和 \hl{U} 也已经被注册为 QMetaType。

\begin{warning}
所有 Qt 的关联型容器已被内置支持,
无需使用此宏进行声明。std::map 容器也已被内置支持。
\end{warning}

下述代码展示了 Q\_DECLARE\_ASSOCIATIVE\_CONTAINER\_METATYPE() 的典型用法:

\begin{lstlisting}[language=C++]
#include <unordered_list>

Q_DECLARE_ASSOCIATIVE_CONTAINER_METATYPE(std::unordered_map)

void someFunc()
{
    std::unordered_map<int, bool> container;
    QVariant var = QVariant::fromValue(container);
    // ...
}
\end{lstlisting}

\begin{quote}
译者注:

用户的自定义类型只需要通过 \hl{Q\_DECLARE\_METATYPE(T)} 注册后,
即可被已注册的所有容器使用,
无需再注册 \hl{Q\_DECLARE\_METATYPE(QMap<QString, T>)}。
\end{quote}

Q\_DECLARE\_METATYPE(\emph{Type})

此宏将类型 \hl{Type} 注册至 QMetaType ,前提是该类型具备一个公共默认构造函数、公共拷贝构造函数和公共析构函数。这是把类型 Type 用于 QVariant 的前提。

此宏要求类型 \hl{T} 在此函数调用时被完整定义;
对于指针类型,同样要求被指向的类型被完整定义(\hl{译者注:即不可为前置声明类型})。
可以使用 Q\_DECLARE\_OPAQUE\_POINTER() 来注册前置声明类型的指针类型。

理想情况下,此宏应被放置在该类型的声明位置之后。若不可行的话,
也可以将其放置在一个私有头文件中,
然后在每次在 QVariant 中使用此类型之前包含该头文件。

Q\_DECLARE\_METATYPE() 使此类型可被所有基于模板的函数使用,
包括 QVariant 中的模板函数。

\begin{notice}
若想在信号槽的队列连接或 QObject 的属性系统中使用此类型,
则还需要调用 qRegisterMetaType(),因为该类型的名称会在运行时被解析。
\end{notice}

此示例为 \hl{Q\_DECLARE\_METATYPE()} 的典型用法:

\begin{lstlisting}[language=C++]
struct MyStruct
{
    int i;
    ...
};

Q_DECLARE_METATYPE(MyStruct)
\end{lstlisting}

若 \hl{MyStruct} 处于命名空间中,则 \hl{Q\_DECLARE\_METATYPE()} 宏必须在命令空间外使用:

\begin{lstlisting}[language=C++]
namespace MyNamespace
{
    ...
}

Q_DECLARE_METATYPE(MyNamespace::MyStruct)
\end{lstlisting}

当 \hl{MyStruct} 被注册至 QMetaType 后,便可将其用于 QVariant 中

\begin{lstlisting}[language=C++]
MyStruct s;
QVariant var;
var.setValue(s); // 将 v 拷贝至 QVariant

...

// 获取类型值
MyStruct s2 = var.value<MyStruct>();
\end{lstlisting}

下述类型已被自动注册,无需使用此宏:

\begin{compactitem}[\arr]
\item 继承自 QObject 的类型的指针;
\item QList<T>,QVector<T>,QQueue<T>,QStack<T>,QSet<T> 或 QLinkedList<T>,前提是 T 被注册为元类型;
\item QHash<T1, T2>,QMap<T1, T2> 或 QPair<T1, T2>,前提是 T1 和 T2 都被注册为元类型;
\item QPointer<T>,QSharedPointer<T>,QWeakPointer<T>,前提是 T 为 QObject 的子类;
\item 通过 Q\_ENUM 或 Q\_FLAG 注册的枚举类型;
\item 包含 Q\_GADGET 宏的类。
\end{compactitem}

\begin{seeAlso}
qRegisterMetaType()。
\end{seeAlso}

%%%%%%%%%%%%%%%%

Q\_DECLARE\_OPAQUE\_POINTER(\hl{PointerType})

此宏使得前置声明类型的指针类型 PointerType 可被 Q\_DECLARE\_METATYPE() 或 qRegisterMetaType() 注册至 QMetaType。

此函数在 Qt 5.0 中被引入。

\begin{seeAlso}
Q\_DECLARE\_METATYPE() 和 qRegisterMetaType()。
\end{seeAlso}

Q\_DECLARE\_SEQUENTIAL\_CONTAINER\_METATYPE(\emph{Container})

此宏令容器类型 \hl{Container} 作为顺序容器被注册至 QMetaType,
即允许将 \hl{Container<T>} 实例存入 QVariant,
前提是 \hl{T} 已经被注册为 QMetaType。

\begin{notice}
所有 Qt 的顺序容器已被内置支持,无需使用此宏进行声明。
\end{notice}

\hl{std::vector} 和 \hl{std::list} 容器已也被内置支持。

下述代码展示了 \hl{Q\_DECLARE\_SEQUENTIAL\_CONTAINER\_METATYPE()} 的典型用法:

\begin{lstlisting}[language=C++]
#include <deque>

Q_DECLARE_SEQUENTIAL_CONTAINER_METATYPE(std::deque)

void someFunc()
{
    std::deque<QFile*> container;
    QVariant var = QVariant::fromValue(container);
    // ...
}
\end{lstlisting}

\begin{quote}
译者注:

用户的自定义类型只需要通过 \hl{Q\_DECLARE\_METATYPE(T)} 注册后,
即可被已注册的所有容器使用,
无需再注册 \hl{Q\_DECLARE\_METATYPE(QVector<T>)}。
\end{quote}

Q\_DECLARE\_SMART\_POINTER\_METATYPE(\emph{SmartPointer})

此宏令智能指针类型 \hl{SmartPointer} 作为智能指针被注册至 QMetaType,
即允许将 \hl{Container<T>} 实例存入 QVariant,
前提是 T 已经被注册为 QMetaType。

\begin{notice}
QWeakPointer,QSharedPointer 和 QPointer 已被内置支持,
无需使用此宏进行声明。
\end{notice}

下述代码展示了 \hl{Q\_DECLARE\_SMART\_POINTER\_METATYPE()} 的典型用法:

\begin{lstlisting}[language=C++]
#include <memory>

Q_DECLARE_SMART_POINTER_METATYPE(std::shared_ptr)

void someFunc()
{
    auto smart_ptr = std::make_shared<QFile>();
    QVariant var = QVariant::fromValue(smart_ptr);
    // ...
    if (var.canConvert<QObject*>()) {
        QObject *sp = var.value<QObject*>();
        qDebug() << sp->metaObject()->className(); // Prints 'QFile'.
    }
}
\end{lstlisting}

\begin{quote}
译者注:

用户继承自 QObject 的自定义类型可直接被已注册的智能指针使用,
无需再注册 \hl{Q\_DECLARE\_METATYPE(QSharedPointer<T>)}。

与容器不同的是,通过 \hl{Q\_DECLARE\_METATYPE(T)} 
注册的自定义类型无法直接被已注册的智能指针使用,
必须单独注册 \hl{Q\_DECLARE\_METATYPE(QSharedPointer<T>)}。
\end{quote}

\section{已废弃成员}

QMetaType 类的以下成员已被废弃。
它们仅为了保证老代码能够运行而保留,
我们强烈反对在新代码中使用。

\subsection{静态公共成员}

\begin{tabular}{|l|l|}
\hline
返回类型 	函数 \\
\hline
\hl{(obsolete)} void * 	& construct(int \emph{type}, const void \emph{*copy} = nullptr) \\
\end{tabular}

\subsection{成员函数文档}

\emph{[static]} void *QMetaType::construct(int \emph{type}, const void \emph{*copy} = nullptr)

在给定的内存地址 \hl{where} 上构造对应 \hl{type} 类型的对象,
该对象是 \hl{copy} 的副本,并返回 \hl{where}。若 \hl{copy} 是空指针,
则执行默认构造。

这是用于显示管理存储 hl{type} 类型对象的内存的底层函数。
若不需要此类底层控制,则考虑使用 create() 函数(也就是指,
使用 \hl{new} 而非 \hl{placement new})。

您必须确保 \hl{where} 指向的内存区域大小足够存储 \hl{type} 对应的数据,
并且 \hl{where} 地址需要对齐,对应类型的大小可通过 sizeOf() 获取。

内存对齐的规则是对齐至类型的自然边界,也就是大于等于类型大小的2的n次方值,
直至平台有效对齐宽度上限为止。对于特定用途来说,
超过 \hl{2 * sizeof(void*)} 的对齐宽度只是某些特定硬件指令所必需的
(例如,x86 平台中对齐后的 SSE 读取和存储)。

此函数在 Qt 5.0 中被引入。

此函数已被废弃,仅为了保证老代码能够运行而保留,
我们强烈反对在新代码中使用。

构造对应 \hl{type} 类型的对象,该对象是 \hl{copy} 的副本。
copy 的默认值是 \hl{nullptr}。

已弃用,该用新的静态函数 QMetaType::create(int type, const void *copy)。
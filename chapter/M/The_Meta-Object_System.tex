\chapter{元对象系统}

Qt 的元对象系统提供了对象间通信的信号槽机制、运行时类型信息,以及动态属性系统。

元对象系统基于以下三者:

\begin{compactenum}
\item QObject 类,提供了便于利用元对象系统的基类;
\item Q\_OBJECT 宏,放置于类声明的私有域,用于激活元对象系统特性,例如动态属性和信号槽;
\item 元对象编译器 (moc),为每个 QObject 的子类提供实现元对象特性的代码生成。
\end{compactenum}

moc 工具读取 C++ 源文件,若在其中找到包含 Q\_OBJECT 宏的类声明,
则会创建另一个 C++ 源文件,并在其中填充用于实现元对象得的代码。
该生成的源文件需要通过 \#include' 包含至对应类的源文件,
或者更常见的是将其加入编译列表,并于对应类的实现一同链接。

\begin{compactitem}[\arr]
\item QObject::metaObject() 返回该类对应的 元对象。
\item QMetaObject::className() 在运行时以字符串形式返回该类的类名,并且不需要依赖 C++ 编译器的运行时类型信息(RTTI)支持;
\item QObject::inherits() 函数返回该对象所属类型,是否派生自 QObject 继承树中的指定类型;
\item QObject::tr() 和 QObject::trUtf8() 为 国际化 支持提供字符串翻译;
\item QObject::setProperty() 和 QObject::property() 用于通过属性名称,动态设置和读取属性值;
\item QMetaObject::newInstance() 构建指定类的新实例。
\end{compactitem}

我们还可以对 QObject 进行 qobject\_cast() 操作,
该函数与标准 C++ 的 dynamic\_cast() 表现类似,
但优点时不需要 RTTI 支持,并且可以跨越动态库边界运作。
它会尝试将输入指针转换为尖括号中的指针类型,
若类型正确则返回非空指针(在运行时作出判断),
若对象类型不兼容则返回 nullptr。

例如,假设 MyWidget 继承自 QWidget 类,并声明了 Q\_OBJECT 宏:

\begin{lstlisting}[language=C++]
QObject *obj = new MyWidget;
\end{lstlisting}

QObject * 类型的变量 obj 实际指向一个 MyWidget 对象,于是我们可以进行如下转换:

\begin{lstlisting}[language=C++]
QWidget *widget = qobject_cast<QWidget *>(obj);
\end{lstlisting}

从 QObject 到 QWidget 的转换成功进行,
因为该对象实际是 QWidget 的子类 MyWidget。
由于我们知道 obj 是 MyWidget 类型,
我们可以将其转换为 MyWidget *:

\begin{lstlisting}[language=C++]
MyWidget *myWidget = qobject_cast<MyWidget *>(obj);
\end{lstlisting}

转换至 MyWidget 的操作可以成功进行,因为 qobject\_cast() 并不会将 Qt 内置类型和自定义类型区别对待。
(译者注:Qt 是在运行时通过读取元对象信息进行动态转换,开发者可通过 Q\_OBJECT 宏让自定义类型支持被 qobject\_cast() 进行转换)

\begin{lstlisting}[language=C++]
QLabel *label = qobject_cast<QLabel *>(obj);
// label is 0
\end{lstlisting}

另一个例子,转换为 QLabel 的操作失败了,该指针会被置零。但此机制也让运行时基于转换结果来区别处理不同类型成为可能:

\begin{lstlisting}[language=C++]
if (QLabel *label = qobject_cast<QLabel *>(obj)) {
    label->setText(tr("Ping"));
} else if (QPushButton *button = qobject_cast<QPushButton *>(obj)) {
    button->setText(tr("Pong!"));
}
\end{lstlisting}

虽然可以使用 QObject 作为基类,
但不定义 Q\_OBJECT 宏,也不生成元对象代码,
但这也意味着信号槽以及本文提到所有的其它机制无法使用。
从元对象系统的视角来看,没有元对象代码的 QObject 的子类
(译者注:即未使用 Q\_OBJECT 宏)等价于它最近的一个包含元对象代码的父类,
这也意味着,例如,QMetaObject::className() 不会返回该类的类名,而是会返回父类的类名。

因此,我们强烈建议在所有 QObject 的子类中都使用 Q\_OBJECT 宏,
无论它们是否用到了信号槽和动态属性。

\begin{seeAlso}
QMetaObject,Qt 的属性系统 以及 信号与槽。
\end{seeAlso}
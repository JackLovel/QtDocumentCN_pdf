\chapter{QMainWindow}

QMainWindow 类用于创建主程序窗口。 更多内容...


\begin{tabular}{|r|l|}
	\hline
	属性 & 方法 \\
	\hline
	头文件 & \#include <QMainWindow>\\      
	\hline
	qmake & QT += widgets\\      
	\hline
	继承: &	QWidget\\
	\hline
\end{tabular}

\begin{compactitem}
\item 列出所有成员函数, 包括继承的成员函数
\end{compactitem}

\section{公共成员类型}

\begin{tabular}{|r|m{25em}|}
	\hline
	类型 & 方法 \\
	\hline
    enum &	DockOption \{ AnimatedDocks, AllowNestedDocks, AllowTabbedDocks, ForceTabbedDocks, VerticalTabs, GroupedDragging \} \\ 
    \hline
    flags &	DockOptions \\ 
	\hline
\end{tabular}

\section{属性}

\begin{tabular}{|r|l|}
	\hline
    属性名 &	类型 \\ 
    \hline
    animated 	& bool \\ 
    \hline
    dockNestingEnabled &	bool\\
    \hline
    dockOptions &	DockOptions\\
    \hline
    documentMode 	&bool\\
    \hline
    iconSize  &	QSize\\
    \hline
    tabShape 	&QTabWidget::TabShape\\
    \hline
    toolButtonStyle &	Qt::ToolButtonStyle\\
    \hline
    unifiedTitleAndToolBarOnMac &	bool\\
	\hline
\end{tabular}

\section{公共成员函数}

\begin{longtable}{|r|m{20em}|}
\hline
返回类型  & 	函数名 \\
\hline
 &QMainWindow (QWidget *parent = nullptr, Qt::WindowFlags flags = Qt::WindowFlags()) \\ 
 \hline
virtual  &	$\sim$QMainWindow () \\
\hline
void 	&addDockWidget (Qt::DockWidgetArea area, QDockWidget *dockwidget) \\ 
\hline
void 	&addDockWidget (Qt::DockWidgetArea area, QDockWidget *dockwidget, Qt::Orientation orientation) \\
\hline
void 	&addToolBar (Qt::ToolBarArea area, QToolBar *toolbar) \\ 
\hline
void 	&addToolBar (QToolBar *toolbar) \\ 
\hline
QToolBar *& 	addToolBar (const QString \&title) \\
\hline
void 	&addToolBarBreak (Qt::ToolBarArea area = Qt::TopToolBarArea) \\
\hline
QWidget *& 	centralWidget () const \\ 
\hline
Qt::DockWidgetArea &	corner (Qt::Corner corner) const \\
\hline
virtual QMenu * &	createPopupMenu () \\
\hline
QMainWindow::DockOptions& 	dockOptions () const\\
\hline
Qt::DockWidgetArea 	&dockWidgetArea (QDockWidget *dockwidget) const\\
\hline
bool &	documentMode () const \\
\hline
QSize &	iconSize () const\\
\hline
void 	&insertToolBar (QToolBar *before, QToolBar *toolbar)\\
\hline
void 	&insertToolBarBreak (QToolBar *before)\\
\hline
bool 	&isAnimated () const\\
\hline
bool 	&isDockNestingEnabled () const\\
\hline
QMenuBar *& 	menuBar () const \\ 
\hline
QWidget * &	menuWidget () const \\ 
\hline
void 	&removeDockWidget (QDockWidget *dockwidget)\\ 
\hline
void 	&removeToolBar (QToolBar *toolbar) \\ 
\hline
void 	&removeToolBarBreak (QToolBar *before) \\ 
\hline
void 	&resizeDocks (const QList<QDockWidget *> \&docks, const QList<int> \&sizes, Qt::Orientation orientation) \\ 
\hline
bool 	&restoreDockWidget (QDockWidget *dockwidget) \\ 
\hline
bool 	&restoreState (const QByteArray \&state, int version = 0) \\ 
\hline
QByteArray &	saveState (int version = 0) const \\ 
\hline
void &	setCentralWidget (QWidget *widget) \\ 
\hline
void &	setCorner (Qt::Corner corner, Qt::DockWidgetArea area) \\ 
\hline
void 	&setDockOptions (QMainWindow::DockOptions options) \\ 
\hline
void &	setDocumentMode (bool enabled) \\ 
\hline
void &	setIconSize (const QSize \&iconSize) \\ 
\hline
void &	setMenuBar (QMenuBar *menuBar) \\ 
\hline
void& 	setMenuWidget (QWidget *menuBar) \\ 
\hline
void &	setStatusBar (QStatusBar *statusbar) \\ 
\hline
void &	setTabPosition (Qt::DockWidgetAreas areas, QTabWidget::TabPosition tabPosition) \\ 
\hline
void &	setTabShape (QTabWidget::TabShape tabShape) \\ 
\hline
void 	&setToolButtonStyle (Qt::ToolButtonStyle toolButtonStyle) \\ 
\hline
void 	&splitDockWidget (QDockWidget *first, QDockWidget *second, Qt::Orientation orientation) \\ 
\hline
QStatusBar * & 	statusBar () const \\ 
\hline
QTabWidget::TabPosition &	tabPosition (Qt::DockWidgetArea area) const \\ 
\hline
QTabWidget::TabShape & 	tabShape () const \\
\hline
QList<QDockWidget *> 	&tabifiedDockWidgets (QDockWidget *dockwidget) const \\
\hline
void &	tabifyDockWidget (QDockWidget *first, QDockWidget *second) \\ 
\hline
QWidget *& 	takeCentralWidget () \\ 
\hline
Qt::ToolBarArea &	toolBarArea (QToolBar *toolbar) const \\ 
\hline
bool &	toolBarBreak (QToolBar *toolbar) const \\ 
\hline
Qt::ToolButtonStyle & 	toolButtonStyle () const \\ 
\hline
bool &	unifiedTitleAndToolBarOnMac () const \\
\hline
\end{longtable}

\section{公共槽}

\begin{tabular}{|r|l|}
\hline
返回类型 &	函数名 \\
\hline
void 	&setAnimated (bool enabled) \\ 
\hline
void 	&setDockNestingEnabled (bool enabled) \\
\hline
void 	&setUnifiedTitleAndToolBarOnMac (bool set) \\
\hline
\end{tabular}

\section{信号}

\begin{tabular}{|r|l|}
    \hline
    返回类型 &	函数名 \\
    \hline
    void &	iconSizeChanged (const QSize \&iconSize) \\ 
    \hline
    void &	tabifiedDockWidgetActivated (QDockWidget *dockWidget) \\ 
    \hline
    void &	toolButtonStyleChanged (Qt::ToolButtonStyle toolButtonStyle) \\
    \hline
\end{tabular}

\section{保护成员函数}

\begin{tabular}{|r|l|}
\hline
返回类型 &	函数名 \\
\hline
virtual void &	contextMenuEvent (QContextMenuEvent *event) override \\
\hline
virtual bool &	event (QEvent *event) override \\
\hline
\end{tabular}

\section{详细描述}
\subsection{Qt 主窗口框架}

主窗口提供了一套创建应用程序用户界面的框架。 Qt 用QMainWindow和 相关类 来管理主窗口。QMainWindow已经定义了一个布局,可以往里添加一些 QToolBar 和 QDockWidget,也可以添加一个 QMenuBar 和一个 QStatusBar。这个布局有一个中央区域,可以放任意部件。如下图所示:

main window layout

\begin{notice}
主窗口必须设置中央部件。
\end{notice}

\subsection{创建主窗口组件}

中央部件通常是标准 Qt 部件,如 QTextEdit 或 QGraphicsView,也可自定义部件。
用setCentralWidget()来设置中央部件。

主窗口可以是单文档界面或多文档界面。 
Qt 中设置 QMdiArea 为中央部件即创建了多文档界面。

下面举例说明主窗口可以添加的部件。

\subsubsection{创建菜单}

Qt 用 QMenu 类实现菜单,主窗口将其放在 QMenuBar。
可以添加 QAction 到QMenu,一个QAction代表菜单中的一个条目。

用menuBar()可以得到主窗口的菜单栏,用 QMenuBar::addMenu 添加菜单。

QMainWindow 默认有一个菜单栏,可以用setMenuBar()自定义一个新的菜单栏。
如果不想用 QMenuBar ,也可以用setMenuWidget()来定制菜单栏。

创建菜单代码示例:

\begin{lstlisting}{language=C++}
void MainWindow::createMenus()
{
    fileMenu = menuBar()->addMenu(tr("&File"));
    fileMenu->addAction(newAct);
    fileMenu->addAction(openAct);
    fileMenu->addAction(saveAct);
}
\end{lstlisting}

createPopupMenu()可以创建弹出式菜单,它会在主窗口收到 context menu 事件时弹出。
停靠部件和菜单栏默认实现了右键菜单,可以重写createPopupMenu()创建自定义菜单。

\subsubsection{创建工具栏}

Qt 用 QToolBar 类实现工具栏,可以用addToolBar()添加工具栏到主窗口。

可以设置 Qt::ToolBarArea 来控制工具栏的初始位置。可以用addToolBarBreak()或insertToolBarBreak()分割工具栏所在的区域,前者可使接下来添加的工具栏换至新的一行,后者添加了一个工具栏分隔符。用 QToolBar::setAllowedAreas 加 QToolBar::setMovable 可以限制用户放工具栏的位置。

工具栏图标的尺寸可以用iconSize()获取,它是平台相关的。可以用setIconSize()设置固定尺寸。用setToolButtonStyle()可以修改工具栏图标外观。

创建工具栏代码示例:

\begin{lstlisting}{language=C++}
void MainWindow::createToolBars()
{
    fileToolBar = addToolBar(tr("File"));
    fileToolBar->addAction(newAct);
}
\end{lstlisting}
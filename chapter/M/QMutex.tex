\chapter{QMutex}

QMutex类提供线程间的访问序列化。\href{https://github.com/QtDocumentCN/QtDocumentCN/blob/master/Src/M/QMutex/QMutex.md#%E8%AF%A6%E7%BB%86%E6%8F%8F%E8%BF%B0}{更多...}

\begin{tabular}{|r|l|}
	\hline
	属性 & 内容 \\
	\hline
    头文件  &	\hl{\#include<QMutex>} \\
    \hline
    qmake: & QT += core    \\
	\hline
    子类 	& QRecursiveMutex \\ 
    \hline
\end{tabular}

\begin{compactitem}
\item 此类中所有函数都是线程安全的。
\end{compactitem}

\section{公共成员类型}


\begin{tabular}{|r|m{28em}|}   
\hline
类型 	& 名称 \\ 
\hline
enmu 	& RecursionMode \{ Recursive, NonRecursive \} \\ 
\hline
\end{tabular}

\section{公共成员函数}

\begin{longtable}{|r|m{28em}|}   
    \hline
    返回类型 	& 函数 \\
    \hline
   & QMutex(QMutex::RecursionMode mode) \\
    \hline
	&QMutex() \\ 
    \hline
	&$\sim$QMutex() \\ 
    \hline
bool &	isRecursive() const \\ 
\hline
void &	lock() \\ 
\hline
bool 	&tryLock(int timeout = 0) \\ 
\hline
bool 	&try\_lock() \\ 
\hline
bool &	try\_lock\_for(std::chrono::duration<Rep, Period> \emph{duration}) \\ 
\hline
bool 	&try\_lock\_until(std::chrono::time\_point<Clock, Duration> \emph{timePoint}) \\ 
\hline
void &	unlock() \\ 
    \hline 
\end{longtable}


\section{详细描述}

QMutex的目的是保护对象、数据结构或代码段,以便一次只有一个线程可以访问它们(这类似于 Java synchronized 关键字)。
最好通过 QMutexLocker 来使用互斥量,这样可以很方便确保成对执行锁和解锁。
例如,假设有一个方法将消息打印到两行上:

\begin{cppcode}
int number = 6;

void method1()
{
    number *= 5;
    number /= 4;
}

void method2()
{
    number *= 3;
    number /= 2;
}
\end{cppcode}

如果连续调用这两个方法,将发生以下情况:

\begin{cppcode}
// method1()
number *= 5;        // number 30
number /= 4;        // number  7

// method2()
number *= 3;        // number 21
number /= 2;        // number 10
\end{cppcode}

如果两个线程同时调用这两个方法,则可能会产生以下结果:

\begin{cppcode}
// 线程 1 调用 method1()
number *= 5;        // number 30

// 线程 2 调用 method2().
//
// 很可能线程 1 已被操作系统置于等待队列
// 操作系统运行线程 2
number *= 3;        // number 90
number /= 2;        // number 45

// 线程 1 执行完毕
number /= 4;        // number 是 11,而不是 10
\end{cppcode}

如果我们添加一个互斥量,我们就能得到我们想要的结果:

\begin{cppcode}
QMutex mutex;
int number = 6;

void method1()
{
    mutex.lock();
    number *= 5;
    number /= 4;
    mutex.unlock();
}

void method2()
{
    mutex.lock();
    number *= 3;
    number /= 2;
    mutex.unlock();
}
\end{cppcode}

在任何给定的时间只有一个线程可以修改 number,并正确执行。
虽然这只是一个微不足道的例子,但也适用于其他需要有序执行的地方。

当你在一个线程中调用 lock() 时,在同一位置尝试调用 lock() 的其他线程将阻塞,
直到获得锁的线程调用 unlock()。
替代 lock() 的非阻塞方法是 tryLock()。

QMutex被优化为在非争用情况 ( the non-contended )下速度更快。
如果互斥体上没有争用,非递归QMutex将不会分配内存。
它的构造和销毁几乎没有任何开销,
这意味着将互斥体作为类的一部分是很好的做法。

\begin{seeAlso}
QRecursiveMutex,QMutexLocker,QReadWriteLock,QSemaphore 和 QWaitCondition。
\end{seeAlso}

\section{成员类型文档}

enum QMutex::RecursionMode

\begin{tabular}{|c|c|m{20em}|}
\hline
常量 	&值& 	描述 \\ 
\hline
QMutex::Recursive &	1 	&在这种模式下,一个线程可以多次锁定同一个互斥量,并且在调用相同数量的 unlock() 之前,互斥体不会被解锁。对于这种情况,您应该使用 QRecursiveMutex。 \\
\hline
QMutex::NonRecursive &	0 	&在这种模式下,一个线程只能锁定一次。\\
\hline
\end{tabular}

\begin{seeAlso}
QMutex(),QRecursiveMutex。
\end{seeAlso}

\section{成员函数文档}

QMutex::QMutex(QMutex::RecursionMode \emph{mode})

构造一个互斥量,初始状态为未上锁。

如果 mode 是 QMutex::Recursive,则线程可以多次锁定同一个互斥量,
并且在调用相同数量的 unlock() 之前,互斥量不会被解锁。
否则,线程只能锁定互斥量一次。默认值为 QMutex::NonRecursive。

\begin{seeAlso}
lock(),unlock()。
\end{seeAlso}

QMutex::QMutex()

构造一个互斥量,初始状态为未上锁。

QMutex::$\sim$QMutex()

析构。

\begin{warning}
销毁锁定的互斥量可能会导致未定义的行为。
\end{warning}

bool QMutex::isRecursive() const

如果互斥量是递归的,则返回 true。

在 Qt 5.7 引入该函数。

void QMutex::lock()

锁定互斥量。如果另一个线程锁定了该互斥量,
那么这个调用将被阻塞,直到锁定线程将其解锁为止。

如果该互斥量是递归互斥量,则允许在同一线程的同一互斥体上多次调用此函数。
如果这个互斥量是非递归互斥量,则当该互斥量递归锁定时,这个函数将死锁。

\begin{seeAlso}
unlock()。
\end{seeAlso}

bool QMutex::tryLock(int \emph{timeout} = 0)

尝试锁定互斥量。如果获得了锁,此函数返回 true;否则返回 false。
如果另一个线程锁定了互斥量,则此函数最多将等待 timeout 毫秒,以尝试获取。

\begin{notice}
传递一个负数给 timeout 相当于调用 lock()。即,如果 timeout 为负数,
这个函数将一直待,直到互斥量被锁定为止。
\end{notice}

如果获得了锁,则必须使用 unlock() 解锁,另一个线程才能成功锁定。

如果该互斥量是递归互斥量,则允许在同一线程的同一互斥量上多次调用此函数。
如果此互斥量是非递归互斥量,则当尝试递归锁定互斥量时,此函数将始终返回 false。

\begin{seeAlso}
lock(),unlock()。
\end{seeAlso}

bool QMutex::try\_lock()

尝试锁定互斥量。如果获得了锁,此函数返回 true;否则返回 false。

提供此函数是为了与可锁定的标准库概念兼容。它相当于 tryLock()。

在 Qt 5.8 引入该函数。

template <typename Rep, typename Period> bool QMutex::try\_lock\_for(std::chrono::duration<Rep, Period> duration)

尝试锁定互斥量。如果获得了锁,此函数返回 true;否则返回 false。
如果另一个线程锁定了互斥量,则此函数最多将等待 duration 这么长时间,
以尝试获取。

\begin{notice}
传递一个负数给 duration 相当于调用 try\_lock()。
\end{notice}

此行为与 tryLock() 不同。

如果获得了锁,则必须使用 unlock() 解锁,另一个线程才能成功锁定。

如果该互斥量是递归互斥量,则允许在同一线程的同一互斥量上多次调用此函数。
如果此互斥量是非递归互斥量,则当尝试递归锁定互斥量时,此函数将始终返回 false。

在 Qt 5.8 引入该函数。

\begin{seeAlso}
lock(),unlock()。
\end{seeAlso}

template <typename Clock, typename Duration> bool QMutex::try\_lock\_until(std::chrono::time\_point<Clock, Duration> \emph{timePoint})

尝试锁定互斥量。如果获得了锁,此函数返回 true;否则返回 false。
如果另一个线程锁定了互斥量,则此函数最多将等待 timePoint 这么长时间,
以尝试获取。

\begin{notice}
传递一个负数给 timePoint 相当于调用 try\_lock()。
\end{notice}

此行为与 tryLock() 不同。

如果获得了锁,则必须使用 unlock() 解锁,另一个线程才能成功锁定。

如果该互斥量是递归互斥量,则允许在同一线程的同一互斥量上多次调用此函数。
如果此互斥量是非递归互斥量,则当尝试递归锁定互斥量时,此函数将始终返回 false。

在 Qt 5.8 引入该函数。

\begin{seeAlso}
lock(),unlock()。
\end{seeAlso}

void QMutex::unlock()

解锁互斥量。试图在不同线程中解锁互斥量会导致错误。
解锁未锁定的互斥量会导致未定义的行为。

\begin{seeAlso}
lock()。
\end{seeAlso}


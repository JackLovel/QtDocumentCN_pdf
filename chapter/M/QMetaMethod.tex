\chapter{QMetaMethod}

QMetaMethod 类提供了对应一个成员函数的元数据。更多内容...

\begin{tabular}{|r|l|}
	\hline
	属性 & 方法 \\
	\hline
	头文件 & \#include <QMetaMethod>\\      
	\hline
	qmake & QT += core\\      
	\hline
\end{tabular}

\section{公共类型}

\begin{tabular}{|r|l|}   
\hline
类型	& 名称 \\ 
\hline
enum	&Access \{ Private, Protected, Public \} \\ 
\hline
enum	& MethodType \{ Method, Signal, Slot, Constructor \} \\
\hline
\end{tabular}

\section{公共成员函数}

\begin{longtable}{|l|m{27em}|} 
\hline
返回类型 & 函数 \\   
\hline
QMetaMethod::Access	& access() const \\ 
\hline
bool &	invoke(QObject *object, Qt::ConnectionType connectionType, QGenericReturnArgument returnValue, QGenericArgument val0 = QGenericArgument(nullptr), QGenericArgument val1 = QGenericArgument(), QGenericArgument val2 = QGenericArgument(), QGenericArgument val3 = QGenericArgument(), QGenericArgument val4 = QGenericArgument(), QGenericArgument val5 = QGenericArgument(), QGenericArgument val6 = QGenericArgument(), QGenericArgument val7 = QGenericArgument(), QGenericArgument val8 = QGenericArgument(), QGenericArgument val9 = QGenericArgument()) const \\
\hline
bool &	invoke(QObject *object, QGenericReturnArgument returnValue, QGenericArgument val0 = QGenericArgument(0), QGenericArgument val1 = QGenericArgument(), QGenericArgument val2 = QGenericArgument(), QGenericArgument val3 = QGenericArgument(), QGenericArgument val4 = QGenericArgument(), QGenericArgument val5 = QGenericArgument(), QGenericArgument val6 = QGenericArgument(), QGenericArgument val7 = QGenericArgument(), QGenericArgument val8 = QGenericArgument(), QGenericArgument val9 = QGenericArgument()) const \\
\hline
bool &	invoke(QObject *object, Qt::ConnectionType connectionType, QGenericArgument val0 = QGenericArgument(0), QGenericArgument val1 = QGenericArgument(), QGenericArgument val2 = QGenericArgument(), QGenericArgument val3 = QGenericArgument(), QGenericArgument val4 = QGenericArgument(), QGenericArgument val5 = QGenericArgument(), QGenericArgument val6 = QGenericArgument(), QGenericArgument val7 = QGenericArgument(), QGenericArgument val8 = QGenericArgument(), QGenericArgument val9 = QGenericArgument()) const \\
\hline
bool &	invoke(QObject *object, QGenericArgument val0 = QGenericArgument(0), QGenericArgument val1 = QGenericArgument(), QGenericArgument val2 = QGenericArgument(), QGenericArgument val3 = QGenericArgument(), QGenericArgument val4 = QGenericArgument(), QGenericArgument val5 = QGenericArgument(), QGenericArgument val6 = QGenericArgument(), QGenericArgument val7 = QGenericArgument(), QGenericArgument val8 = QGenericArgument(), QGenericArgument val9 = QGenericArgument()) const \\
\hline
bool &	invokeOnGadget(void *gadget, QGenericReturnArgument returnValue, QGenericArgument val0 = QGenericArgument(nullptr), QGenericArgument val1 = QGenericArgument(), QGenericArgument val2 = QGenericArgument(), QGenericArgument val3 = QGenericArgument(), QGenericArgument val4 = QGenericArgument(), QGenericArgument val5 = QGenericArgument(), QGenericArgument val6 = QGenericArgument(), QGenericArgument val7 = QGenericArgument(), QGenericArgument val8 = QGenericArgument(), QGenericArgument val9 = QGenericArgument()) const \\
\hline
bool &	invokeOnGadget(void *gadget, QGenericArgument val0 = QGenericArgument(0), QGenericArgument val1 = QGenericArgument(), QGenericArgument val2 = QGenericArgument(), QGenericArgument val3 = QGenericArgument(), QGenericArgument val4 = QGenericArgument(), QGenericArgument val5 = QGenericArgument(), QGenericArgument val6 = QGenericArgument(), QGenericArgument val7 = QGenericArgument(), QGenericArgument val8 = QGenericArgument(), QGenericArgument val9 = QGenericArgument()) const \\
\hline
const char * 	&name() const \\
\hline
const char * 	& value() const \\ 
\hline
bool & 	isValid() const \\ 
\hline
int	& methodIndex() const \\ 
\hline
QByteArray &	methodSignature() const \\ 
\hline
QMetaMethod::MethodType	 & methodType() const \\
\hline
QByteArray & 	name() const \\ 
\hline
int	& parameterCount() const \\
\hline
QList<QByteArray>	& parameterNames() const \\
\hline
int	& parameterType(int index) const \\
\hline
QList<QByteArray> &	parameterTypes() const \\
\hline
int	& returnType() const \\
\hline
int	& revision() const \\ 
\hline
const char *	& tag() const \\ 
\hline
const char *	& typeName() const \\ 
\hline
\end{longtable}


\section{静态公共成员}


\begin{tabular}{|l|l|}
\hline
返回类型 &	函数 \\ 
\hline
QMetaMethod 	&fromSignal(PointerToMemberFunction \emph{signal}) \\ 
\hline
\end{tabular}

\section{相关非成员函数}

\begin{tabular}{|l|l|}
\hline
返回类型 &	函数 \\ 
\hline
bool  &	operator!=(const QMetaMethod \emph{\&m1}, const QMetaMethod \emph{\&m2}) \\
\hline 
bool  &	operator==(const QMetaMethod \emph{\&m1}, const QMetaMethod \emph{\&m2})  \\
\hline
\end{tabular}

\section{宏定义}

\begin{tabular}{|c|c|}
	\hline
	宏定义 \\ 
	\hline
Q\_METAMETHOD\_INVOKE\_MAX\_ARGS \\ 
	\hline
	\end{tabular}

\section{详细描述}

QMetaMethod 类具有一个 methodType()、一个 methodSignature()、一组 parameterTypes() 和 parameterNames()、返回值的 typeName()、一个 tag()、一个 access() 描述符。
可以通过 invoke() 来执行任意 QObject 的方法。

\begin{seeAlso}
QMetaObject、QMetaEnum、QMetaProperty 和 Qt 属性系统。
\end{seeAlso}

\section{成员类型文档}

enum QMetaMethod::Access

此枚举描述某方法的访问权限,遵循 C++ 相关公约。


\begin{tabular}{|c|c|}
	\hline
	常量 	& 数值  \\
	\hline
	QMetaMethod::Private  &	0 \\ 
	\hline
	QMetaMethod::Protected &	1 \\ 
	\hline
	QMetaMethod::Public &	2 \\ 
	\hline
	\end{tabular}

enum QMetaMethod::MethodType	

\begin{tabular}{|c|c|c|}
	\hline
	常量  &	数值 	& 描述  \\ 
	\hline
	QMetaMethod::Method &	0 &	该函数是普通的成员函数。 \\
	\hline
	QMetaMethod::Signal &	1 	&该函数是信号函数。 \\
	\hline
	QMetaMethod::Slot &	2 &	该函数是槽函数。 \\ 
	\hline
	QMetaMethod::Constructor &	3 &	该函数是构造函数。 \\ 
	\hline
	\end{tabular}

\section{成员函数文档}

QMetaMethod::Access QMetaMethod::access() const

返回该方法的访问权限(private、protected 或 `public)。

\begin{notice}
信号永远是公共的,但应将此认为是实现细节。在类外发射该类的信号通常是个坏主意。
\end{notice}

\begin{seeAlso}
methodType()。
\end{seeAlso}

[static] template <typename PointerToMemberFunction> QMetaMethod QMetaMethod::fromSignal(PointerToMemberFunction signal)

返回对应给定 signal 的元方法,若 signal 并非信号,则返回无效的 QMetaMethod 对象。

范例:

\begin{lstlisting}[language=C++]
QMetaMethod destroyedSignal = QMetaMethod::fromSignal(&QObject::destroyed);
\end{lstlisting}


本函数在 Qt 5.0 中被引入。

bool QMetaMethod::invoke(QObject *object, Qt::ConnectionType connectionType, QGenericReturnArgument returnValue, 
   QGenericArgument val0 = QGenericArgument(nullptr), 
   QGenericArgument val1 = QGenericArgument(), 
QGenericArgument val2 = QGenericArgument(), 
QGenericArgument val3 = QGenericArgument(), 
QGenericArgument val4 = QGenericArgument(), 
QGenericArgument val5 = QGenericArgument(), 
QGenericArgument val6 = QGenericArgument(), 
QGenericArgument val7 = QGenericArgument(),
 QGenericArgument val8 = QGenericArgument(), 
 QGenericArgument val9 = QGenericArgument()) const

通过 object 对象动态调用本方法。若可被动态调用则返回 tue,若该对象无此方法或参数不匹配则返回 false。

该动态调用可以是同步或异步的,由 connectionType 决定:

\begin{compactitem}
\item 若 type 是 Qt::DirectConnection,则该方法会被立即执行。
\item 若 type 是 Qt::QueuedConnection,则会发送一个 QEvent ,该方法会在应用进入该对象所属线程的主事件循环后执行。
\item 若 type 是 Qt::AutoConnection,当 object 与调用者处于相同线程中时,该方法会被同步执行,否则会被异步执行。
\end{compactitem}

返回值会被存放在 returnvalue 中。若调用方式是异步,则返回值无法被获取。
最多可以传递十个参数 (val0, val1, val2, val3, val4, val5, val6, val7, val8 和 val9) 至本方法。

QGenericArgument 和 QGenericReturnArgument 是内部的辅助类。
为了动态调用信号槽,您需要将参数通过 Q\_ARG() 和 Q\_RETURN\_ARG() 宏进行封装。
Q\_ARG() 接受一个类型名称和一个该类型的不可变引用;
Q\_RETURN\_ARG() 接受一个类型名称和一个该类型的可变引用。

通过异步方式动态调用 QPushButton 的 animateClick() 槽:

\begin{lstlisting}[language=C++]
int methodIndex = pushButton->metaObject()->indexOfMethod("animateClick()");
QMetaMethod method = metaObject->method(methodIndex);
method.invoke(pushButton, Qt::QueuedConnection);
\end{lstlisting}

当异步调用方法时,传递的参数必须被 Qt 的元对象系统所知悉,因为 Qt 需要在后台事件中拷贝并保存它们。
如果您使用队列连接时遇到下述错误信息:

\begin{lstlisting}
QMetaMethod::invoke: Unable to handle unregistered datatype 'MyType'	
\end{lstlisting}

则在调用 invokeMethod() 之前通过 qRegisterMetaType() 来注册该数据类型。

若想通过 obj 对象同步调用 \hl{compute(QString, int, double)} 槽,则代码如下:

\begin{lstlisting}[language=C++]
QString retVal;
QByteArray normalizedSignature = QMetaObject::normalizedSignature("compute(QString, int, double)");
int methodIndex = obj->metaObject()->indexOfMethod(normalizedSignature);
QMetaMethod method = obj->metaObject()->method(methodIndex);
method.invoke(obj,
              Qt::DirectConnection,
              Q_RETURN_ARG(QString, retVal),
              Q_ARG(QString, "sqrt"),
              Q_ARG(int, 42),
              Q_ARG(double, 9.7));
\end{lstlisting}


此处使用了 QMetaObject::normalizedSignature() 来确保函数签名符合 invoke() 期望的格式,即将多余的空白移除。

若 compute 槽通过特定顺序没有完整获取到一个 QString、一个 int 和一个 double,则此调用会失败。

\begin{warning}
此方法不会验证参数的有效性,object 必须是创建 QMetaMethod 的 QMetaObject 的类型实例,参数列表必须与该方法的保持一直,否则会导致未定义行为。
\end{warning}

\begin{seeAlso}
Q\_ARG()、Q\_RETURN\_ARG()、qRegisterMetaType() 和 QMetaObject::invokeMethod()。
\end{seeAlso}


%%%%%%%%%

bool QMetaMethod::invoke(QObject *object, QGenericReturnArgument returnValue, QGenericArgument val0 = QGenericArgument(0), QGenericArgument val1 = QGenericArgument(), QGenericArgument val2 = QGenericArgument(), QGenericArgument val3 = QGenericArgument(), QGenericArgument val4 = QGenericArgument(), QGenericArgument val5 = QGenericArgument(), QGenericArgument val6 = QGenericArgument(), QGenericArgument val7 = QGenericArgument(), QGenericArgument val8 = QGenericArgument(), QGenericArgument val9 = QGenericArgument()) const

此函数是 invoke() 的重载。

此重载始终通过 Qt::AutoConnection 调用本方法。

bool QMetaMethod::invoke(QObject *object, Qt::ConnectionType connectionType, QGenericArgument val0 = QGenericArgument(0), QGenericArgument val1 = QGenericArgument(), QGenericArgument val2 = QGenericArgument(), QGenericArgument val3 = QGenericArgument(), QGenericArgument val4 = QGenericArgument(), QGenericArgument val5 = QGenericArgument(), QGenericArgument val6 = QGenericArgument(), QGenericArgument val7 = QGenericArgument(), QGenericArgument val8 = QGenericArgument(), QGenericArgument val9 = QGenericArgument()) const

此函数是 invoke() 的重载。

此重载用于不关心对返回值的场合。

bool QMetaMethod::invoke(QObject *object, QGenericArgument val0 = QGenericArgument(0), QGenericArgument val1 = QGenericArgument(), QGenericArgument val2 = QGenericArgument(), QGenericArgument val3 = QGenericArgument(), QGenericArgument val4 = QGenericArgument(), QGenericArgument val5 = QGenericArgument(), QGenericArgument val6 = QGenericArgument(), QGenericArgument val7 = QGenericArgument(), QGenericArgument val8 = QGenericArgument(), QGenericArgument val9 = QGenericArgument()) const

此函数是 invoke() 的重载。

此重载通过 Qt::AutoConnection 调用本方法,并忽略返回值。

bool QMetaMethod::invokeOnGadget(void \emph{*gadget}, QGenericReturnArgument \emph{returnValue}, QGenericArgument \emph{val0} = QGenericArgument(nullptr), QGenericArgument val1 = QGenericArgument(), QGenericArgument val2 = QGenericArgument(), QGenericArgument val3 = QGenericArgument(), QGenericArgument val4 = QGenericArgument(), QGenericArgument val5 = QGenericArgument(), QGenericArgument val6 = QGenericArgument(), QGenericArgument val7 = QGenericArgument(), QGenericArgument val8 = QGenericArgument(), QGenericArgument val9 = QGenericArgument()) const

通过 Q\_GADGET 对象动态调用本方法。
若可被动态调用则返回 tue,若该对象无此方法或参数不匹配则返回 false。

gadget 指针必须是 Q\_GADGET 类的实例。

该调用始终是同步的。

返回值会被存放在 returnvalue 中。若调用方式是异步,则返回值无法被获取。最多可以传递十个参数 (val0, val1, val2, val3, val4, val5, val6, val7, val8 和 val9) 至本方法。

\begin{seeAlso}
此方法不会验证参数的有效性,gadget 必须是创建 QMetaMethod 的 QMetaObject 的类型实例,参数列表必须与该方法的保持一直,否则会导致未定义行为。
\end{seeAlso}

本函数在 Qt 5.5 中被引入。

\begin{seeAlso}
Q\_ARG()、Q\_RETURN\_ARG()、qRegisterMetaType() 和 QMetaObject::invokeMethod()。
\end{seeAlso}

bool QMetaMethod::invokeOnGadget(void \*gadget, QGenericArgument val0 = QGenericArgument(0), QGenericArgument val1 = QGenericArgument(), QGenericArgument val2 = QGenericArgument(), QGenericArgument val3 = QGenericArgument(), QGenericArgument val4 = QGenericArgument(), QGenericArgument val5 = QGenericArgument(), QGenericArgument val6 = QGenericArgument(), QGenericArgument val7 = QGenericArgument(), QGenericArgument val8 = QGenericArgument(), QGenericArgument val9 = QGenericArgument()) const

这是一个重载函数。

此重载函数通过 gadget 对象动态调用本方法,并忽略返回值。

本函数在 Qt 5.5 中被引入。

bool QMetaMethod::isValid() const

若此方法有效则返回 true(即可被自省并动态调用),否则返回 false。

本函数在 Qt 5.0 中被引入。

int QMetaMethod::methodIndex() const

返回本方法的索引编号。

本函数在 Qt 4.6 中被引入。

QByteArray QMetaMethod::methodSignature() const

返回本方法的签名(例如 setValue(double))。

本函数在 Qt 5.0 中被引入。

\begin{seeAlso}
parameterTypes() 和 parameterNames()。
\end{seeAlso}

QMetaMethod::MethodType QMetaMethod::methodType() const

返回本方法的类型(signal、slot、method 或 constructor)。

\begin{seeAlso}
access()。
\end{seeAlso}

QByteArray QMetaMethod::name() const

返回本方法的名字。

本函数在 Qt 5.0 中被引入。

\begin{seeAlso}
methodSignature() 和 parameterCount()。
\end{seeAlso}

int QMetaMethod::parameterCount() const

返回本方法的参数个数。

本函数在 Qt 5.0 中被引入。

\begin{seeAlso}
parameterType() 和 parameterNames()。
\end{seeAlso}

QList<QByteArray> QMetaMethod::parameterNames() const

返回参数名称列表。

\begin{seeAlso}
parameterTypes() 和 methodSignature()。
\end{seeAlso}

int QMetaMethod::parameterType(int \emph{index}) const

返回对应 index 的参数类型。

返回值是 QMetaType 中注册的类型之一,若该类型未被注册则返回 QMetaType::UnknownType。

本函数在 Qt 5.0 中被引入。

\begin{seeAlso}
parameterCount()、returnType() 和 QMetaType。
\end{seeAlso}

QList<QByteArray> QMetaMethod::parameterTypes() const

返回参数类型的列表。

\begin{seeAlso}
parameterNames() 和 methodSignature()。
\end{seeAlso}

int QMetaMethod::returnType() const

返回本方法返回值的类型。

返回

返回值是 QMetaType 中注册的类型之一,若该类型未被注册则返回 QMetaType::UnknownType。

本函数在 Qt 5.0 中被引入。

\begin{seeAlso}
parameterType()、QMetaType 和 typeName()。
\end{seeAlso}

int QMetaMethod::revision() const

返回通过 Q\_REVISION 注明的版本,若未注明则返回 0。

本函数在 Qt 5.1 中被引入。

const char *QMetaMethod::tag() const

返回与本方法关联的标签。

标签在此处指的是可以被 moc 识别的特定宏,用于为方法添加附加信息。

标签信息可以用如下方式添加到函数声明中:

\begin{lstlisting}[language=C++]
// 在 MainWindow 类声明中
#ifndef Q_MOC_RUN
// 将标签内容定义为空,以确保对编译器不可见
#  define MY_CUSTOM_TAG
#endif
...
private slots:
	MY_CUSTOM_TAG void testFunc();
\end{lstlisting}

相关信息可通过如下方式获取:

\begin{lstlisting}[language=C++]
MainWindow win;
win.show();

int functionIndex = win.metaObject()->indexOfSlot("testFunc()");
QMetaMethod mm = win.metaObject()->method(functionIndex);
qDebug() << mm.tag(); // 将打印 MY_CUSTOM_TAG
\end{lstlisting}

%%%%%%%%

此时,moc 会提取并记录所有标签,但不会对它们做任何特殊处理。您可以用标签来标注不同的方法,并在您的应用中通过特定的用途来使用它们。

\begin{notice}
自 Qt 5.0 开始,moc 会展开预编译宏,所以必须将标签定义包含在 \#ifndef Q\_MOC\_RUN 中,正如上文范例。
这在 Qt 4 中不是必要的,但上述代码在 Qt 4 中同样也能生效。
\end{notice}

const char *QMetaMethod::typeName() const

返回本方法的返回类型名称。

\begin{seeAlso}
returnType() 和 QMetaType::type()。
\end{seeAlso}

\section{相关非成员函数}

bool operator!=(const QMetaMethod \emph{\&m1}, const QMetaMethod\emph{\&m2})

这是一个重载函数。

若 m1 与 m2 不相等则返回 true,否则返回 false。

本函数在 Qt 5.0 中被引入。

bool operator==(const QMetaMethod \emph{\&m1}, const QMetaMethod \emph{\&m2})

这是一个重载函数。

若 m1 与 m2 相等则返回 true,否则返回 false。

本函数在 Qt 5.0 中被引入。

\section{宏定义文档}

Q\_METAMETHOD\_INVOKE\_MAX\_ARGS

此宏的数值等于通过 QMetaMethod::invoke() 执行方法时,可以传递的最大参数个数。
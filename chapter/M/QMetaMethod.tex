\chapter{QMetaMethod}

QMetaMethod 类提供了对应一个成员函数的元数据。更多内容...

\begin{tabular}{|r|l|}
	\hline
	属性 & 方法 \\
	\hline
	头文件 & \#include <QMetaMethod>\\      
	\hline
	qmake & QT += core\\      
	\hline
\end{tabular}

\section{公共类型}

\begin{tabular}{|r|l|}   
\hline
类型	& 名称 \\ 
\hline
enum	&Access \{ Private, Protected, Public \} \\ 
\hline
enum	& MethodType \{ Method, Signal, Slot, Constructor \} \\
\hline
\end{tabular}

\section{公共成员函数}

\begin{longtable}{|l|m{27em}|} 
\hline
返回类型 & 函数 \\   
\hline
QMetaMethod::Access	& access() const \\ 
\hline
bool &	invoke(QObject *object, Qt::ConnectionType connectionType, QGenericReturnArgument returnValue, QGenericArgument val0 = QGenericArgument(nullptr), QGenericArgument val1 = QGenericArgument(), QGenericArgument val2 = QGenericArgument(), QGenericArgument val3 = QGenericArgument(), QGenericArgument val4 = QGenericArgument(), QGenericArgument val5 = QGenericArgument(), QGenericArgument val6 = QGenericArgument(), QGenericArgument val7 = QGenericArgument(), QGenericArgument val8 = QGenericArgument(), QGenericArgument val9 = QGenericArgument()) const \\
\hline
bool &	invoke(QObject *object, QGenericReturnArgument returnValue, QGenericArgument val0 = QGenericArgument(0), QGenericArgument val1 = QGenericArgument(), QGenericArgument val2 = QGenericArgument(), QGenericArgument val3 = QGenericArgument(), QGenericArgument val4 = QGenericArgument(), QGenericArgument val5 = QGenericArgument(), QGenericArgument val6 = QGenericArgument(), QGenericArgument val7 = QGenericArgument(), QGenericArgument val8 = QGenericArgument(), QGenericArgument val9 = QGenericArgument()) const \\
\hline
bool &	invoke(QObject *object, Qt::ConnectionType connectionType, QGenericArgument val0 = QGenericArgument(0), QGenericArgument val1 = QGenericArgument(), QGenericArgument val2 = QGenericArgument(), QGenericArgument val3 = QGenericArgument(), QGenericArgument val4 = QGenericArgument(), QGenericArgument val5 = QGenericArgument(), QGenericArgument val6 = QGenericArgument(), QGenericArgument val7 = QGenericArgument(), QGenericArgument val8 = QGenericArgument(), QGenericArgument val9 = QGenericArgument()) const \\
\hline
bool &	invoke(QObject *object, QGenericArgument val0 = QGenericArgument(0), QGenericArgument val1 = QGenericArgument(), QGenericArgument val2 = QGenericArgument(), QGenericArgument val3 = QGenericArgument(), QGenericArgument val4 = QGenericArgument(), QGenericArgument val5 = QGenericArgument(), QGenericArgument val6 = QGenericArgument(), QGenericArgument val7 = QGenericArgument(), QGenericArgument val8 = QGenericArgument(), QGenericArgument val9 = QGenericArgument()) const \\
\hline
bool &	invokeOnGadget(void *gadget, QGenericReturnArgument returnValue, QGenericArgument val0 = QGenericArgument(nullptr), QGenericArgument val1 = QGenericArgument(), QGenericArgument val2 = QGenericArgument(), QGenericArgument val3 = QGenericArgument(), QGenericArgument val4 = QGenericArgument(), QGenericArgument val5 = QGenericArgument(), QGenericArgument val6 = QGenericArgument(), QGenericArgument val7 = QGenericArgument(), QGenericArgument val8 = QGenericArgument(), QGenericArgument val9 = QGenericArgument()) const \\
\hline
bool &	invokeOnGadget(void *gadget, QGenericArgument val0 = QGenericArgument(0), QGenericArgument val1 = QGenericArgument(), QGenericArgument val2 = QGenericArgument(), QGenericArgument val3 = QGenericArgument(), QGenericArgument val4 = QGenericArgument(), QGenericArgument val5 = QGenericArgument(), QGenericArgument val6 = QGenericArgument(), QGenericArgument val7 = QGenericArgument(), QGenericArgument val8 = QGenericArgument(), QGenericArgument val9 = QGenericArgument()) const \\
\hline
const char * 	&name() const \\
\hline
const char * 	& value() const \\ 
\hline
bool & 	isValid() const \\ 
\hline
int	& methodIndex() const \\ 
\hline
QByteArray &	methodSignature() const \\ 
\hline
QMetaMethod::MethodType	 & methodType() const \\
\hline
QByteArray & 	name() const \\ 
\hline
int	& parameterCount() const \\
\hline
QList<QByteArray>	& parameterNames() const \\
\hline
int	& parameterType(int index) const \\
\hline
QList<QByteArray> &	parameterTypes() const \\
\hline
int	& returnType() const \\
\hline
int	& revision() const \\ 
\hline
const char *	& tag() const \\ 
\hline
const char *	& typeName() const \\ 
\hline
\end{longtable}


\section{静态公共成员}


\begin{tabular}{|l|l|}
\hline
返回类型 &	函数 \\ 
\hline
QMetaMethod 	&fromSignal(PointerToMemberFunction \emph{signal}) \\ 
\hline
\end{tabular}

\section{相关非成员函数}

\begin{tabular}{|l|l|}
\hline
返回类型 &	函数 \\ 
\hline
bool  &	operator!=(const QMetaMethod \emph{\&m1}, const QMetaMethod \emph{\&m2}) \\
\hline 
bool  &	operator==(const QMetaMethod \emph{\&m1}, const QMetaMethod \emph{\&m2})  \\
\hline
\end{tabular}

\section{宏定义}

\begin{tabular}{|c|c|}
	\hline
	宏定义 \\ 
	\hline
Q\_METAMETHOD\_INVOKE\_MAX\_ARGS \\ 
	\hline
	\end{tabular}

\section{详细描述}

QMetaMethod 类具有一个 methodType()、一个 methodSignature()、一组 parameterTypes() 和 parameterNames()、返回值的 typeName()、一个 tag()、一个 access() 描述符。
可以通过 invoke() 来执行任意 QObject 的方法。

\begin{notice}[另请参阅]
QMetaObject、QMetaEnum、QMetaProperty 和 Qt 属性系统。
\end{notice}

\section{成员类型文档}

enum QMetaMethod::Access

此枚举描述某方法的访问权限,遵循 C++ 相关公约。


\begin{tabular}{|c|c|}
	\hline
	常量 	& 数值  \\
	\hline
	QMetaMethod::Private  &	0 \\ 
	\hline
	QMetaMethod::Protected &	1 \\ 
	\hline
	QMetaMethod::Public &	2 \\ 
	\hline
	\end{tabular}

enum QMetaMethod::MethodType	

\begin{tabular}{|c|c|c|}
	\hline
	常量  &	数值 	& 描述  \\ 
	\hline
	QMetaMethod::Method &	0 &	该函数是普通的成员函数。 \\
	\hline
	QMetaMethod::Signal &	1 	&该函数是信号函数。 \\
	\hline
	QMetaMethod::Slot &	2 &	该函数是槽函数。 \\ 
	\hline
	QMetaMethod::Constructor &	3 &	该函数是构造函数。 \\ 
	\hline
	\end{tabular}

\section{成员函数文档}

QMetaMethod::Access QMetaMethod::access() const

返回该方法的访问权限(private、protected 或 `public)。

\begin{notice}
信号永远是公共的,但应将此认为是实现细节。在类外发射该类的信号通常是个坏主意。
\end{notice}

\begin{notice}[另请参阅]
methodType()。
\end{notice}

[static] template <typename PointerToMemberFunction> QMetaMethod QMetaMethod::fromSignal(PointerToMemberFunction signal)

返回对应给定 signal 的元方法,若 signal 并非信号,则返回无效的 QMetaMethod 对象。

范例:

\begin{lstlisting}[language=C++]
QMetaMethod destroyedSignal = QMetaMethod::fromSignal(&QObject::destroyed);
\end{lstlisting}


本函数在 Qt 5.0 中被引入。

bool QMetaMethod::invoke(QObject *object, Qt::ConnectionType connectionType, QGenericReturnArgument returnValue, 
   QGenericArgument val0 = QGenericArgument(nullptr), 
   QGenericArgument val1 = QGenericArgument(), 
QGenericArgument val2 = QGenericArgument(), 
QGenericArgument val3 = QGenericArgument(), 
QGenericArgument val4 = QGenericArgument(), 
QGenericArgument val5 = QGenericArgument(), 
QGenericArgument val6 = QGenericArgument(), 
QGenericArgument val7 = QGenericArgument(),
 QGenericArgument val8 = QGenericArgument(), 
 QGenericArgument val9 = QGenericArgument()) const

通过 object 对象动态调用本方法。若可被动态调用则返回 tue,若该对象无此方法或参数不匹配则返回 false。

该动态调用可以是同步或异步的,由 connectionType 决定:

\begin{compactitem}
\item 若 type 是 Qt::DirectConnection,则该方法会被立即执行。
\item 若 type 是 Qt::QueuedConnection,则会发送一个 QEvent ,该方法会在应用进入该对象所属线程的主事件循环后执行。
\item 若 type 是 Qt::AutoConnection,当 object 与调用者处于相同线程中时,该方法会被同步执行,否则会被异步执行。
\end{compactitem}
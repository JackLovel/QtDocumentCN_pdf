\chapter{Qt 中的多线程技术}

Qt 提供一系列的类与函数来处理多线程。Qt 开发者们可以使用下面四种方法来实现多线程应用。

\section{QThread: 低级 API 与可选的事件循环}

作为 Qt 进行线程控制的基石,每一个 QThread 实例都代表并控制着一个线程。

您可以直接实例化 QThread,或建立子类。实例化一个 QThread 将附带一个并行事件循环,
允许 QObject 槽函数在子线程执行。若子类化一个 QThread,程序可以在事件循环启动前初始化这个新线程;
或者在无事件循环下运行并行代码。

\begin{seeAlso}
QThread 类文档 以及示例代码 多线程范例 来了解如何使用 QThread。
\end{seeAlso}

\section{QThreadPool 与 QRunnable: 线程重用}

如果频繁地创建与销毁线程,资源开销将会非常大。为了减少这样额外的开销,可以重复使用一些现成的线程来执行新的任务。QThreadPool 就是这样一个保存着可重用的 QThead 的集合。

为了将代码放入 QThreadPool 的线程中运行,可以重写 QRunnable::run() 函数并实例化继承自 QRunnable 的子类。调用 QThreadPool::start() 函数可将 QRunnable 添加到 QThreadPool 的运行队列。一旦出现了一个可用的线程,它将会执行 QRunnable::run() 里的代码。

每一个 Qt 程序都会自带一个公共线程池,可以通过调用 QThreadPool::globalInstance() 来获取。公共线程池会自动维持着一定数量的线程,线程数为基于 CPU 核心数计算的最佳值。不过,您也可以显式创建并管理一个独立的 QThreadPool 。

\begin{seeAlso}
Qt Concurrent 模块文档以获取各个函数的详细信息。
\end{seeAlso}
    
\section{Qt Concurrent: 使用高级 API}

Qt Concurrent 模块提供了数个高级函数,用于处理一些常见的并行计算模式:map、filter 和 reduce。
不同于使用 QThread 与 QRunnable,这些高级函数不需要使用底层线程原语,比如互斥锁与信号量。
取而代之的是返回一个 QFuture 对象,它能够在传入的函数返回值就绪后检索该结果。
QFuture 既可以用来查询计算进度,也可以暂停/恢复/取消计算。
方便起见,QFutureWatcher 可以让您通过信号槽与 QFuture 进行交互。

Qt Concurrent 的 map、filter 和 reduce 算法会自动将计算过程分配到可用的处理器核心,
由此,当下编写的程序在以后部署到更多核心的系统上时会被自动扩展。

此模块还提供了 QtConcurrent::run() 函数,可以将任何函数在另一个线程中运行。
不过,QtConcurrent::run() 仅提供 map 、 filter 和 reduce 函数的一部分功能。
QFuture 可以用于检索函数返回值,也可以用于查看线程是否处于运行中。
然而,调用 QtConcurrent::run() 时只会使用一个线程,并且无法暂停/恢复/取消,也不能查询计算进度。



\section{WorkerScript: QML中的多线程}

QML 类型 WorkerScript 可将 JavaScript 代码与 GUI 线程并行运行。

每个 WorkerScript 实例可附加一个 .js 脚本。当调用 WorkerScript.sendMessage() 时,
脚本将会运行在一个独立的线程中(伴随一个独立的 QML 上下文)。
在脚本运行结束后,WorkerScript 将会向 GUI 线程发送回复,
后者会调用 WorkerScript.onMessage() 信号处理函数。

使用 WorkerScript,很像使用一个移入子线程工作的 QObject,数据通过信号槽在线程间进行传输。

\begin{seeAlso}
WorkerScript 文档以获得实现脚本的详细信息,以及能够在线程间传输的数据类型列表。
\end{seeAlso}

\section{选择合适的方法}

如上文所述,Qt 提供了开发多线程应用的不同解决方案。
对一个给定的程序,需要根据新线程的用途与线程的生命周期来决定正确的方案。
下面是一组 Qt 多线程技术的功能对比表,以及对于一些范例较为推荐的解决方案。


\section{解决方案对比}

\begin{longtable}{|l|l|l|l|l|l|}
\hline
特性  &	QThread &	QRunnable 与 QThreadPool 	& QtConcurrent::run() &	Qt Concurrent(Map, Filter, Reduce) & 	WorkerScript \\ 
\hline
语言 &	C++ &	C++ &	C++ &	C++ &	QML \\ 
\hline
可以指定线程优先级 & 	是 & 	是   & & & \\ 
\hline
可以在线程中运行运行事件循环 	&是 &&&& 		 \\ 
\hline
可以通过信号获取数据更新 	&是(通过 QObject 对象接收) & & & &				是(通过 WorkerScript 接收) \\ 
\hline
可以使用信号控制线程 	& 是(通过 QThread 接收) & & 			是(通过 QFutureWatcher 接收) &&	 \\ 
\hline
可以通过QFuture 监视线程 	&&		部分适用(译者注:仅可监视返回值,不可监视执行进度) 	&是 && 	 \\ 
\hline
内建的暂停/恢复/取消功能 		& & &		是 & &	 \\ 
\hline
\end{longtable}

\section{示例用例}


\begin{longtable}{|l|m{10em}|m{15em}|}
\hline
线程生命周期 	& 操作 	 & 解决方案   \\ 
\hline
单次调用 	& 在另一个线程中运行新的线性函数,可选地在运行期间进行更新进度。 &	Qt 提供不同解决方案:
1. 重写 QThread::run() 并将函数放入其中,启动 QThread ,发射信号来更新进度。
2. 重写 QRunnable::run() ,将函数放入其中。将 QRunnable 添加到 QThreadPool 中。修改线程安全的变量以更新进度;
3. 使用 QtConcurrent::run() 运行函数。修改线程安全变量以更新进度。\\ 
\hline
单次调用 &	在另一个线程中运行一个现成的函数,并取得它的返回值。 	& 使用 QtConcurrent::run() 运行函数。让 QFutureWatcher 在函数返回时发射 finish() 信号,并调用 QFutureWatcher::result() 来获取函数的返回值。 \\ 
\hline
单次调用 &	对一个容器的所有元素执行一个操作,使用所有可用的 CPU 核心。例如,从一组图片生成缩略图。 &	使用 Qt Concurrent 的 QtConcurrent::filter() 函数选择容器元素,并用 QtConcurrent::map() 函数对每个元素执行操作。要将输出合并为单个结果,可以使用 QtConcurrent:: filteredReduced() 和 QtConcurrent::mappedReduced()。 \\ 
\hline
单次调用/长期存在 &	在纯 QML 应用中执行耗时计算,并在结果就绪时更新 GUI 。 &	将计算代码放在 .js 脚本中,将它附加在 WorkerScript 实例上。调用 WorkerScript.sendMessage() 启动新线程执行计算。脚本也同样调用 sendMessage(),将结果传回 GUI 线程。在 onMessage 中处理结果并且更新 GUI 。 \\ 
\hline
长期存在 & 对象位于另一个线程中,可以根据请求执行不同的任务和/或接收要使用的新数据。 &	子类化一个 QObject 以创建一个工作类。实例化这个工作类对象与一个 QThread 。将工作类移入新线程。通过队列信号槽的连接,将命令与数据传递给工作类对象。 \\ 
\hline
长期存在 &	在另一个不需要接收任何信号或事件的线程中,重复执行资源开销大的操作。 & 	直接重写 QThread::run() 并创建死循环。启动无事件循环的线程。可以发送信号将数据送回 GUI 线程(译者注:发送信号不需要依赖事件循环,接收并执行槽函数才需要)。 \\
\hline
\end{longtable}
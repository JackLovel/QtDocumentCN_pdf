\chapter{QMetaObject 结构体}

QMetaObject 类包含了 Qt 对象的元信息。更多内容...。

\begin{tabular}{|r|l|}
	\hline
	属性 & 方法 \\
	\hline
	头文件 & \#include <QMetaObject>\\      
	\hline
	qmake & QT += core\\      
	\hline
\end{tabular}

\section{公共成员类型}


\begin{tabular}{|r|l|}   
\hline
类型	& 名称 \\ 
\hline
class	& Connection \\ 
\hline
\end{tabular}


\section{公共成员函数}

\begin{longtable}{|r|m{28em}|}   
\hline
QMetaClassInfo	& classInfo(int \emph{index}) const \\ 
\hline
int &	classInfoCount() const \\
\hline
int	&classInfoOffset() const\\
\hline
const char *	&className() const\\
\hline
QMetaMethod	&constructor(int index) const\\
\hline
int	&constructorCount() const\\
\hline
QMetaEnum	&enumerator(int index) const\\
\hline
int	&enumeratorCount() const\\
\hline
int	&enumeratorOffset() const\\
\hline
int&	indexOfClassInfo(const char *name) const\\
\hline
int	&indexOfConstructor(const char *constructor) const\\
\hline
int	&indexOfEnumerator(const char *name) const\\
\hline
int	&indexOfMethod(const char *method) const\\
\hline
int	&indexOfProperty(const char *name) const\\
\hline
int	&indexOfSignal(const char *signal) const\\
\hline
int&	indexOfSlot(const char *slot) const\\
\hline
bool	&inherits(const QMetaObject *metaObject) const\\
\hline
QMetaMethod	&method(int index) const\\
\hline
int	&methodCount() const\\
\hline
int	&methodOffset() const\\
\hline
QObject *	&newInstance(QGenericArgument val0 = QGenericArgument(nullptr), QGenericArgument val1 = QGenericArgument(), QGenericArgument val2 = QGenericArgument(), QGenericArgument val3 = QGenericArgument(), QGenericArgument val4 = QGenericArgument(), QGenericArgument val5 = QGenericArgument(), QGenericArgument val6 = QGenericArgument(), QGenericArgument val7 = QGenericArgument(), QGenericArgument val8 = QGenericArgument(), QGenericArgument val9 = QGenericArgument()) const\\
\hline
QMetaProperty	&property(int index) const\\
\hline
int&	propertyCount() const\\
\hline
int	&propertyOffset() const\\
\hline
const QMetaObject *	&superClass() const\\
\hline
QMetaProperty	&userProperty() const\\
\hline
\end{longtable}



\section{静态公共成员}

\begin{longtable}{|l|m{28em}|}
\hline
返回类型 &	函数 \\ 
\hline
bool	&checkConnectArgs(const char *signal, const char *method) \\
\hline
bool	&checkConnectArgs(const QMetaMethod \&signal, const QMetaMethod \&method) \\
\hline
void	&connectSlotsByName(QObject *object) \\
\hline
bool	&invokeMethod(QObject *obj, const char *member, Qt::ConnectionType type, QGenericReturnArgument ret, QGenericArgument val0 = QGenericArgument(nullptr), QGenericArgument val1 = QGenericArgument(), QGenericArgument val2 = QGenericArgument(), QGenericArgument val3 = QGenericArgument(), QGenericArgument val4 = QGenericArgument(), QGenericArgument val5 = QGenericArgument(), QGenericArgument val6 = QGenericArgument(), QGenericArgument val7 = QGenericArgument(), QGenericArgument val8 = QGenericArgument(), QGenericArgument val9 = QGenericArgument())\\
\hline
bool	&invokeMethod(QObject *obj, const char *member, QGenericReturnArgument ret, QGenericArgument val0 = QGenericArgument(0), QGenericArgument val1 = QGenericArgument(), QGenericArgument val2 = QGenericArgument(), QGenericArgument val3 = QGenericArgument(), QGenericArgument val4 = QGenericArgument(), QGenericArgument val5 = QGenericArgument(), QGenericArgument val6 = QGenericArgument(), QGenericArgument val7 = QGenericArgument(), QGenericArgument val8 = QGenericArgument(), QGenericArgument val9 = QGenericArgument())\\
\hline
bool	&invokeMethod(QObject *obj, const char *member, Qt::ConnectionType type, QGenericArgument val0 = QGenericArgument(0), QGenericArgument val1 = QGenericArgument(), QGenericArgument val2 = QGenericArgument(), QGenericArgument val3 = QGenericArgument(), QGenericArgument val4 = QGenericArgument(), QGenericArgument val5 = QGenericArgument(), QGenericArgument val6 = QGenericArgument(), QGenericArgument val7 = QGenericArgument(), QGenericArgument val8 = QGenericArgument(), QGenericArgument val9 = QGenericArgument())\\
\hline
bool	&invokeMethod(QObject *obj, const char *member, QGenericArgument val0 = QGenericArgument(0), QGenericArgument val1 = QGenericArgument(), QGenericArgument val2 = QGenericArgument(), QGenericArgument val3 = QGenericArgument(), QGenericArgument val4 = QGenericArgument(), QGenericArgument val5 = QGenericArgument(), QGenericArgument val6 = QGenericArgument(), QGenericArgument val7 = QGenericArgument(), QGenericArgument val8 = QGenericArgument(), QGenericArgument val9 = QGenericArgument())\\
\hline
bool	&invokeMethod(QObject *context, Functor function, Qt::ConnectionType type = Qt::AutoConnection, FunctorReturnType *ret = nullptr) \\
\hline
bool	&invokeMethod(QObject *context, Functor function, FunctorReturnType *ret) \\
\hline
QByteArray&	normalizedSignature(const char *method) \\
\hline
QByteArray&	normalizedType(const char *type)\\
\hline
\end{longtable}

\section{宏定义}

\begin{tabular}{|l|l|}
\hline
返回类型	& 宏定义 \\
\hline
QGenericArgument	& Q\_ARG(Type, const Type \&value) \\
\hline
QGenericReturnArgument&	Q\_RETURN\_ARG(Type, Type \&value) \\
\hline
\end{tabular}



\section{详细描述}

Qt 的元对象系统负责信号槽跨对象通信机制、运行时类型信息和 Qt 的属性系统。
应用中的每个 QObject 子类都有一个唯一的 QMetaObject 实例(译者注:与类一一对应,即同一个 QObject 子类的任意对象,都使用同一个 QMetaObject),其中保存了这个 QObject 子类的所有元信息,可以通过 QObject::metaObject() 获取。

QMetaObject 在应用编写中通常不需要,但在进行元编程时会非常有用,例如脚本引擎或者用户界面生成器。

最常用的成员函数有:

\begin{compactitem}
\item className() 返回类的名称。
\item superClass() 返回父类对应的元对象。
\item method() 和 methodCount() 提供关于一个类的元方法(信号、槽和其它可动态调用的成员方法)。
\item enumerator() 和 enumeratorCount() 提供一个类的枚举类型的信息。
\item propertyCount() 和 property() 提供一个类的属性的信息。
\item constructor() 和 constructorCount() 提供一个类的元构造函数的信息。
\end{compactitem}


索引函数 indexOfConstructor()、indexOfMethod()、indexOfEnumerator() 和 indexOfProperty() 将构造函数、成员函数、枚举类型和属性的名称映射为索引。例如,当连接信号槽时,Qt 内部使用 indexOfMethod() 进行检索。

类可以拥有一系列 名称—数值 格式的附加信息,保存在 QMetaClassInfo 对象中。信息条目数量可通过 classInfoCount()查询, classInfo()返回单条信息,也可通过 indexOfClassInfo() 检索信息条目。

\begin{notice}
元对象系统的操作通常是线程安全的,比如元对象是在编译期生成的静态只读实例。
然而,如果元对象在被程序动态修改了(如通过 QQmlPropertyMap),应用需要显示地同步对相关对象的访问。
\end{notice}

\begin{notice}[另请参阅]
QMetaClassInfo、QMetaEnum、QMetaMethod、QMetaProperty、QMetaType 和 Meta-Object System。
\end{notice}

\section{成员函数文档}

[static] bool QMetaObject::checkConnectArgs(const char \emph{*signal}, const char \emph{*method})

如果 signal 和 method 的参数能够匹配则返回 true,否则返回 false。

signal 和 method 都被假设是已经规范化的。

\begin{notice}[另请参阅]
normalizedSignature()。
\end{notice}

[static] bool QMetaObject::checkConnectArgs(const QMetaMethod \emph{\&signal}, const QMetaMethod \emph{\&method})

这是一个重载函数。

如果 signal 和 method 的参数能够匹配则返回 true,否则返回 false。

本函数在 Qt 5.0 中被引入。

QMetaClassInfo QMetaObject::classInfo(int \emph{index}) const

返回对应 index 的类型信息的元数据对象。

范例:

\begin{lstlisting}[language=C++]
class MyClass : public QObject
 {
     Q_OBJECT
     Q_CLASSINFO("author", "Sabrina Schweinsteiger")
     Q_CLASSINFO("url", "http://doc.moosesoft.co.uk/1.0/")

 public:
     ...
 };
\end{lstlisting}


%%%%%%%%%%%%%%%%

\begin{notice}[另请参阅]
classInfoCount()、classInfoOffset() 和 indexOfClassInfo()。
\end{notice}

int QMetaObject::classInfoCount() const

返回该类信息条目数量。

\begin{notice}[另请参阅]
classInfo()、classInfoOffset() 和 indexOfClassInfo()。
\end{notice}

int QMetaObject::classInfoOffset() const

返回类信息在该类中的偏移量,即第一条类信息的编号。

若该类没有包含类信息的父类,则偏移量为 0,否则偏移量是所有父类的类信息数量的总和。

\begin{notice}[另请参阅]
classInfo()、classInfoCount() 和 indexOfClassInfo()。
\end{notice}

const char *QMetaObject::className() const

返回该类的名称。

\begin{notice}[另请参阅]
superClass()。
\end{notice}

[static] void QMetaObject::connectSlotsByName(QObject \emph{*object})

递归检索 object 和所有子对象,将它们的信号连接至 object 中匹配的槽,匹配格式如下:

\begin{lstlisting}
void on_<对象名>_<信号名>(<信号参数>);
\end{lstlisting}

假设有一个对象名 为 button1 的 QPushButton 类型的子对象,则捕获它的 clicked() 信号的槽应为:

\begin{lstlisting}
void on_button1_clicked();
\end{lstlisting}

若 object 对象自身的名字已设置,则它自己的信号也会被连接至对应的槽。

\begin{notice}[另请参阅]
QObject::setObjectName().
\end{notice}

%%%%%%%%%%%
QMetaMethod QMetaObject::constructor(int \emph{index}) const

返回指定 index 的构造函数的元数据。

该函数在 Qt 4.5 中被引入。

\begin{notice}[另请参阅]
constructorCount() 和 newInstance()。
\end{notice}

int QMetaObject::constructorCount() const

返回此类的构造函数个数。

该函数在 Qt 4.5 中被引入。

\begin{notice}[另请参阅]
constructor() 和 indexOfConstructor()。
\end{notice}

QMetaEnum QMetaObject::enumerator(int \emph{index}) const

返回指定 index 的枚举类型的元数据。

\begin{notice}[另请参阅]
enumeratorCount()、enumeratorOffset() 和 indexOfEnumerator()。
\end{notice}

int QMetaObject::enumeratorCount() const

返回该类的枚举类型的个数。

\begin{notice}[另请参阅]
enumerator()、enumeratorOffset() 和 indexOfEnumerator()。
\end{notice}
	
int QMetaObject::enumeratorOffset() const

返回该类的枚举类型偏移量,即首个枚举变量的编号。

若该类没有包含枚举类型的父类,则偏移量为 0,否则偏移量是所有父类的枚举类型数量的总和。

\begin{notice}[另请参阅]
enumerator()、enumeratorCount() 和 indexOfEnumerator()。
\end{notice}

int QMetaObject::indexOfClassInfo(const char \emph{*name}) const

查找名为 name 的类型信息条目并返回其编号,未找到则返回-1。

\begin{notice}[另请参阅]
classInfo()、classInfoCount() 和 classInfoOffset()。
\end{notice}

int QMetaObject::indexOfConstructor(const char \emph{*constructor}) const

查找名为 constructor 的构造函数并返回其编号,未找到则返回-1。

注意:constructor 需要为规范化的格式,如 normalizedSignature() 的返回值。

该函数在 Qt 4.5 中被引入。

另请参阅:constructor()、constructorCount() 和 normalizedSignature()。

int QMetaObject::indexOfEnumerator(const char *name) const

查找名为 name 的枚举类型并返回其编号,未找到则返回-1。

另请参阅:enumerator(), enumeratorCount() 和 enumeratorOffset().

int QMetaObject::indexOfMethod(const char *method) const

查找名为 method 的方法并返回其编号,未找到则返回-1。

注意:method 需要为规范化的格式,如 normalizedSignature() 的返回值。

另请参阅:method()、methodCount()、methodOffset() 和 normalizedSignature()。

int QMetaObject::indexOfProperty(const char *name) const

查找名为 name 的属性并返回其编号,未找到则返回-1。

\begin{notice}[另请参阅]
property()、propertyCount() 和 propertyOffset()。	
\end{notice}

int QMetaObject::indexOfSignal(const char \emph{*signal}) const

查找名为 name 的信号并返回其编号,未找到则返回-1。

此方法与 indexOfMethod() 相似,区别是若该方法存在但并非信号函数,则会返回 -1。

\begin{notice}
signal 需要为规范化的格式,如 normalizedSignature() 的返回值。	
\end{notice}

\begin{notice}[另请参阅]
indexOfMethod()、normalizedSignature(), method()、methodCount() 和 methodOffset()。
\end{notice}

int QMetaObject::indexOfSlot(const char \emph{*slot}) const

查找名为 name 的槽并返回其编号,未找到则返回-1。

此方法与 indexOfMethod() 相似,区别是若该方法存在但并非槽函数,则会返回 -1。

\begin{notice}[另请参阅]
indexOfMethod()、method()、methodCount() 和 methodOffset()。
\end{notice}


bool QMetaObject::inherits(const QMetaObject \emph{*metaObject}) const

若该 QMetaObject 继承自 metaObject 描述的类型,则返回 true,否则返回 false。

一个类型被认为是继承自它自己的。

该函数在 Qt 5.7 中被引入。
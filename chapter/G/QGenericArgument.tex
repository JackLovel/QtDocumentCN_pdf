\chapter{QGenericArgument }

QGenericArgument 类

QGenericArgument 类是用于序列化参数的内部辅助类。\href{https://github.com/JackLovel/QtDocumentCN/blob/master/Src/G/QGenericArgument/QGenericArgument.md#%E8%AF%A6%E7%BB%86%E6%8F%8F%E8%BF%B0}{更多信息...}

\begin{tabular}{|l|l|}
\hline
属性& 	内容\\
\hline
头文件& 	\hl{\#include <QGenericArgument>}\\
\hline
qmake& 	\hl{QT += core}\\
\hline
被继承& 	QGenericReturnArgument\\
\hline
\end{tabular}


\splitLine

公共成员函数


\begin{tabular}{|l|m{30em}|}
\hline
返回类型& 	函数\\
\hline
	&QGenericArgument(const char *name = nullptr, const void * data =
   nullptr)\\
\hline
void * 	&data() const\\
\hline
const char *& 	name() const\\
\hline
\end{tabular}

\splitLine

详细描述

该类绝不应该被主动使用,请通过 Q\_ARG() 来使用。

\textbf{另请参阅}:Q\_ARG()、QMetaObject::invokeMethod() 和 QGenericReturnArgument。

\splitLine

成员函数描述

QGenericArgument::QGenericArgument(const char *name = nullptr, const void *data = nullptr)

通过给定的 name 和 data 构造 QGenericArgument 对象。

void *QGenericArgument::data() const

返回构造函数中设置的数据对象指针。

const char *QGenericArgument::name() const

返回构造函数中设置的名称。

%%% Local Variables:
%%% mode: latex
%%% TeX-master: "../../master"
%%% End: